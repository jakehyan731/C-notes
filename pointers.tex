\section{Pointers}

\subsection{Pointer arithmetic}

Как однажды сказал Дональд Кнут в интервью\cite{KnuthInterview1993}, то как в Си устроены указатели, это является
очень удачной инновацией в языках программирования по тем временам.

Простой пример:

\lstinputlisting{src/phonebook1.c}

Мы объяляем глобальный массив из структур. Если скомпилировать это в GCC с ключом -S либо в MSVC с ключом
/Fa, мы увидим листинг на ассемблере и то, как компилятор расположил эти строки. 

Расположил он их как линейный массив указателей на строки, вот так:

\begin{center}
\begin{tabular}{ | l | l | }
\hline
  ячейка 0    & ``Kirk'' \\
  ячейка 1    & ``Hammett'' \\
  ячейка 2    & ``555-1234'' \\
  ячейка 3    & ``Lars'' \\
  ячейка 4    & ``Ulrich'' \\
  ячейка 5    & ``555-5678'' \\
  ячейка 6    & ``James'' \\
  ячейка 7    & ``Hetfield'' \\
  ячейка 8    & ``555-1122'' \\
  ячейка 9    & ``Robert'' \\
  ячейка 10   & ``Trujillo'' \\
  ячейка 11   & ``555-7788'' \\
  ячейка 12   & NULL \\
  ячейка 13   & NULL \\
  ячейка 14   & NULL \\
\hline
\end{tabular}
\end{center}

Ф-ции dump1() и dump2() эквивалентны.

Но в первой итератор i начинается с 0 и к нему прибавляется 1 на каждой итерации.

Во второй ф-ции итератор i указывает на начало массива и затем, к нему прибавляется длина структуры 
(а не 1 байт, как можно ошибочно подумать),
таким образом, на каждой итерации, i указывает на следующий элемент массива.

