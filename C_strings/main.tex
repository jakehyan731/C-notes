\chapter{Строки в Си}

В Си нет встроенных возможностей для удобной работы со строками, такими, какие имеются в ЯП более
высокого уровня.

Часто жалуются на неудобную
конкатенацию строк (то есть, склеивание) в Си при помощи функции strcat(). Также, многих раздражает sprintf(),
под которых нельзя толком зараннее предсказать, сколько нужно выделять памяти. Копирование строк при помощи
strcpy() также неудобно --- нужно думать, сколько же выделить байт под буфер. Помимо всего прочего, неудобная
работа со строками в Си, это источник огромного количества уязвимостей в ПО, связанных с переполнениями буфера\cite[1.14.2]{REBook}.

Прежде всего, нужно задать себе вопрос, какие операции со строками нам нужны.
Конкатенация (склеивание) нужна чтобы 1) выдавать в лог сообщения; 2) конструировать строки и записывать их куда-то.

Для 1) можно использовать потоки (streams) --- не конструируя строку, выдавать её по порциям, например:

\begin{lstlisting}
printf ("Date: ");
dump_date(stdout, date);
printf (" a=");
dump_a(stdout, a);
printf ("\n");
\end{lstlisting}

Подобное заменяется в Си++ выводом в ostream:

\begin{lstlisting}
cout << "Date: " << Date_ToString(date) << " a=" << a_ToString(a) << "\n";
\end{lstlisting}

Так быстрее и меньше требуется памяти для конструирования строк.

Кстати, ошибкой является писать так:

\begin{lstlisting}
cout << "Date: " + Date_ToString(date) + " a=" + a_ToString(a) + "\n";
\end{lstlisting}

Для неспешного вывода в лог небольшого кол-ва сообщений это нормально, но если таких строк очень много, то будут
накладные расходы на их конкатенацию. \\
\\
Но все же строки иногда конструировать надо.

Есть какие-то библиотеки для этого.
К примеру, в Glib\footnote{\url{https://developer.gnome.org/glib/}} есть 
gstring.h\footnote{\url{https://github.com/GNOME/glib/blob/master/glib/gstring.h}}/
gstring.c\footnote{\url{https://github.com/GNOME/glib/blob/master/glib/gstring.c}}. 

\label{strbuf}
А в исходниках git можно найти strbuf.h\footnote{\url{https://github.com/git/git/blob/master/strbuf.h}}/
strbuf.c\footnote{\url{https://github.com/git/git/blob/master/strbuf.c}}. Собственно,
подобные Си-библиотеки очень похожи: они обеспечивают структуру данных, в которой есть некоторый буфер для строки, текущий размер буфера
и текущий размер строки в буфере. При помощи отдельных функций, можно добавлять новые строки или символы
в буфер, который, в свою очередь, будет автоматически увеличиваться или даже уменьшаться.

В \IT{strbuf.c} из git есть даже ф-ция \IT{strbuf\_addf()}, работающая как \IT{sprintf()}, 
но добавляющая строку-результат в буфер.

Так пользователь освобождается от головной боли связанной с выделением памяти.
При работе с этими библиотеками, практически невозможна ситуация переполнения буфера, если только не начать
работать со структурой самостоятельно.

Типичная последовательность работы с такими библиотеками, выглядит так:

\begin{itemize}
\item
Инициализация структуры strbuf или GString.

\item
Добавление строк и/или символов.

\item
Имеем сконструированную строку. Используем как обычную Си-строку, записываем куда-то в файл, передаем по сети, итд.

\item
Освобождаем структуру.
\end{itemize}

Кстати, конструирование строк чем-то напоминает 
Buffer\footnote{\url{http://docs.oracle.com/javase/7/docs/api/java/nio/Buffer.html}}, 
ByteBuffer\footnote{\url{http://docs.oracle.com/javase/7/docs/api/java/nio/ByteBuffer.html}} и 
CharBuffer\footnote{\url{http://docs.oracle.com/javase/7/docs/api/java/nio/CharBuffer.html}} в Java.

\section{Хранение длины строки}

Всегда хранить длину строки --- это было принято в реализациях ЯП Pascal. 
Не смотря на исходы святых войн\footnote{holy wars} между приверженцами Си и Pascal, все же, почти все библиотеки
для хранения строк и работы с ними, хранят также и текущую длину --- потому что удобства от этого перевешивают
необходимость пересчитывать это значение.

Например, \IT{strlen()} (подсчет длины строки) больше не нужен вообще, длина все время известна.
Конкатенация строк работает намного быстрее, потому что не нужно вычислять длину первой строки.
Ф-ция сравнения строк в самом начале может сравнить длины строк и если они не равны, тут же вернуть false,
не начиная сравнивание самих строк.

В Oracle RDBMS, в сетевых библиотеках, в функции работы со строками, зачастую передается строка и, 
отдельным аргументом, её длина\footnote{\url{http://blog.yurichev.com/node/64}}.
Это не очень эстетично, это выглядит избыточно, зато очень удобно.
Например, у нас есть некоторая ф-ция, которой нужно в начале узнать, какую строку ей передали:

\lstinputlisting{C_strings/strcmp1.c}

А вот если бы эта ф-ция имела длину входной строки, её можно было бы переписать так:

\lstinputlisting{C_strings/strcmp2.c}

Конечно, с эстетической точки зрения, код выглядит ужасно.
Тем не менее, мы здорово сократили количество необходимых сравнений строк! Вероятно, для тех ситуаций, когда 
нужно как можно быстрее обрабатывать текстовые строки, такой подход может улучшить ситуацию.

\section{Возврат строки}

Если некая ф-ция должна вернуть строку, имеются такие возможности:

\begin{itemize}
\item
Возврат строки-константы, это самое простое и быстрое.

\item
Возврат строки через глобальный массив символов. Недостаток: массив один и каждый вызов ф-ции перезаписывает
его содержимое.

\item
Возврат строки через буфер, заданный в аргументах ф-ции. Недостаток: нужно также передавать и длину буфера.

\item
Выделяем буфер нужного размера сами, записываем туда строку, возвращаем указатель. Недостаток: тратятся ресурсы
на выделение памяти.

\item
Записываем строку в уже рассмотренный strbuf или GString или иную другую структуру, указатель на которую был
передан в аргументах.

\end{itemize}

\subsection{Через строку-константу}

Первый вариант очень прост. Например:

\begin{lstlisting}
const char* get_month_name (int month)
{
	switch (month)
	{
	case 1: return "January";
	case 2: return "February";
	case 3: return "March";
	case 4: return "April";
	case 5: return "May";
	case 6: return "June";
	case 7: return "July";
	case 8: return "August";
	case 9: return "September";
	case 10: return "October";
	case 11: return "November";
	case 12: return "December";
	default: return "Unknown month!";
	};
};
\end{lstlisting}

Можно даже еще проще:

\begin{lstlisting}
const char* month_names[]={
	"January", "February", "March", "April", "May", "June", "July", "August",
	"September", "October", "November", "December" };

const char* get_month_name (int month)
{
	if (month>=1 && month<=12)
		return month_names[month-1];

	return "Unknown month!";
};
\end{lstlisting}

\subsection{Через глобальный массив символов}

Так делает стандартная ф-ция asctime(). Следует помнить, что нужно использовать возвращенную строку
перед каждым следующим вызовом asctime. 

Например, это правильно:

\begin{lstlisting}
printf("date1: %s\n", asctime(&date1));
printf("date2: %s\n", asctime(&date2));
\end{lstlisting}

А это нет:

\begin{lstlisting}
char *date1=asctime(&date1);
char *date2=asctime(&date2);
printf("date1: %s\n", date1);
printf("date2: %s\n", date2);
\end{lstlisting}

... ведь указатели date1 и date2 будут указывать на одно и то же место, и вывод printf() будет одинаковым. \\
\\
В git в hex.c\footnote{\url{https://github.com/git/git/blob/master/hex.c}} можно найти такое:

\lstinputlisting{C_strings/git_hex.c}

Строка возвращается фактически через глобальную переменную, объявление её как static внутри ф-ции просто напросто
обеспечивает доступ к ней только из этой ф-ции. Но вот недостаток: после вызова \IT{sha1\_to\_hex()} вы не можете
вызвать её повторно для получения второй строки до тех пор, пока не используете как-то первую, ведь она
затрется! Для того чтобы решить эту проблему здесь, по видимому, сделали сразу 4 буфера и каждый раз строка
возвращается в следующем. Но имейте ввиду --- так можно делать если только вы уверены в том что вы делаете,
это код на уровне ``грязного хака''.
Если вы
вызовете эту ф-цию 5 раз и вам нужно будет использовать как-то строку полученную при первом вызове, это может
привести к трудновыявляемой ошибке.

Кстати, обратите также внимание на то что переменная \IT{bufno} не инициализируется, потому что используются только 
2 младших её бита, к тому же, не важно, какое значение переменная будет содержать в самом начале!


\section{Определение строк}

\label{heredoc}
Малоизвестная возможность Си, длинные строки можно определять так:

\begin{lstlisting}
const char* long_line="line 1"
	"line 2"
	"line 3"
	"line 4"
	"line 5";

...

printf ("Some Utility v0.1\n"
	"Usage: %s parameters\n"
	"\n"
	"Authors:...\n", argv[0]);
\end{lstlisting}

Это отдаленно напоминает ``here document''\footnote{\url{https://en.wikipedia.org/wiki/Here_document}} в 
UNIX-шеллах и Perl.

\section{Стандартные ф-ции для работы со строками}

\subsection{strstr() и memmem()}

strstr() применяется для поиска строки в другой строке, либо чтобы узнать, есть ли там такая строка вообще.

memmem() можно применять с этими же целями, но для поиска по буферу, в котором могут быть нули,
либо по части строки.

\label{memchr}
\subsection{strchr() и memchr()}

strchr() применяется для поиска символа в строке, либо чтобы узнать, есть ли там такой символ вообще.

memchr() можно применять с этими же целями, но для поиска по части строки.

\subsection{atoi(), atof(), strtod(), strtof()}

Ф-ции atoi/atof не могут сигнализировать об ошибке, а strtod/strtof, делая то же самое --- могут.

\subsection{scanf(), fscanf(), sscanf()}

Извечный спор, что лучше, текстовые файлы или бинарные. С бинарными быстрее и проще работать, зато текстовые
легче просматривать и редактировать в обычном текстовом редакторе, к тому же, в UNIX имеется огрмоный арсенал
утилит для обработки текстов и строк. Но текстовые файлы нужно парсить.

Ф-ции scanf()\cite[7.19.6.2]{C99TC3} конечно же, регулярные выражения не поддерживают, 
однако при их помощи некоторые простые последовательности строк можно парсить. 

\subsection{Пример \#1}

Генерируемый ядром Linux файл \TT{/proc/meminfo}, начинается примерно так:

\begin{lstlisting}
MemTotal:        1026268 kB
MemFree:          119324 kB
Buffers:          170796 kB
Cached:           263736 kB
SwapCached:        11428 kB
...
\end{lstlisting}

Предположим, нам нужно узнать первое и третье число, игнорируя второе и остальные.
Так это можно сделать:

\begin{lstlisting}
void read_proc_meminfo()
{
	FILE *f=fopen("/proc/meminfo", "r");
	assert(f);
	unsigned result1, result2;
	if (fscanf (f, "MemTotal:\t%d kB\n"
			"MemFree:\t%*d kB\n"
			"Buffers:\t%d kB\n", 
			&result1, &result2)==2)
		printf ("results: %d %d\n", result1, result2);
	fclose(f);
};
\end{lstlisting}

Строка формата расходится на три строки, в реальности это одна\ref{heredoc}.
Обратите внимание на \TT{\textbackslash{}n}, так мы задаем перевод строки.

\TT{*} в модификаторе scanf-строки указывает что число будет прочитано, но никуда записано не будет.
Таким образом, это поле игнорируется. scanf()-функции возвращают кол-во не прочитанных полей (здесь
их будет 3) а кол-во записанных полей (2).

\subsection{Пример \#2}

Имеется текстовый файл с парами в каждой строке (ключ-значение):

\begin{lstlisting}
some_param1=some_value
some_param2=Lazy fox etc etc.
param3=Lorem Ipsum etc etc.
space here=value containing space
too long param, we should fail here=value
\end{lstlisting}

Нужно просто читать оба поля.

\begin{lstlisting}
int main(int argc, char *argv[])
{
	assert(argc==2);
	assert(argv[1]);
	FILE *f=fopen (argv[1], "r");
	assert(f);
	int line=1;
	do
	{
		char param[16];
		char value[60];
		if (fscanf (f, "%16[^=]=%60[^\n]\n", param, value)==2)
		{
			printf ("param=%s\n", param);
			printf ("value=%s\n", value);
		}
		else
		{
			printf ("error at line %d\n", line);
			return 0;
		};
		line++;
	} while (!feof(f) && !ferror(f));
};
\end{lstlisting}

\TT{\%16[\^{}=]} --- это отдаленно напоминает регулярные выражения. Означает, читать 16 любых символов, кроме
знака ``равно'' (=). Затем, мы указываем scanf()-у, что далее должен быть этот самый знак (=). Затем
пусть он читает 60 любых символов, кроме символа перевода строки. В конце читаем символ перевода строки.

Это работает, и поля ограничены длиной 16 и 60 символов. Поэтому на 5-й строке предсказуемо происходит ошибка,
ведь там длина парамера длиннее.

Так можно парсить несложные форматы, даже CSV
\footnote{Comma-separated values: \url{https://en.wikipedia.org/wiki/Comma-separated_values}}.

Однако, нельзя забывать о том что scanf()-функции не способны прочитать пустую строку там где задается \%s.
Поэтому, этим методом невозможно парсить файл с ключами-значениями, где есть отсутствующие ключи или значения.

\subsubsection{Засада \#1}

Если использовать \%d в строке формата, scanf() подразумевает что это 32-битный int. 

Ошибкой является подобное:

\begin{lstlisting}
char a[10];

scanf ("%d %d %d %d", &a[0], &a[1], &a[2], &[3]);
\end{lstlisting}

Символы (или байты) лежат ``в притык'' друг к другу. Когда scanf() будет обрабатывать первое значение, он будет считать
его за 32-битный int, и ``затрет'' остальные три, рядом лежащие. И так далее.


\subsection{strspn(), strcspn()}

strspn() часто применяется для того чтобы удостовериться, что некая строка полностью состоит из
нужных символов:
    
\begin{lstlisting}
if (strspn(s, "1234567890") == strlen(s)) ... OK
...
if (strspn(IPv4, "1234567890.") == strlen(IPv4)) ... OK
...
if (strspn(IPv6, "0123456789AaBbCcDdEeFf:.") == strlen(IPv6)) ... OK
\end{lstlisting}

Либо для того чтобы пропустить начало строки:

\begin{lstlisting}
const char *whitespaces = " \n\r\t";
*buf += strspn(*buf, whitespaces); // skip whitespaces at start
\end{lstlisting}

strcspn() это обратная ф-ция, её можно использовать для пропуска всех символов в начале строки, не попадающих
под множество символов:

\begin{lstlisting}
s += strcspn(s, whitespaces); // first, skip anything till whitespaces
s += strspn(s, whitespaces); // then skip shitespaces
\end{lstlisting}

\subsection{strtok() и strpbrk()}

Обе ф-ции служат для разбиения строки на подстроки, отделенные друг от друга разделительными символами
\footnote{delimiter}.
Только strtok() модифицирует исходную строку (и таким образом, подстроку сразу можно использовать
как отдельную Си-строку), а strpbrk() нет, он только возвращает указатель на следующую подстроку.



\section{Unicode}

Unicode это важно! Наиболее популярные способы его применения это:

\begin{itemize}
\item UTF-8
Популярно в UNIX-системах. Сильное приемущество: можно продолжать пользоваться стандартными ф-циями для
обработки строк.

\item UTF-16
Используется в Windows API.
\end{itemize}

\subsection{UTF-16}

Под каждый символ отводят 16-битный тип \IT{wchar\_t}.

Для объявления строк с таким типом, используется макрос L:

\begin{lstlisting}
L"hello world"
\end{lstlisting}

Для работы с wchar\_t вместо char, имеется целый класс функций-двойников с символом w в названии,
например: fwprintf(), wcscmp(), wcslen(), iswalpha().

\subsubsection{Windows}

В Windows, если некто хочет писать программу сразу в двух версиях, с использованием Unicode и без,
для этого есть тип tchar, в зависимости от объявленной переменной препроцессора UNICODE, 
он будет либо char либо wchar\_t\footnote{Сборка с Unicode и без была популярна во времена популярности
как Windows NT/2000/XP так и Windows 95/98/ME. Вторая линейка плохо поддерживала Unicode}.
Для этого же имеется макрос \TT{\_T(...)}:

\begin{lstlisting}
_T("hello world")
\end{lstlisting}

В зависимости от выставленной переменной препроцессора UNICODE, она будет объявлена как char либо wchar\_t.

В заголовочном файле tchar.h есть масса ф-ций, меняющих свою функцию в зависимости от этой переменной.

\section{Списки строк}

Самый простой список строк, это просто набор строк оканчивающийся нулем.
Например, в Windows API, в библиотеке Common Dialogs, 
так\footnote{\url{http://msdn.microsoft.com/en-us/library/windows/desktop/ms646829(v=vs.85).aspx}} 
передаются список допустимых расширений файлов для диалогового окна:

\begin{lstlisting}
// Initialize OPENFILENAME
ZeroMemory(&ofn, sizeof(ofn));
...
ofn.lpstrFilter = "All\0*.*\0Text\0*.TXT\0";
...

// Display the Open dialog box. 

if (GetOpenFileName(&ofn)==TRUE) 
	...
\end{lstlisting}

