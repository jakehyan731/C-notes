\section{Стандартные ф-ции для работы со строками}

\subsection{strstr() и memmem()}

strstr() применяется для поиска строки в другой строке, либо чтобы узнать, есть ли там такая строка вообще.

memmem() можно применять с этими же целями, но для поиска по буферу, в котором могут быть нули,
либо по части строки.

\label{memchr}
\subsection{strchr() и memchr()}

strchr() применяется для поиска символа в строке, либо чтобы узнать, есть ли там такой символ вообще.

memchr() можно применять с этими же целями, но для поиска по части строки.

\subsection{atoi(), atof(), strtod(), strtof()}

Ф-ции atoi/atof не могут сигнализировать об ошибке, а strtod/strtof, делая то же самое --- могут.

\subsection{scanf(), fscanf(), sscanf()}

Извечный спор, что лучше, текстовые файлы или бинарные. С бинарными быстрее и проще работать, зато текстовые
легче просматривать и редактировать в обычном текстовом редакторе, к тому же, в UNIX имеется огрмоный арсенал
утилит для обработки текстов и строк. Но текстовые файлы нужно парсить.

Ф-ции scanf()\cite[7.19.6.2]{C99TC3} конечно же, регулярные выражения не поддерживают, 
однако при их помощи некоторые простые последовательности строк можно парсить. 

\subsection{Пример \#1}

Генерируемый ядром Linux файл \TT{/proc/meminfo}, начинается примерно так:

\begin{lstlisting}
MemTotal:        1026268 kB
MemFree:          119324 kB
Buffers:          170796 kB
Cached:           263736 kB
SwapCached:        11428 kB
...
\end{lstlisting}

Предположим, нам нужно узнать первое и третье число, игнорируя второе и остальные.
Так это можно сделать:

\begin{lstlisting}
void read_proc_meminfo()
{
	FILE *f=fopen("/proc/meminfo", "r");
	assert(f);
	unsigned result1, result2;
	if (fscanf (f, "MemTotal:\t%d kB\n"
			"MemFree:\t%*d kB\n"
			"Buffers:\t%d kB\n", 
			&result1, &result2)==2)
		printf ("results: %d %d\n", result1, result2);
	fclose(f);
};
\end{lstlisting}

Строка формата расходится на три строки, в реальности это одна\ref{heredoc}.
Обратите внимание на \TT{\textbackslash{}n}, так мы задаем перевод строки.

\TT{*} в модификаторе scanf-строки указывает что число будет прочитано, но никуда записано не будет.
Таким образом, это поле игнорируется. scanf()-функции возвращают кол-во не прочитанных полей (здесь
их будет 3) а кол-во записанных полей (2).

\subsection{Пример \#2}

Имеется текстовый файл с парами в каждой строке (ключ-значение):

\begin{lstlisting}
some_param1=some_value
some_param2=Lazy fox etc etc.
param3=Lorem Ipsum etc etc.
space here=value containing space
too long param, we should fail here=value
\end{lstlisting}

Нужно просто читать оба поля.

\begin{lstlisting}
int main(int argc, char *argv[])
{
	assert(argc==2);
	assert(argv[1]);
	FILE *f=fopen (argv[1], "r");
	assert(f);
	int line=1;
	do
	{
		char param[16];
		char value[60];
		if (fscanf (f, "%16[^=]=%60[^\n]\n", param, value)==2)
		{
			printf ("param=%s\n", param);
			printf ("value=%s\n", value);
		}
		else
		{
			printf ("error at line %d\n", line);
			return 0;
		};
		line++;
	} while (!feof(f) && !ferror(f));
};
\end{lstlisting}

\TT{\%16[\^{}=]} --- это отдаленно напоминает регулярные выражения. Означает, читать 16 любых символов, кроме
знака ``равно'' (=). Затем, мы указываем scanf()-у, что далее должен быть этот самый знак (=). Затем
пусть он читает 60 любых символов, кроме символа перевода строки. В конце читаем символ перевода строки.

Это работает, и поля ограничены длиной 16 и 60 символов. Поэтому на 5-й строке предсказуемо происходит ошибка,
ведь там длина парамера длиннее.

Так можно парсить несложные форматы, даже CSV
\footnote{Comma-separated values: \url{https://en.wikipedia.org/wiki/Comma-separated_values}}.

Однако, нельзя забывать о том что scanf()-функции не способны прочитать пустую строку там где задается \%s.
Поэтому, этим методом невозможно парсить файл с ключами-значениями, где есть отсутствующие ключи или значения.

\subsubsection{Засада \#1}

Если использовать \%d в строке формата, scanf() подразумевает что это 32-битный int. 

Ошибкой является подобное:

\begin{lstlisting}
char a[10];

scanf ("%d %d %d %d", &a[0], &a[1], &a[2], &[3]);
\end{lstlisting}

Символы (или байты) лежат ``в притык'' друг к другу. Когда scanf() будет обрабатывать первое значение, он будет считать
его за 32-битный int, и ``затрет'' остальные три, рядом лежащие. И так далее.


\subsection{strspn(), strcspn()}

strspn() часто применяется для того чтобы удостовериться, что некая строка полностью состоит из
нужных символов:
    
\begin{lstlisting}
if (strspn(s, "1234567890") == strlen(s)) ... OK
...
if (strspn(IPv4, "1234567890.") == strlen(IPv4)) ... OK
...
if (strspn(IPv6, "0123456789AaBbCcDdEeFf:.") == strlen(IPv6)) ... OK
\end{lstlisting}

Либо для того чтобы пропустить начало строки:

\begin{lstlisting}
const char *whitespaces = " \n\r\t";
*buf += strspn(*buf, whitespaces); // skip whitespaces at start
\end{lstlisting}

strcspn() это обратная ф-ция, её можно использовать для пропуска всех символов в начале строки, не попадающих
под множество символов:

\begin{lstlisting}
s += strcspn(s, whitespaces); // first, skip anything till whitespaces
s += strspn(s, whitespaces); // then skip shitespaces
\end{lstlisting}

\subsection{strtok() и strpbrk()}

Обе ф-ции служат для разбиения строки на подстроки, отделенные друг от друга разделительными символами
\footnote{delimiter}.
Только strtok() модифицирует исходную строку (и таким образом, подстроку сразу можно использовать
как отдельную Си-строку), а strpbrk() нет, он только возвращает указатель на следующую подстроку.

