\section{-Wall}

Стоит ли постоянно держать включенным ключ \TT{-Wall} в GCC или \TT{/Wall} в MSVC, то есть, чтобы выводить
все возможные предупреждения (warnings)?
Да, однозначно стоит, так можно зараннее найти мелкие ошибки.
Можно даже в GCC включить \TT{-Werror} или \TT{/WX} в MSVC ~--- 
тогда предупреждения будут трактоваться как ошибки.

Вот простой пример:

\begin{lstlisting}
#include <stdio.h>

int f1(int a, int b, int c)
{
	printf ("(in %s) %d\n", __FUNCTION__, a*b+c);
	// return a*b+c; // OOPS, accidentally I forgot to add this
};

int main()
{
	printf ("(in %s) %d\n", __FUNCTION__, f1(123,456,789));
};
\end{lstlisting}

Автор ``забыл'' дописать return в ф-ции f1(.)
Тем не менее, GCC 4.8.1 компилирует этот пример молча.
При запуске мы увидим:

\begin{lstlisting}
(in f1) 56877
(in main) 14
\end{lstlisting}

Откуда взялось число $14$? Это то что вернула ф-ция printf() вызванная из f1(). Возвращаемые результаты
ф-ций интегральных типов\footnote{То есть, содержащих челочисленное число: int, short, char}
остаются в регистре EAX/RAX. В ф-ции main() берется значение из регистра EAX/RAX
и передается дальше во второй printf()
\footnote{Больше о том, как возвращаются результаты ф-ций через регистры, можно почитать в \cite{REBook}}.

Если компилировать с опцией -Wall, GCC скажет:

\begin{lstlisting}
1.c: In function 'f1':
1.c:7:1: warning: control reaches end of non-void function [-Wreturn-type]
 };
 ^
1.c: In function 'main':
1.c:12:1: warning: control reaches end of non-void function [-Wreturn-type]
 };
 ^
\end{lstlisting}

... хотя всё равно скомпилирует.

MSVC 2010 тоже генерирует код, работающий точно также, хотя и выводит предупреждение:

\begin{lstlisting}
...\1.c(7) : warning C4716: 'f1' : must return a value
\end{lstlisting}

Как видно, ошибка почти критическая, вызванная, можно сказать, опечаткой, но предупреждения компилятора
либо не видно вовсе, либо можно и не заметить.

