\section{\IFRU{Предупреждения компилятора}{Compiler warnings}}

\IFRU{Стоит ли постоянно держать включенным ключ}{Is it worth to turn on} \TT{-Wall} \InENRU GCC 
\OrENRU \TT{/Wall} \InENRU MSVC, \IFRU{то есть, чтобы выводить
все возможные предупреждения (warnings)}{in other words, to dump all possible warnings}?
\IFRU{Да, однозначно стоит, так можно зараннее найти мелкие ошибки}{Yes, it is worth to do it,
in order to determine quickly small errors}.
\IFRU{Можно даже в GCC включить}{In GCC it is even possible to turn on} \TT{-Werror} \OrENRU \TT{/WX} 
\InENRU MSVC ~--- 
\IFRU{тогда предупреждения будут трактоваться как ошибки}{then warning will be treated as errors}.

\IFRU{Вот простой пример}{Here is a simple example}:

\begin{lstlisting}
#include <stdio.h>

int f1(int a, int b, int c)
{
	printf ("(in %s) %d\n", __FUNCTION__, a*b+c);
	// return a*b+c; // OOPS, accidentally I forgot to add this
};

int main()
{
	printf ("(in %s) %d\n", __FUNCTION__, f1(123,456,789));
};
\end{lstlisting}

\IFRU{Автор ``забыл'' дописать \IT{return} в ф-ции f1()}
{The author ``forgot'' to add \IT{return} in f1() function}.
\IFRU{Тем не менее}{Nevertheless}, GCC 4.8.1 \IFRU{компилирует этот пример молча}{compiles this silently}.
\IFRU{При запуске мы увидим это}{After running we will see this}:

\begin{lstlisting}
(in f1) 56877
(in main) 14
\end{lstlisting}

\IFRU{Откуда взялось число $14$}{Where the $14$ number is came from}?
\IFRU{Это то что вернула ф-ция}{This is what returns the} printf() \IFRU{вызванная из}{called from} f1().
\IFRU{Возвращаемые результаты ф-ций}{Returned functions results of} 
\glslink{integral type}{\IFRU{интегральных типов}{integral types}}
\IFRU{остаются в регистре EAX/RAX}{are leaved in the EAX/RAX registers}.
\IFRU{В ф-ции main() берется значение из регистра EAX/RAX и передается дальше во второй}
{The value from the EAX/RAX register is taken in the main() function and then passed into the second} printf()
\footnote{
\IFRU{Больше о том, как возвращаются результаты ф-ций через регистры, можно почитать в}
{About how results are returned via registers, you may read more here}
\cite{REBook}}.

\IFRU{Если компилировать с опцией}{If to compile with the} \TT{-Wall}\IFRU{}{ option}, \ac{GCC} 
\IFRU{скажет}{will tell}:

\begin{lstlisting}
1.c: In function 'f1':
1.c:7:1: warning: control reaches end of non-void function [-Wreturn-type]
 };
 ^
1.c: In function 'main':
1.c:12:1: warning: control reaches end of non-void function [-Wreturn-type]
 };
 ^
\end{lstlisting}

... \IFRU{хотя всё равно скомпилирует}{but will compile anyway}.

MSVC 2010 \IFRU{генерирует код, работающий точно также, хотя и выводит предупреждение}
{generates the code working just the same, with warning, though}:

\begin{lstlisting}
...\1.c(7) : warning C4716: 'f1' : must return a value
\end{lstlisting}

\IFRU{Как видно, ошибка почти критическая, вызванная, можно сказать, опечаткой, но предупреждения компилятора
либо не видно вовсе, либо можно и не заметить}
{As you may see, the error is almost critical, caused by, it can be said, type, but there were no
compiler warning, or it was inconspicuous}.

