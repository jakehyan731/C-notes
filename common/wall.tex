\section{\IFRU{Предупреждения компилятора}{Compiler warnings}}

\IFRU{Стоит ли постоянно держать включенным ключ}{Is it worth to turn on} \TT{-Wall} \InENRU GCC 
\OrENRU \TT{/Wall} \InENRU MSVC, \IFRU{то есть, чтобы выводить
все возможные предупреждения (warnings)}{in other words, to dump all possible warnings}?
\IFRU{Да, однозначно стоит, так можно заранее найти мелкие ошибки}{Yes, it is worth to do it,
in order to determine quickly small errors}.
\IFRU{Можно даже в GCC включить}{In GCC it is even possible to turn on} \TT{-Werror} \OrENRU \TT{/WX} 
\InENRU MSVC ~--- 
\IFRU{тогда все предупреждения будут трактоваться как ошибки}
{then all warning will be treated as errors}.

\subsection{\IFRU{Пример}{Example} \#1}

\label{no_return}
\begin{lstlisting}
#include <stdio.h>

int f1(int a, int b, int c)
{
	printf ("(in %s) %d\n", __FUNCTION__, a*b+c);
	// return a*b+c; // OOPS, accidentally I forgot to add this
};

int main()
{
	printf ("(in %s) %d\n", __FUNCTION__, f1(123,456,789));
};
\end{lstlisting}

\IFRU{Автор ``забыл'' дописать \IT{return} в ф-ции f1()}
{The author ``forgot'' to add \IT{return} in f1() function}.
\IFRU{Тем не менее}{Nevertheless}, GCC 4.8.1 \IFRU{компилирует этот пример молча}{compiles this silently}.

\index{C99}
\IFRU{Это связано с тем что в стандарте Си}{It is because in the both C standard} 
(\cite[6.9.1/12]{C99TC3}) \IFRU{и в}{and in}
\CPP (\cite[6.6.3/2]{CPP11}) \IFRU{допустимо если ф-ция не возвращает значение}{is okay if a function
does not return a value when it should}.

\IFRU{При запуске мы увидим это}{After running we will see this}:

\begin{lstlisting}
(in f1) 56877
(in main) 14
\end{lstlisting}

\index{printf()}
\IFRU{Откуда взялось число $14$}{Where the $14$ number is came from}?
\IFRU{Это то что вернула ф-ция}{This is what returns the} printf() \IFRU{вызванная из}{called from} f1().
\IFRU{Возвращаемые результаты ф-ций}{Returned functions results of} 
\glslink{integral type}{\IFRU{интегральных типов}{integral types}}
\IFRU{остаются в регистре EAX/RAX}{are leaved in the EAX/RAX registers}.
\IFRU{В ф-ции main() берется значение из регистра EAX/RAX и передается дальше во второй}
{The value from the EAX/RAX register is taken in the main() function and then passed into the second} printf()
\footnote{
\IFRU{Больше о том, как возвращаются результаты ф-ций через регистры, можно почитать в}
{About how results are returned via registers, you may read more here}
\cite{REBook}}.

\IFRU{Если компилировать с опцией}{If to compile with the} \TT{-Wall}\IFRU{}{ option}, \ac{GCC} 
\IFRU{скажет}{will tell}:

\begin{lstlisting}
1.c: In function 'f1':
1.c:7:1: warning: control reaches end of non-void function [-Wreturn-type]
 };
 ^
1.c: In function 'main':
1.c:12:1: warning: control reaches end of non-void function [-Wreturn-type]
 };
 ^
\end{lstlisting}

... \IFRU{хотя всё равно скомпилирует}{but will compile anyway}.

MSVC 2010 \IFRU{генерирует код, работающий точно также, хотя и выводит предупреждение}
{generates the code running likewise, with warning, though}:

\begin{lstlisting}
...\1.c(7) : warning C4716: 'f1' : must return a value
\end{lstlisting}

\IFRU{Как видно, ошибка почти критическая, вызванная, можно сказать, опечаткой, но предупреждения компилятора
либо не видно вовсе, либо можно и не заметить}
{As you may see, the error is almost critical, caused by, it can be said, type, but there were no
compiler warning, or it was inconspicuous}.

\subsection{\IFRU{Пример}{Example} \#2}

\index{C99!bool}
\IFRU{В Си-стандарте}{In the} C99 \IFRU{появился тип}{standard, new type} \IT{bool}
\IFRU{, и согласно этому стандарту, он должен быть достаточным, чтобы
хранить там по крайней мере один бит}{ and according to standard, it should be big enough to store at least
one bit}.
\IFRU{В GCC bool это байт}{It is byte in GCC}.\\
\\
\IFRU{Если GCC не находит объявления некоей используемой ф-ции,
он по умолчанию считает его возвращаемый тип за \IT{int}, о чем
предупреждает}{If GCC cannot find declaration of some function we going to use, it considers its returning
type as \IT{int} by default and warns about it}.\\
\\
\IFRU{Далее, представим что имеем два файла}{Now let's consider we have two files}:

\begin{lstlisting}[caption=file1.c]
bool f1()
{
	...

	return cond ? true : false;
};
\end{lstlisting}

\begin{lstlisting}[caption=file2.c]
...

if (f1())
	do_something();

...
\end{lstlisting}

\IFRU{При компиляции file2.c, у GCC нет информации о f1() и он считает что возвращается \IT{int}}
{GCC has not information about f1() while compiling file2, so it considers its return type as \IT{int}}.
\IFRU{При компиляции file1.c GCC знает что нужно вернуть bool, но достаточно просто байт}
{GCC knows it should return \IT{bool} while compiling file1.c, but byte is enough}.
\IFRU{Переменные}{Variables of} \glslink{integral type}{\IFRU{интегральных типов}{integral types}}
\IFRU{возвращаются через регистр EAX или RAX x86-процессоров}
{are returned via EAX or RAX registers of x86-processors}
\footnote{\IFRU{о том как возвращаются переменные интегральных типов из ф-ций, можно прочитать 
еще здесь\cite[1.6]{REBook}}
{read more here \cite[1.6]{REBook} on how integral type variables are returned from functions}}, 
\IFRU{так что GCC генерирует код, который
выставляет только младший байт этого регистра (AL) в 1 или 0, а остальную часть регистра он может вообще
не трогать, и там может оставаться случайный мусор от исполнения другого кода}
{so GCC generates a code which can set only low byte of the register (AL) to 1 or 0 and do not touch the rest
regsiter part, so there might be random noise leaved from an other code execution}.
\IFRU{Так что сгенерированный код в f1() может возвращать false записывая 0 в младший байт регистра EAX/RAX,
тогда как в остальных битах будет мусор}
{So, the generated code of f1() may return false by writing 0 into the lowest byte of register EAX/RAX,
while other bits will contain noise}.
\IFRU{С точки зрения file2.c, где принимается что f1() возвращает \IT{int}, возвращаемое число может выглядеть так}
{From the point of view of file2.c where returning type of f1() is considered to be \IT{int}, the returning
value may looks like}:
\TT{0x??????00}, \IFRU{где}{where} \IT{?} ~--- \IFRU{случайные биты}{random bits}.
\IFRU{Поэтому, даже если f1() возвращает false, условие if() может срабатывать почти всегда}
{So even if when f1() returning false, if() condition may be triggered almost always}.\\
\\
\IFRU{Автору этих строк удалось найти такую ошибку только заглянув в отладчик и ассемблерные листинги этих ф-ций,
и пришлось потратить несколько часов}{This notes author once spent several hours for bug-hunting of such error,
and he had to dive into the debugger and assembly listings}.

