\label{typedef}
\subsubsection{typedef}

typedef вводит \IT{синоним} для типа. Часто это используется для структур, чтобы каждый раз не писать \IT{struct}
перед её именем, например:

\begin{lstlisting}
typedef struct _node
{
	node *prev;
	node *next;
	void *data;
} node;
\end{lstlisting}

Такого очень много в ``заголовочных'' файлах в Windows SDK (Windows API).

Тем не менее, \IT{typedef} также можно использовать не только для структур, но и для обычных, 
интегральных\footnote{приводимых к числу}, типов, например:

\begin{lstlisting}
typedef int age;
int compute_mean (age wife, age husband);

typedef int coord;
void draw (coord X, coord Y, coord Z);

typedef uint32_t address;
void write_memory (address a, size_t size, byte *buf);
\end{lstlisting}

Как видно, \IT{typedef} здесь может помочь в документировании исходного кода, так он легче читается.

Например, тип \IT{time\_t} (время в формате UNIX time, то что возвращает стандартная функция localtime(), 
например), это на самом деле
обычное 32-битное число, но этот тип объявленен в time.h обычно так:

\begin{lstlisting}
typedef long __time32_t;
\end{lstlisting}

Здесь вплне можно было бы использовать директиву препроцессора \TT{\#define} (многие так и делают),
но это хуже с точки зрения обработки ошибок во время компиляции.

\paragraph{Критика typedef}

Тем не менее, такие известные и опытные программисты как Линус Торвальдс, против использования typedef:
\cite{Torvalds:2002}.

