\subsection{union}

\TT{union} \IFRU{часто используется, когда в каком-то месте структуры нужно хранить разные типы на выбор}
{is often used when in some place of the structure one need to store various data types by choice}.
\IFRU{К примеру}{For example}:

\begin{lstlisting}
union
{
	int i; // 4 bytes
	float f; // 4 bytes
	double d; // 8 bytes
} u;
\end{lstlisting}

\IFRU{Такой}{Such} \TT{union} \IFRU{позволяет хранить одну из этих трех переменных на выбор}{allows to
store one of these variables by choice}.
\IFRU{Занимать он будет места столько же,
сколько максимальный элемент (double) ~--- 8 байт}{It will require the same ammount of space as a largest
element (double) ~--- 8 bytes}.

\TT{union} \IFRU{часто используют для обращения к какому-то типу данных как к другому}
{is often used as a way to access some data type as another data type}.

\IFRU{Например, как известно, каждый XMM-регистр в SSE может представлять собой 16 байт, 8 16-битных слов,
4 32-битных слова, 2 64-битных слова, 4 float-значения и 2 double-значения}{For example, as we know,
each XMM-register in SSE may be represented as 16 bytes, 8 16-bit words, 3 32-bit words, 2 64-bit words,
4 float variables and 2 double variables}.
\IFRU{Так можно описать его}{This is how it can be declared}:

\begin{lstlisting}
union
{
	double d[2];
	float f[4];
	uint8_t b[16];
	uint16_t w[8];
	uint32_t i[4];
	uint64_t q[2];
} XMM_register;

union XMM_register reg1;

reg.u.d[0]=123.4567;
reg.u.d[1]=89.12345;

// here we can use reg.u.b[...]

\end{lstlisting}

\IFRU{Это также очень удобно использовать вместе со структурой, где поля имеют битовую гранулярность}
{It is also handy to use it with a structure where fields has bit granularity}.
\IFRU{Как флаги x86-процессора}{As x86-CPU flags}:

\begin{lstlisting}
typedef struct _s_EFLAGS
{
    unsigned CF : 1;
    unsigned reserved1 : 1;
    unsigned PF : 1;
    unsigned reserved2 : 1;
    unsigned AF : 1;
    unsigned reserved3 : 1;
    unsigned ZF : 1;
    unsigned SF : 1;
    unsigned TF : 1;
    unsigned IF : 1;
    unsigned DF : 1;
    unsigned OF : 1;
    unsigned IOPL : 2;
    unsigned NT : 1;
    unsigned reserved4 : 1;
    unsigned RF : 1;
    unsigned VM : 1;
    unsigned AC : 1;
    unsigned VIF : 1;
    unsigned VIP : 1;
    unsigned ID : 1;
} s_EFLAGS;

typedef union _u_EFLAGS
{
    uint32_t flags;
    s_EFLAGS s;
} u_EFLAGS;
\end{lstlisting}

\IFRU{Можно таким образом загрузить флаги как 32-битное значение в поле \IT{flags},
а затем из поля \IT{(s)} обращаться к отдельным битам}
{Thus it possible to load flags as a 32-bit value into \IT{flags} field and then, from the \IT{(s)} field
to access specific bits}.
\IFRU{Либо наоборот, модифицировать биты, затем прочитать поле}{Or conversely, to modify bits
and then to read the} \IT{flags}\IFRU{}{ field}.
\IFRU{Такого очень много в исходниках ядра Linux}{A lot of such examples you may find in the Linux kernel}.

%NOTTRANSLATED
Еще один пример использования union для определения порядка байтов (endianness
\begin{lstlisting}
int is_big_endian(void)
{
    union {
        uint32_t i;
        char c[4];
    } bint = {0x01020304};

    return bint.c[0] == 1; 
}
\end{lstlisting}\footnote{пример взят отсюда: 
\url{http://stackoverflow.com/questions/1001307/detecting-endianness-programmatically-in-a-c-program}}

\subsubsection{tagged union}

\IFRU{Это}{It is} \TT{union} \IFRU{плюс флаг}{plus flag} (tag), \IFRU{определяющий тип}{defines type of} \TT{union}.
\IFRU{К примеру, если нам нужна какая-то переменная, которая может быть как числом,
так и числом с плавающей точкой, так и текстовой строкой}
{For example, if we need a variable which can be a number, a float point number, a text string} 
(\IFRU{как переменные в динамически-типизированных}{as in dynamically typed} \ac{PL}
\footnote{\IFRU{в Visual Basic это называется также}{in Visual Basic it also called} ``variant type''}),
\IFRU{то мы можем объявить такую структуру}{then we may declare such structure}:

\begin{lstlisting}

enum var_type
{
	INT,
	DOUBLE,
	STRING
};

struct
{
	enum var_type tag; // 4 bytes
	union
	{
		int i; // 4 bytes
		double d; // 8 bytes
		char *string; // 4 bytes (on 32-bit architecture)
	} u;
} variable;
\end{lstlisting}

\IFRU{Суммарная длина такой структуры будет}{The whole size of the structure is} $8+4=12$ 
\IFRU{байт}{byte}. \IFRU{В любом случае, это компактнее, чем выделять
поля для переменной каждого возможного типа}{It is much more compact than to allocate fields for the variable
of each type}.

\IFRU{Начиная с}{Beginning with} C11\cite{C11}, \TT{(u)} 
\IFRU{можно не указывать, это называется}{may not be specified, it is called} ``\IFRU{анонимный}{anonymous} union'':

\begin{lstlisting}
struct
{
	enum var_type tag; // 4 bytes
	union
	{
		int i; // 4 bytes
		double d; // 8 bytes
		char *string; // 4 bytes (on 32-bit architecture)
	};
} variable;
\end{lstlisting}

... \IFRU{и обращаться к полям union просто как к}{and to access them as} \TT{variable.i}, \TT{variable.d}, 
\IFRU{итд}{etc}.

\subsubsection{\IFRU{Тэггированные указатели}{tagged pointers}}

\IFRU{Вернемся к примеру объявления переменной, которая может быть как числом, так и числом с плавающей запятой,
так и текстовой строкой}{Let's back to the example of variable declaration, which can be a number,
a floating point number, and a text string}.
\IFRU{Самый большой тип}{Largest type} ~--- double (8 \IFRU{байт}{bytes}), 
\IFRU{следовательно, имея много таких блоков в памяти рядом, указатель на каждый
блок будет всегда выровнен по 8-байтной границе}{this means, by storing a lot of such blocks in memory
back to back, a pointer to each block will always be aligned on 8-byte border}.
\IFRU{И даже более того, в malloc() glibc всегда выделяет блоки
по 8-байтной границе}{Even more than that, glibc malloc() always allocates memory blocks by
8-byte border}. 
\IFRU{Следовательно, указатель на подобный блок всегда будет иметь нули в трех младших битах}
{Hence, the pointer to such block will always have zero in 3 lowest bits}.
\IFRU{А раз так, то эти младшие биты можно использовать для чего-то}
{And if so, these lowest bits may be used for something}.
\IFRU{Одна из возможностей, это хранить там тип}{One chance is to store there the type of} \IT{union}.
\IFRU{У нас всего три различных типа переменных, для хранения числа в диапазоне 0..2 нужно только 2 бита}
{We have only 3 variable types, so we need only 2 bits for storing a number in 0..2 range}.

\IFRU{Это называется тэггированный указатель (tagged pointer)}{This is what is called \IT{tagger pointer}}.
\IFRU{Так можно сэкономить на памяти и убрать из структуры \IT{enum} 
указывающий на тип}
{That is the way to make memory footprint smaller and to get rid of data type defining \IT{enum}
from the structure}.

\IFRU{Это очень популярно в LISP-интерпретаторах и компиляторах, потому что атомы в LISP-е это как раз вот такие
структуры, описывающие каждую переменную, их может быть очень много, и использование младших бит указателя
здорово экономит память}
{This approach is very popular in LISP-interpreters and compilers because LISP atoms is a similar
variable defining structures, a there are may be a lot of them in memory, so it can reduce memory footprint
by using lowest bits of pointers}.

\IFRU{Точно так же, любая другая информация может хранится в этих битах указателя}
{Likewise, any other information maybe stored in the tagger pointer bits}.

\IFRU{В качестве негативной стороны, нужно всегда помнить о том что указатель теперь не совсем указатель,
а содержит еще информацию}{As a negative side, one should always keep in mind that this is not an usual
pointer, but has more information}.
\IFRU{Отладчики не смогут работать с такими указателями корректно}{Debuggers will not be able to
work with such pointers correctly}.

