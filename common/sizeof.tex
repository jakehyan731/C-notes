\subsection{sizeof}

\index{sizeof()}
\IFRU{Обычно}{Usually}, sizeof() \IFRU{применяют к \glslink{integral type}{интегральным типам}}
{is applied to \glslink{integral type}{integral types}}
\IFRU{либо к структурам}{or to structures},
\IFRU{тем не менее, его можно применять и к массивам, к примеру}
{but nevertheless it is possible to apply it to arrays as well}:

\begin{lstlisting}
	char buf[1024];
	snprintf(buf, sizeof(buf), "...");
\end{lstlisting}

\index{snprintf()}
\IFRU{В противном случае, если указывать длину массива}{Otherwise, if to specify array length} ($1024$) 
\IFRU{в двух местах}{in both places} 
(\IFRU{в определении \TT{buf} и как второй аргумент \TT{snprintf()}}
{in \TT{buf} declaration and as a second argument of \TT{snprintf()}}),
\IFRU{то и изменять это значение придется каждый раз в обоих местах, а об этом легко забыть}
{then the value is have to be changed at the both places each time, and it is easy to forget about this}.

\index{wchar\_t}
\IFRU{Если нужны wide-строки, то}{If one need wide-strings, then} sizeof() 
\IFRU{можно применять к}{can be applied to} \IT{wchar\_t} 
(\IFRU{который, на самом деле, 16-битный тип данных \IT{short}}
{which is in turn, 16-bit data type \IT{short}}):

\begin{lstlisting}
	wchar_t buf[1024];
	swprintf(buf, sizeof(buf)/sizeof(wchar_t), "...");
\end{lstlisting}

sizeof() \IFRU{возвращает длину в байтах, так что здесь он будет равен}{returns the size in bytes, so it will
be here} $1024*2$, \IFRU{т.е.}{i.e.}, $2048$. 
\IFRU{Но мы можем разделить это значение на длину одного элемента массива}
{But we can divide this value by length of one array element} (\IT{wchar\_t})
\IFRU{в байтах ($2$)}{is $2$ in bytes},
\IFRU{чтобы получить количество элементов в массиве}{in order to get elements number in array} ($1024$).

sizeof() \IFRU{можно применять и к массивам структур}{can be applied to array of structures}:

\begin{lstlisting}
struct phonebook_entry
{
	char *name;
	char *surname;
	char *tel;
};

struct phonebook_entry phonebook[]=
{
	{ "Kirk", "Hammett", "555-1234" },
	{ "Lars", "Ulrich", "555-5678" },
	{ "James", "Hetfield", "555-1122" },
	{ "Robert", "Trujillo", "555-7788" }
};

void dump (struct phonebook_entry* input)
{
	for (int i=0; i<sizeof(phonebook)/sizeof(struct phonebook_entry); i++)
		printf ("%s %s - %s\n", input[i].name, input[i].surname, input[i].tel);
};
\end{lstlisting}

sizeof(phonebook) ~--- \IFRU{это размер всего массива структур в байтах}{is a size of the whole array
of structures in bytes}.
\TT{sizeof(struct phonebook\_entry)} ~--- \IFRU{это размер одной структуры в байтах}{is a size of one structure
in bytes}.
\IFRU{Делением мы узнаем количество структур в массиве}{By division we get number of structures in an array}.

