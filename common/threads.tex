\section{\IFRU{Треды}{Threads}}

\index{C++11}
\IFRU{В}{In the} C++11 \IFRU{ввели модификатор}{standard, a new} \IT{thread\_local} 
\IFRU{показывающий что каждый тред будет иметь свою версию этой переменной}
{modifier was added, showing that each thread will have its own version of the variable},
\IFRU{и её можно инициализировать, и она расположена в}{it can be initialized, and it is located in the} \ac{TLS}
%\footnote{
%\index{C11}
%\IFRU{В C11 также есть поддержка тредов, хотя и опциональная}
%{C11 also has thread support, optional though}}
:

\begin{lstlisting}[caption=C++11]
#include <iostream>
#include <thread>

thread_local int tmp=3;

int main()
{
	std::cout << tmp << std::endl;
};
\end{lstlisting}
\footnote{\IFRU{Компилируется в}{Compiled in} GCC 4.8.1, \IFRU{но не в}{but not in} MSVC 2012}

\IFRU{В исполняемом файле значение}{In the resulting executable file, the} \IT{tmp} 
\IFRU{будет именно в}{variable will be stored in the} \ac{TLS}.

\index{errno}
\IFRU{Это удобно например для хранения глобальных переменных вроде}
{It is useful for storing global variables like} \IT{errno}, 
\IFRU{которая не может быть одна для всех тредов}
{which cannot be one single variable for all threads}.

