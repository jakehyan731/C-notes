\subsection{switch}

\index{switch()}
\IFRU{Иногда можно устать писать одно и то же}{It is sometimes boring to write the same again and again}:

\begin{lstlisting}
switch(...)
{
	case 0:
	case 1:
	case 2:
	case 3:
		fn1();
		break;
	case 4:
	case 5:
	case 6:
	case 7:
		fn2();
		break;
};
\end{lstlisting}

\IFRU{А вот это нестандартное расширение GCC}{And this non-standard GCC extension}
\footnote{\url{http://gcc.gnu.org/onlinedocs/gcc/Case-Ranges.html}} \IFRU{может немного всё упростить}
{may make things somewhat simpler}:

\begin{lstlisting}
switch(...)
{
	case 0 ... 3:
		fn1();
		break;
	case 4 ... 7:
		fn2();
		break;
};
\end{lstlisting}

\IFRU{Так что если в планах имеется использовать только компилятор \ac{GCC}, то можно делать так}
{So if you plan to use only \ac{GCC} compiler, it is possible to do so}.

\subsubsection{\IFRU{Объявление переменных внутри}{Variable declarations inside} switch()}

\index{C99}
\IFRU{Этого делать нельзя, но зато можно открывать новый блок и объявлять их уже там (в \CPP или начиная с C99)}
{It is not possible, but it is possible to open a new block and to declare them there (in \CPP or starting from
C99)}:

\begin{lstlisting}
switch(...)
{
	case 0:
		{
			int x=1,y=2;
			fn1(x, y);
		};
		break;
	case 1:
	case 2:

	...
};
\end{lstlisting}

%TODO: fall through with examples.

