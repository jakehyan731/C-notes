\section{\IFRU{Ф-ция main()}{main() function}}

\IFRU{Стандартное определение}{Standard declaration}:

\begin{lstlisting}
int main(int argc, char* argv[], char* envp[])
\end{lstlisting}

\IT{argc} \IFRU{будет}{will be} 1 \IFRU{при отсутствии аргументов}{if no arguments present}, 
2 ~--- \IFRU{при одном аргументе}{if one argument}, 3 ~--- \IFRU{если два}{if two}, \IFRU{итд}{etc}.

\begin{itemize}
\item argv[0] ~--- \IFRU{имя текущей запущенной программы}{current running program name}.
\item argv[1] ~--- \IFRU{первый аргумент}{first argument}.
\item argv[2] ~--- \IFRU{второй аргумент}{second argument}.
\item \IFRU{итд}{etc}.
\end{itemize}

\IFRU{элементы в }{}\IT{argv} \IFRU{можно перечислять в цикле}{can be enumerated in loop}.
\IFRU{К примеру, программа может принимать список файлов в командной строке}
{For example, the program may take a file list in command line} (\IFRU{как это делает утилита}{like} UNIX \IT{cat} 
\IFRU{итд}{utility does, etc}).
\IFRU{Опции с дефисом в начале могут добавляться для отличия их от имен файлов}
{Dashed options may be supplied in order to distinguish them from file names}.

\IFRU{В аргументах \IT{main()}, \IT{envp[]} может быть пропущено, но и \IT{argc/argv[]}, и это корректно}
{Both \IT{envp[]} and \IT{argc/argv[]} can be omitted in the main() function argument list, and it is correct}.
\IFRU{Почему это корректно, можно прочитать в}{Read more here on why it is correct:} \cite[1.2.1]{REBook}.

\IFRU{Выражение return может быть пропущено начиная с}{Return clause can be omitted in functions as of}
C99 (\ref{no_return}) (\IFRU{тогда ф-ция}{then the} \IT{main()} 
\IFRU{будет возвращать 0}{function will return 0}
\footnote{\IFRU{это исключение из правил существует только для}{this rule exception is present only for}
\IT{main()}}).

\IFRU{Возвращаемое значение ф-ции}{Return value of} \IT{main()}
\IFRU{в итоге передается в ф-цию}{function is eventually passed to the} \IT{exit()} 
\IFRU{либо в}{function or} \IT{ExitProcess()} \InENRU win32.
\IFRU{Обычно это возвращаемый код ошибки, который можно проверять в шеллах, итд}
{It is usually a return error code which may be checked in command shells, etc}.
0 \IFRU{обычно означает успех, хотя, разумеется, автор сам может определять (и переопределять) свои возвращаемые 
коды}
{is usually means success, but of course, it is up to author to define (or redefine) its own return codes}.

