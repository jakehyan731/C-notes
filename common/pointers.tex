\subsection{\IFRU{Указатели}{Pointers}}

\IFRU{Как однажды сказал Дональд Кнут в интервью}{As Donald Knuth once said in the interview}
\cite{KnuthInterview1993}, \IFRU{то как в Си устроены указатели, это является
очень удачной инновацией в языках программирования по тем временам}{the way C handles pointers, was
a brilliant innovation at the time}.

\IFRU{Итак, определимся с терминологией}{So let us fix terminology}.
\IFRU{Указатель это просто адрес какого-то элемента в памяти}{A pointer is a just an address
of some element in memory}.
\IFRU{Указатели настолько популярны,
потому что в какую-то функцию намного проще передать просто адрес объекта в памяти, 
вместо того чтобы передавать весь объект ~--- ведь это абсурдно}
{The reason pointers are so popular is that an address of object is much easier to pass into a function
instead of passing the whole object ~--- because it is absurdly}.

\IFRU{К тому же, вызываемая функция, например, обрабатывающая ваш массив данных,
просто изменит что-то в нем, вместо того чтобы возвращать новый, измененный массив данных, что тоже абсурдно}
{Besides, calling function, for example, processing a data array, will just change something in it instead
of returning new one, which is absurdical too}.

\IFRU{Возьмем простой пример}{Let's take a simple example}.
\IFRU{Стандартная функция}{The standard C function} \IT{strtok()} 
\IFRU{делит строку на подстроки, используя заданный символ как разделитель}{just divide string by substrings
using specified character as delimiter}.
\IFRU{К примеру, мы можем подать на вход строку}{For example, we may specify the string}
\TT{The quick brown fox jumps over the lazy dog} 
\IFRU{и задать пробел в качестве разделителя}{and set the space as a delimiter}.

\lstinputlisting{common/strtok_ex1.c}

\IFRU{Мы в итоге получим на выходе}{What we got on output}:

\begin{lstlisting}
The
quick
brown
fox
jumps
over
the
lazy
dog
\end{lstlisting}

\IFRU{Что тут в реальности происходит, это то что ф-ция}
{What is going on here is that the} \IT{strtok()} 
\IFRU{просто находит в заданной строке следующий пробел}{just searching for the next space in the input string}
(\IFRU{либо иной заданный разделитель}{or any other delimiter set}),
\IFRU{записывает туда}{writes} $0$\IFRU{}{ to it} 
(\IFRU{что по соглашениям текстовых строк в Си является концом строки}
{this is string terminator by C conventions}) 
\IFRU{и возвращает указатель на это место}{and returns a pointer to that place}.

\IFRU{В качестве недостатка}{As a shortcoming, it can be said that the} \IT{strtok()} 
\IFRU{можно отметить, что эта ф-ция ``портит'' входную строку, записывая нули на месте разделителей}
{function ``garbles'' input string, writing zeros at the delimiter's places}.

\IFRU{Но вот что важно заметить: никакие строки или подстроки не копируются в памяти}
{What is worth to note: no strings or substrings copied in memory}.
\IFRU{Входная строка остается там же где и лежала}{The input string is still on its own place}.

\IFRU{В \IT{strtok()} передается только указатель на нее, или, её адрес}
{It is only pointer to the string (or its address) is passed to the \IT{strtok()} function}.

\IFRU{Эта ф-ция затем, после того как записывает $0$, возвращает \IT{адрес} каждого следующего ``слова''}
{The function then after it writes $0$, returns \IT{address} of each consecutive ``word''}.

\IFRU{Адрес ``слова'' затем подается на вход в}{The address of the ``word'' is then passed to the}
\IT{printf()}, 
\IFRU{где происходит его вывод на консоль}{where it dumped to the console}.

\IFRU{Обратите также внимание на то что в исходнике присутствует и некорректное определение \IT{str}}
{Please also take a note that an incorrect declaration of \IT{str} is present in the source code}.

\IFRU{Оно тем некорректное что в Си строка имеет тип}{It is incorrect in that sense that the C string
has type} \IT{const char*}, \IFRU{то есть, распологается в константном сегменте данных, 
защищенным от записи}{i.e., it is located in the constant data segment, write-protected}.

\IFRU{Если так сделать, то}{If do so, then the} \IT{strtok()} 
\IFRU{не сможет модифицировать входную строку записывая туда нули и процесс ``упадет''}
{will not be able modify the input string by writing zeros there and the process will crash}.

\IFRU{Так что, в нашем примере, строка выделяется как массив}
{So, in our example, the string is allocated as an array of} \IT{char}
\IFRU{а не массив}{instead of array of} \IT{const char}.

\IFRU{Обобщая, скажем что работа со строками в Си происходит только лишь используя адреса этих строк}
{Generalizing, we may say all standard C strings functions works with them using only their addresses}.

\IFRU{К примеру, ф-ция сравнения строк}{For example, the function of string comparison} \IT{strcmp()} 
\IFRU{берет на вход два адреса двух строк и по одному символу сравнивает их}
{takes addresses of two strings and compare them by one character}.
\IFRU{Было бы очень абсурдно копировать куда-то эти две строки лишний раз, чтобы}
{It would be absurdical to copy these strings to some other place so the} \IT{strcmp()} 
\IFRU{обработала их}{may process them}.

\IFRU{Трудность понимания указателей в Си связана с тем, что указатель это ``часть'' объекта}
{The difficulty of C pointers understanding is in the fact that pointer is a ``part'' of an object}.
\IFRU{Указатель на строку, это не сама строка}{The pointer to the string is not the string itselfs}.
\IFRU{Сама строка еще должна где-то в памяти хранится, под нее нужно перед этим выделять место, итд}
{The string should be placed somewhere in memory, a memory should be allocated for it before, etc}.

\IFRU{В}{In higher level} \ac{PL} 
\IFRU{более высокого уровня, 
объект и указатель на него могут быть представлены как единое целое, что облегчает понимание}
{an object and a pointer may be represented as a single whole, and that is makes understanding simpler}.

\IFRU{Впрочем, это не значит что в этих \ac{PL} строки и 
иные объекты неразумно копируются много раз при передаче в другие функции ~---
там точно так же как и в Си используются указатели, но просто эта механика скрыта от программиста}
{It is however not mean that a strings and other objects are copied misspendinly in these \ac{PL} ~---
a pointers are used there internally just as in C, and this mechanisms are hidden from the programmer}.

\subsubsection{\IFRU{Синтаксический сахар для}{Syntactic sugar for} array[index]}

\IFRU{Ради упрощения, можно сказать что в Си нет массивов вообще,
а есть только синтаксический сахар для выражений вроде}
{For the sake of simplification, it could be said that C has not arrays at all,
it has only syntactic sugar for expressions like} \IT{array[index]}.

\IFRU{К примеру, возможно вы видели такой трюк}{For example, perhaps you saw this trick}:

\begin{lstlisting}
printf ("%c", 3["hello"]);
\end{lstlisting}

\IFRU{Это выдаст}{It outputs} 'l'. 

\IFRU{Это происходит, потому что любое выражение}{This happens because the expression} \IT{a[i]}, 
\IFRU{на самом деле преобразовывается в}{is in fact translating into} \IT{*(a+i)}
\cite[6.5.2.1]{C99TC3}.
\IT{3["hello"]} \IFRU{преобразовывается в}{is translated into} \IT{*(3+"hello")},
\IFRU{а}{and} \IT{"hello"} \IFRU{это просто указатель на массив символов, типа}{is just
a pointer to array of characters like} \IT{const char*}.

\IT{3+"hello"} \IFRU{это в итоге указатель на часть строки, то есть}{as a result is a pointer
to the part of string}, \IT{"lo"}. \IFRU{А}{And} \IT{*("lo")} \IFRU{это cимвол}{is a} 'l'. 
\IFRU{Вот почему это работает}{That is why it works}.

\IFRU{Но так врядли стоит писать, если вы конечно не готовите программу на конкурс}
{It is not advisable to write such things unless your intentions is to participate in}
The International Obfuscated C Code Contest\footnote{\url{http://www.ioccc.org/}}.
\IFRU{Так что я привел этот пример, чтобы наглядно показать, 
что выражения вроде}{So I demostrated the trick here in order to explain that the} \IT{a[i]}
\IFRU{это синтаксический сахар}{is a syntactic sugar}.

\IFRU{При некотором упорстве, в Си вообще можно обойтись без индексации массивов,
хотя выглядеть это будет не очень эстетично}
{With some persistence, it is possible not to use indexed arrays in C at all, but it will not be
very aesthetical though}.

\IFRU{Кстати, так легко понять как работают отрицательные индексы массивов}
{By the way, now it is easy to understand how negative array indexes works}.
\IT{a[-3]} \IFRU{просто преобразуется в}{is translating into} \IT{*(a-3)}, 
\IFRU{так адресуется элемент лежащий перед самим массивом}{and that is how the element before
array itself is addressed}.
\IFRU{И хотя это вполне возможно, так можно делать только если вы точно знаете, что вы делаете}
{Despite it is possible, one should use this only if one exactly knows what one does}.

\IFRU{Еще один трюк связанный с негативными индексами: например, когда вы привыкли адресовать
массивы начиная не с 0, а с 1 (как в FORTRAN), тогда можно сделать такое}
{Another array negative indexes trick: for example, if you used to address arrays starting
not from 0 but from 1 (like in FORTRAN), then you may do something like this}:

\begin{lstlisting}
void f (int *a)
{
	a[1]=...; // first element
	a[2]=...; // second element
};

int main()
{
	int array[10];
	f(&a[-1]); // passing a pointer to the one int element before array
};
\end{lstlisting}

\IFRU{Хотя снова нельзя с уверенностью сказать что использование таких трюков оправдано}
{But again it is hard to say if the trick is justified}.

\IFRU{Так что в Си массив это, в каком-то смысле, это просто место в памяти под массив плюс указатель,
указывающий на него}{So C array in some sense is just a memory block plus a pointer to it}.

\IFRU{Вот почему имя массива в Си можно считать за указатель}{That is why array name in C may be treated
as a pointer}:

\IFRU{Если вы объявите глобальную переменную}{If to declare global variable} \IT{int a[10]},
\IFRU{то}{then} \IT{(a)} \IFRU{будет иметь тип}{will have the type} \IT{int*}.
\IFRU{Позже, когда где-то в коде вы укажете}{When further the following expression will be appeared:}
\IT{x=a[5]}, \IFRU{выражение будет преобразовано в}{the expression will be translated into} \IT{x=*(a+5)}.
\IFRU{От начала массива (то есть, первого элемента массива), будет отсчитано 5 элементов,
затем оттуда прочитается элемент для записи в}{From the array start (i.e., from the first array element)
5 elements will be counted, then the element will be read from that point for the storing it into} \IT{(x)}.

\label{PtrArith}
\subsubsection{\IFRU{Арифметика указателей}{Pointer arithmetic}}

\IFRU{Простой пример}{Simple example}:

\lstinputlisting{common/phonebook1.c}

Мы объявляем глобальный массив из структур.
Если скомпилировать это в GCC с ключом \IT{-S} либо в MSVC с ключом
\IT{/Fa}, мы увидим листинг на ассемблере и то, как компилятор расположил эти строки. 

Расположил он их как линейный массив указателей на строки, вот так:

\begin{center}
\begin{tabular}{ | l | l | }
\hline
  ячейка 0    & адрес строки ``Kirk'' \\
  ячейка 1    & адрес строки ``Hammett'' \\
  ячейка 2    & адрес строки ``555-1234'' \\
  ячейка 3    & адрес строки ``Lars'' \\
  ячейка 4    & адрес строки ``Ulrich'' \\
  ячейка 5    & адрес строки ``555-5678'' \\
  ячейка 6    & адрес строки ``James'' \\
  ячейка 7    & адрес строки ``Hetfield'' \\
  ячейка 8    & адрес строки ``555-1122'' \\
  ячейка 9    & адрес строки ``Robert'' \\
  ячейка 10   & адрес строки ``Trujillo'' \\
  ячейка 11   & адрес строки ``555-7788'' \\
  ячейка 12   & 0 \\
  ячейка 13   & 0 \\
  ячейка 14   & 0 \\
\hline
\end{tabular}
\end{center}

Ф-ции \IT{dump1()} и \IT{dump2()} эквивалентны.

Но в первой итератор \IT{i} начинается с 0 и к нему прибавляется 1 на каждой итерации.

Во второй ф-ции итератор \IT{i} указывает на начало массива и затем, к нему прибавляется длина структуры 
(а не 1 байт, как можно поначалу ошибочно подумать),
таким образом, на каждой итерации, \IT{i} указывает на следующий элемент массива.

\subsubsection{Указатели на функции}

Часто используются для callback-в.

Из-за того что можно напрямую задавать адрес функции, в embedded-программировании,
так можно сделать переход по нужному адресу:

\begin{lstlisting}
void (*func_ptr)(void) = (void (*)(void))0x12345678;
func_ptr();
\end{lstlisting}

Впрочем, нужно помнить, что это не совсем аналог безусловного перехода, потому что в стеке 
сохраняется адрес возврата.

