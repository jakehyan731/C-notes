\subsection{\IFRU{Операторы}{Operators}}

\subsubsection{==}

\IFRU{Очень неприятные ошибки возникают если в условии}
{Somewhat unpleasant mistakes may appear if in} \IT{if(a==3)} 
\IFRU{опечататься и написать}{condition become} \IT{if(a=3)}\IFRU{}{ in result of typo}.
\IFRU{Ведь выражение}{Because the statement} \IT{a=3} ``\IFRU{возвращает}{returns}'' 3,
\IFRU{а}{and} 3 \IFRU{это не}{is not a} 0, \IFRU{поэтому условие \IT{if()} всегда будет 
срабатывать}{so the \IT{if()} condition will always trigger}.

\IFRU{Раньше, для защиты от подобных ошибок, была мода писать наоборот}
{It was fashionable in past to protect from such mistakes by writing}: \IT{if(3==a)}, 
\IFRU{таким образом}{and thus},
\IFRU{если опечататься, выйдет}{we will get a} \IT{if(3=a)}\IFRU{, компилятор тут же выдаст ошибку}
{ in case of typo and the compiler will report error instantly}.

\IFRU{Тем не менее, в наше время, компиляторы обычно предупреждают если написать}
{Nevertheless, in modern times, compilers are usually warns if to write} \IT{if(a=3)}, 
\IFRU{так что, наверное, менять местами элементы выражения уже не обязательно}
{so elements swapping in conditions is probably not necessary these days}.

\subsubsection{Short-circuit evaluation 
\IFRU{и артефакт приоритетов операций}{and operator precedence artefact}}

\IFRU{Разберем что такое}{Let's see what is} \IT{short-circuit}
\IFRU{\footnote{дословный перевод на русский: ``короткое замыкание''}} \IT{evaluation}.

\IFRU{Это когда в выражении}{It is when in the expression} \IT{if(a \&\& b \&\& c)},
\IFRU{часть}{the part} \IT{(b)} \IFRU{будет вычисляться только если}{will be calculated only if} 
\IT{(a)} ~--- \IFRU{истинна}{is true},
\IFRU{а}{and} \IT{(c)}
\IFRU{будет вычисляться только если}{will be calculated only if} \IT{(a)} \AndENRU \IT{(b)} 
~--- \IFRU{оба истинны}{are both true}.
\IFRU{И вычисляться они будут именно в таком порядке, как указано}{and they will be computed
exactly in the same order as specified}.

\IFRU{Иногда можно встретить подобное}{Sometimes we can see expression like}: 
\IT{if (p!=NULL \&\& p->field==123)} ~--- \IFRU{и это совершенно правильно}{and this is completely
correct}.
\IFRU{Поле}{The field} \IT{field} \IFRU{в структуре, на которую указывает}{in the structure to which} 
\IT{(p)}\IFRU{, будет вычисляться только если указатель}{ points will be computed only if the pointer}
\IT{(p)} \IFRU{не равен}{not equals to} \IT{NULL}.

\IFRU{То же касается и операции}{The same story about} ``\IFRU{или}{or}'', 
\IFRU{если в выражении}{if in the expression} \IT{if (a || b || c)} \IFRU{подвыражение}{subexpression} 
\IT{(a)} \IFRU{будет ``истинно''}{will be ``true''},
\IFRU{остальные вычисляться не будут}{others will not be computed}.

\IFRU{Это может быть удобно для вызова нескольких ф-ций}{It is useful when one need to call several
functions}:
\IT{if (get\_flagsA() || get\_flagsB() || get\_flagsC())} ~--- 
\IFRU{если первая или вторая ф-ция вернет}{if first or second will return} \IT{true}, 
\IFRU{остальные даже не будут вызываться}{others will not be called at all}.

\IFRU{Эта особенность есть не только в Си/Си++}{This feature is not unique for C/C++}
\footnote{\IFRU{Здесь список}{Here is a list of} \ac{PL} \IFRU{где присутствует}{where}
\IT{short-circuit evaluation}\IFRU{}{ exist}\url{https://en.wikipedia.org/wiki/Short-circuit_evaluation}.
\IFRU{Кстати, хотя это и не про Си, но все же интересно}{It is not about C, but interesting nevertheless}:
\IFRU{в bash если писать}{if to write in bash} \IT{cmd1 \&\& cmd2 \&\& cmd3}, 
\IFRU{то каждая следующая команда будет исполняться только если предыдущая закончилась с успехом}
{then each next command will be executed only if the previous was executed with success}.
\IFRU{Это также}{It is also} \IT{short-circuit}.}.

\IFRU{Когда-то давно}{Some time ago}\cite{dmr:1995}, 
\IFRU{в языках B и BCPL (предтечи Си) не было операторов}{there was no operators} 
\IT{\&\&} \AndENRU \IT{||}\IFRU{}{ in B and BCPL (C precursors)}, 
\IFRU{но чтобы реализовать в них}{but in order to implement}
\IT{short-circuit evaluation}\IFRU{}{ in them}, 
\IFRU{приоритет операций}{the priority of the operators} \IT{\&} \AndENRU \IT{|} 
\IFRU{сделали больше, чем, например, у}{was made higher than in} \IT{\^} \OrENRU \IT{+}
\footnote{\IFRU{Приоритет операций в Си++}{C++ Operator Precedence}: \url{http://en.cppreference.com/w/cpp/language/operator_precedence}}.

\IFRU{Это позволяло писать что-то вроде}{That allowed to write something like} \IT{if (a==1 \& b==c)} 
\IFRU{используя}{while using} \IT{\&} \IFRU{вместо}{instead of} \IT{\&\&}.
\IFRU{Вот откуда взялся этот артефакт в приоритетах}{That is where that artefact came from}. \\
\\
\IFRU{Так что, нередкая ошибка это забывать о высоком приоритете этих операций и писать, например}
{So one often mistake is to forget about higher priority of these operators and to write, for example},
\IT{if (a\&1==0)}, \IFRU{в то время как это нужно брать в скобки}{which should be taken
in brackets}: \IT{if ((a\&1)==0)}.

\subsubsection{! \AndENRU \~{}}

\~{} (\IFRU{тильда}{tilde}) \IFRU{это побитовое инвертирование всех бит в значении}
{is a bitwise inversion of all bits in a value}.

\IFRU{Эта операция часто используется для инвертирования результатов действия ф-ций}
{The operation is often used for function results invertion}.
\IFRU{Например}{For example}, strcmp() \IFRU{в случае равенства строк возвращает}{in case of strings
equivalence, returns} 0.
\IFRU{Поэтому можно писать}{So we can write}:

\begin{lstlisting}
if (!strcmp(str1, str2))
{
	// do something in case of strings equivalence
};
\end{lstlisting}

... \IFRU{вместо}{instead of} \TT{if (strcmp (...)==0)}. \\
\\
\IFRU{Также, два подряд восклицательных знака применяется для трасформирования любого значения в тип bool
по правилу}{Also, two consecutive exclamation points can be used for
transforming any value into \IT{bool} type}: 0 ~--- false (0); \IFRU{не ноль}{not zero} ~--- true (1).

\IFRU{Например}{For example}:

\begin{lstlisting}
bool some_object_present=!!struct->object;
\end{lstlisting}

\IFRU{Или}{Or}:

\begin{lstlisting}
#define FLAG 0x00001000
bool FLAG_present=!!(value & FLAG);
\end{lstlisting}

\IFRU{А также}{And also}:

\begin{lstlisting}
bool bit_7_set=!!(value & (1<<7));
\end{lstlisting}

