\subsection{const}

Объявлять переменные, аргументы ф-ций и методы классов в Си++ как const очень полезно, потому что:

\begin{itemize}
\item
Документация кода --- сразу видно что это элемент только для чтения.

\item
Защита от ошибок: в случае с глобальной переменной-const, при попытке записать в нее, 
сработает защита и процесс упадет.
А если пытаться модифицировать const-аргумент ф-ции, компилятор выдаст ошибку.

\item
Оптимизация: компилятор, зная что элемент всегда ``только для чтения'', может сделать работу с ним эффективнее.
\end{itemize}

Желательно все аргументы ф-ции, которые вы не собираетесь модифицировать, объявлять как const.
Например, ф-ция strcmp() ничего не меняет во входных аргументах, так что их обычно оба объявляют как const.
А, например, strcat() ничего не меняет во втором аргументе, но меняет в первом, поэтому обычно она объявляется
с const во втором аргументе.

\subsubsection{\CPP}

В \CPP, желательно все методы класса, которые не изменяют ничего в объекте, 
также объявлять как const.

const-методы класса называют также аксессорами (accessors), 
а не-const-методы --- манипуляторами (manipulators)\cite{Lakos}.
