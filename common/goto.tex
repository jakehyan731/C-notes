\subsection{goto}

Использование оператора \IT{goto}\footnote{statement} считается плохим тоном и вредным вообще\cite{Dijkstra:1968:LEG:362929.362947}\cite{Dijkstra:1979:GSC:1241515.1241518}, 
тем не менее, использование его в разумных дозах\cite{Knuth:1974:SPG:356635.356640} может облегчить жизнь.

Частый пример, это выход из функции.

\begin{lstlisting}
void f(...)
{
	byte* buf1=malloc(...);
	byte* buf2=malloc(...);

	...

	if (something_goes_wrong_1)
		goto cleanup_and_exit;

	...
	
	if (something_goes_wrong_2)
		goto cleanup_and_exit;

	...

cleanup_and_exit:
	free(buf1);
	free(buf2);
	return;
};
\end{lstlisting}

Более сложный пример:

\begin{lstlisting}
void f(...)
{
	byte* buf1=malloc(...);
	byte* buf2=malloc(...);

	FILE* f=fopen(...);
	if (f==NULL)
		goto cleanup_and_exit;

	...

	if (something_goes_wrong_1)
		goto close_file_cleanup_and_exit;

	...
	
	if (something_goes_wrong_2)
		goto close_file_cleanup_and_exit;

	...

close_file_cleanup_and_exit:
	fclose(f);

cleanup_and_exit:
	free(buf1);
	free(buf2);
	return;
};
\end{lstlisting}

Если в данных примерах отказаться от \IT{goto}, то придется вызывать \IT{free()} и \IT{fclose()}
перед каждым выходом из функции (\IT{return}), что здорово замусорит весь код.

Использование \IT{goto} в таких случаях одобряется, например, в \cite{LinuxKernelCodingStyle}.

