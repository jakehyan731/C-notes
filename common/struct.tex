\subsection{struct}

\index{C99}
\IFRU{В}{In the} C99(\ref{C99}) \IFRU{можно инициализировать отдельные поля структур}{it is possible
to initialize specific structure fields}.
\IFRU{Пропущенные будут заполнены нулями}{Fields not set will be filled by zeroes}.
\IFRU{Такого очень много в ядре Linux}{A lot of such examples can be found in Linux kernel}.

\begin{lstlisting}
struct color
{
	int R;
	int G;
	int B;
};

struct color blue={ .B=255 };
\end{lstlisting}

\IFRU{И даже более того, можно создавать структуру прямо в аргументах ф-ции, например}
{And even more than that, it is possible to create a structure right in the function arguments, e.g.}:

\begin{lstlisting}
struct color
{
	int R;
	int G;
	int B;
};

void print_color_info (struct color *c)
{
	printf ("%d %d %d\n", c->R, c->G, c->B);
};

int main()
{
	print_color_info(&blue);
	print_color_info(&(struct color){ .G=255 });
};
\end{lstlisting}

\IFRU{Точно также стурктуру можно и возвращать из ф-ции}
{The structure is also can be returned from the function in the same way}:

\begin{lstlisting}
struct pair
{
	int a;
	int b;
};

struct pair f1(int a, int b)
{
	return (struct pair) {.a=a, .b=b};
};
\end{lstlisting}

\IFRU{Помимо всего прочего, о структурах также много есть в разделе}
{Read more about structures in the section} ``\COOPname''(\ref{COOP}).

\subsubsection{\IFRU{Расположение полей в структурах}{Structure fields placement}}

\IFRU{В современных x86-микропроцессорах (как Intel, так и AMD) имеется кеш-память разных уровней}
{In modern x86 CPUs (both Intel and AMD) a multi-level cache-memory present}.
\IFRU{Самая быстрая кеш-память (L1),
разделена на 64-байтные элементы (кеш-линии) 
и любое обращение к памяти заполняет сразу всю линию}
{The fastest cache-memory (L1) is divided by 64-byte elements (cache-lines) and any
memory access resulting in filling a whole line}\cite{AgnerFog}.

\IFRU{Можно сказать, что любое обращение к памяти (по выровненным адресам) подтягивает в кеш сразу 64 байта}
{It can be said that any memory access (on aligned boundary) fetches 64 bytes into cache at once}.

\IFRU{Поэтому, если некая структура данных имеет размер более 64-х байт, очень важно разделить её на две части:
наиболее востребованные поля и менее востребованные}
{So if a data structure is larger than 64 bytes, it is very important to divide it by 2 parts:
the most demanded fields and the less ones}.
\IFRU{Самые востребованные поля желательно разместить в пределах первых 64-х байт}
{It is desirable to place the most demanded fields in the first 64 bytes}.

\IFRU{Это же касается и классов в \CPP}{\CPP classes are also concerned}.

