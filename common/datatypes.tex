\subsubsection{\IFRU{Типы данных}{Datatypes}}

\paragraph{long double}

\index{float}
\index{double}
\index{long double}
\index{IEEE 754}
\IT{float} \IFRU{это}{is} 32-\IFRU{битное число в формате}{bit number in} IEEE 754\IFRU{}{ format},
\IT{double} \IFRU{это 64-битное, но FPU-сопроцессор в x86 оперирует 80-битными числами}
{is 64-bit variable but x86 FPU-coprocessor is in fact operating 80-bit numbers}.
\IFRU{Для них имеется тип}{There is another type for those:} \IT{long double}, 
\IFRU{он поддерживается в}{it is supported in the} \ac{GCC} \IFRU{но не в}{but not in} \ac{MSVC}.

\paragraph{int}

\IFRU{В}{The} \IT{int} 
\IFRU{обычно столько же бит сколько и разрядность регистров общего назначения процессора}
{usually occupies the same number of bits as general purpose CPU registers}.
\index{int}
\index{x86-84}
\IFRU{Однако, в}{However, in the} x86-64, 
\IFRU{для лучшей обратной совместимости, ширина}{for better backward compatibility, } \IT{int} 
\IFRU{так и осталась 32 бита}{width is still 32 bits}.

\paragraph{short, long \AndENRU long long}

\index{short}
\index{long}
\index{long long}
\IFRU{По крайней мере в}{At least in the} \ac{MSVC} \AndENRU \ac{GCC} \IT{short} 16-\IFRU{битный}{bit},
\IT{long} ~--- 32-\IFRU{битный}{bit},
\IFRU{а}{and} \IT{long long} ~--- 64-\IFRU{битный}{bit}. \\
\\
\index{stdint.h}
\index{C99}
\IFRU{Во избежании путанницы, в}{In order to avoid confusion,} \IT{stdint.h} \IFRU{(по крайней мере в C99) имеются типы}{(at least in C99) has new
types:} \IT{int8\_t}, \IT{uint8\_t},
\IT{int16\_t}, \IT{uint16\_t},
\IT{int32\_t}, \IT{uint32\_t},
\IT{int64\_t}, \IT{uint64\_t}.

\paragraph{bool}

\index{C++!bool}
\index{C99!bool}
bool \IFRU{есть в}{is present in the} \CPP, \IFRU{но также он есть и в Си, начиная с}
{but also in the C starting at} C99(\ref{C99}) (stdbool.h).

\IFRU{И в}{Both in the} MSVC \IFRU{и в}{and} GCC, bool \IFRU{занимает}{occupies} 1 \IFRU{байт}{byte}.

\index{Windows API!BOOL}
\IFRU{В Windows API принят тип BOOL, это синоним \IT{int}}
{There is a synonymous to the \IT{int} type in Windows API ~--- BOOL}.

\paragraph{\IFRU{Знаковые или беззнаковые}{Signed or unsigned}?}

\index{char}
\IFRU{Знаковые типы}{Signed types} (int, char) 
\IFRU{используются куда чаще беззнаковых}{are used much more often than unsigned} (unsigned int, unsigned char)
\footnote{\IFRU{Стандартном не закреплен тип}{The data type of} \IT{char}\IFRU{}{ is not fixed in the standard}, \IFRU{но в}{but in} \ac{GCC} \AndENRU \ac{MSVC} \IFRU{по умолчанию это именно знаковый тип}
{it is exactly signed type by default}.
\IFRU{При желании в}{In can be changed in the} GCC \IFRU{можно изменить это задав ключ}{by adding key}
\TT{-funsigned-char} \IFRU{а в}{and in} MSVC \IFRU{ключ}{key} \TT{/J}.}.

\IFRU{Но с точки зрения документации кода, если вы определяете переменную, которая никогда не будет хранить отрицательное
значение, в т.ч., индексы массивов, наверное лучше применять беззнаковый тип}
{However, in the sense of the code self-documenting, if you declare a variable which will not be assigned to
a negative value, including array indices, perhaps, unsigned type is better}.
\index{LLVM}
\IFRU{Например, в LLVM очень часто используется \IT{unsigned} там где обычно можно было бы использовать \IT{int}}
{For example, \IT{unsigned} type is often used in LLVM at the places where \IT{int} might be used}.

\IFRU{Если вы работаете с байтами, например, с байтами в памяти, то наверное лучше применять именно}
{If you work with bytes, e.g. with bytes in memory, then perhaps}
\IT{unsigned char}
\IFRU{}{ is better}.

\index{Integer overflow}
\IFRU{К тому же, это может немного защититься от ошибок связанных с}
{Aside from that, this may help protecting from the errors related to} integer overflow\cite{Phrack3C0A}.

\IFRU{В качестве очень простого примера}{As a simple example}:

\begin{lstlisting}
#define MAX_BUFFER_SIZE 1024

void f(int size)
{
	if (size>MAX_BUFFER_SIZE)
		die ("Too large!");
	void *p=malloc (size);
	...
};
\end{lstlisting}

\index{malloc()}
\IFRU{Если}{If} \IT{size} \IFRU{будет, например}{will be, for example}, $-1$, 
\IFRU{то}{then} malloc() \IFRU{вызовется с аргументом}{will be called with an argument}
\TT{0xffffffff} (\IFRU{это}{it is} $4294967295$).
\IFRU{Конечно, нужно было бы добавить вторую проверку}{Of course, we need to add a second sanitizing check}:
\IT{if (size<0)}, \IFRU{но такая проверка выглядит здесь абсурдной}{but such check here will have absurdical look}.

\IFRU{Таким образом, здесь нужно было бы применить}{So, the type} 
\IT{unsigned}\IFRU{, либо даже тип}{should be used here, maybe even} \IT{size\_t}. 
\IT{size\_t} \IFRU{определяет тип, достаточно большой, способный хранить размер любого,
сколько угодно большого блока памяти}{declares a big enough type able to store a size of any, big enough memory block}.
\IFRU{На 32-битных архитектурах это обычно}{It is} \IT{unsigned int}\IFRU{}{ on 32-bit architectures},
\IFRU{а на 64-битных это}{and} \IT{unsigned int64}\IFRU{}{ on 64-bit ones}.

\paragraph{char \OrENRU uint8\_t \IFRU{вместо}{instead of} int?}

\index{char}
\index{int}
\IFRU{Может показаться что если какая-то переменная всегда будет в пределах}
{One may think that is a value will always be in} $0..100$\IFRU{, то незачем выделять под нее 32-битный}
{ limits, then it is not necessary to allocate the whole 32-bit}
\IT{int}, \IFRU{а можно обойтись типом}{smaller types may be enough like} \IT{char} \OrENRU \IT{unsigned char}.
\IFRU{К тому же, такая переменная будет занимать в памяти в 4 раза меньше}{Besides, it will require
less memory}.

\IFRU{Это не так}{It is not so}.
\IFRU{Из-за выравнивания по 4-байтной границе (а в 64-битных архитектурах ~--- по 8-байтной), 
определяемые переменные типа \IT{char}, занимают столько же места сколько и}
{Because of aligning by 4-bytes border (or by 8-bytes border in 64-bit architectures),
the variables declared with the type \IT{char}, requires as much space as}
\IT{int}.

\IFRU{Конечно, компилятор мог бы отводить под}{Of course the compiler may allocate only 1 bytes for the}
\IT{char}\IFRU{ только один байт}{},
\IFRU{но тогда \ac{CPU} тратил бы больше времени на обращение к ``невыровненным'' по границе байтам}
{but then \ac{CPU} will spent more time for accessing ``unaligned'' by border bytes}.

\IFRU{Работа с отдельными байтами может быть ``дороже'' и медленнее чем работа с 32-битными или 64-битными 
значениями потому что
регистры \ac{CPU} обычно имеют ту же ширину что и разрядность процессора}
{Specific bytes processing may be more ``expensive'' and slower then processing 32-bit or 64-bit values
because \ac{CPU} registers are usually has the same width as CPU bits}.
\index{RISC}
\index{ARM}
\IFRU{И даже более того}{Even more than that}, \ac{RISC}-\IFRU{процессоры}{units}
(\IFRU{например ARM}{e.g. ARM})
\IFRU{вообще могут быть неспособны работать с отдельными байтами внутренне,
потому что имеют только 32-битные регистры}
{may not be able to work with specific bytes internally at all because they have only 32-bit registers}.

\IFRU{Таким образом, если вы раздумываете над типом для локальной переменной, то \IT{int/unsigned int} может быть лучше}
{So if you considering about type for the local variable, \IT{int/unsigned int} may be better}.

\IFRU{С другой стороны, переменные каких типов лучше использовать в структурах}
{On the other hand, which types are better suited for a structures}?
\IFRU{Это вопрос поиск баланса между скоростью и компактностью}
{This is a question of seeking balance between speed and compactness}.
\IFRU{С одной стороны, можно отвести}{On the one hand, one may use} \IT{char} 
\IFRU{под небольшие переменные, под флаги, под enum, итд, но не следует
забывать, что доступ к этим переменным будет чуть медленнее}{for a small variables, flags, bitfields, enums, etc,
but one should not forget that access to these variables will be slower}.
\IFRU{С другой стороны, под все переменные можно отводить}{On the other hand, if to assign}
\IT{int}\IFRU{, тогда работа со структурой будет быстрее, но она будет занимать в памяти больше места}
{ to each variables, working with a structure will be faster, but it will require more space in memory}.

\IFRU{Например}{For example}:

\begin{lstlisting}
struct
{
	char some_flags; // 1 byte
	void* ptr; // 4/8 bytes, offset: +1
} s;
\end{lstlisting}

\IFRU{Если скомпилировать это с упаковкой полей по 1-байтной границе, 
то доступ к}{If to compile this with structure packing by 1-byte border, access to the} \IT{some\_flags}
\IFRU{в памяти будет возможно даже быстрее чем доступ к}{in the memory may be even faster then to} \IT{ptr}, 
\IFRU{потому что первое поле выровнено по 4-байтной границе, а второе нет}
{because the first field is aligned by 4-byte border, while the second is not}.

\IFRU{А если компилировать это с упаковкой полей по умолчанию, то компилятор отведет под первое поле 4 байта и смещение
у второго поля будет +4}{If to compile this by default structure packing, then 4 bytes will be allocated for 
the first field and the offset of the second field will be +4}.

\IFRU{Резюмируя: если компактность и экономия памяти для вас важнее скорости, тогда нужно использовать}
{Summarizing: if compactness and memory footprint is important, then}
\IT{char}, \IT{uint16\_t}, \IFRU{итд}{etc, may be used}.

\paragraph{x86-64 \OrENRU AMD64}

\index{x86-64}
\IFRU{На новых 64-битных x86-процессорах, тип}{On the new 64-bit x86 CPUs, the} \IT{int/unsigned int} 
\IFRU{оставили 32-битным, вероятно, в целях совместимости}{is still 32-bit, perhaps, for compatibility}.
\IFRU{Так что если вы хотите использовать 64-битные значения, можно использовать}
{So if one need 64-bit variables, one may use} \IT{uint64\_t} \OrENRU \IT{int64\_t}.

\IFRU{А указатели теперь, конечно, 64-битные}{But pointers, of course, has 64-bit width}.
