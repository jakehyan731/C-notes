\section{\IFRU{Объявления в Си/Си++}{C/C++ declarations}}
\subsection{Объявления переменных внутри ф-ции}

Раньше, в Си можно было объявлять переменные только в начале ф-ции. А в Си++ --- где угодно.

К тому же, нельзя было объявлять итератор в for() (а в Си++ также можно было):

\begin{lstlisting}
for(int i=0; i<10; i++)
	...
\end{lstlisting}

Новый стандарт C99\ref{C99} позволяет делать это.

\subsubsection{static}

Обычно глобальные переменные (или ф-ции) объявляются как static, так их область видимости ограничивается 
данным файлом. Но локальные переменные внутри ф-ции также можно объявлять как static, тогда эта переменная
будет не локальной, а глобальной, но её область видимости будет ограничена только этой ф-цией.

Например:

\begin{lstlisting}
void fn(...)
{
	for(int x=0; x<100; x++)
	{
		static int times_executed = 0;
		times_executed++;
	}
};
\end{lstlisting}

К примеру, это помогло бы для реализации strtok(), ведь этой ф-ции что-то нужно хранить у себя между вызовами.

\label{forwarddeclaration}
\subsection{forward declaration}

\IFRU{Как известно, в заголовочных файлах (headers) обычно содержатся декларации ф-ций, то есть, 
имя ф-ции, аргументы и типы, тип возвращаемого значения, но нет тела ф-ций}
{As it is well-known, in the header files (headers) function declarations are usually present,
i.e., function names, arguments and types, returning type, but no function bodies}.
\IFRU{Так делается для того, чтобы компилятор мог знать, с чем имеет дело,
не углубляясь в тонкости реализации ф-ций}{This is done for the compiler so it will have information,
what it is working with, without delving into the intricacies of function implementations}.

\IFRU{То же самое можно делать и для типов}{The same can be done for types}.
\IFRU{Для того чтобы не включать при помощи \#include файл с описаниями
какого либо класса в другой заголовочный файл, можно обойтись указанием, что он вообще существует}
{In order not to include with the help of \#include the file with a class definitions into
the other header file, one can just declare the type presence}.

\IFRU{Например, вы работаете с комплексными числами и у вас где-то есть такая структура}
{For example, you work with complex numbers and you have a such structure somewhere}:

\begin{lstlisting}
struct complex
{
	double real;
	double imag;
};
\end{lstlisting}

\IFRU{И, например, она определена в файле}{And let's say it is defined in the file} my\_complex.h.

\IFRU{Безусловно, вам нужно включить этот файл, если вы собиретесь работать с переменными типа 
\IT{complex}, с отдельными полями структуры}
{Of course, one should include the file if one have intention to work with variables of \IT{complex} type
and specific structure fields}.
\IFRU{Но если вы описываете свои ф-ции для работы с этой структурой в отдельном заголовочном файле,
то включать там}
{But if you declare your functions using the structure in other header file, you may not include}
my\_complex.h \IFRU{не обязательно, компилятору достаточно просто знать что \IT{complex} это структура}
{there, compiler just needs to know that the \IT{complex} is a structre}:

\begin{lstlisting}
struct complex;

void sum(struct complex *x, struct complex *y, struct complex *out);
void pow(struct complex *x, struct complex *y, struct complex *out);
\end{lstlisting}

\IFRU{Это позволяет увеличить скорость компиляции, а также бороться с циркулярными зависимостями, когда
в двух модулях используются типы и ф-ции друг друга}
{This may speed up the compilation process and also solve circular dependencies, when two modules
uses functions and type of each other}.

 % subsection

\subsection{C++11: auto}

Пользуясь \ac{STL}, иногда надоедает каждый раз объявлять тип итератора вроде:

\begin{lstlisting}
for (std::list<int>::iterator it=list.begin(); it!=list.end(); it++)
\end{lstlisting}

Тип it вполне можно получить из list.begin(), поэтому, в начиная со стандарта C++11, можно использовать auto:

\begin{lstlisting}
for (auto it=list.begin(); it!=list.end(); it++)
\end{lstlisting}

