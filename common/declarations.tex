\subsection{\IFRU{Определения в Си/Си++}{C/C++ declarations}}
\subsubsection{\IFRU{Определения локальных переменных}{Local variable declarations}}

\IFRU{Раньше, в Си можно было определять переменные только в начале ф-ции}
{It was possible to declare variables only at the function beginning in C}.
\IFRU{А в}{And anywhere in} \CPP\IFRU{ ~--- где угодно}{}.

\IFRU{К тому же, нельзя было определять счетчик или \glslink{iterator}{итератор} в}{It was also not possible to declare
counter or \gls{iterator} in} for() (\IFRU{а в}{it was possible in} \CPP\IFRU{ также можно было}{}):

\begin{lstlisting}
for(int i=0; i<10; i++)
	...
\end{lstlisting}

\index{C99}
\IFRU{Новый стандарт}{The new} C99(\ref{C99}) \IFRU{позволяет делать это}{ standard allows this}.

\paragraph{static}

\IFRU{Если глобальные переменные (или ф-ции) определяются как \IT{static}, так их область видимости ограничивается 
данным файлом}{If the global variables (or functions) are declared as \IT{static}, its scope is limited
by current file}.
\IFRU{Но локальные переменные внутри ф-ции также можно определять как}{However, local variables
inside a function may also be declared as} \IT{static}, \IFRU{тогда эта переменная
будет не локальной, а глобальной, но её область видимости будет ограничена только этой ф-цией}
{then this variable will be global instead of local, but its scope will be limited by a function}.

\IFRU{Например}{For example}:

\begin{lstlisting}
void fn(...)
{
	for(int x=0; x<100; x++)
	{
		static int times_executed = 0;
		times_executed++;
	}
};
\end{lstlisting}

\IFRU{К примеру, это помогло бы для реализации}
{For example, it might be helpful for the} \TT{strtok()}\IFRU{, ведь этой ф-ции что-то нужно
хранить у себя между вызовами}{ implementing, because this function should store something between calls}.

\label{forwarddeclaration}
\subsection{forward declaration}

\IFRU{Как известно, в заголовочных файлах (headers) обычно содержатся декларации ф-ций, то есть, 
имя ф-ции, аргументы и типы, тип возвращаемого значения, но нет тела ф-ций}
{As it is well-known, in the header files (headers) function declarations are usually present,
i.e., function names, arguments and types, returning type, but no function bodies}.
\IFRU{Так делается для того, чтобы компилятор мог знать, с чем имеет дело,
не углубляясь в тонкости реализации ф-ций}{This is done for the compiler so it will have information,
what it is working with, without delving into the intricacies of function implementations}.

\IFRU{То же самое можно делать и для типов}{The same can be done for types}.
\IFRU{Для того чтобы не включать при помощи \#include файл с описаниями
какого либо класса в другой заголовочный файл, можно обойтись указанием, что он вообще существует}
{In order not to include with the help of \#include the file with a class definitions into
the other header file, one can just declare the type presence}.

\IFRU{Например, вы работаете с комплексными числами и у вас где-то есть такая структура}
{For example, you work with complex numbers and you have a such structure somewhere}:

\begin{lstlisting}
struct complex
{
	double real;
	double imag;
};
\end{lstlisting}

\IFRU{И, например, она определена в файле}{And let's say it is defined in the file} my\_complex.h.

\IFRU{Безусловно, вам нужно включить этот файл, если вы собиретесь работать с переменными типа 
\IT{complex}, с отдельными полями структуры}
{Of course, one should include the file if one have intention to work with variables of \IT{complex} type
and specific structure fields}.
\IFRU{Но если вы описываете свои ф-ции для работы с этой структурой в отдельном заголовочном файле,
то включать там}
{But if you declare your functions using the structure in other header file, you may not include}
my\_complex.h \IFRU{не обязательно, компилятору достаточно просто знать что \IT{complex} это структура}
{there, compiler just needs to know that the \IT{complex} is a structre}:

\begin{lstlisting}
struct complex;

void sum(struct complex *x, struct complex *y, struct complex *out);
void pow(struct complex *x, struct complex *y, struct complex *out);
\end{lstlisting}

\IFRU{Это позволяет увеличить скорость компиляции, а также бороться с циркулярными зависимостями, когда
в двух модулях используются типы и ф-ции друг друга}
{This may speed up the compilation process and also solve circular dependencies, when two modules
uses functions and type of each other}.

 % subsection
\subsubsection{\IFRU{Частые ошибки}{Frequent caveats}}

\IFRU{Чтобы объявить два указателя на \IT{char}, можно, по инерции, написать}
{In order to declare two pointers to \IT{char}, one may write by inertion}:

\begin{lstlisting}
char* p1, p2;
\end{lstlisting}

\IFRU{Это не верно, потому что компилятор распознает это описание так}
{It is not correct, because the compiler treat this declaration as}:

\begin{lstlisting}
char *p1, p2;
\end{lstlisting}

... \IFRU{и определяет указатель на}{and declares the pointer to} \IT{char}
\IFRU{и просто}{and just} \IT{char}.

\IFRU{Правильно так}{This one is correct}:

\begin{lstlisting}
char *p1, *p2;
\end{lstlisting}

\subsection{const}

Объявлять переменные, аргументы ф-ций и методы классов в Си++ как const очень полезно, потому что:

\begin{itemize}
\item
Документация кода ~--- сразу видно что это элемент только для чтения.

\item
Защита от ошибок: в случае с глобальной переменной-const, при попытке записать в нее, 
сработает защита и процесс упадет.
А если пытаться модифицировать const-аргумент ф-ции, компилятор выдаст ошибку.

\item
Оптимизация: компилятор, зная что элемент всегда ``только для чтения'', может сделать работу с ним эффективнее.
\end{itemize}

Желательно все аргументы ф-ции, которые вы не собираетесь модифицировать, объявлять как const.
Например, ф-ция strcmp() ничего не меняет во входных аргументах, так что их обычно оба объявляют как const.
А, например, strcat() ничего не меняет во втором аргументе, но меняет в первом, поэтому обычно она объявляется
с const во втором аргументе.

\subsubsection{\CPP}

В \CPP, желательно все методы класса, которые не изменяют ничего в объекте, 
также объявлять как const.

const-методы класса называют также аксессорами (accessors), 
а не-const-методы ~--- манипуляторами (manipulators)\cite{Lakos}.

\subsection{\IFRU{Типы данных}{Datatypes}}

\subsubsection{bool}

bool \IFRU{есть в}{is present in the} \CPP, \IFRU{но также он есть и в Си, начиная с}
{but also in the C starting at} C99\ref{C99} (stdbool.h).

\IFRU{В Windows API принят тип BOOL, это синоним \IT{int}}
{There are synonymous to the \IT{int} type in Windows API ~--- BOOL}.

\subsubsection{\IFRU{Знаковые или беззнаковые}{Signed or unsigned}?}

\IFRU{Знаковые типы}{Signed types} (int, char) 
\IFRU{используются куда чаще беззнаковых}{are used much more often than unsigned} (unsigned int, unsigned char).

\IFRU{Но с точки зрения документации кода, если вы объявляете переменную, которая никогда не будет хранить отрицательное
значение, в т.ч., индексы массивов, наверное лучше применять беззнаковый тип}
{However, in the sense of the code self-documenting, if you declare a variable which will not be assigned to
a negative value, including array indices, perhaps, unsigned type is better}.
\IFRU{Например, в LLVM очень часто используется \IT{unsigned}}
{For example, \IT{unsigned} type is often used in LLVM}.

\IFRU{Если вы работаете с байтами, например, с байтами в памяти, то наверное лучше применять именно}
{If you work with bytes, for example, with bytes in memory, then perhaps}
\IT{unsigned char}
\IFRU{}{ is better}.

\IFRU{К тому же, это может немного защититься от ошибок связанных с}
{Aside from that, this may help protecting from the errors related to} integer overflow\cite{Phrack3C0A}.

\IFRU{В качестве очень простого примера}{As a simple example}:

\begin{lstlisting}
#define MAX_BUFFER 1024

void f(int size)
{
	if (size>MAX_BUFFER_SIZE)
		die ("Too large!");
	void *p=malloc (size);
	...
};
\end{lstlisting}

\IFRU{Если}{If} \IT{size} \IFRU{будет, например}{will be, for example}, $-1$, 
\IFRU{то}{then} malloc() \IFRU{вызовется с аргументом}{will be called with an argument}
\TT{0xffffffff} (\IFRU{это}{it is} $4294967295$).
\IFRU{Конечно, нужно было бы добавить вторую проверку}{Of course, we need to add a second sanitizing check}:
\IT{if (size<0)}, \IFRU{но такая проверка выглядит здесь абсурдной}{but such check here will have absurdical look}.

\IFRU{Таким образом, здесь нужно было бы применить}{So, the type} 
\IT{unsigned}\IFRU{, либо даже тип}{should be used here, maybe even} \IT{size\_t}. 
\IT{size\_t} \IFRU{определяет тип, достаточно большой, способный хранить размер любого,
сколько угодно большого блока памяти}{defines a big enough type able to store a size of any, big enough memory block}.
\IFRU{На 32-битных архитектурах это обычно}{It is} \IT{unsigned int}\IFRU{}{ on 32-bit architectures},
\IFRU{а на 64-битных это}{and} \IT{unsigned int64}\IFRU{}{ on 64-bit ones}.

\subsubsection{char \OrENRU uint8\_t \IFRU{вместо}{instead of} int?}

\IFRU{Может показаться что если какая-то переменная всегда будет в пределах}
{One may think that is a value will always be in} $0..100$\IFRU{, то незачем выделять под нее 32-битный}
{ limits, then it is not necessary to allocate the whole 32-bit}
\IT{int}, \IFRU{а можно обойтись типом}{smaller types may be enough like} \IT{char} \OrENRU \IT{unsigned char}.
\IFRU{К тому же, такая переменная будет занимать в памяти в 4 раза меньше}{Besides, it will require
less memory}.

\IFRU{Это не так}{It is not so}.
\IFRU{Из-за выравнивания по 4-байтной границе (а в 64-битных архитектурах ~--- по 8-байтной), 
определяемые переменные типа \IT{char}, занимают столько же места сколько и}
{Because of aligning by 4-bytes border (or by 8-bytes border in 64-bit architectures),
the variables declared with the type \IT{char}, requires as much space as}
\IT{int}.

\IFRU{Конечно, компилятор мог бы отводить под}{Of course the compiler may allocate only 1 bytes for the}
\IT{char}\IFRU{ только один байт}{},
\IFRU{но тогда \ac{CPU} тратил бы больше времени на обращение к ``невыровненным'' по границе байтам}
{but then \ac{CPU} will spent more time for accessing ``unaligned'' by border bytes}.

\IFRU{Работа с отдельными байтами может быть ``дороже'' и медленнее чем работа с 32-битными или 64-битными 
значениями потому что
регистры \ac{CPU} обычно имеют ту же ширину что и разрядность процессора}
{Specific bytes processing may be more ``expensive'' and slower then processing 32-bit or 64-bit values
because \ac{CPU} registers are usually has the same width as CPU bits}.
\IFRU{И даже более того}{Even more than that}, \ac{RISC}-\IFRU{процессоры}{units}
(\IFRU{например ARM}{ARM for example})
\IFRU{вообще могут быть неспособны работать с отдельными байтами внутренне,
потому что имеют только 32-битные регистры}
{may not be able to work with specific bytes internally at all because they have only 32-bit registers}.

\IFRU{Таким образом, если вы раздумываете над типом для локальной переменной, то \IT{int/unsigned int} может быть лучше}
{So if you considering about type for the local variable, \IT{int/unsigned int} may be better}.

\IFRU{С другой стороны, переменные каких типов лучше использовать в структурах}
{On the other hand, which types are better suited for a structures}?
\IFRU{Это вопрос поиск баланса между скоростью и компактностью}
{This is a question of seeking balance between speed and compactness}.
\IFRU{С одной стороны, можно отвести}{On the one hand, one may use} \IT{char} 
\IFRU{под небольшие переменные, под флаги, под enum, итд, но не следует
забывать, что доступ к этим переменным будет чуть медленнее}{for a small variables, flags, bitfields, enums, etc,
but one should not forget that access to these variables will be slower}.
\IFRU{С другой стороны, под все переменные можно отводить}{On the other hand, if to assign}
\IT{int}\IFRU{, тогда работа со структурой будет быстрее, но она будет занимать в памяти больше места}
{ to each variables, working with a structure will be faster, but it will require more space in memory}.

\IFRU{Например}{For example}:

\begin{lstlisting}
struct
{
	char some_flags; // 1 byte
	void* ptr; // 4/8 bytes, offset: +1
} s;
\end{lstlisting}

\IFRU{Если скомпилировать это с упаковкой полей по 1-байтной границе, 
то доступ к}{If to compile this with structure packing by 1-byte border, access to the} \IT{some\_flags}
\IFRU{в памяти будет возможно даже быстрее чем доступ к}{in the memory may be even faster then to} \IT{ptr}, 
\IFRU{потому что первое поле выровнено по 4-байтной границе, а второе нет}
{because the first field is aligned by 4-byte border, while the second is not}.

\IFRU{А если компилировать это с упаковкой полей по умолчанию, то компилятор отведет под первое поле 4 байта и смещение
у второго поля будет +4}{If to compile this by default structure packing, then 4 bytes will be allocated for 
the first field and the offset of the second field will be +4}.

\IFRU{Резюмируя: если компактность и экономия памяти для вас важнее скорости, тогда нужно использовать}
{Summarizing: if compactness and memory footprint is important, then}
\IT{char}, \IT{uint16\_t}, \IFRU{итд}{etc, may be used}.

\subsubsection{x86-64 \OrENRU AMD64}

\IFRU{На новых 64-битных x86-процессорах, тип}{On the new 64-bit x86 CPUs, the} \IT{int/unsigned int} 
\IFRU{оставили 32-битным, вероятно, в целях совместимости}{is still 32-bit, perhaps, for compatibility}.
\IFRU{Так что если вы хотите использовать 64-битные значения, можно использовать}
{So if one need 64-bit variables, one may use} \IT{uint64\_t} \OrENRU \IT{int64\_t}.

\IFRU{А указатели теперь, конечно, 64-битные}{But pointers, of course, has 64-bit width}.

\label{typedef}
\subsubsection{typedef}

\TT{typedef} \IFRU{вводит}{introduces} \IT{\IFRU{синоним}{synonym}} \IFRU{для типа данных}{for a data type}.
\IFRU{Часто это используется для структур, чтобы каждый раз не писать}
{It is often used for structures, for the reason not to write} \IT{struct}
\IFRU{перед её именем, например}{each time before its name, for example}:

\begin{lstlisting}
typedef struct _node
{
	node *prev;
	node *next;
	void *data;
} node;
\end{lstlisting}

\IFRU{Такого очень много в ``заголовочных'' файлах в}
{A lot of such examples may be found in header files in the} Windows SDK (Windows API).

\IFRU{Тем не менее}{Nevertheless}, \IT{typedef} 
\IFRU{также можно использовать не только для структур, но и для обычных, 
интегральных}{can be also used not only for structures, but also for usual integral}
\footnote{\IFRU{приводимых к числу}{which may be converted to a number}}, \IFRU{типов, например}{types like}:

\begin{lstlisting}
typedef int age;
int compute_mean (age wife, age husband);

typedef int coord;
void draw (coord X, coord Y, coord Z);

typedef uint32_t address;
void write_memory (address a, size_t size, byte *buf);
\end{lstlisting}

\IFRU{Как видно}{As we can see}, \IT{typedef} 
\IFRU{здесь может помочь в документировании исходного кода, так он легче читается}{here may help with
code documentation, it is now easier to read}.

\IFRU{Например, тип}{For example, the} \IT{time\_t} 
(\IFRU{время в формате}{The time in the} UNIX time
\IFRU{, то что возвращает стандартная функция}{ format, for example, what the}
localtime()
\IFRU{, например}{returns}), \IFRU{это на самом деле
обычное 32-битное число, но этот тип определен в}{it is in fact 32-bit number, but the type is defined in the}
\IT{time.h} \IFRU{обычно так}{file usually as}:

\begin{lstlisting}
typedef long __time32_t;
\end{lstlisting}

\IFRU{Здесь вполне можно было бы использовать директиву препроцессора}
{A preprocessor directive} \TT{\#define} \IFRU{(многие так и делают),
но это хуже с точки зрения обработки ошибок во время компиляции}{may be used here (many do so),
but it is worse in the sense of errors handling during compilation}.

\paragraph{\IFRU{Критика }{}typedef\IFRU{}{ criticism}}

\IFRU{Тем не менее, такой известный и опытный программист как Линус Торвальдс,
против использования typedef}{Nevertheless, such well-known and experiences programmer as Linus
Torvalds is against typedef usage}:
\cite{Torvalds:2002}.


