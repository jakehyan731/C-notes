\subsection{for}

\IFRU{В}{The} for()\IFRU{, как известно, три выражения:
первое вычисляется перед началом всех итераций,
второе вычисляется перед каждой итерацией,
третье ~--- после каждой итерации}
{ statement, as we know, has 3 expressions:
1st computing before all iterations begin,
2nd computing before each iteration
and the 3rd ~--- after each iteration}.

\IFRU{И конечно же, там можно указывать что-то отличное от обычного счетчика}
{And of course, there might be written something different from the usual counter}.

\subsubsection{\IFRU{Засада}{Caveat} \#1}

\IFRU{Если написать такое}{If to write this}:

\lstinputlisting{common/for_strlen.cpp}

... \IFRU{то это наверное будет ошибкой}{perhaps this is a mistake}:
\TT{strlen(s)} \IFRU{будет вызываться перед каждой итерацией}{will be called before each iteration} 
~--- \IFRU{такой код генерирует}{that is the code} MSVC 2010\IFRU{}{ generated}.
\IFRU{Впрочем}{However}, GCC 4.8.1 \IFRU{вызывает}{calls} \TT{strlen(s)} 
\IFRU{только один раз, в начале цикла}{only once, at the loop beginning}.

\subsubsection{\IFRU{Запятая}{Comma}}

\IFRU{Запятая}{Comma}\cite[6.5.17]{C99TC3} ~--- \IFRU{не самая понятная для всех штука в Си, 
однако, их очень удобно использовать в определениях в for()}
{is not widely understood C feature, however, it is very useful for using in a for() declarations}.

\IFRU{Например, может пригодится использовать в цикле два итератора одновременно}
{For example, it is useful to have two iterators simultaneously}.
\IFRU{Пусть один просто отсчитывает от 0, прибавляя 1 при каждой итерации,
а второй итератор указывает на элемент в списке}
{Let the first iterator just counts from 0 adding 1 at each iteration, and the second iterator
points to the list element}:

\lstinputlisting{common/for_comma.cpp}

\IFRU{Это выдаст предсказуемый результат}{This will dumps predictable result}:

\begin{lstlisting}
0: 123
1: 456
2: 789
3: 1
\end{lstlisting}

\IFRU{Но к сожалению, определять итераторы вместе с типами в теле самого for() вот так нельзя}
{However, it is not possible to declare iterators with its types in for() clause}:

\begin{lstlisting}
	for (int i=0, std::list<int>::iterator it=l.begin(); it!=l.end(); i++, it++)
\end{lstlisting}

\subsubsection{continue}

\IT{continue} \IFRU{это безусловный переход на конец тела цикла}{is unconditional goto to the end
of loop body}.

\IFRU{Это может быть очень полезно, например, в подобном коде}
{This may be very useful, for example, in such code}:

\begin{lstlisting}
for (...)
{
	if (is_element_satisfied_criteria_1(...)==true)
	{
		// do something need in is_element_satisfied_criteria_2()

		if (is_element_satisfied_criteria_2(...)==true)
		{
			do_something_1();
			do_something_2();
			do_something_3();
		};

	};
};
\end{lstlisting}

... \IFRU{всё это можно легко заменить на более опрятное}{it is all can be replaced by neat}:

\begin{lstlisting}
for (...)
{
	if (is_element_satisfied_criteria_1(...)==false)
		continue;

	// do something need in is_element_satisfied_criteria_2()

	if (is_element_satisfied_criteria_2(...)==false)
		continue;

	do_something_1();
	do_something_2();
	do_something_3();
};
\end{lstlisting}

