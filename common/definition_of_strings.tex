\subsubsection{\IFRU{Определение строк}{String declarations}}

\paragraph{\IFRU{Последовательности символов используемые в строках}{Character sequences used in strings}}

\begin{center}
\begin{tabular}{ | l | l | l | }
\hline
\textbackslash{}0 & 0x00 & \IFRU{нулевой байт}{zero byte} \\
\hline
\textbackslash{}a & 0x07 & \IFRU{звонок}{bell}
\footnote{\IFRU{используется для подачи звукового сигнала в терминале}{used for beeping to a terminal}} \\
\hline
\textbackslash{}t & 0x09 & \IFRU{табуляция}{tabulation} \\
\hline
\textbackslash{}n & 0x0A & line feed (LF) \\
\hline
\textbackslash{}r & 0x0D & carriage return (CR) \\
\hline
\end{tabular}
\end{center}

\IFRU{Разница между LF и CR в том, что в старых матричных принтерах,
LF означал протягивание бумаги на одну строку вниз, а CR перевод каретки влево до края бумаги}
{The difference between LF and CR is that in old dot-matrix printers LF mean line feed by one line down,
and CR carriage return to the left margin of paper}.
\IFRU{Так что в принтер нужно было передавать оба символа}{So both characters must be transmitted to the
printer}.

\IFRU{Вывод CR без LF дает возможность перезаписывать текующую строку в консоли}{Outputting CR without LF
allows to rewrite current string in the console}:

\begin{lstlisting}
for (;;)
{
	// do something
	// how much we processed?
	percents=ammount_of_work/total_work*100;
	printf ("%d%% complete\r", percents);
};
\end{lstlisting}

\IFRU{Это часто используется например в архиваторах для отображения текущего статуса и}
{This is often used in the file archivers for displaying current status and} wget.

\IFRU{В Си}{In C} \AndENRU UNIX \IFRU{принят LF как символ новой строки}
{LF is traditionally accepted as newline symbol}.

\IFRU{В}{In the} DOS \IFRU{и затем в}{and then} Windows ~--- CR+LF.

\paragraph{\IFRU{Строка определенная в нескольких строках}{The string defined as multi-string}}
\label{heredoc}
\IFRU{Малоизвестная возможность Си, длинные строки можно объявлять так}
{Not widely known C feature, long strings can be defined as}:

\begin{lstlisting}
const char* long_line="line 1"
	"line 2"
	"line 3"
	"line 4"
	"line 5";

...

printf ("Some Utility v0.1\n"
	"Usage: %s parameters\n"
	"\n"
	"Authors:...\n", argv[0]);
\end{lstlisting}

\IFRU{Это отдаленно напоминает}{It is somewhat resembling}
``here document''\footnote{\url{https://en.wikipedia.org/wiki/Here_document}} 
\IFRU{в UNIX-шеллах}{in UNIX-shells} \AndENRU Perl.

