\section{\IFRU{Препроцессор}{Preprocessor}}

\IFRU{Препроцессор обрабатывает директивы начинающиеся с}
{The preprocessor handles directives started with} \# ~--- \#define, \#include, \#if, \IFRU{итд}{etc}.

\begin{lstlisting}[caption=``\IFRU{или}{or}'']
#if defined(LINUX) || defined(ANDROID)
\end{lstlisting}

\section{\IFRU{Стандартные для компиляторов и}
{Standard values for compilers and} \ac{OS}\IFRU{ значения}{}}

\begin{itemize}
\item \TT{\_DEBUG} --- \IFRU{отладочная сборка}{debugging build}.
\item \TT{NDEBUG} --- \IFRU{неотладочная}{non-debugging} (release) \IFRU{сборка}{build}.
\item \TT{\_\_linux\_\_} --- \ac{OS} Linux.
\item \TT{\_WIN32} --- \ac{OS} Windows. \IFRU{Присутствует как и в x86-проектах, так и в x64}
{Present in both x86 and x64 builds}.
\item \TT{\_WIN64} --- \IFRU{Присутствует в x64-проектах для}{Present in x64 builds for} \ac{OS} Windows.
\item \TT{\_\_cplusplus} --- \IFRU{присутствует в}{Present in a} \CPP \IFRU{проектах}{projects}.
\item \TT{\_MSC\_VER} --- \IFRU{компилятор }{}\ac{MSVC}\IFRU{}{ compiler}.
\item \TT{\_\_GNUC\_\_} --- \IFRU{компилятор }{}\ac{GCC}\IFRU{}{ compiler}.
\item \TT{\_\_APPLE\_\_} --- \IFRU{компиляция под устройства Apple}{compilation for Apple devices}.
\item \TT{\_\_ppc\_\_} --- \IFRU{компиляция под}{compilation for} PowerPC 32-bit.
\item \TT{\_\_ppc64\_\_} --- \IFRU{компиляция под}{compilation for} PowerPC 64-bit.
\item \TT{\_\_LP64\_\_} \OrENRU \TT{\_LP64} --- \ac{GCC}: \IFRU{компиляция под 64-битный режим}
{compilation for 64-bit architecture}
\footnote{\IFRU{Дословно: long pointer, т.е., указатель требует 64 бита для хранения}
{LP mean Long Pointer, i.e., pointer require 64 bits for storage}}.
\end{itemize}

\IFRU{Так можно писать разные участки кода для разных компиляторов и}
{That is how it is possible to write various code pieces for various compilers and} \ac{OS}.

\IFRU{Прочие макросы разных \ac{OS}}{Other \ac{OS}-related macros}:
\url{http://sourceforge.net/p/predef/wiki/OperatingSystems/}

\subsection{``\IFRU{Пустой}{Empty}'' \IFRU{макрос}{macro}}

\IFRU{Всем известны макросы не объявляющие никаких значений, например \IT{\_DEBUG}}
{\IT{\_DEBUG} is well-known macro without any value}.
\IFRU{Обычно, только проверяется наличие его или отсутствие}{It is usually checked its presence
or absence}.
\IFRU{Вот еще пример полезного ``пустого'' макроса}
{Here is another example of useful ``empty'' macro}:

\IFRU{В заголовочных файлах}{In the} Windows API \IFRU{мы можем увидеть такое}{header files we
can find this}:

\begin{lstlisting}
typedef NTSTATUS
(NTAPI *TDI_REGISTER_CALLBACK)(
  IN PUNICODE_STRING DeviceName,
  OUT HANDLE *TdiHandle);

...

typedef NDIS_STATUS
(NTAPI *CM_CLOSE_CALL_HANDLER)(
  IN NDIS_HANDLE  CallMgrVcContext,
  IN NDIS_HANDLE  CallMgrPartyContext  OPTIONAL,
  IN PVOID  CloseData  OPTIONAL,
  IN UINT  Size  OPTIONAL);
\end{lstlisting}

IN, OUT \AndENRU OPTIONAL ~--- \IFRU{это}{are} ``\IFRU{пустые}{empty}'' \IFRU{макросы объявленные так}
{macros defined as}:

\begin{lstlisting}
#ifndef IN
#define IN
#endif
#ifndef OUT
#define OUT
#endif
#ifndef OPTIONAL
#define OPTIONAL
#endif
\end{lstlisting}

\IFRU{Для компилятора они никакой информации не несут,
они предназначены только для документирования, показать,
какие параметры ф-ций зачем нужны}
{They carry no information for compiler at all, they are intended for documenting purposes,
to mark function arguments}.

\subsection{\IFRU{Частые ошибки}{Frequent caveats}}

\subsubsection{\#1}

\IFRU{К примеру, вы хотите создать макрос для возведения числа в квадрат}
{For example, you may want to define a macro for taking the power of a number}:

\begin{lstlisting}
#define square(x)      x*x
\end{lstlisting}

\IFRU{Это ошибка, потому что выражение}{It is a mistake because the expression} \IT{square(a+b)} 
\IFRU{в итоге ``развернется'' в}{will ``unfold'' into} $a+b*a+b$, 
\IFRU{что, разумеется, совсем не то что хотелось}{and that is not what you probably wanted}.
\IFRU{Поэтому в определнии макроса все переменные, и сам макрос, нужно ``изолировать'' скобками}
{So, all variables in the macro definition, and also macro itself, should be parenthesized}:

\begin{lstlisting}
#define square(x)      ((x)*(x))
\end{lstlisting}

\IFRU{Пример из файла}{Example from the} minmax.h \IFRU{из}{file from} MinGW:

\begin{lstlisting}
#define max(a,b) (((a) > (b)) ? (a) : (b))
...
#define min(a,b) (((a) < (b)) ? (a) : (b))
\end{lstlisting}

\subsubsection{\#2}

\IFRU{Если вы где-то определяете какую-то константу}{If you define a constant somewhere}:

\begin{lstlisting}
#define N 1234
\end{lstlisting}

... \IFRU{затем где-то дальше переопределяете её снова}{and then redefine it somewhere}, 
\IFRU{то компилятор промолчит, и это приведет к трудновыявляемой
ошибке}{compiler will be silent and that may lead to hard-to-find bug}.

\IFRU{Поэтому константы желательнее определять как глобальные переменные с модификатором}
{So that is why it is advisable to define constants as a global variables with a}
\IT{const}\IFRU{}{ modifier}.

