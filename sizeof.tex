\section{sizeof}

% array of struct phonebook

Обычно, sizeof() применяют к интегральным\footnote{типы отражающие числа} типам либо к структурам, тем не менее,
его можно применять и к массивам:

% snprintf, wchar_t...

И к массивам структур:

\begin{lstlisting}
struct phonebook_entry
{
	char *name;
	char *surname;
	char *tel;
};

struct phonebook_entry phonebook[]=
{
	{ "Kirk", "Hammett", "555-1234" },
	{ "Lars", "Ulrich", "555-5678" },
	{ "James", "Hetfield", "555-1122" },
	{ "Robert", "Trujillo", "555-7788" }
};

void dump (struct phonebook_entry* input)
{
	for (int i=0; i<sizeof(phonebook)/sizeof(struct phonebook_entry); i++)
		printf ("%s %s - %s\n", input[i].name, input[i].surname, input[i].tel);
};
\end{lstlisting}

sizeof(phonebook) -- это размер всего массива структур в байтах. sizeof(struct phonebook\_entry) -- это размер
одной структуры в байтах. Делением мы узнаем количество структур в массиве.

