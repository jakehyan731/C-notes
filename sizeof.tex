\section{sizeof}

Обычно, sizeof() применяют к интегральным\footnote{типы отражающие числа} типам либо к структурам, 
тем не менее, его можно применять и к массивам, к примеру:

% snprintf, wchar_t...

\begin{lstlisting}
	char buf[1024];
	snprintf(buf, sizeof(buf), "...");
\end{lstlisting}

В противном случае, если указывать длину массива (1024) в двух местах (в объявлении buf и как второй 
аргумент snprintf()), то и изменять это значение придется каждый раз в обоих местах, а об этом легко забыть.

Если нужны wide-строки, то sizeof можно применять из \IT{wchar\_t} (который, на самом деле, 16-битный тип short):

\begin{lstlisting}
	wchar_t buf[1024];
	swprintf(buf, sizeof(buf)/sizeof(wchar_t), "...");
\end{lstlisting}

sizeof() возвращает длину в байтах, так что здесь он будет равен $1024*2$, т.е., $2048$. Но мы можем
разделить это значение на длину одного элемента массива (\IT{wchar\_t}) в байтах ($2$), 
чтобы получить количество элементов в массиве ($1024$).

sizeof() можно применять и к массивам структур:

\begin{lstlisting}
struct phonebook_entry
{
	char *name;
	char *surname;
	char *tel;
};

struct phonebook_entry phonebook[]=
{
	{ "Kirk", "Hammett", "555-1234" },
	{ "Lars", "Ulrich", "555-5678" },
	{ "James", "Hetfield", "555-1122" },
	{ "Robert", "Trujillo", "555-7788" }
};

void dump (struct phonebook_entry* input)
{
	for (int i=0; i<sizeof(phonebook)/sizeof(struct phonebook_entry); i++)
		printf ("%s %s - %s\n", input[i].name, input[i].surname, input[i].tel);
};
\end{lstlisting}

sizeof(phonebook) -- это размер всего массива структур в байтах. sizeof(struct phonebook\_entry) -- 
это размер одной структуры в байтах.
Делением мы узнаем количество структур в массиве.

