\section{Комментарии}

Их иногда удобно вставлять прямо в вызов ф-ции, чтобы где-то на виду держать пометку,
что означает каждый аргумент:

\begin{lstlisting}
f (val1, /* a very special flag! */ false, /* another special flag here */ true);
\end{lstlisting}

Целый блок кода можно откомментировать при помощи \#if\footnote{директива препроцессора}:

\begin{lstlisting}
	ta	= aemif_calc_rate(t->ta, clkrate, TA_MAX);
	rhold	= aemif_calc_rate(t->rhold, clkrate, RHOLD_MAX);
#if 0	
	rstrobe	= aemif_calc_rate(t->rstrobe, clkrate, RSTROBE_MAX);
	rsetup	= aemif_calc_rate(t->rsetup, clkrate, RSETUP_MAX);
	whold	= aemif_calc_rate(t->whold, clkrate, WHOLD_MAX);
#endif	
	wstrobe	= aemif_calc_rate(t->wstrobe, clkrate, WSTROBE_MAX);
	wsetup	= aemif_calc_rate(t->wsetup, clkrate, WSETUP_MAX);
\end{lstlisting}

Это может быть удобнее чем традиционный способ потому что текстовый редактор или IDE в этом случае
не ``сломает'' отступы при выравнивании.

