\section{struct}

В C99\ref{C99} можно инициализировать отдельные поля структур. Пропущенные будут заполнены нулями. Такого очень
много в ядре Linux. 

\begin{lstlisting}
struct color
{
	int R;
	int G;
	int B;
};

struct color blue={ .B=255 };
\end{lstlisting}

И даже более того, можно создавать структуру прямо в аргументах ф-ции, например:

\begin{lstlisting}
struct color
{
	int R;
	int G;
	int B;
};

void print_color_info (struct color *c)
{
	printf ("%d %d %d\n", c->R, c->G, c->B);
};

int main()
{
	print_color_info(&blue);
	print_color_info(&(struct color){ .G=255 });
};
\end{lstlisting}

Помимо всего прочего, о структурах также много есть в разделе ``\COOPname''\ref{COOP}.

\subsection{Расположение полей в структурах}

В современных x86-микропроцессорах (как Intel, так и AMD) имеется кеш-память разных уровней. 
Самая быстрая кеш-память (первого уровня),
разделена на 64-байтные элементы (кеш-линии) и любое обращение к памяти заполняет сразу всю линию\cite{AgnerFog}.

Можно сказать, что любое обращение к памяти (по выровненным адресам) подтягивает в кеш сразу 64 байта.

Поэтому, если некая структура данных имеет размер более 64-х байт, очень важно разделить её на две части:
наиболее востребованные поля и менее востребованные. 
Самые востребованные поля желательно разместить в пределах первых 64-х байт.

Это же касается и классов в Си++.

