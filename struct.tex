\section{struct}

В современных x86-микропроцессорах имеется кеш-память разных уровней. Самая быстрая кеш-память (первого уровня),
разделена на 64-байтные элементы (кеш-линии) и любое обращение к памяти заполняет сразу всю линию.

Можно сказать, что любое обращение к памяти (по выровненным адресам) подтягивает в кеш сразу 64 байта.

Поэтому, если некая структура данных имеет размер более 64-х байт, очень важно разделить её на две части:
наиболее востребованные поля и менее востребованные. Самые востребованные поля желательно разместить в пределах
первых 64-х байт.

Помимо всего прочего, о структурах еще много описано в разделе ``\COOPname''\ref{COOP}.

