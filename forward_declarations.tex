\label{forwarddeclaration}
\section{forward declaration}

Как известно, в заголовочных файлах (headers) обычно содержатся декларации ф-ций, то есть, 
имя ф-ции, аргументы и типы, тип возвращаемого значения, но нет тела ф-ций. Так делается для того,
чтобы компилятор мог знать, с чем имеет дело, не углубляясь в тонкости реализации ф-ций.

То же самое можно делать и для типов. Для того чтобы не включать при помощи \#include файл с описаниями
какого либо класса в другой заголовочный файл, можно обойтись указанием, что он вообще существует.

Например, вы работаете с комплексными числами и у вас где-то есть такая структура:

\begin{lstlisting}
struct complex
{
	double real;
	double imag;
};
\end{lstlisting}

И она определена в файле my\_complex.h.

Безусловно, вам нужно включить этот файл, если вы собиретесь работать с переменными типа complex, 
с отдельными полями структуры.
Но если вы описываете свои ф-ции для работы с этой структурой в отдельном заголовочном файле, то включать там
my\_complex.h не обязательно, компилятору достаточно просто знать что complex это структура:

\begin{lstlisting}
struct complex;

void sum(struct complex *x, struct complex *y, struct complex *out);
void pow(struct complex *x, struct complex *y, struct complex *out);
\end{lstlisting}

Это позволяет увеличить скорость компиляции, а также бороться с циркулярными зависимостями, когда
в двух модулях используются типы и ф-ции друг друга.
