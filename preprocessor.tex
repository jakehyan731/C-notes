\chapter{\IFRU{Препроцессор}{Preprocessor}}

Препроцессор обрабатывает директивы начинающиеся с \# --- \#define, \#include, \#if, итд.

\section{``Пустой'' макрос}

Всем известны макросы не объявляющие никаких значений, например \IT{\_DEBUG}.
Обычно, только проверяется наличие его или отсутствие.
Вот еще пример полезного ``пустого'' макроса:

В заголовочных файлах Windows API мы можем увидеть такое:

\begin{lstlisting}
typedef NTSTATUS
(NTAPI *TDI_REGISTER_CALLBACK)(
  IN PUNICODE_STRING DeviceName,
  OUT HANDLE *TdiHandle);

...

typedef NDIS_STATUS
(NTAPI *CM_CLOSE_CALL_HANDLER)(
  IN NDIS_HANDLE  CallMgrVcContext,
  IN NDIS_HANDLE  CallMgrPartyContext  OPTIONAL,
  IN PVOID  CloseData  OPTIONAL,
  IN UINT  Size  OPTIONAL);
\end{lstlisting}

IN, OUT и OPTIONAL --- это ``пустые'' макросы объявленные так:

\begin{lstlisting}
#ifndef IN
#define IN
#endif
#ifndef OUT
#define OUT
#endif
#ifndef OPTIONAL
#define OPTIONAL
#endif
\end{lstlisting}

Для компилятора они никакой информации не несут, они предназначены только для документирования, показать,
какие параметры ф-ций зачем нужны.

\section{Частая ошибка}

К примеру, вы хотите создать макрос для возведения числа в квадрат:

\begin{lstlisting}
#define square(x)      x*x
\end{lstlisting}

Это ошибка, потому что выражение \IT{square(a+b)} в итоге ``развернется'' в $a+b*a+b$, что, разумеется, совсем
не то что хотелось. Поэтому в определнии макроса все переменные, и сам макрос, нужно ``изолировать'' скобками:

\begin{lstlisting}
#define square(x)      ((x)*(x))
\end{lstlisting}

Пример из файла minmax.h из MinGW:

\begin{lstlisting}
#define max(a,b) (((a) > (b)) ? (a) : (b))
...
#define min(a,b) (((a) < (b)) ? (a) : (b))
\end{lstlisting}


