\chapter{Прочее}

Что хранится в объектных и бинарных (.exe, .dll) файлах?

Обычно только данные (глобальные переменные) и тела ф-ций (включая методы классов).

Информации о типах (классы, структуры, typedef\ref{typedef}-ы) там нет. 
Это может помочь в понимании того как всё устроено.

В этом заключается одна из больших проблем декомпиляции --- отсутствие информации о типах.

% FIXME: add about name mangling

О том как всё компилируется в машинный код, подробнее можно почитать здесь:\cite{REBook}.

\section{Глобальные переменные}

Мода на ООП и всякие такие штуки утвердила что глобальные переменные это плохо, тем не менее, в каких-то разумных
случаях, это можно использовать, например, для возврата большего количества информации из ф-ций. 

Так, некоторые стандартные ф-ции Си возвращают код ошибки через глобальную переменную errno, которая, в наше
время, фактически не глобальная, а находится внутри \ac{TLS}.

В Windows API код ошибки можно узнать вызвав GetLastError(), 
которая, на самом деле, выдает значение из \ac{TIB}.

В кодегенераторе компилятора OpenWatcom, всё находится в глобальных переменных,
а самая главная ф-ция выглядит так:

\begin{lstlisting}[caption=bld\textbackslash{}cg\textbackslash{}c\textbackslash{}generate.c]
extern  void    Generate( bool routine_done )
/*******************************************/
/* The big one - here's where most of code generation happens.
 * Follow this routine to see the transformation of code unfold.
 */
{
    if( BGInInline() ) return;
    HaveLiveInfo = FALSE;
    HaveDominatorInfo = FALSE;
    #if ( _TARGET & ( _TARG_370 | _TARG_RISC ) ) == 0
        /* if we want to go fast, generate statement at a time */
        if( _IsModel( NO_OPTIMIZATION ) ) {
            if( !BlockByBlock ) {
                InitStackMap();
                BlockByBlock = TRUE;
            }
            LNBlip( SrcLine );
            FlushBlocks( FALSE );
            FreeExtraSyms( LastTemp );
            if( _MemLow ) {
                BlowAwayFreeLists();
            }
            return;
        }
    #endif

    /* if we couldn't get the whole procedure in memory, generate part of it */
    if( BlockByBlock ) {
        if( _MemLow || routine_done ) {
            GenPartialRoutine( routine_done );
        } else {
            BlkTooBig();
        }
        return;
    }

    /* if we're low on memory, go into BlockByBlock mode */
    if( _MemLow ) {
        InitStackMap();
        GenPartialRoutine( routine_done );
        BlowAwayFreeLists();
        return;
    }

    /* check to see that no basic block gets too unwieldy */
    if( routine_done == FALSE ) {
        BlkTooBig();
        return;
    }

    /* The routine is all in memory. Optimize and generate it */
    FixReturns();
    FixEdges();
    Renumber();
    BlockTrim();
    FindReferences();
    TailRecursion();
    NullConflicts( USE_IN_ANOTHER_BLOCK );
    InsDead();
    FixMemRefs();
    FindReferences();
    PreOptimize();
    PropNullInfo();
    MemtoBaseTemp();
    if( _MemCritical ) {
        Panic( FALSE );
        return;
    }
    MakeConflicts();
    if( _IsModel( LOOP_OPTIMIZATION ) ) {
        SplitVars();
    }
    AddCacheRegs();
    MakeLiveInfo();
    HaveLiveInfo = TRUE;
    AxeDeadCode();
    /* AxeDeadCode() may have emptied some blocks. Run BlockTrim() to get rid
     * of useless conditionals, then redo conflicts etc. if any blocks died.
     */
    if( BlockTrim() ) {
        FreeConflicts();
        NullConflicts( EMPTY );
        FindReferences();
        MakeConflicts();
        MakeLiveInfo();
    }
    FixIndex();
    FixSegments();
    FPRegAlloc();
    if( RegAlloc( FALSE ) == FALSE ) {
        Panic( TRUE );
        HaveLiveInfo = FALSE;
        return;
    }
    FPParms();
    FixMemBases();
    PostOptimize();
    InitStackMap();
    AssignTemps();
    FiniStackMap();
    FreeConflicts();
    SortBlocks();
    if( CalcDominatorInfo() ) {
        HaveDominatorInfo = TRUE;
    }
    GenProlog();
    UnFixEdges();
    OptSegs();
    GenObject();
    if( ( CurrProc->prolog_state & GENERATED_EPILOG ) == 0 ) {
        GenEpilog();
    }
    FreeProc();
    HaveLiveInfo = FALSE;
#if _TARGET & _TARG_INTEL
    if( _IsModel( NEW_P5_PROFILING ) ) {
        FlushQueue();
    }
#else
    FlushQueue();
#endif
}
\end{lstlisting}

\section{Битовые поля}

Это очень популярная штука в Си, да и не только.

Для задания булевых значений ``true'' или ``false'', можно передавать 1 или 0 в байте или в 32-битном регистре,
или в типе int, но это очень не экономно в плане расхода памяти. Намного удобнее передавать такие значения в отдельных
битах.

К примеру, стандартная ф-ция findfirst() возвращает структуру о найденном файле, где аттрибуты файла передаются
такими флагами:

\begin{lstlisting}
#define _A_NORMAL 0x00
#define _A_RDONLY 0x01
#define _A_HIDDEN 0x02
#define _A_SYSTEM 0x04
#define _A_SUBDIR 0x10
#define _A_ARCH 0x20
\end{lstlisting}

Конечно, передавать каждый аттрибут отдельной переменной bool было бы очень неэкономично.

И напротив, для указания флагов в ф-цию можно исползовать битовые поля. 
Например, CreateFile
\footnote{\url{http://msdn.microsoft.com/en-us/library/windows/desktop/aa363858(v=vs.85).aspx}} из Windows API.

Для задания флагов, чтобы не запутаться и не опечататься в значениях каждого бита, можно писать так:

\begin{lstlisting}
#define FLAG1 (1<<0)
#define FLAG2 (1<<1)
#define FLAG3 (1<<2)
#define FLAG4 (1<<3)
#define FLAG5 (1<<4)
\end{lstlisting}

(Компилятор все равно всё это легко соптимизирует).

А чтобы было удобнее выставлять, удалять и проверять отдельный бит/флаг, можно использовать подобные макросы:

\begin{lstlisting}
#define IS_SET(flag, bit)       (((flag) & (bit)) ? true : false)
#define SET_BIT(var, bit)       ((var) |= (bit))
#define REMOVE_BIT(var, bit)    ((var) &= ~(bit))
\end{lstlisting}

С другой стороны, необходимо помнить, что операции выделения отдельного бита из значения типа int 
обычно даются процессору ``дороже'', чем работа с типом bool в 32-битном регистре. 
Так что, если скорость для вас намного критичнее чем
экономия памяти, можно попробовать использовать тип bool.



\section{Интересные проекты для изучения}

\subsection{Си}

\begin{itemize}
\item
Go Compiler \url{http://golang.org/doc/install/source}

\item
Git \url{https://github.com/git/git}
\end{itemize}

\subsection{Си++}

\begin{itemize}
\item
LLVM \url{http://llvm.org/releases/download.html}

\item
Google Chrome \url{http://www.chromium.org/developers/how-tos/get-the-code}
\end{itemize}

