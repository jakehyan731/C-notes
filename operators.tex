\section{Operators}

\subsection{==}

Очень неприятные ошибки возникают если в условии \IT{if(a==3)} опечататься и написать \IT{if(a=3)}.
Ведь выражение \IT{a=3} ``возвращает'' 3, а 3 это не 0, поэтому условие if() всегда будет 
срабатывать.

Раньше, для защиты от подобных ошибок, была мода писать наоборот: \IT{if(3==a)}, таким образом,
если опечататься, выйдет \IT{if(3=a)}, компилятор тут же выдаст ошибку.

Тем не менее, в наше время, компиляторы обычно предупреждают если написать \IT{if(a==3)}, 
так что, наверное, менять местами элементы выражения уже не обязательно.

\subsection{Short-circuit и артефакты приоритетов операций в Си}

Разберем что такое short-circuit\footnote{дословный перевод на русский: ``короткое замыкание''}.

Это когда в выражении \IT{if(a \&\& b \&\& c)}, часть b будет вычисляться только если a -- истинна,
а c будет вычисляться
только если a и b -- истинны. И вычисляться они будут именно в таком порядке, как указано.

Иногда можно встретить подобное: \IT{if (p!=NULL \&\& p->field==123)} -- и это совершенно правильно.
Поле field в структуре,
на которую указывает p, будет вычисляться только если указатель p не равен NULL.

