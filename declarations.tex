\chapter{\IFRU{Объявления в Си/Си++}{C/C++ declarations}}
\section{Объявления переменных внутри ф-ции}

Раньше, в Си можно было объявлять переменные только в начале ф-ции. А в Си++ --- где угодно.

К тому же, нельзя было объявлять итератор в for() (а в Си++ также можно было):

\begin{lstlisting}
for(int i=0; i<10; i++)
	...
\end{lstlisting}

Новый стандарт C99\ref{C99} позволяет делать это.

\subsection{static}

Обычно глобальные переменные (или ф-ции) объявляются как static, так их область видимости ограничивается 
данным файлом. Но локальные переменные внутри ф-ции также можно объявлять как static, тогда эта переменная
будет не локальной, а глобальной, но её область видимости будет ограничена только этой ф-цией.

К примеру, это помогло бы для реализации strtok(), ведь этой ф-ции что-то нужно хранить у себя между вызовами.

\section{forward declaration}

Как известно, в заголовочных файлах (headers) обычно содержатся декларации ф-ций, то есть, 
имя ф-ции, аргументы и типы, тип возвращаемого значения, но нет тела ф-ций. Так делается для того,
чтобы компилятор мог знать, с чем имеет дело, не углубляясь в тонкости реализации ф-ций.

То же самое можно делать и для типов. Для того чтобы не включать при помощи \#include файл с описаниями
какого либо класса в другой заголовочный файл, можно обойтись указанием, что он вообще существует.

Например, вы работаете с комплексными числами и у вас где-то есть такая структура:

\begin{lstlisting}
struct complex
{
	double real;
	double imag;
};
\end{lstlisting}

И она определена в файле my\_complex.h.

Безусловно, вам нужно включить этот файл, если вы собиретесь работать с переменными типа complex, 
с отдельными полями структуры.
Но если вы описываете свои ф-ции для работы с этой структурой в отдельном заголовочном файле, то включать там
my\_complex.h не обязательно, компилятору достаточно просто знать что complex это структура:

\begin{lstlisting}
struct complex;

void sum(struct complex *x, struct complex *y, struct complex *out);
void pow(struct complex *x, struct complex *y, struct complex *out);
\end{lstlisting}

Это позволяет увеличить скорость компиляции, а также бороться с циркулярными зависимостями, когда
в двух модулях используются типы и ф-ции друг друга.
 % section

\section{C++11: auto}

Пользуясь \ac{STL}, иногда надоедает каждый раз объявлять тип итератора вроде:

\begin{lstlisting}
for (std::list<int>::iterator it=list.begin(); it!=list.end(); it++)
\end{lstlisting}

Тип it вполне можно получить из list.begin(), поэтому, в начиная со стандарта C++11, можно использовать auto:

\begin{lstlisting}
for (auto it=list.begin(); it!=list.end(); it++)
\end{lstlisting}

