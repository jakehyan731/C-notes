\section{for}

В for(), как известно, три выражения, первое вычисляется перед началом, второе вычисляется перед каждой итерацией, третье --- после каждой итерации.

И конечно же, там можно указывать что-то отличное от обычного счетчика.

\subsection{\IFRU{Засада}{Pitfall} \#1}

Если написать такое:

\lstinputlisting{src/for_strlen.cpp}

... то это наверное будет ошибкой: strlen(s) будет вызываться перед каждой итерацией --- такой код генерирует
MSVC 2010. Впрочем, GCC 4.8.1 вызывает strlen(s) только один раз, в начале цикла.

\subsection{Запятая}

Запятая\cite[6.5.17]{C99TC3} --- не самая понятная для всех штука в Си, 
однако, их очень удобно использовать в объявлении for().

Например, может пригодится использовать в цикле два итератора одновременно. Пусть один просто отсчитывает
от 0, прибавляя 1 при каждой итерации, а второй итератор указывает на элемент в списке:

\lstinputlisting{src/for_comma.cpp}

Это выдаст предсказуемое:

\begin{lstlisting}
0: 123
1: 456
2: 789
3: 1
\end{lstlisting}

Но к сожалению, объявлять итераторы в теле самого for() вот так нельзя:

\begin{lstlisting}
	for (int i=0, std::list<int>::iterator it=l.begin(); it!=l.end(); i++, it++)
\end{lstlisting}

\subsection{continue}

continue это безусловный переход на конец тела цикла.

Это может быть очень полезно, например, в подобном коде:

\begin{lstlisting}
for (...)
{
	if (is_element_satisfied_criteria_1(...)==true)
	{
		// do something need in is_element_satisfied_criteria_2()

		if (is_element_satisfied_criteria_2(...)==true)
		{
			do_something_1();
			do_something_2();
			do_something_3();
		};

	};
};
\end{lstlisting}

Всё это можно легко заменить на более опрятное:

\begin{lstlisting}
for (...)
{
	if (is_element_satisfied_criteria_1(...)==false)
		continue;

	// do something need in is_element_satisfied_criteria_2()

	if (is_element_satisfied_criteria_2(...)==false)
		continue;

	do_something_1();
	do_something_2();
	do_something_3();
};
\end{lstlisting}

