\section{for}

\subsection{Запятая}

Запятая --- не самая понятная штука в Си, однако, их удобно использовать в объявлении for().

Например, может пригодится использовать в цикле два итератора одновременно. Пусть один просто отсчитывает
от 0, прибавляя 1 при каждой итерации, а второй итератор указывает на элемент в списке:

\lstinputlisting{src/for_comma.cpp}

Это выдаст предсказуемое:

\begin{lstlisting}
0: 123
1: 456
2: 789
3: 1
\end{lstlisting}

Но к сожалению, объявлять итераторы в теле самого for() вот так нельзя:

\begin{lstlisting}
	for (int i=0, std::list<int>::iterator it=l.begin(); it!=l.end(); i++, it++)
\end{lstlisting}

