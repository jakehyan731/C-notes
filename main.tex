\documentclass[11pt,a4paper,oneside]{book}

\usepackage{cmap}

\ifdefined\RUSSIAN
\usepackage[english,russian]{babel}
\usepackage[T2A]{fontenc}
\usepackage{paratype}
\renewcommand*\familydefault{\sfdefault}
% http://www.emerson.emory.edu/services/latex/latex_169.html
\newcommand{\lstlistingsize}{\scriptsize}
\else
\usepackage[russian,english]{babel}
\usepackage[T2A]{fontenc}
\usepackage[default]{sourcesanspro}
\newcommand{\lstlistingsize}{\footnotesize}
\fi

\usepackage[utf8]{inputenc}
\usepackage{listings}
\usepackage{ulem}
\usepackage{url} % \url
\usepackage{graphicx}
\usepackage{listingsutf8}
\usepackage{makeidx}
\usepackage{cite}
\usepackage[cm]{fullpage}
\usepackage{color}
\usepackage{fancyvrb}
\usepackage{xspace}
\usepackage{framed}
\usepackage{ccicons}
\usepackage[nottoc]{tocbibind}
\usepackage{amsmath}
\usepackage[footnote,printonlyused,withpage]{acronym}
\usepackage[table]{xcolor}% http://ctan.org/pkg/xcolor
\usepackage[]{hyperref} % \href. should be last
%\usepackage{tikz}

\definecolor{lstbgcolor}{rgb}{0.94,0.94,0.94}
\makeindex

\newcommand{\TT}[1]{\texttt{#1}}
\newcommand{\IT}[1]{\textit{#1}}
\newcommand{\IFRU}[2]{\iflanguage{russian}{#1}{#2}}


\newcommand{\TITLE}{\IFRU{Заметки о языке программирования Си/Си++}
{C/C++ programming language notes}}
\newcommand{\AUTHOR}{\IFRU{Денис Юричев}{Dennis Yurichev}}
\newcommand{\EMAIL}{dennis@yurichev.com}

\hypersetup{
    pdftex,
    colorlinks=true,
    allcolors=blue,
    pdfauthor={\AUTHOR},
    pdftitle={\TITLE}
    }

\selectlanguage{english}

\lstset{
    backgroundcolor=\color{lstbgcolor},
    basicstyle=\ttfamily\lstlistingsize, 
    breaklines=true,
    frame=single,
    inputencoding=cp1251,
    columns=fullflexible,keepspaces,
}

\newcommand{\COOPname}{\IFRU{Объектно-ориентированное программирование в Си}{Object-oriented programming in C}}
\newcommand{\AndENRU}{\IFRU{и}{and}\xspace}
\newcommand{\OrENRU}{\IFRU{или}{or}\xspace}
\newcommand{\InENRU}{\IFRU{в}{in}\xspace}
\newcommand{\CPP}{\IFRU{Си++}{C++}\xspace}



\begin{document}

\VerbatimFootnotes

\frontmatter

\begin{titlepage}
\begin{center}
\vspace*{\fill}
\LARGE \TITLE

\vspace*{\fill}

\large \AUTHOR

\large \TT{<\EMAIL>}
\vspace*{\fill}
\vfill

\ccbyncnd

\textcopyright 2013, \AUTHOR. 

\IFRU{Это произведение доступно по лицензии Creative Commons «Attribution-NonCommercial-NoDerivs» 
(«Атрибуция — Некоммерческое использование — Без производных произведений») 3.0 Непортированная. 
Чтобы увидеть копию этой лицензии, посетите}
{This work is licensed under the Creative Commons Attribution-NonCommercial-NoDerivs 3.0 Unported License. 
To view a copy of this license, visit} \url{http://creativecommons.org/licenses/by-nc-nd/3.0/}.

\IFRU{Версия этого текста}{Text version} ({\large \today}).

\IFRU{Возможно, более новая версии текста, а так же англоязычная версия, также доступна по ссылке}
{There is probably a newer version of this text, and also Russian language version also accessible at}
\url{http://yurichev.com/C-book.html}

\IFRU{Вы также можете подписаться на мой twitter для получения информации о новых версиях этого текста, итд:
\href{https://twitter.com/yurichev_ru}{@yurichev\_ru}, либо подписаться на \href{http://yurichev.com/mailing_lists.html}{список рассылки}}
{You may also subscribe to my twitter, to get information about updates of this text, etc: 
\href{https://twitter.com/yurichev}{@yurichev}, or to subscribe to \href{http://yurichev.com/mailing_lists.html}{mailing list}}.
\end{center}
\end{titlepage}

\tableofcontents
\cleardoublepage

\begin{center}
\vspace*{\fill}

\IFRU{Эта страница сдается в аренду для рекламы}{This page can be rented for advertisement}.

\TT{<\EMAIL>}

\vspace*{\fill}
\vfill
\end{center}


\cleardoublepage

\chapter{\IFRU{Введение}{Preface}}

\IFRU{Сейчас, в 2013-м году, если некто желает написать}{Today, in year 2013, if one wants to write} 
1) \IFRU{как можно более быстро работающую программу}{as fast program as possible};
2) \IFRU{либо как можно более компактную для встраиваемых систем либо маломощных микроконтроллеров}
{or as compact as possible for embedded systems or low-cost microcontrollers},
\IFRU{то выбор очень ограниченный}{the choice is very limited}:
\IFRU{Си}{C}, \CPP \IFRU{либо ассемблер}{or assembly language}.
\IFRU{И альтернативы этим старым но популярным языкам, в обозримом будущем, пока что не видно}
{And as it seems in the near future, there are no alternative to these old but popular programming languages}. \\
\\
\IFRU{О ``чистом Си'' также не стоит забывать, огромное количество больших программ продолжаются писаться на нем, 
например}{``Pure C'' should be still considered, a huge number of large programs are still developed in it, e.g.}
\IFRU{ядро Linux}{Linux kernel}, \IFRU{ядра линейки Windows NT}{Windows NT OS line kernels}, Oracle RDBMS, \IFRU{итд}{etc}.

\section{\IFRU{Целевая аудитория}{Target audience}}

\IFRU{Этот сборник заметок предназначен не для начинающих, но и не для экспертов, а скорее для тех, 
кто хочет освежить свои знания по Си/Си++}{This notes collections is not intended for beginners, neither for experts,
it is rather for those who wants to fresh their C/\CPP knowledge}.

\section{\IFRU{Об авторе}{About author}}

\IFRU{Денис Юричев ~--- опытный программист, reverse engineer.
С его резюме можно ознакомиться \href{http://yurichev.com/Dennis_Yurichev.pdf}{здесь}.}
{Dennis Yurichev is an experienced programmer, reverse engineer.
His CV is available \href{http://yurichev.com/Dennis_Yurichev.pdf}{here}.}

\section{\IFRU{Благодарности}{Thanks}}

\IFRU{Андрей ''herm1t'' Баранович, Слава ''Avid'' Казаков}
{Andrey ''herm1t'' Baranovich, Slava ''Avid'' Kazakov}.

%\section{\IFRU{Краудфандинг}{Crowdfunding}}

\IFRU{Эта книга является свободной, находится в свободном доступе, и доступна в виде исходных кодов}
{This book is free, available freely and available in source code form}\footnote{\url{https://github.com/dennis714/RE-for-beginners}} (LaTeX), 
\IFRU{и всегда будет оставаться таковой}{and it will be so forever}.

\IFRU{В мои текущие планы насчет этой книги входит добавление информации на эти темы:}
{My current plans for this books is to add a lot of information about} C++11, flex/bison.

\IFRU{Если вы хотите чтобы я продолжал свою работу и писал на эти темы,
вы можете рассмотреть идею краудфандинга}
{If you want me to continue writing on all these topics, you may consider crowdfunding}.

\IFRU{Со способами краудфандинга можно ознакомиться на странице}
{Ways to crowdfund are available on the page:} \url{http://yurichev.com/crowdfunding.html}

%\subsection{\IFRU{Жертвователи}{Donors}}



\chapter{\IFRU{Целевая аудитория}{Target audience}}

Этот сборник заметок предназначен не для начинающих, но и не для экспертов, а скорее для тех, 
кто хочет освежить свои знания по Си/Си++.

\chapter{\IFRU{Об авторе}{About author}}

\IFRU{Денис Юричев ~--- опытный программист, свободный для найма как программист, reverse engineer, консультант, тренер. 
С его резюме можно ознакомиться \href{http://yurichev.com/Dennis_Yurichev.pdf}{здесь}.}
{Dennis Yurichev is an experienced programmer, available for hire as programmer, reverse engineer, consultant or trainer. 
His CV is available \href{http://yurichev.com/Dennis_Yurichev.pdf}{here}.}

\chapter{\IFRU{Благодарности}{Thanks}}

\IFRU{Андрей ''herm1t'' Баранович, Слава ''Avid'' Казаков}
{Andrey ''herm1t'' Baranovich, Slava ''Avid'' Kazakov}.

\mainmatter

% only chapters here!

\chapter{\IFRU{Общее для Си}{Common for C} \AndENRU \CPP}
% secions here
\section{\IFRU{Определения и объявления}{Definitions and declarations}}

\IFRU{Разница между \IT{определениями (declaration)} и \IT{объявлениями (definition)}}
{The difference between \IT{declarations} and \IT{definitions} is}:

\begin{itemize}
\item
\IFRU{\IT{Определение (declaration)} определяет имя/тип переменной либо имя и типы аргументов а также 
возвращаемого значения ф-ции или метода}
{\IT{Declaration} declares name/type of variable or name and argument types and also returning value type of function
or method}.
\IFRU{Обычно, определения перечисляются в заголовочных .h или .hpp-файлах, чтобы при компиляции отдельного
.c или .cpp-файла, компилятор имел информацию об именах и типах внешних (обычно глобальных) переменных и ф-ций/методов}
{Declarations are usually enlisted in the header .h or .hpp-files, so the compiler while compiling individual
.c or .cpp-file may have an access to the information about all names and types of external (usually global) variables
and functions/methods}.

\IFRU{Типы данных также \IT{определяются}}{Data types are also \IT{declared}}.

\item
\IFRU{\IT{Объявление (definition)} объявляет значение (обычно глобальной) переменной либо тело ф-ции/метода.
Обычно это происходит в одном из .c или .cpp-файлах}
{\IT{Definition} defines a value of (usually global) variable or a body of function/method.
It is usually take place in the individual .c or .cpp-file}.
\end{itemize}

\section{\IFRU{Объявления в Си/Си++}{C/C++ declarations}}
\subsection{Объявления переменных внутри ф-ции}

Раньше, в Си можно было объявлять переменные только в начале ф-ции. А в Си++ --- где угодно.

К тому же, нельзя было объявлять итератор в for() (а в Си++ также можно было):

\begin{lstlisting}
for(int i=0; i<10; i++)
	...
\end{lstlisting}

Новый стандарт C99\ref{C99} позволяет делать это.

\subsubsection{static}

Обычно глобальные переменные (или ф-ции) объявляются как static, так их область видимости ограничивается 
данным файлом. Но локальные переменные внутри ф-ции также можно объявлять как static, тогда эта переменная
будет не локальной, а глобальной, но её область видимости будет ограничена только этой ф-цией.

Например:

\begin{lstlisting}
void fn(...)
{
	for(int x=0; x<100; x++)
	{
		static int times_executed = 0;
		times_executed++;
	}
};
\end{lstlisting}

К примеру, это помогло бы для реализации strtok(), ведь этой ф-ции что-то нужно хранить у себя между вызовами.

\input{common/forward_declarations} % subsection

\subsection{C++11: auto}

Пользуясь \ac{STL}, иногда надоедает каждый раз объявлять тип итератора вроде:

\begin{lstlisting}
for (std::list<int>::iterator it=list.begin(); it!=list.end(); it++)
\end{lstlisting}

Тип it вполне можно получить из list.begin(), поэтому, в начиная со стандарта C++11, можно использовать auto:

\begin{lstlisting}
for (auto it=list.begin(); it!=list.end(); it++)
\end{lstlisting}



\subsection{\IFRU{Описания (definitions)}{Definitions}}
\input{common/definition_of_strings}


\section{\IFRU{Элементы языка}{Language elements}}
\subsection{\IFRU{Комментарии}{Comments}}

\IFRU{Их иногда удобно вставлять прямо в вызов ф-ции, чтобы где-то на виду держать пометку,
что означает некий аргумент}
{It is sometimes useful to insert them right into a function call, in order to have a visual note
about meaning of an argument}:

\begin{lstlisting}
f (val1, /* a very special flag! */ false, /* another special flag here */ true);
\end{lstlisting}

\IFRU{Целый блок кода можно откомментировать при помощи}
{The whole code block can be commented with the help of} \#if
\footnote{\IFRU{директива препроцессора}{preprocessor directive}}:

\begin{lstlisting}
	ta	= aemif_calc_rate(t->ta, clkrate, TA_MAX);
	rhold	= aemif_calc_rate(t->rhold, clkrate, RHOLD_MAX);
#if 0	
	rstrobe	= aemif_calc_rate(t->rstrobe, clkrate, RSTROBE_MAX);
	rsetup	= aemif_calc_rate(t->rsetup, clkrate, RSETUP_MAX);
	whold	= aemif_calc_rate(t->whold, clkrate, WHOLD_MAX);
#endif	
	wstrobe	= aemif_calc_rate(t->wstrobe, clkrate, WSTROBE_MAX);
	wsetup	= aemif_calc_rate(t->wsetup, clkrate, WSETUP_MAX);
\end{lstlisting}

\IFRU{Это может быть удобнее чем традиционный способ потому что текстовый редактор или \ac{IDE} в этом случае
не ``сломает'' отступы при выравнивании}
{This might be more convenient then usual way because the text editor or \ac{IDE} in this case will not ``break''
indentation while auto-indentation}.


\subsection{goto}

\index{goto}
\IFRU{Использование}{Usage of} \IT{goto}\footnote{statement} 
\IFRU{считается плохим тоном и вредным вообще}{is considered as bad taste and harmful}
\cite{Dijkstra:1968:LEG:362929.362947}\cite{Dijkstra:1979:GSC:1241515.1241518}, 
\IFRU{тем не менее, использование его в разумных дозах}{nevertheless, its usage in reasonable
doses}\cite{Knuth:1974:SPG:356635.356640} \IFRU{может облегчить жизнь}{may be very helpful}.

\IFRU{Частый пример, это выход из функции}{One frequent example is return from a function}:

\begin{lstlisting}
void f(...)
{
	byte* buf1=malloc(...);
	byte* buf2=malloc(...);

	...

	if (something_goes_wrong_1)
		goto cleanup_and_exit;

	...
	
	if (something_goes_wrong_2)
		goto cleanup_and_exit;

	...

cleanup_and_exit:
	free(buf1);
	free(buf2);
	return;
};
\end{lstlisting}

\IFRU{Более сложный пример}{More complex example}:

\begin{lstlisting}
void f(...)
{
	byte* buf1=malloc(...);
	byte* buf2=malloc(...);

	FILE* f=fopen(...);
	if (f==NULL)
		goto cleanup_and_exit;

	...

	if (something_goes_wrong_1)
		goto close_file_cleanup_and_exit;

	...
	
	if (something_goes_wrong_2)
		goto close_file_cleanup_and_exit;

	...

close_file_cleanup_and_exit:
	fclose(f);

cleanup_and_exit:
	free(buf1);
	free(buf2);
	return;
};
\end{lstlisting}

\IFRU{Если в данных примерах отказаться от}{If to remove all} \IT{goto}\IFRU{, то придется вызывать}
{ in these examples, one will need to call} \IT{free()} \AndENRU \IT{fclose()}
\IFRU{перед каждым выходом из функции}{before each return from the function} 
(\IT{return})\IFRU{, что здорово замусорит весь код}{which adds a lot of mess}.

\index{Linux}
\IFRU{Использование}{Usage of} \IT{goto} 
\IFRU{в таких случаях одобряется, например, в}{is, for example, approved in} \cite{LinuxKernelCodingStyle}.

%Примеры более ``harmful'' но эффективного использования goto можно найти в исходниках nginx.
% example?


\subsection{for}

\IFRU{В}{The} for()\IFRU{, как известно, три выражения:
первое вычисляется перед началом всех итераций,
второе вычисляется перед каждой итерацией,
третье ~--- после каждой итерации}
{ statement, as we know, has 3 expressions:
1st computing before all iterations begin,
2nd computing before each iteration
and the 3rd ~--- after each iteration}.

\IFRU{И конечно же, там можно указывать что-то отличное от обычного счетчика}
{And of course, there might be written something different from the usual counter}.

\subsubsection{\IFRU{Засада}{Caveat} \#1}

\IFRU{Если написать такое}{If to write this}:

\lstinputlisting{common/for_strlen.cpp}

... \IFRU{то это наверное будет ошибкой}{perhaps this is a mistake}:
\TT{strlen(s)} \IFRU{будет вызываться перед каждой итерацией}{will be called before each iteration} 
~--- \IFRU{такой код генерирует}{that is the code} MSVC 2010\IFRU{}{ generated}.
\IFRU{Впрочем}{However}, GCC 4.8.1 \IFRU{вызывает}{calls} \TT{strlen(s)} 
\IFRU{только один раз, в начале цикла}{only once, at the loop beginning}.

\subsubsection{\IFRU{Запятая}{Comma}}

\IFRU{Запятая}{Comma}\cite[6.5.17]{C99TC3} ~--- \IFRU{не самая понятная для всех штука в Си, 
однако, их очень удобно использовать в определениях в for()}
{is not widely understood C feature, however, it is very useful for using in a for() declarations}.

\IFRU{Например, может пригодится использовать в цикле два итератора одновременно}
{For example, it is useful to have two iterators simultaneously}.
\IFRU{Пусть один просто отсчитывает от 0, прибавляя 1 при каждой итерации,
а второй итератор указывает на элемент в списке}
{Let the first iterator just counts from 0 adding 1 at each iteration, and the second iterator
points to the list element}:

\lstinputlisting{common/for_comma.cpp}

\IFRU{Это выдаст предсказуемый результат}{This will dumps predictable result}:

\begin{lstlisting}
0: 123
1: 456
2: 789
3: 1
\end{lstlisting}

\IFRU{Но к сожалению, определять итераторы вместе с типами в теле самого for() вот так нельзя}
{However, it is not possible to declare iterators with its types in for() clause}:

\begin{lstlisting}
	for (int i=0, std::list<int>::iterator it=l.begin(); it!=l.end(); i++, it++)
\end{lstlisting}

\subsubsection{continue}

\IT{continue} \IFRU{это безусловный переход на конец тела цикла}{is unconditional goto to the end
of loop body}.

\IFRU{Это может быть очень полезно, например, в подобном коде}
{This may be very useful, for example, in such code}:

\begin{lstlisting}
for (...)
{
	if (is_element_satisfied_criteria_1(...)==true)
	{
		// do something need in is_element_satisfied_criteria_2()

		if (is_element_satisfied_criteria_2(...)==true)
		{
			do_something_1();
			do_something_2();
			do_something_3();
		};

	};
};
\end{lstlisting}

... \IFRU{всё это можно легко заменить на более опрятное}{it is all can be replaced by neat}:

\begin{lstlisting}
for (...)
{
	if (is_element_satisfied_criteria_1(...)==false)
		continue;

	// do something need in is_element_satisfied_criteria_2()

	if (is_element_satisfied_criteria_2(...)==false)
		continue;

	do_something_1();
	do_something_2();
	do_something_3();
};
\end{lstlisting}


\subsection{if}

\IFRU{Вместо коротких}{Instead short} \IT{if} \IFRU{желательно использовать}{the shorter} 
\TT{?:}\IFRU{, например}{ clause is advisable, for example}:

\begin{lstlisting}
char* get_name (struct data *s)
{
	return s->name==NULL ? "<name_unknown>" : s->name;
};

...

printf ("val=%s\n", val ? "true" : "false");
\end{lstlisting}

\subsubsection{\CPP: \IFRU{определения переменных в}{variable declarations in} if()}

\IFRU{Это доступно как минимум в стандарте}{It is avaiable at least in} C++03\IFRU{}{ standard}:

\begin{lstlisting}
if (int a=fn(...))
{
     ...
     cout << a;
     ...
};
\end{lstlisting}

\IFRU{Точно также их можно определять и в}{They can be declared likewise also in} switch().


\subsection{switch}

\IFRU{Иногда можно устать писать одно и то же}{It is sometimes boring to write the same again and again}:

\begin{lstlisting}
switch(...)
{
	case 0:
	case 1:
	case 2:
	case 3:
		fn1();
		break;
	case 4:
	case 5:
	case 6:
	case 7:
		fn2();
		break;
};
\end{lstlisting}

\IFRU{А вот это нестандартное расширение GCC}{And this non-standard GCC extension}
\footnote{\url{http://gcc.gnu.org/onlinedocs/gcc/Case-Ranges.html}} \IFRU{может немного всё упростить}
{may make things somewhat simpler}:

\begin{lstlisting}
switch(...)
{
	case 0 ... 3:
		fn1();
		break;
	case 4 ... 7:
		fn2();
		break;
};
\end{lstlisting}

\IFRU{Так что если в планах имеется использовать только компилятор \ac{GCC}, то можно делать так}
{So if you plan to use only \ac{GCC} compiler, it is possible to do so}.

%NOTTRANSLATED
\subsubsection{Объявление переменных внутри switch}

Этого делать нельзя, но зато можно открывать новый блок и объявлять их уже там (в \CPP или начиная с C99):

\begin{lstlisting}
switch(...)
{
	case 0:
		{
			int x=1,y=2;
			fn1(x, y);
		};
		break;
	case 1:
	case 2:

	...
};
\end{lstlisting}


\subsection{sizeof}

\IFRU{Обычно}{Usually}, sizeof() \IFRU{применяют к \glslink{integral type}{интегральным типам}}
{is applied to \glslink{integral type}{integral types}}
\IFRU{либо к структурам}{or to structures},
\IFRU{тем не менее, его можно применять и к массивам, к примеру}
{but nevertheless it is possible to apply it to arrays as well}:

\begin{lstlisting}
	char buf[1024];
	snprintf(buf, sizeof(buf), "...");
\end{lstlisting}

\IFRU{В противном случае, если указывать длину массива}{Otherwise, if to specify array length} ($1024$) 
\IFRU{в двух местах}{in both places} 
(\IFRU{в определении \TT{buf} и как второй аргумент \TT{snprintf()}}
{in \TT{buf} declaration and as a second argument of \TT{snprintf()}}),
\IFRU{то и изменять это значение придется каждый раз в обоих местах, а об этом легко забыть}
{then the value is have to be changed at the both places each time, and it is easy to forget about this}.

\IFRU{Если нужны wide-строки, то}{If one need wide-strings, then} sizeof() 
\IFRU{можно применять к}{can be applied to} \IT{wchar\_t} 
(\IFRU{который, на самом деле, 16-битный тип данных \IT{short}}
{which is in turn, 16-bit data type \IT{short}}):

\begin{lstlisting}
	wchar_t buf[1024];
	swprintf(buf, sizeof(buf)/sizeof(wchar_t), "...");
\end{lstlisting}

sizeof() \IFRU{возвращает длину в байтах, так что здесь он будет равен}{returns the size in bytes, so it will
be here} $1024*2$, \IFRU{т.е.}{i.e.}, $2048$. 
\IFRU{Но мы можем разделить это значение на длину одного элемента массива}
{But we can divide this value by length of one array element} (\IT{wchar\_t})
\IFRU{в байтах ($2$)}{is $2$ in bytes},
\IFRU{чтобы получить количество элементов в массиве}{in order to get elements number in array} ($1024$).

sizeof() \IFRU{можно применять и к массивам структур}{can be applied to array of structures}:

\begin{lstlisting}
struct phonebook_entry
{
	char *name;
	char *surname;
	char *tel;
};

struct phonebook_entry phonebook[]=
{
	{ "Kirk", "Hammett", "555-1234" },
	{ "Lars", "Ulrich", "555-5678" },
	{ "James", "Hetfield", "555-1122" },
	{ "Robert", "Trujillo", "555-7788" }
};

void dump (struct phonebook_entry* input)
{
	for (int i=0; i<sizeof(phonebook)/sizeof(struct phonebook_entry); i++)
		printf ("%s %s - %s\n", input[i].name, input[i].surname, input[i].tel);
};
\end{lstlisting}

sizeof(phonebook) ~--- \IFRU{это размер всего массива структур в байтах}{is a size of the whole array
of structures in bytes}.
\TT{sizeof(struct phonebook\_entry)} ~--- \IFRU{это размер одной структуры в байтах}{is a size of one structure
in bytes}.
\IFRU{Делением мы узнаем количество структур в массиве}{By division we get number of structures in an array}.


\subsection{\IFRU{Указатели}{Pointers}}
\label{pointers}

\IFRU{Как однажды сказал Дональд Кнут в интервью}{As Donald Knuth once said in the interview}
\cite{KnuthInterview1993}, \IFRU{то как в Си устроены указатели, это является
очень удачной инновацией в языках программирования по тем временам}{the way C handles pointers, was
a brilliant innovation at the time}.

\IFRU{Итак, определимся с терминологией}{So let us fix terminology}.
\IFRU{Указатель это просто адрес какого-то элемента в памяти}{A pointer is a just an address
of some element in memory}.
\IFRU{Указатели настолько популярны,
потому что в какую-то функцию намного проще передать просто адрес объекта в памяти, 
вместо того чтобы передавать весь объект ~--- ведь это абсурдно}
{The reason pointers are so popular is that an address of object is much easier to pass into a function
instead of passing the whole object ~--- because it is absurdly}.

\IFRU{К тому же, вызываемая функция, например, обрабатывающая ваш массив данных,
просто изменит что-то в нем, вместо того чтобы возвращать новый, измененный массив данных, что тоже абсурдно}
{Besides, calling function, for example, processing a data array, will just change something in it instead
of returning new one, which is absurdly too}.

\IFRU{Возьмем простой пример}{Let's take a simple example}.
\IFRU{Стандартная функция}{The standard C function} \IT{strtok()} 
\IFRU{делит строку на подстроки, используя заданный символ как разделитель}{just divide string by substrings
using specified character as delimiter}.
\IFRU{К примеру, мы можем подать на вход строку}{For example, we may specify the string}
\TT{The quick brown fox jumps over the lazy dog} 
\IFRU{и задать пробел в качестве разделителя}{and set the space as a delimiter}.

\lstinputlisting{common/strtok_ex1.c}

\IFRU{Мы в итоге получим на выходе}{What we got on output}:

\begin{lstlisting}
The
quick
brown
fox
jumps
over
the
lazy
dog
\end{lstlisting}

\IFRU{Что тут в реальности происходит, это то что ф-ция}
{What is going on here is that the} \IT{strtok()} 
\IFRU{просто находит в заданной строке следующий пробел}{just searching for the next space in the input string}
(\IFRU{либо иной заданный разделитель}{or any other delimiter set}),
\IFRU{записывает туда}{writes} $0$\IFRU{}{ to it} 
(\IFRU{что по соглашениям текстовых строк в Си является концом строки}
{this is string terminator by C conventions}) 
\IFRU{и возвращает указатель на это место}{and returns a pointer to that place}.

\IFRU{В качестве недостатка}{As a shortcoming, it can be said that the} \IT{strtok()} 
\IFRU{можно отметить, что эта ф-ция ``портит'' входную строку, записывая нули на месте разделителей}
{function ``garbles'' input string, writing zeros at the delimiter's places}.

\IFRU{Но вот что важно заметить: никакие строки или подстроки не копируются в памяти}
{What is worth to note: no strings or substrings copied in memory}.
\IFRU{Входная строка остается там же где и лежала}{The input string is still on its own place}.

\IFRU{В \IT{strtok()} передается только указатель на нее, или, её адрес}
{It is only pointer to the string (or its address) is passed to the \IT{strtok()} function}.

\IFRU{Эта ф-ция затем, после того как записывает $0$, возвращает \IT{адрес} каждого следующего ``слова''}
{The function then after it writes $0$, returns \IT{address} of each consecutive ``word''}.

\IFRU{Адрес ``слова'' затем подается на вход в}{The address of the ``word'' is then passed to the}
\IT{printf()}, 
\IFRU{где происходит его вывод на консоль}{where it dumped to the console}.

\IFRU{Обратите также внимание на то что в исходнике присутствует и некорректное определение \IT{str}}
{Please also take a note that an incorrect declaration of \IT{str} is present in the source code}.

\IFRU{Оно тем некорректное что в Си строка имеет тип}{It is incorrect in that sense that the C string
has type} \IT{const char*}, \IFRU{то есть, распологается в константном сегменте данных, 
защищенным от записи}{i.e., it is located in the constant data segment, write-protected}.

\IFRU{Если так сделать, то}{If do so, then the} \IT{strtok()} 
\IFRU{не сможет модифицировать входную строку записывая туда нули и процесс ``упадет''}
{will not be able modify the input string by writing zeros there and the process will crash}.

\IFRU{Так что, в нашем примере, строка выделяется как массив}
{So, in our example, the string is allocated as an array of} \IT{char}
\IFRU{а не массив}{instead of array of} \IT{const char}.

\IFRU{Обобщая, скажем что работа со строками в Си происходит только лишь используя адреса этих строк}
{Generalizing, we may say all standard C strings functions works with them using only their addresses}.

\IFRU{К примеру, ф-ция сравнения строк}{For example, the function of string comparison} \IT{strcmp()} 
\IFRU{берет на вход два адреса двух строк и по одному символу сравнивает их}
{takes addresses of two strings and compare them by one character}.
\IFRU{Было бы очень абсурдно копировать куда-то эти две строки лишний раз, чтобы}
{It would be absurdly to copy these strings to some other place so the} \IT{strcmp()} 
\IFRU{обработала их}{may process them}.

\IFRU{Трудность понимания указателей в Си связана с тем, что указатель это ``часть'' объекта}
{The difficulty of C pointers understanding is in the fact that pointer is a ``part'' of an object}.
\IFRU{Указатель на строку, это не сама строка}{The pointer to the string is not the string itselfs}.
\IFRU{Сама строка еще должна где-то в памяти хранится, под нее нужно перед этим выделять место, итд}
{The string should be placed somewhere in memory, a memory should be allocated for it before, etc}.

\IFRU{В}{In higher level} \ac{PL} 
\IFRU{более высокого уровня, 
объект и указатель на него могут быть представлены как единое целое, что облегчает понимание}
{an object and a pointer may be represented as a single whole, and that is makes understanding simpler}.

\IFRU{Впрочем, это не значит что в этих \ac{PL} строки и 
иные объекты неразумно копируются много раз при передаче в другие функции ~---
там точно так же как и в Си используются указатели, но просто эта механика скрыта от программиста}
{It is however not mean that a strings and other objects are copied misspendinly in these \ac{PL} ~---
a pointers are used there internally likewise as in C, but this mechanisms are hidden from the programmer}.

\subsubsection{\IFRU{Синтаксический сахар для}{Syntactic sugar for} array[index]}

\IFRU{Ради упрощения, можно сказать что в Си нет массивов вообще,
а есть только синтаксический сахар для выражений вроде}
{For the sake of simplification, it could be said that C has not arrays at all,
it has only syntactic sugar for expressions like} \IT{array[index]}.

\IFRU{К примеру, возможно вы видели такой трюк}{For example, perhaps you saw this trick}:

\begin{lstlisting}
printf ("%c", 3["hello"]);
\end{lstlisting}

\IFRU{Это выдаст}{It outputs} 'l'. 

\IFRU{Это происходит, потому что любое выражение}{This happens because the expression} \IT{a[i]}, 
\IFRU{на самом деле преобразовывается в}{is in fact translating into} \IT{*(a+i)}
\cite[6.5.2/1]{C99TC3}.
\IT{3["hello"]} \IFRU{преобразовывается в}{is translated into} \IT{*(3+"hello")},
\IFRU{а}{and} \IT{"hello"} \IFRU{это просто указатель на массив символов, типа}{is just
a pointer to array of characters like} \IT{const char*}.

\IT{3+"hello"} \IFRU{это в итоге указатель на часть строки, то есть}{as a result is a pointer
to the part of string}, \IT{"lo"}. \IFRU{А}{And} \IT{*("lo")} \IFRU{это cимвол}{is a} 'l'. 
\IFRU{Вот почему это работает}{That is why it works}.

\IFRU{Но так врядли стоит писать, если вы конечно не готовите программу на конкурс}
{It is not advisable to write such things unless your intentions is to participate in}
\ac{IOCCC}\footnote{\url{http://www.ioccc.org/}}.
\IFRU{Так что я привел этот пример, чтобы наглядно показать, 
что выражения вроде}{So I demostrated the trick here in order to explain that the} \IT{a[i]}
\IFRU{это синтаксический сахар}{is a syntactic sugar}.

\IFRU{При некотором упорстве, в Си вообще можно обойтись без индексации массивов,
хотя выглядеть это будет не очень эстетично}
{With some persistence, it is possible not to use indexed arrays in C at all, but it will not be
very aesthetical though}.

\IFRU{Кстати, так легко понять как работают отрицательные индексы массивов}
{By the way, now it is easy to understand how negative array indexes works}.
\IT{a[-3]} \IFRU{просто преобразуется в}{is translating into} \IT{*(a-3)}, 
\IFRU{так адресуется элемент лежащий перед самим массивом}{and that is how the element before
array itself is addressed}.
\IFRU{И хотя это вполне возможно, так можно делать только если вы точно знаете, что вы делаете}
{Despite it is possible, one should use this only if one exactly knows what one does}.

\IFRU{Еще один трюк связанный с негативными индексами: например, когда вы привыкли адресовать
массивы начиная не с 0, а с 1 (как в FORTRAN), тогда можно сделать такое}
{Another array negative indexes trick: for example, if you used to address arrays starting
not from 0 but from 1 (like in FORTRAN), then you may do something like this}:

\begin{lstlisting}
void f (int *a)
{
	a[1]=...; // first element
	a[2]=...; // second element
};

int main()
{
	int array[10];
	f(&a[-1]); // passing a pointer to the one int element before array
};
\end{lstlisting}

\IFRU{Хотя снова нельзя с уверенностью сказать что использование таких трюков оправдано}
{But again it is hard to say if the trick is justified}.

\IFRU{Так что в Си массив это, в каком-то смысле, это просто место в памяти под массив плюс указатель,
указывающий на него}{So C array in some sense is just a memory block plus a pointer to it}.

\IFRU{Вот почему имя массива в Си можно считать за указатель}{That is why array name in C may be treated
as a pointer}:

\IFRU{Если вы объявите глобальную переменную}{If to declare global variable} \IT{int a[10]},
\IFRU{то}{then} \IT{(a)} \IFRU{будет иметь тип}{will have the type} \IT{int*}.
\IFRU{Позже, когда где-то в коде вы укажете}{When further the following expression will be appeared:}
\IT{x=a[5]}, \IFRU{выражение будет преобразовано в}{the expression will be translated into} \IT{x=*(a+5)}.
\IFRU{От начала массива (то есть, первого элемента массива), будет отсчитано 5 элементов,
затем оттуда прочитается элемент для записи в}{From the array start (i.e., from the first array element)
5 elements will be counted, then the element will be read from that point for the storing it into} \IT{(x)}.

\label{PtrArith}
\subsubsection{\IFRU{Арифметика указателей}{Pointer arithmetic}}

\IFRU{Простой пример}{Simple example}:

\lstinputlisting{common/phonebook1.c}

\IFRU{Мы объявляем глобальный массив из структур}{We define a global array of structures}.
\IFRU{Если скомпилировать это в GCC с ключом}{If to compile this in GCC with
a} \IT{-S} \IFRU{либо в MSVC с ключом}{key or in MSVC with a}
\IT{/Fa}\IFRU{, мы увидим в листинге на ассемблере то, как компилятор расположил эти строки}
{ key, we will see in assembly language listing how the compiler placed these strings}.

\IFRU{Расположил он их как линейный массив указателей на строки, вот так}
{The compiler placed them as a linear array of string pointers, that is how}:

\ifdefined\Cell
    ERROR: \textbackslash Cell is already defined
\fi
\ifdefined\StringAddress
    ERROR: \textbackslash StringAddress is already defined
\fi

\newcommand{\Cell}{\IFRU{ячейка}{cell}\xspace}
\newcommand{\StringAddress}{\IFRU{адрес строки}{string address}\xspace}

\begin{center}
\begin{tabular}{ | l | l | }
\hline
  \Cell 0    & \StringAddress ``Kirk'' \\
  \Cell 1    & \StringAddress ``Hammett'' \\
  \Cell 2    & \StringAddress ``555-1234'' \\
  \Cell 3    & \StringAddress ``Lars'' \\
  \Cell 4    & \StringAddress ``Ulrich'' \\
  \Cell 5    & \StringAddress ``555-5678'' \\
  \Cell 6    & \StringAddress ``James'' \\
  \Cell 7    & \StringAddress ``Hetfield'' \\
  \Cell 8    & \StringAddress ``555-1122'' \\
  \Cell 9    & \StringAddress ``Robert'' \\
  \Cell 10   & \StringAddress ``Trujillo'' \\
  \Cell 11   & \StringAddress ``555-7788'' \\
  \Cell 12   & 0 \\
  \Cell 13   & 0 \\
  \Cell 14   & 0 \\
\hline
\end{tabular}
\end{center}

\IFRU{Ф-ции}{The functions} \IT{dump1()} \AndENRU \IT{dump2()}
\IFRU{эквивалентны}{are equivalent to each other}.

\IFRU{Но в первой счетчик}{But in the first function counter} \IT{(i)} 
\IFRU{начинается с 0 и к нему прибавляется 1 на каждой итерации}{is beginning at 0 and 1 is added
to it at each iteration}.

\IFRU{Во второй ф-ции \glslink{iterator}{итератор}}{In the second function \gls{iterator}} \IT{(i)} 
\IFRU{указывает на начало массива и затем, к нему прибавляется длина структуры}
{points to the beginning of the array and then size of structure is added to it}
(\IFRU{а не 1 байт, как можно поначалу ошибочно подумать}{instead of 1 byte,
how one can mistakenly think}),
\IFRU{таким образом, на каждой итерации}{this mean, at each iteration},
\IT{(i)} \IFRU{указывает на следующий элемент массива}{points to the next element of array}.

\subsubsection{\IFRU{Указатели на функции}{Pointer to functions}}

\IFRU{Часто используются для callback-в}{Often used for callbacks}.

\IFRU{Из-за того что можно напрямую задавать адрес функции, в embedded-программировании,
так можно сделать переход по нужному адресу}{The address of function can be set directly,
so it is possible to jump to an arbitrary address, it is useful in embedded-programming}:

\begin{lstlisting}
void (*func_ptr)(void) = (void (*)(void))0x12345678;
func_ptr();
\end{lstlisting}

\IFRU{Впрочем, нужно помнить, что это не совсем аналог безусловного перехода, потому что в стеке 
сохраняется адрес возврата, может быть что-то еще}
{However, it should be noted that this is not the same thing
as unconditional jump, because return address is saved in the stack, maybe something else}.


\subsection{\IFRU{Операторы}{Operators}}

\subsubsection{==}

\IFRU{Очень неприятные ошибки возникают если в условии}
{Somewhat unpleasant mistakes may appear if in} \IT{if(a==3)} 
\IFRU{опечататься и написать}{condition become} \IT{if(a=3)}\IFRU{}{ in result of typo}.
\IFRU{Ведь выражение}{Because the statement} \IT{a=3} ``\IFRU{возвращает}{returns}'' 3,
\IFRU{а}{and} 3 \IFRU{это не}{is not a} 0, \IFRU{поэтому условие \IT{if()} всегда будет 
срабатывать}{so the \IT{if()} condition will always trigger}.

\IFRU{Раньше, для защиты от подобных ошибок, была мода писать наоборот}
{It was fashionable in past to protect from such mistakes by writing}: \IT{if(3==a)}, 
\IFRU{таким образом}{and thus},
\IFRU{если опечататься, выйдет}{we will get a} \IT{if(3=a)}\IFRU{, компилятор тут же выдаст ошибку}
{ in case of typo and the compiler will report error instantly}.

\IFRU{Тем не менее, в наше время, компиляторы обычно предупреждают если написать}
{Nevertheless, in modern times, compilers are usually warns if to write} \IT{if(a=3)}, 
\IFRU{так что, наверное, менять местами элементы выражения уже не обязательно}
{so elements swapping in conditions is probably not necessary these days}.

\subsubsection{Short-circuit evaluation 
\IFRU{и артефакт приоритетов операций}{and operator precedence artefact}}

\IFRU{Разберем что такое}{Let's see what is} \IT{short-circuit}
\IFRU{\footnote{дословный перевод на русский: ``короткое замыкание''}} \IT{evaluation}.

\IFRU{Это когда в выражении}{It is when in the expression} \IT{if(a \&\& b \&\& c)},
\IFRU{часть}{the part} \IT{(b)} \IFRU{будет вычисляться только если}{will be calculated only if} 
\IT{(a)} ~--- \IFRU{истинна}{is true},
\IFRU{а}{and} \IT{(c)}
\IFRU{будет вычисляться только если}{will be calculated only if} \IT{(a)} \AndENRU \IT{(b)} 
~--- \IFRU{оба истинны}{are both true}.
\IFRU{И вычисляться они будут именно в таком порядке, как указано}{and they will be computed
exactly in the same order as specified}.

\IFRU{Иногда можно встретить подобное}{Sometimes we can see expression like}: 
\IT{if (p!=NULL \&\& p->field==123)} ~--- \IFRU{и это совершенно правильно}{and this is completely
correct}.
\IFRU{Поле}{The field} \IT{field} \IFRU{в структуре, на которую указывает}{in the structure to which} 
\IT{(p)}\IFRU{, будет вычисляться только если указатель}{ points will be computed only if the pointer}
\IT{(p)} \IFRU{не равен}{not equals to} \IT{NULL}.

\IFRU{То же касается и операции}{The same story about} ``\IFRU{или}{or}'', 
\IFRU{если в выражении}{if in the expression} \IT{if (a || b || c)} \IFRU{подвыражение}{subexpression} 
\IT{(a)} \IFRU{будет ``истинно''}{will be ``true''},
\IFRU{остальные вычисляться не будут}{others will not be computed}.

\IFRU{Это может быть удобно для вызова нескольких ф-ций}{It is useful when one need to call several
functions}:
\IT{if (get\_flagsA() || get\_flagsB() || get\_flagsC())} ~--- 
\IFRU{если первая или вторая ф-ция вернет}{if first or second will return} \IT{true}, 
\IFRU{остальные даже не будут вызываться}{others will not be called at all}.

\IFRU{Эта особенность есть не только в Си/Си++}{This feature is not unique for C/C++}
\footnote{\IFRU{Здесь список}{Here is a list of} \ac{PL} \IFRU{где присутствует}{where}
\IT{short-circuit evaluation}\IFRU{}{ exist}\url{https://en.wikipedia.org/wiki/Short-circuit_evaluation}.
\IFRU{Кстати, хотя это и не про Си, но все же интересно}{It is not about C, but interesting nevertheless}:
\IFRU{в bash если писать}{if to write in bash} \IT{cmd1 \&\& cmd2 \&\& cmd3}, 
\IFRU{то каждая следующая команда будет исполняться только если предыдущая закончилась с успехом}
{then each next command will be executed only if the previous was executed with success}.
\IFRU{Это также}{It is also} \IT{short-circuit}.}.

\IFRU{Когда-то давно}{Some time ago}\cite{dmr:1995}, 
\IFRU{в языках B и BCPL (предтечи Си) не было операторов}{there was no operators} 
\IT{\&\&} \AndENRU \IT{||}\IFRU{}{ in B and BCPL (C precursors)}, 
\IFRU{но чтобы реализовать в них}{but in order to implement}
\IT{short-circuit evaluation}\IFRU{}{ in them}, 
\IFRU{приоритет операций}{the priority of the operators} \IT{\&} \AndENRU \IT{|} 
\IFRU{сделали больше, чем, например, у}{was made higher than in} \IT{\^} \OrENRU \IT{+}
\footnote{\IFRU{Приоритет операций в Си++}{C++ Operator Precedence}: \url{http://en.cppreference.com/w/cpp/language/operator_precedence}}.

\IFRU{Это позволяло писать что-то вроде}{That allowed to write something like} \IT{if (a==1 \& b==c)} 
\IFRU{используя}{while using} \IT{\&} \IFRU{вместо}{instead of} \IT{\&\&}.
\IFRU{Вот откуда взялся этот артефакт в приоритетах}{That is where that artefact came from}. \\
\\
\IFRU{Так что, нередкая ошибка это забывать о высоком приоритете этих операций и писать, например}
{So one often mistake is to forget about higher priority of these operators and to write e.g.},
\IT{if (a\&1==0)}, \IFRU{в то время как это нужно брать в скобки}{which should be taken
in brackets}: \IT{if ((a\&1)==0)}.

\subsubsection{! \AndENRU \~{}}

\~{} (\IFRU{тильда}{tilde}) \IFRU{это побитовое инвертирование всех бит в значении}
{is a bitwise inversion of all bits in a value}.

\IFRU{Эта операция часто используется для инвертирования результатов действия ф-ций}
{The operation is often used for function results invertion}.
\IFRU{Например}{For example}, strcmp() \IFRU{в случае равенства строк возвращает}{in case of strings
equivalence, returns} 0.
\IFRU{Поэтому можно писать}{So we can write}:

\begin{lstlisting}
if (!strcmp(str1, str2))
{
	// do something in case of strings equivalence
};
\end{lstlisting}

... \IFRU{вместо}{instead of} \TT{if (strcmp (...)==0)}. \\
\\
\IFRU{Также, два подряд восклицательных знака применяется для трасформирования любого значения в тип bool
по правилу}{Also, two consecutive exclamation points can be used for
transforming any value into \IT{bool} type}: 0 ~--- false (0); \IFRU{не ноль}{not zero} ~--- true (1).

\IFRU{Например}{For example}:

\begin{lstlisting}
bool some_object_present=!!struct->object;
\end{lstlisting}

\IFRU{Или}{Or}:

\begin{lstlisting}
#define FLAG 0x00001000
bool FLAG_present=!!(value & FLAG);
\end{lstlisting}

\IFRU{А также}{And also}:

\begin{lstlisting}
bool bit_7_set=!!(value & (1<<7));
\end{lstlisting}


\subsection{\IFRU{Массивы}{Arrays}}

\IFRU{В}{In} C99(\ref{C99})
\IFRU{можно передавать массив в аргументах ф-ции}{it is possible to pass array in the function arguments}.

\IFRU{Собственно, массив байт можно было передавать и в более старых стандартах Си,
кодируя байты в строке, включая ноль, примерно так}
{Strictly speaking, array of bytes can be passed in the older C standards, by encoding all bytes
including zero in a string}
(\IFRU{узнать, встречается ли байт}{let's determine if the byte} (\IT{c})
\IFRU{в массиве байт}{is present in a byte array})(\ref{memchr}):

\begin{lstlisting}
if (memchr ("\x12\x34\x56\x78\x00\xAB", c, 6))
	...
\end{lstlisting}

\IFRU{Байты после ноля нормально кодируются}{The bytes after zero is encoded finely}.

\IFRU{Но в C99 теперь можно передавать массив значений других типов, например}
{However, it is possible in C99 to pass an array of other types, like}
unsigned int:

\begin{lstlisting}
unsigned int find_max_value (unsigned int *array, size_t array_size);

unsigned int max_value=find_max_value ((unsigned[]){ 0x123, 0x456, 0x789, 0xF00 }, 4);
\end{lstlisting}

\IFRU{Поиск в массиве можно реализовать при помощи ф-ций}
{Search for the element in the array can be implemented with the help of} bsearch() \OrENRU lfind()(\ref{bsearch_lfind}),
\IFRU{поиск и вставку при помощи}{search and insertion with the help of} lsearch()
\footnote{\IFRU{работает также как и}{works like} lfind(), 
\IFRU{но при отсутствии искомого элемента, добавляет его в массив}
{but if the element is absent there, it also inserts it}}.

\subsubsection{\IFRU{Инициализация}{Initialization}}

\IFRU{В GCC можно}{It is possible in GCC}
\footnote{\url{http://gcc.gnu.org/onlinedocs/gcc/Designated-Inits.html}} \IFRU{инициализировать части массивов}
{to initialize array parts}:

\begin{lstlisting}
struct a
{
	int f1;
	int f2;
};
 
struct a tbl[8] =
{
[0x03] =	{ 1,6 },
[0x07] =	{ 5,2 } 
};
\end{lstlisting}

... \IFRU{но это нестандартное расширение}{but it is non-standard extension}.


\subsection{struct}

\index{C99}
\IFRU{В}{In the} C99(\ref{C99}) \IFRU{можно инициализировать отдельные поля структур}{it is possible
to initialize specific structure fields}.
\IFRU{Пропущенные будут заполнены нулями}{Fields not set will be filled by zeroes}.
\IFRU{Такого очень много в ядре Linux}{A lot of such examples can be found in Linux kernel}.

\begin{lstlisting}
struct color
{
	int R;
	int G;
	int B;
};

struct color blue={ .B=255 };
\end{lstlisting}

\IFRU{И даже более того, можно создавать структуру прямо в аргументах ф-ции, например}
{And even more than that, it is possible to create a structure right in the function arguments, e.g.}:

\begin{lstlisting}
struct color
{
	int R;
	int G;
	int B;
};

void print_color_info (struct color *c)
{
	printf ("%d %d %d\n", c->R, c->G, c->B);
};

int main()
{
	print_color_info(&blue);
	print_color_info(&(struct color){ .G=255 });
};
\end{lstlisting}

\IFRU{Точно также стурктуру можно и возвращать из ф-ции}
{The structure is also can be returned from the function in the same way}:

\begin{lstlisting}
struct pair
{
	int a;
	int b;
};

struct pair f1(int a, int b)
{
	return (struct pair) {.a=a, .b=b};
};
\end{lstlisting}

\IFRU{Помимо всего прочего, о структурах также много есть в разделе}
{Read more about structures in the section} ``\COOPname''(\ref{COOP}).

\subsubsection{\IFRU{Расположение полей в структурах}{Structure fields placement} (cache locality)}

\IFRU{В современных x86-микропроцессорах (как Intel, так и AMD) имеется кеш-память разных уровней}
{In modern x86 CPUs (both Intel and AMD) a multi-level cache-memory present}.
\IFRU{Самая быстрая кеш-память (L1),
разделена на 64-байтные элементы (кеш-линии) 
и любое обращение к памяти заполняет сразу всю линию}
{The fastest cache-memory (L1) is divided by 64-byte elements (cache-lines) and any
memory access resulting in filling a whole line}\cite{AgnerFog}.

\IFRU{Можно сказать, что любое обращение к памяти (по выровненным адресам) подтягивает в кеш сразу 64 байта}
{It can be said that any memory access (on aligned boundary) fetches 64 bytes into cache at once}.

\IFRU{Поэтому, если некая структура данных имеет размер более 64-х байт, очень важно разделить её на две части:
наиболее востребованные поля и менее востребованные}
{So if a data structure is larger than 64 bytes, it is very important to divide it by 2 parts:
the most demanded fields and the less ones}.
\IFRU{Самые востребованные поля желательно разместить в пределах первых 64-х байт}
{It is desirable to place the most demanded fields in the first 64 bytes}.

\IFRU{Это же касается и классов в \CPP}{\CPP classes are also concerned}.


\subsection{union}

union часто используется, когда в каком-то месте структуры можно хранить разные типы на выбор.
К примеру:

\begin{lstlisting}
union
{
	int i; // 4 bytes
	float f; // 4 bytes
	double d; // 8 bytes
} u;
\end{lstlisting}

Такой union позволяет хранить одну из этих трех переменных на выбор. Занимать он будет места столько же,
сколько максимальный элемент (double) --- 8 байт.

union часто используют для обращения к какому-то типу данных как к другому.

Например, как известно, каждый XMM-регистр в SSE может представлять собой 16 байт, 8 16-битных слов,
4 32-битных слова, 2 64-битных слова, 4 float-значения и 2 double-значения. Так можно описать его:

\begin{lstlisting}
union
{
	double d[2];
	float f[4];
	uint8_t b[16];
	uint16_t w[8];
	uint32_t i[4];
	uint64_t q[2];
} XMM_register;

union XMM_register reg1;

reg.u.d[0]=123.4567;
reg.u.d[1]=89.12345;

// here we can use reg.u.b[...]

\end{lstlisting}

Это также очень удобно использовать вместе со структурой, где поля имеют битовую гранулярность.
Это флаги x86-процессора:

\begin{lstlisting}
typedef struct _s_EFLAGS
{
    unsigned CF : 1;
    unsigned reserved1 : 1;
    unsigned PF : 1;
    unsigned reserved2 : 1;
    unsigned AF : 1;
    unsigned reserved3 : 1;
    unsigned ZF : 1;
    unsigned SF : 1;
    unsigned TF : 1;
    unsigned IF : 1;
    unsigned DF : 1;
    unsigned OF : 1;
    unsigned IOPL : 2;
    unsigned NT : 1;
    unsigned reserved4 : 1;
    unsigned RF : 1;
    unsigned VM : 1;
    unsigned AC : 1;
    unsigned VIF : 1;
    unsigned VIP : 1;
    unsigned ID : 1;
} s_EFLAGS;

typedef union _u_EFLAGS
{
    uint32_t flags;
    s_EFLAGS s;
} u_EFLAGS;
\end{lstlisting}

Можно таким образом загрузить флаги как 32-битное значение в поле flags, а затем из поля s обращаться
к отдельным битам. Либо наоборот, модифицировать биты, затем прочитать поле flags. Такого очень много
в исходниках ядра Linux.

\subsubsection{tagged union}

Это union плюс флаг (tag), определяющий тип union. К примеру, если нам нужна какая-то переменная,
которая может быть как числом, так и числом с плавающей точкой, так и текстовой строкой (как переменные
в динамически-типизированных ЯП), то мы можем объявить такую структуру:

\begin{lstlisting}

enum var_type
{
	INT,
	DOUBLE,
	STRING
};

struct
{
	enum var_type tag; // 4 bytes
	union
	{
		int i; // 4 bytes
		double d; // 8 bytes
		char *string; // 4 bytes (on 32-bit architecture)
	} u;
} variable;
\end{lstlisting}

Суммарная длина такой структуры будет $8+4=12$ байт. В любом случае, это компактнее, чем выделять
поля для переменной каждого возможного типа.

Начиная с C11\cite{C11}, \TT{u} можно не указывать, это называется ``анонимный union'':

\begin{lstlisting}
struct
{
	enum var_type tag; // 4 bytes
	union
	{
		int i; // 4 bytes
		double d; // 8 bytes
		char *string; // 4 bytes (on 32-bit architecture)
	};
} variable;
\end{lstlisting}

... и обращаться к полям union просто как к \TT{variable.i}, \TT{variable.d}, итд.



\section{\IFRU{Препроцессор}{Preprocessor}}

Препроцессор обрабатывает директивы начинающиеся с \# --- \#define, \#include, \#if, итд.

\section{Стандартные для компиляторов и ОС значения}

\begin{itemize}
\item \TT{\_DEBUG} --- отладочная сборка.
\item \TT{NDEBUG} --- неотладочная (release) сборка.
\item \TT{\_\_linux\_\_} --- ОС Linux.
\item \TT{\_WIN32} --- ОС Windows. Присутствует как и в x86-проектах, так и в x64.
\item \TT{\_WIN64} --- Присутствует в x64-проектах для ОС Windows.
\item \TT{\_\_cplusplus} --- присутствует в Си++ проектах.
\item \TT{\_MSC\_VER} --- компилятор MSVC.
\item \TT{\_\_GNUC\_\_} --- компилятор GCC.
\end{itemize}

Так можно писать разные участки кода для разных компиляторов и ОС.

\subsection{``Пустой'' макрос}

Всем известны макросы не объявляющие никаких значений, например \IT{\_DEBUG}.
Обычно, только проверяется наличие его или отсутствие.
Вот еще пример полезного ``пустого'' макроса:

В заголовочных файлах Windows API мы можем увидеть такое:

\begin{lstlisting}
typedef NTSTATUS
(NTAPI *TDI_REGISTER_CALLBACK)(
  IN PUNICODE_STRING DeviceName,
  OUT HANDLE *TdiHandle);

...

typedef NDIS_STATUS
(NTAPI *CM_CLOSE_CALL_HANDLER)(
  IN NDIS_HANDLE  CallMgrVcContext,
  IN NDIS_HANDLE  CallMgrPartyContext  OPTIONAL,
  IN PVOID  CloseData  OPTIONAL,
  IN UINT  Size  OPTIONAL);
\end{lstlisting}

IN, OUT и OPTIONAL --- это ``пустые'' макросы объявленные так:

\begin{lstlisting}
#ifndef IN
#define IN
#endif
#ifndef OUT
#define OUT
#endif
#ifndef OPTIONAL
#define OPTIONAL
#endif
\end{lstlisting}

Для компилятора они никакой информации не несут, они предназначены только для документирования, показать,
какие параметры ф-ций зачем нужны.

\subsection{Частые ошибки}

\subsubsection{\#1}

К примеру, вы хотите создать макрос для возведения числа в квадрат:

\begin{lstlisting}
#define square(x)      x*x
\end{lstlisting}

Это ошибка, потому что выражение \IT{square(a+b)} в итоге ``развернется'' в $a+b*a+b$, что, разумеется, совсем
не то что хотелось. Поэтому в определнии макроса все переменные, и сам макрос, нужно ``изолировать'' скобками:

\begin{lstlisting}
#define square(x)      ((x)*(x))
\end{lstlisting}

Пример из файла minmax.h из MinGW:

\begin{lstlisting}
#define max(a,b) (((a) > (b)) ? (a) : (b))
...
#define min(a,b) (((a) < (b)) ? (a) : (b))
\end{lstlisting}

\subsubsection{\#2}

Если вы где-то определяете какую-то константу:

\begin{lstlisting}
#define N 1234
\end{lstlisting}

... затем где-то дальше переопределяете её снова, то компилятор промолчит, и это приведет к трудновыявляемой
ошибке.

Поэтому константы желательнее определять как глобальные переменные с модификатором const.


\section{\IFRU{Предупреждения компилятора}{Compiler warnings}}

\IFRU{Стоит ли постоянно держать включенным ключ}{Is it worth to turn on} \TT{-Wall} \InENRU GCC 
\OrENRU \TT{/Wall} \InENRU MSVC, \IFRU{то есть, чтобы выводить
все возможные предупреждения (warnings)}{in other words, to dump all possible warnings}?
\IFRU{Да, однозначно стоит, так можно зараннее найти мелкие ошибки}{Yes, it is worth to do it,
in order to determine quickly small errors}.
\IFRU{Можно даже в GCC включить}{In GCC it is even possible to turn on} \TT{-Werror} \OrENRU \TT{/WX} 
\InENRU MSVC ~--- 
\IFRU{тогда все предупреждения будут трактоваться как ошибки}
{then all warning will be treated as errors}.

\IFRU{Вот простой пример}{Here is a simple example}:

\begin{lstlisting}
#include <stdio.h>

int f1(int a, int b, int c)
{
	printf ("(in %s) %d\n", __FUNCTION__, a*b+c);
	// return a*b+c; // OOPS, accidentally I forgot to add this
};

int main()
{
	printf ("(in %s) %d\n", __FUNCTION__, f1(123,456,789));
};
\end{lstlisting}

\IFRU{Автор ``забыл'' дописать \IT{return} в ф-ции f1()}
{The author ``forgot'' to add \IT{return} in f1() function}.
\IFRU{Тем не менее}{Nevertheless}, GCC 4.8.1 \IFRU{компилирует этот пример молча}{compiles this silently}.

\IFRU{Это связано с тем что в стандарте Си}{It is because in the both C standard} 
(\cite[6.9.1/12]{C99TC3}) \IFRU{и в}{and in}
\CPP (\cite[6.6.3/2]{CPP11}) \IFRU{допустимо если ф-ция не возвращает значение}{is okay if a function
does not return a value when it should}.

\IFRU{При запуске мы увидим это}{After running we will see this}:

\begin{lstlisting}
(in f1) 56877
(in main) 14
\end{lstlisting}

\IFRU{Откуда взялось число $14$}{Where the $14$ number is came from}?
\IFRU{Это то что вернула ф-ция}{This is what returns the} printf() \IFRU{вызванная из}{called from} f1().
\IFRU{Возвращаемые результаты ф-ций}{Returned functions results of} 
\glslink{integral type}{\IFRU{интегральных типов}{integral types}}
\IFRU{остаются в регистре EAX/RAX}{are leaved in the EAX/RAX registers}.
\IFRU{В ф-ции main() берется значение из регистра EAX/RAX и передается дальше во второй}
{The value from the EAX/RAX register is taken in the main() function and then passed into the second} printf()
\footnote{
\IFRU{Больше о том, как возвращаются результаты ф-ций через регистры, можно почитать в}
{About how results are returned via registers, you may read more here}
\cite{REBook}}.

\IFRU{Если компилировать с опцией}{If to compile with the} \TT{-Wall}\IFRU{}{ option}, \ac{GCC} 
\IFRU{скажет}{will tell}:

\begin{lstlisting}
1.c: In function 'f1':
1.c:7:1: warning: control reaches end of non-void function [-Wreturn-type]
 };
 ^
1.c: In function 'main':
1.c:12:1: warning: control reaches end of non-void function [-Wreturn-type]
 };
 ^
\end{lstlisting}

... \IFRU{хотя всё равно скомпилирует}{but will compile anyway}.

MSVC 2010 \IFRU{генерирует код, работающий точно также, хотя и выводит предупреждение}
{generates the code running likewise, with warning, though}:

\begin{lstlisting}
...\1.c(7) : warning C4716: 'f1' : must return a value
\end{lstlisting}

\IFRU{Как видно, ошибка почти критическая, вызванная, можно сказать, опечаткой, но предупреждения компилятора
либо не видно вовсе, либо можно и не заметить}
{As you may see, the error is almost critical, caused by, it can be said, type, but there were no
compiler warning, or it was inconspicuous}.


\section{\IFRU{Треды}{Threads}}

\index{C++11}
\IFRU{В}{In the} C++11 \IFRU{ввели модификатор}{standard, a new} \IT{thread\_local} 
\IFRU{показывающий что каждый тред будет иметь свою версию этой переменной}
{modifier was added, showing that each thread will have its own version of the variable},
\IFRU{и её можно инициализировать, и она расположена в}{it can be initialized, and it is located in the} \ac{TLS}
%\footnote{
%\index{C11}
%\IFRU{В C11 также есть поддержка тредов, хотя и опциональная}
%{C11 also has thread support, optional though}}
:

\begin{lstlisting}[caption=C++11]
#include <iostream>
#include <thread>

thread_local int tmp=3;

int main()
{
	std::cout << tmp << std::endl;
};
\end{lstlisting}
\footnote{\IFRU{Компилируется в}{Compiled in} GCC 4.8.1, \IFRU{но не в}{but not in} MSVC 2012}

\IFRU{В исполняемом файле значение}{In the resulting executable file, the} \IT{tmp} 
\IFRU{будет именно в}{variable will be stored in the} \ac{TLS}.

\index{errno}
\IFRU{Это удобно например для хранения глобальных переменных вроде}
{It is useful for storing global variables like} \IT{errno}, 
\IFRU{которая не может быть одна для всех тредов}
{which cannot be one single variable for all threads}.


\section{\IFRU{Ф-ция main()}{main() function}}

\IFRU{Стандартное определение}{Standard declaration}:

\begin{lstlisting}
int main(int argc, char* argv[], char* envp[])
\end{lstlisting}

\IT{argc} \IFRU{будет}{will be} 1 \IFRU{при отсутствии аргументов}{if no arguments present}, 
2 ~--- \IFRU{при одном аргументе}{if one argument}, 3 ~--- \IFRU{если два}{if two}, \IFRU{итд}{etc}.

\begin{itemize}
\item argv[0] ~--- \IFRU{имя текущей запущенной программы}{current running program name}.
\item argv[1] ~--- \IFRU{первый аргумент}{first argument}.
\item argv[2] ~--- \IFRU{второй аргумент}{second argument}.
\item \IFRU{итд}{etc}.
\end{itemize}

\IFRU{элементы в }{}\IT{argv} \IFRU{можно перечислять в цикле}{can be enumerated in loop}.
\IFRU{К примеру, программа может принимать список файлов в командной строке}
{For example, the program may take a file list in command line}
\index{UNIX!cat}
(\IFRU{как это делает утилита}{like} UNIX \IT{cat} \IFRU{итд}{utility does, etc}).
\IFRU{Опции с дефисом в начале могут добавляться для отличия их от имен файлов}
{Dashed options may be supplied in order to distinguish them from file names}.

\IFRU{В аргументах \IT{main()}, \IT{envp[]} может быть пропущено, но и \IT{argc/argv[]}, и это корректно}
{Both \IT{envp[]} and \IT{argc/argv[]} can be omitted in the main() function argument list, and it is correct}.
\IFRU{Почему это корректно, можно прочитать в}{Read more here on why it is correct:} \cite[1.2.1]{REBook}.

\index{C99}
\IFRU{Выражение return может быть пропущено начиная с}{Return clause can be omitted in functions as of}
C99 (\ref{no_return}) (\IFRU{тогда ф-ция}{then the} \IT{main()} 
\IFRU{будет возвращать 0}{function will return 0}
\footnote{\IFRU{это исключение из правил существует только для}{this rule exception is present only for}
\IT{main()}}).

\index{exit()}
\index{Windows API!ExitProcess()}
\InENRU \ac{CRT} \IFRU{возвращаемое значение ф-ции}{the return value of} \IT{main()}
\IFRU{в итоге передается в ф-цию}{function is eventually passed to the} \IT{exit()} 
\IFRU{либо в}{function or} \IT{ExitProcess()} \InENRU win32.
\IFRU{Обычно это возвращаемый код ошибки, который можно проверять в шеллах, итд}
{It is usually a return error code which may be checked in command shells, etc}.
0 \IFRU{обычно означает успех, хотя, разумеется, автор сам может определять (и переопределять) свои возвращаемые 
коды}
{is usually means success, but of course, it is up to author to define (or redefine) its own return codes}.



\chapter{\IFRU{Си}{C}}
% sections here
\section{\IFRU{Работа с памятью Си}{Memory in C}}

\IFRU{Есть наверное только два основных типа в памяти, предоставляемых программисту на Си}
{Probably, there are two most common memory types available for a programmer in C}.

\begin{itemize}
\item
\IFRU{Память выделяемая в локальном стеке}{Memory space in the local stack}. 
\index{alloca()}
\IFRU{Это локальные переменные, память выделенная при помощи}{It is local variables, a memory allocated
with the help of} alloca().
\IFRU{Обычно это очень быстро выделяемая память}{It is usually a memory very fast to allocate}.

\item
\IFRU{Куча\footnote{heap}}{Heap}.
\index{malloc()}
\IFRU{То что выделается при помощи}{It is what is allocated with the help of} malloc().
\end{itemize}

% subsections here
\subsection{\IFRU{Локальный стек}{Local stack}}

\IFRU{Когда вы определяете что-то вроде}{If you declare something like} \TT{char a[1024]}, 
\IFRU{выделения памяти как такового не происходит, происходит
просто перемещение указателя стека на $1024$ байта назад}
{there are no memory allocation happens, it is just stack pointer moving back for $1024$ bytes}
\cite[1.2.3]{REBook}. \IFRU{Это очень быстрая операция}{This is very fast operation}.

\IFRU{Освобождать эту память никак не надо, в конце работы ф-ции, это происходит автоматически, с возвратом указателя стека}
{One not need to free that memory, it is happen automatically at the function end, with the stack pointer restoring}.

\IFRU{В качестве обратной стороны медали, вам нужно знать заранее, сколько места нужно выделить, а также, размер
этого блока нельзя изменить, освободить и выделить заново его также нельзя}
{As the flip side, one need to know exactly how much space to allocate, and also, the block cannot be shrinked or expanded,
freed and reallocated again}.

\IFRU{Выделение локальных переменных происходит простым сдвигом указателя стека назад}
{Local variables allocated in the local stack by simple shifting stack pointer back}\cite[1.2.1][REBook].
\IFRU{При этом с выделенной областью памати ничего больше не происходит, в новых выделенных переменных
будет содержаться то, что находилось в это время
в этом месте в стеке, скорее всего, что-то от работы предыдущих ф-ций}
{During that, nothing else is happen, new variables will contain the values which were at the place in stack, most likely,
what was leaved there from previous functions execution}.


\label{alloca}
\index{alloca()}
\subsection{alloca()}

\IFRU{Ф-ция alloca() выделяет блок памяти в локальном стеке точно также,
отодвигая указатель стека}
{alloca() function likewise allocates a memory block in the local stack,
shifting stack pointer}\cite[1.2.4]{REBook}.
\IFRU{Блок памяти будет освобождена в конце ф-ции автоматически}
{The memory block will be freed at the function finish automatically}.

\IFRU{В стандарте}{In the} C99(\ref{C99})\IFRU{, использовать alloca() 
уже не обязательно, там можно просто писать}{ standard,
it is necessary to use alloca(), one can write just}:

\begin{lstlisting}
void f(size_t s, ...)
{
	char a[s];
};
\end{lstlisting}

\index{Variable length array}
\IFRU{Это называется}{This is called} variable length array.

\IFRU{Впрочем, внутри, это работает так же как и}
{Internally it works just as} alloca()\IFRU{}{ however}.

\IFRU{Критика: Линус Торвальдс против использования}
{Criticism: Linus Torvalds against usage of} alloca()\cite{Torvalds:2003}.

\subsection{\IFRU{Выделение памяти в куче}{Allocating memory in heap}}

\IFRU{Куча (heap) это какая-то часть памяти выделенная \ac{OS} процессу,
где процесс может уже сам делить эту часть как хочет}
{The heap is an area of memory allocated by \ac{OS} to the process,
where it can divide it within its sole discretion}.
\IFRU{После завершения процесса (в т.ч., некорректного),
куча автоматически аннулируется и \ac{OS} не нужно разбирать 
по одному все выделенные процессом блоки}
{After terminating of the process (including process crash),
the heap is annuled automatically and \ac{OS} will not need to free
all allocated blocks one by one}.

\index{malloc()}
\index{calloc()}
\index{realloc()}
\index{free()}
\index{C++!new}
\index{C++!delete}
\IFRU{Для работы с кучей есть стандартные библиотечные ф-ции}
{There are standard C functions to work with the heap:}
malloc(), calloc(), realloc(), free(), 
\IFRU{а в Си++ ~--- new/delete}{and new/delete in C++}.

\label{HeapOverhead}
\IFRU{Очевидно, чтобы поддерживать информацию о выделенных блоках в куче,
нужна масса связных друг с другом структур}
{Apparently, heap manager must use a lot of interconnected structures
in order to preserve information about allocated blocks}.
\IFRU{Отсюда имеется вполне осязаемые накладные расходы (overhead)}
{So that's why quite tangible overhead is present}.
\IFRU{Вы можете выделить блок размером 8 байт, 
но еще как минимум 8 байт\footnote{MSVC, 32-битная Windows, 
примерно то же самое и в Linux}
будет задействованы для хранения информации о выделенном блоке}
{
You can allocate memory block of size 8 bytes,
but at least more 8 bytes\footnote{MSVC, 32-bit Windows, 
almost the same in Linux}
will be used for preserving information about allocated block
}
\footnote{\IFRU{Это еще называют ``метаданными'', то есть, данные о данных}
{It is also called ``metadata'', i.e., data about data}}.
\IFRU{В 64-битных ОС указатели занимают в два раза больше,
так что информация о каждом блоке будет занимать как минимум
16 байт}{In 64-bit \ac{OS} pointers requiring twices as much space,
so information about each block will require at least 16 bytes}.
\IFRU{В свете этого, чтобы эффективнее использовать память компьютера,
блоки должны быть побольше, либо
сама организация данных должна быть иная}
{In the light of this, in order to effectively use as much memory as possible,
the blocks should be as large as possible, or, the orgranization of data
must be different}.

\IFRU{Использование кучи требует некоторой программистской дисциплины,
без которой легко наделать ошибок}
{Heap using requires a programmer's discipline, it is easy to make a lot
of mistakes without one}.
\index{RAII}
\IFRU{Возможно поэтому, считается что \ac{PL} с \ac{RAII} как Си++
либо \ac{PL} со сборщиками мусора (Python, Ruby) легче}
{Probably because of this, it is widely considered that \ac{PL} with 
\ac{RAII} like C++
or \ac{PL} with garbage collector (Python, Ruby) are easier}.

\subsubsection{\IFRU{Одна из основных ошибок: утечки памяти}
{One of the common mistakes: memory leaks}}

\IFRU{Память была выделена, но её забыли освободить через}
{Memory was allocated, but we forgot to free it via} free().
\IFRU{Эта проблема довольно легко решается своей собственной
надстройкой над ф-циями malloc()/free()}
{This problem is easily solved by thunk functions on top of
malloc()/free()}.
\IFRU{Пусть эта надстройка ведет учет выделенных блоков, а также, где и когда
(и для чего) был выделен тот или иной блок}
{Let this thunk to keep a records about blocks allocated, and also,
where and when (and for what) each block was allocated}.

\IFRU{Я сделал это в своей библиотеке octothorpe}{I made this in my octothorpe library}
\footnote{\url{https://github.com/dennis714/octothorpe/blob/master/dmalloc.c}}. 
\IFRU{Макрос DMALLOC вызывает ф-цию dmalloc(), передавая
ей имя файла, имя вызывающей ф-ции, номер строки, а также комментарий (имя блока)}
{DMALLOC macro calls dmalloc() function passing it the file name, name of the calling function,
line number and comment (block name)}.
\IFRU{В конце работы программы, вызываем}
{At the end of program, we call}
\TT{dump\_unfreed\_blocks()} \IFRU{и он покажет список блоков, которые забыли освободить}
{and it will dump the list of blocks we forgot to free}:

\begin{lstlisting}
seq_n:2, size: 124, filename: dmalloc_test.c:31, func: main, struct: block124
seq_n:3, size: 12, filename: dmalloc_test.c:33, func: main, struct: block12
seq_n:4, size: 555, filename: dmalloc_test.c:35, func: main, struct: block555
\end{lstlisting}

\IFRU{У каждого блока есть также номер}{Each block also has a number}.
\IFRU{Это для того чтобы можно было установить брякпоинт по номеру выделяемого блока ~--- 
тогда отладчик сработает в тот момент, когда этот блок будет выделяться и вы увидите,
где и при каких условиях это происходит}
{This is helpful because one can set a breakpoint by a block number and debugger will trigger at the moment
the block is being allocated, and you can see, where and under what conditions it is occurring}.

\IFRU{Писать в коде комментарии для каждого выделяемого блока памяти нудно, но очень полезно}
{It is boring to write a comment for each block allocated, but very useful}.
\IFRU{Потом легко увидеть, под что была выделена память}{Then it is easy to see, what was memory allocated for}.
\index{Oracle RDBMS}
\IFRU{Я впервые увидел эту идею в}{I first saw this idea in the} Oracle RDBMS.
\IFRU{Помимо всего прочего, там еще и ведется
статистика, под какие блоки было выделено больше памяти, её можно легко увидеть}
{Aside from that, it also keeps statistics of block types,
how many memory was allocated for each, and it is easy to see it}:

\begin{lstlisting}
SQL> select * from v$sgastat;

POOL         NAME                            BYTES     CON_ID
------------ -------------------------- ---------- ----------
shared pool  AQ Slave list                    1224          1
shared pool  KQR L PO                       653312          2
shared pool  KQR X SO                       635808          2
shared pool  RULEC                           20688          1
shared pool  KQR M SO                         7168          2
shared pool  work area table entry           12240          2
shared pool  kglsim object batch              3864          2
large pool   PX msg pool                    860160          1
large pool   free memory                  30523392          0
large pool   SWRF Metric CHBs              1802240          2
large pool   SWRF Metric Eidbuf             368640          2
\end{lstlisting}

\IFRU{Подобная штука также присутствует и в ядре Windows, там это называется}
{The same thing present in the Windows kernel, it is called} \IT{tagging}.

\IFRU{При выделении памяти
в ядре или драйвере, нужно указывать также 32-битный тег (обычно, четырехбуквенное сокращение, означающее
подсистему Windows)}
{When one allocates memory in the kernel or driver, a 32-bit tag may be set (usually, it is a four-letter
abbreviation, indicating Windows subsystem)}.
\IFRU{Затем в отладчике можно увидеть статистику, под что выделено больше всего памяти}
{Then it is possible to see statistics in a debugger, how much memory is allocated what for}:

\begin{lstlisting}
kd> !poolused 4
   Sorting by  Paged Pool Consumed

  Pool Used:
            NonPaged            Paged
 Tag    Allocs     Used    Allocs     Used
 CM25        0        0       935  4124672	Internal Configuration manager allocations , Binary: nt!cm
 Gh05        0        0       268  3291016	GDITAG_HMGR_SPRITE_TYPE , Binary: win32k.sys
 MmSt        0        0      2119  2936752	Mm section object prototype ptes , Binary: nt!mm
 CM35        0        0        91  2150400	Internal Configuration manager allocations , Binary: nt!cm
 vmfb        0        0        13  2148752	UNKNOWN pooltag 'vmfb', please update pooltag.txt
 Ntff        5     1040      1287  1070784	FCB_DATA , Binary: ntfs.sys
 ArbA        0        0       108   442368	ARBITER_ALLOCATION_STATE_TAG , Binary: nt!arb
 NtfF        0        0       457   431408	FCB_INDEX , Binary: ntfs.sys
 CM16        0        0        62   331776	Internal Configuration manager allocations , Binary: nt!cm
 IoNm        0        0      2022   267288	Io parsing names , Binary: nt!io
 Ttfd        0        0       159   253976	TrueType Font driver 
 Ifs         0        0         4   249968	Default file system allocations (user's of ntifs.h) 
 CM29        0        0        26   212992	Internal Configuration manager allocations , Binary: nt!cm
\end{lstlisting}

\IFRU{Конечно, можно возразить, что для этого нужно хранить еще больше информации о выделенных блоках, а еще
и теги, названия блоков}
{Of course, one may argue the heap manager will require much more space about allocated blocks,
including their names or tags}.
\IFRU{И это еще сильнее замедляет работу программы. Конечно.}
{And it is slowing down the program much more. That is for sure.}
\IFRU{Поэтому пусть это будет работать только в отладочных (debug) сборках, 
а в release-сборках, DMALLOC() становится обычной \IT{пустой} функцией-переходником\footnote{thunk function} для}
{Then we may use it only in debug builds, and in the release-builds DMALLOC() will be
simple \IT{empty} thunk-function for} malloc().
\IFRU{Ну а в ядре Windows это вообще по умолчанию отключено, и нужно включать при помощи утилиты GFlags}
{It is also turned off by default in Windows and it must be turned on with the help of GFlags utility}
\footnote{\url{http://msdn.microsoft.com/en-us/library/windows/hardware/ff549557(v=vs.85).aspx}}
\IFRU{Помимо всего прочего, подобное есть и в MSVC}{Aside from that, something similar present in MSVC}
\footnote{\IFRU{читайте больше о ф-циях}{read more about the} 
\href{http://msdn.microsoft.com/en-us/library/5at7yxcs.aspx}{\_CrtSetDbgFlag}
\IFRU{и}{and}
\href{http://msdn.microsoft.com/en-us/library/d41t22sb.aspx}{\_CrtDumpMemoryLeaks}\IFRU{}{ functions}}.

\subsubsection{\IFRU{Одна из основных ошибок: разрушение кучи}
{One of the common mistakes: heap corruption}}

\IFRU{Нетрудно выделить память под 4 байта, но по ошибке дописать туда пятый}
{It is easy to allocate a memory for 4 bytes, but write there a fifth by accident}.
\IFRU{Скорее всего, сразу это никак не проявится, но фактически это очень опасная мина замедленного действия,
опасная, потому что ведет к трудновыявляемым ошибкам}
{Most likely, it will not come out instantly, but in fact, it is a very dangerous time bomb, dangerous because
it is hard-to-find bug}.
\IFRU{Байт следующий за выделенным вами блоком может не использоваться вовсе, но также там уже
может начинаться какая-то структура менеджера памяти, хранящая информацию о каком-то выделенном блоке,
а может даже об этом самом}
{The byte next after block you allocated, most likely, is not used at all,
but there may begin a heap manager structure, keeping the information about some other allocated block, or maybe
even that block}.
\IFRU{Если какую-то из таких структур сознателно разрушить,
перезаписать, то тогда следующие вызовы malloc() или free() не смогут корректно работать}
{If some of these structures get corrupted or rewriten intentionally, consequent malloc() or free() calls
will not work properly}.
\IFRU{Иногда это проявляется в выводе ошибок вроде}{Sometimes it is manifested in errors like} 
(\IFRU{в}{in} Windows):

\begin{lstlisting}
HEAP[Application.exe]: HEAP: Free Heap block 211a10 modified at 211af8 after it was freed
\end{lstlisting}

\index{\IFRU{Переполнение кучи}{Heap overflow}}
\IFRU{Подобные ошибки эксплуатируются авторами эксплоитов}
{Such errors are exploited by exploit authors}: 
\IFRU{если знать что вы можете изменить структуры данных
менеджера памяти нужным вам образом, вы можете добиться какого-то нужного вам поведения программы}
{if you know you can alter heap manager structures in a way you need, you may achieve some 
specific program behaviour you need}
(\IFRU{это называется переполнение кучи (heap overflow)}{this is called heap overflow}
\footnote{\url{https://en.wikipedia.org/wiki/Heap_overflow}}).

\IFRU{Довольно распространенный метод борьбы с подобными ошибками: это просто дописывать ``guard''-ы с обоих сторон
блока, например, 4-байтного размера}
{Widely used protection from such errors: just to write ``guard'' values (e.g. of 32-bit size) 
at the both sides of the block}.
\IFRU{Например, я сделал это в своем}{For example, I did it in} DMALLOC.
\IFRU{При каждом вызове free(),
проверяется целостность guard-ов (это могут быть просто какие-то фиксированные значения вроде 0x12345678),
и если кто-то или что-то затерло один из них, можно тут же сообщить об этом}
{At each free() call, integrity of both guards are checked (these may be a fixed values like 0x12345678),
and if something or someone wrote to it, that fact can be reported instantly}.

%\subsubsection{Приемущества своего собственного менеджера памяти или надстройки над стандартным}
%... может выдать размер выделенного блока, хотя нафик оно надо...

\subsubsection{\IFRU{Одна из основных ошибок: непроверка результата malloc()}
{One of the common mistakes: not checking malloc() result}}

\IFRU{При успешном выполнении, malloc() возвращает указатель на только что выделенный блок,
который можно использовать, либо NULL если памяти не хватает}
{If malloc() finishes successfully, it returns a pointer to the newly allocated block that can be used,
or NULL in case of memory shortage}.
\IFRU{Конечно, в наше время дешевой памяти эта проблема становится редкой, тем не менее,
если вы используете много памяти, думать об этом все же надо}
{Of course, in our time of cheap memory, this is rare problem, nevertheless, if one uses it a lot,
one should consider it}.
\index{xmalloc()}
\index{xrealloc()}
\IFRU{Проверять возвращаемый указатель после каждого вызова
malloc() неудобно, так что довольно популярный метод это писать свои функции-переходники с названием 
xmalloc(), xrealloc(), вызывающие malloc()/realloc(), но проверяющие их результат и падающие в случае
ошибки}
{It is not handy to check the returned pointer after each malloc() call, so there are a popular technique
to write own thunk functions named xmalloc(), xrealloc() calling malloc()/realloc(), which checks
returning result and exiting in case of error}.

\index{git}
\IFRU{Интересно упомянуть, как ведет себя}{It is interesting to note how} xmalloc() \IFRU{в}{behaves in} git:

\begin{lstlisting}
void *xmalloc(size_t size)
{
	void *ret;

	memory_limit_check(size);
	ret = malloc(size);
	if (!ret && !size)
		ret = malloc(1);
	if (!ret) {
		try_to_free_routine(size);
		ret = malloc(size);
		if (!ret && !size)
			ret = malloc(1);
		if (!ret)
			die("Out of memory, malloc failed (tried to allocate %lu bytes)",
			    (unsigned long)size);
	}
#ifdef XMALLOC_POISON
	memset(ret, 0xA5, size);
#endif
	return ret;
}
\end{lstlisting}
\footnote{\url{https://github.com/git/git/blob/master/wrapper.c}}

\IFRU{Если}{If} malloc() \IFRU{не успешен, он пытается освободить какие-то уже выделенные (и не очень нужные) 
блоки при помощи}{is not successful, it tries to free some already allocated (but not very needed)
blocks with the help of}
try\_to\_free\_routine(), \IFRU{а затем вызвать}{and then to call} malloc() \IFRU{снова}{again}.

\IFRU{Помимо всего прочего, если определен}
{Aside from that, if} XMALLOC\_POISON\IFRU{, все байты в выделенном блоке заполняются}
{ macro is defined, all bytes in the block allocated is filled with} 0xA5.

\IFRU{Это может помочь визуально, на глаз, увидеть когда вы, например, выделили память под структуру,
а затем используете какое-то поле из нее до того как инициализировали}
{This may help to see, visually, when you use some value from the block before its initialization}.

\IFRU{Значение}{The value of} \TT{0xA5A5A5A5} \IFRU{будет
бросаться в глаза в отладчике, ну или просто если вы захотите где-то в дампе вывести его в шестнадцатеричной
форме}{will easily be spotted in the debugger, or, in some place in dump where it will be printed in hexadecimal
form}.
\IFRU{В MSVC для этой же цели служит константа}{There are the constant for the same purpose in MSVC:}
\TT{0xbaadf00d}.

\IFRU{И даже более того: после вызова free(), освобожденный блок может маркироваться уже какой-то другой константой,
чтобы если кто-то захочет использовать что-то оттуда после освобождения блока, это также было видно, хотя
бы визуально}
{Even more than that: after call of free(), freed block may be marked by some other constant, in order
to spot visually if someone attempting to use some data from the block after it has been freed}.

\IFRU{Некоторые константы от}{Some constants from} Microsoft:

\index{Magic numbers}
\begin{lstlisting}
* 0xABABABAB : Used by Microsoft's HeapAlloc() to mark "no man's land" guard bytes after allocated heap memory
* 0xABADCAFE : A startup to this value to initialize all free memory to catch errant pointers
* 0xBAADF00D : Used by Microsoft's LocalAlloc(LMEM_FIXED) to mark uninitialised allocated heap memory
* 0xCCCCCCCC : Used by Microsoft's C++ debugging runtime library to mark uninitialised stack memory
* 0xCDCDCDCD : Used by Microsoft's C++ debugging runtime library to mark uninitialised heap memory
* 0xFDFDFDFD : Used by Microsoft's C++ debugging heap to mark "no man's land" guard bytes before and after allocated heap memory
* 0xFEEEFEEE : Used by Microsoft's HeapFree() to mark freed heap memory
\end{lstlisting}
\footnote{\url{https://en.wikipedia.org/wiki/Magic_number_(programming)}}

\subsubsection{\IFRU{Еще частые ошибки}{Other common mistakes}}

\index{stdlib.h}
\IFRU{Если не включить заголовочный файл}
{If you do not include header file} stdlib.h, 
GCC \IFRU{считает возвращаемое значение неизвестной ф-ции}{treat the returning value of} malloc() 
\IFRU{за}{as} \IT{int} \IFRU{и постоянно ругается на приведение типов}{and warns about types}.

\IFRU{С другой стороны}{On the other hand}, \InENRU \CPP, 
\IFRU{результат malloc() все же нужно приводить к нужному вам типу}{malloc() result should be casted
anyway to right type}:

\begin{lstlisting}
int *a=(int*) malloc(...);
\end{lstlisting}

\IFRU{Еще одна ошибка, которая может попортить нервов,
это выделить один и тот же блок памяти в одном месте 
больше одного раза (предыдущие вызовы ``теряются'' из вида)}
{Another mistake may cost your nerves is to allocate the same block of memory at one place 
more than one time (previous calls are ``hiding'' from the sight)}.

\subsubsection{\IFRU{Еще методы борьбы}{Other bug hunting techniques}}

\index{Valgrind}
\IFRU{Однако, может оказаться так, что ошибки в программе у вас есть, а перекомпилировать её по каким-то причинам
вы не можете}
{However, you may be in the situation, when there are bugs in program, but you are not able to recompile it
for some reason}. \IFRU{Тогда может помочь, например, valgrind}
{As an example, valgrind\footnote{\url{http://valgrind.org/}} may help then}.


\subsection{\IFRU{Локальный стек или куча}{Local stack or heap}?}

\IFRU{Конечно, в локальном стеке выделение памяти происходит намного быстрее}
{Of course, allocation in the local stack is much faster}.

\IFRU{Например: в}{For example: in} tracer
\footnote{\url{\IFRU{http://yurichev.com/tracer-ru.html}{http://yurichev.com/tracer-en.html}}}
\IFRU{у меня есть дизассемблер}{i have disassembler}\footnote{\url{https://github.com/dennis714/x86_disasm}} 
\IFRU{и эмулятор x86-процессора}{and also x86 CPU emulator}
\footnote{\url{https://github.com/dennis714/bolt/blob/master/X86_emu.c}}.
\IFRU{Когда я писал дизассемблер на Си (я делал это после того как длительное время писал на \ac{PL} 
более высокого уровня ~--- Python), 
я думал, что было бы неплохо, чтобы он сам выделял память под структуру, заполнял её
и возвращал указатель на нее, а в случае ошибки дизассемблирования, возвращал бы NULL}
{When I was writing disassembled in C (I did it after a long period of programming in higher level 
\ac{PL} ~---
Python), I thought, it is a good idea for disassembler to allocate the memory for the structure it returns,
fill it and return a pointer to it, or NULL in case of disassembling error}.

\IFRU{Эстетически, это 
неплохо смотрится, в стиле высокоуровневых \ac{PL}, к тому же, такой код наверное легче читается}
{\AE{}sthetically it looks good, in style of high-level \ac{PL}, aside from that, such code is easier to read}.
\IFRU{Однако, дизассемблер и эмулятор x86-процессора
работают в цикле, огромное количество раз в секунду и эффективность здесь более чем важна}
{However, disassembler and x86 CPU emulator works in loop, a huge number of iterations per second and efficiency
is crucial}.
\IFRU{Так что, основной цикл у меня выглядит примерно так}
{So the main loop I wrote in that way}:

\begin{lstlisting}
while(true)
{
	struct disassembled_instruction DA;

	bool DA_success=disassemble(&DA...);
	if (DA_success==false)
		break;

	bool emulate_success=try_to_emulate(&dDA);
	if (emulate_success==false)
		break;

};
\end{lstlisting}

\IFRU{Затрат на выделение памяти под структуры, описывающие дизассемблированную инструкцию, нет вовсе}
{There are no costs for allocating disassembler structures at all}.
\IFRU{А иначе, нужно было бы на каждой итерации цикла вызывать malloc()/free(), каждая из которых, каждый раз,
работала бы со структурами кучи, и т.д.}
{Otherwise, we need to call malloc()/free() at teach loop iteration, each of which will also work with heap
data structures, etc}.

\IFRU{Как известно, у x86-инструкций может быть вплоть до трех операндов, так что, в моей структуре, помимо
кода инструкции, есть также и информация о трех операндах}
{As we know, x86-instructions may have up to 3 operands, so, in my structure, aside from instruction code,
there are also information about 3 operands}.
\IFRU{Конечно, можно было бы оформить её примерно так}{Of course, I could do it like}:

\begin{lstlisting}
struct disassembled_instruction
{
	int instruction_code;
	struct operand *op1;
	struct operand *op2;
	struct operand *op3;
};
\end{lstlisting}

... \IFRU{а в случае отсутствия какого либо операнда, пусть там будет NULL}
{and let it be NULL there in case of absence of some operand}.
\IFRU{Тем не менее, это снова выделение памяти в куче}{Nevertheless, it is still heap memory allocations}.

\IFRU{Так что у меня сделано примерно так}{So I did it like}:

\begin{lstlisting}
struct disassembled_instruction
{
	int instruction_code;
	int operands_total;
	struct operand op[3];
};
\end{lstlisting}

\IFRU{Такая структура занимает больше места в памяти}
{Such structure requires more memory}.
\IFRU{К тому же, трехоперандные инструкции очень редки в x86-коде,
а здесь у меня пустой третий операнд хранится всегда}
{Aside from that, 3-operand instructions are rare in x86-code, but third operand is stored here always}.
\IFRU{Однако, лишних манипуляций с памятью не происходит}
{However, there are no extra manipulations with memory}.

\IFRU{Ну а если уж так сильно хочется сэкономить на третьем операнде,
то можно не хранить третий операнд вовсе}{But if one like to save a space by not storing third operator,
it may be not stored at all}: \IFRU{нетрудно
вычислить размер структуры без одного операнда}{it is easy to calculate structure size
without one operatnd}: \TT{sizeof(disassembled\_instruction) - sizeof(struct operand)} 
\IFRU{и скопировать её куда-то, где она должен храниться}{and copy it to the some place where it must be
stored}.
\IFRU{Ведь никто не запрещает нам использовать (и хранить) не всю структуру а только её часть}
{Because no one prohibits to use (and store) not the whole structure, but only its part}.
\IFRU{А ф-ции работы с этой структурой могут не трогать в памяти третий операнд вовсе и,
таким образом, ошибок не будет}{Besides, the functions which works with the structure, may not touch
third operand at all, and that will work correctly}.

\IFRU{И даже более того: я специально сделал свой дизассемблер именно так,
чтобы он мог принимать на вход не инициализированную
структуру, и мог работать даже если там осталась информация от предыдущих вызовов}
{Even more than that: I made my disassembler intentionally in that way that it can take
not initialized structure and may work even if there are still some information leaved from
the previous calls}.

\IFRU{Возможно, это уже слишком, но вы поняли идею}{Maybe it is overkill, but you got the idea}.

\IFRU{Таким образом, если вы выделяете память под небольшие структуры заранее известного размера, или если скорость
очень важна, то лучше подумать насчет выделения в локальном стеке}
{Thus, if you allocate small structures of known size and if speed is crucial, you may consider allocating
them in the local stack}.



\section{\IFRU{Строки в Си}{String in C}}

\IFRU{В Си нет встроенных возможностей для удобной работы со строками, такими, какие имеются в ЯП более
высокого уровня}{There are no features in C to handle strings like those present in higher level PLs}.

\IFRU{Часто жалуются на неудобную
конкатенацию строк (то есть, склеивание) в Си при помощи функции strcat().
Также, многих раздражает sprintf(), под который нельзя толком зараннее предсказать, 
сколько нужно выделять памяти}{People often complain about awkward string concatenation (i.e., glueling together).
Also irritating sprintf(), for which is hard to predict how much space will need}.

\IFRU{Копирование строк при помощи
strcpy() также неудобно ~--- нужно думать, сколько же выделить байт под буфер}
{Strings copying with strcpy() is not easy as well ~--- one need to think how many bytes should be allocated
for buffer}.
\IFRU{Помимо всего прочего, неудобная
работа со строками в Си, это источник огромного количества уязвимостей в ПО, связанных с переполнениями буфера}
{Aside from that, awkward C strings is the source of huge number of vulnerabilities related
to buffer overflow}\cite[1.14.2]{REBook}.

\IFRU{Прежде всего, нужно задать себе вопрос, какие операции со строками нам нужны}
{In the first place, we should ask ourselves, which string operations we need}.
\IFRU{Конкатенация (склеивание) нужна чтобы}{Concatenation (glueling) is needed for} 
1) \IFRU{выдавать в лог сообщения}{output messages to log};
2) \IFRU{конструировать строки и затем передавать (или записывать) их куда-то}
{construction of strings and then to pass (or write) them to some place}.

\IFRU{Для 1) можно использовать потоки (streams) ~--- не конструируя строку, выдавать её по порциям, например}
{For 1) it is possible to use streams ~--- without string construction just to output it by portions, for example}:

\begin{lstlisting}
printf ("Date: ");
dump_date(stdout, date);
printf (" a=");
dump_a(stdout, a);
printf ("\n");
\end{lstlisting}

\IFRU{Подобное заменяется в Си++ выводом в \IT{ostream}}{This is what \IT{ostream} in C++ is intended for}:

\begin{lstlisting}
cout << "Date: " << Date_ToString(date) << " a=" << a_ToString(a) << "\n";
\end{lstlisting}

\IFRU{Так быстрее и меньше требуется памяти для конструирования строк}
{It is faster, and requries less memory for string construction}.

\IFRU{Кстати, ошибкой является писать так}{By the way, it is a mistake to write like}:

\begin{lstlisting}
cout << "Date: " + Date_ToString(date) + " a=" + a_ToString(a) + "\n";
\end{lstlisting}

\IFRU{Для неспешного вывода в лог небольшого кол-ва сообщений это нормально,
но если таких сообщений очень много, то будут накладные расходы на их конкатенацию}
{At an easy pace, it is good enough to write messages to log, however, if there are a lot
of such messages, there may be string concatenation overhead}. \\
\\
\IFRU{Но все же строки иногда конструировать надо}{Anyway, sometimes strings should be constructed}.

\IFRU{Есть какие-то библиотеки для этого}{There are some libraries for this}.
\IFRU{К примеру, в}{For example, in} Glib\footnote{\url{https://developer.gnome.org/glib/}} 
\IFRU{есть}{there are}
gstring.h\footnote{\url{https://github.com/GNOME/glib/blob/master/glib/gstring.h}}/
gstring.c\footnote{\url{https://github.com/GNOME/glib/blob/master/glib/gstring.c}}. 

\label{strbuf}
\IFRU{А в исходниках git можно найти}
{In the git source code we may find} strbuf.h\footnote{\url{https://github.com/git/git/blob/master/strbuf.h}}/
strbuf.c\footnote{\url{https://github.com/git/git/blob/master/strbuf.c}}. 
\IFRU{Собственно,
подобные Си-библиотеки очень похожи: они обеспечивают структуру данных, 
в которой есть некоторый буфер для строки, текущий размер буфера
и текущий размер строки в буфере}
{Strictly speaking, such C-libraries are very similar: they provide a data structure with a string buffer
in it, current buffer length and current string in buffer length}.
\IFRU{При помощи отдельных функций, можно добавлять новые строки или символы
в буфер, который, в свою очередь, будет автоматически увеличиваться или даже уменьшаться}
{With the help of various functions, it is possible to add to buffer other string or characters,
which, in turn, will grow or shrink}.

\IFRU{В}{In} \IT{strbuf.c} \IFRU{из}{from} git 
\IFRU{есть в том числе и ф-ция}{there are also function} \IT{strbuf\_addf()}, 
\IFRU{работающая как}{working just like} \IT{sprintf()}, 
\IFRU{но добавляющая строку-результат в буфер}{but adding resulting string into the buffer}.

\IFRU{Так программист освобождается от головной боли связанной с выделением памяти}
{Thus a programmer may get rid of headache related to memory allocation}.
\IFRU{При работе с этими библиотеками, практически невозможна ситуация переполнения буфера, если только не
работать со структурой данных самостоятельно}{While using such libraries,
buffer overflows are virtually impossibly if not to work with the structures by himself}.

\IFRU{Типичная последовательность работы с такими библиотеками, выглядит так}
{The typical sequence of using such libraries is looks like}:

\begin{itemize}
\item
\IFRU{Инициализация структуры}{Structure} strbuf \IFRU{или}{or} GString\IFRU{}{ initialization}.

\item
\IFRU{Добавление строк и/или символов}{Adding strings and/or characters}.

\item
\IFRU{Имеем сконструированную строку}{Now we have constructed string}.

\item
\IFRU{Модифицируем её если нужно}{Modifying it if need}.

\item
\IFRU{Используем её как обычную Си-строку, записываем куда-то в файл, передаем по сети, итд}
{Using it as usual C-string, writing it to to some file, send it by network, etc}.

\item
\IFRU{Освобождаем структуру}{Structure deinitialization}.
\end{itemize}

\IFRU{Кстати, конструирование строк чем-то напоминает}{By the way, string construction is resembling somehow}
Buffer\footnote{\url{http://docs.oracle.com/javase/7/docs/api/java/nio/Buffer.html}}, 
ByteBuffer\footnote{\url{http://docs.oracle.com/javase/7/docs/api/java/nio/ByteBuffer.html}} \IFRU{и}{and}
CharBuffer\footnote{\url{http://docs.oracle.com/javase/7/docs/api/java/nio/CharBuffer.html}} \IFRU{в}{in} Java.

\subsection{Хранение длины строки}

\IFRU{Всегда хранить длину строки ~--- это было принято в реализациях ЯП Pascal}
{Storing always string length ~--- it was done in Pascal PL implementations}.
\IFRU{Не смотря на исходы святых войн\footnote{holy wars} между приверженцами Си и Pascal}
{Aside from holy wars outcomes between both PL devotees}, 
\IFRU{все же, почти все библиотеки
для хранения строк и работы с ними, хранят также и текущую длину}
{nevertheless, almost all string libraries keep current string length} ~--- 
\IFRU{просто потому что удобства от этого перевешивают необходимость пересчитывать это значение после
каждой модификации}
{just because conveniences outweigh the need of length value recalculation after each modification}.

\IFRU{Например}{For example}, \IT{strlen()}
\footnote{\IFRU{подсчет длины строки}{string length calculation}} 
\IFRU{больше не нужен вообще, длина строки известна всегда}{is not needed at all, string length is always known}.
\IFRU{Конкатенация строк работает намного быстрее, потому что не нужно вычислять длину первой строки}
{String concatenation is also much faster, because we do not need to calculate length of the first string}.
\IFRU{Ф-ция сравнения строк в самом начале может сравнить длины строк и если они не равны, тут же вернуть 
\IT{false},
не начиная сравнивание символов в строках}
{The function of strings comparing may just compare string lengths 
at the beginning and if they are not equals to each other, return \IT{false} without starting to compare
characters in the strings}.

\IFRU{В всетевых библиотеках}{In the network libraries of} Oracle RDBMS, 
\IFRU{в функции работы со строками, зачастую передается строка и, 
отдельным аргументом, её длина}
{to the various string functions often passed string with its length, as separate argument}
\footnote{\url{http://blog.yurichev.com/node/64}}.
\IFRU{Это не очень эстетично, это выглядит избыточно, зато очень удобно}
{Not very aesthetical, looks redundant, but very useful}.
\IFRU{Например, у нас есть некоторая ф-ция, которой нужно в начале узнать, какую строку ей передали}
{For example, we have a function, which needs to know, which string was passed to it}:

\lstinputlisting{C/strings/strcmp1.c}

\IFRU{А вот если бы эта ф-ция имела длину входной строки, её можно было бы переписать так}
{However, if this function have length of the input string, it may be rewritten like}:

\lstinputlisting{C/strings/strcmp2.c}

\IFRU{Конечно, с эстетической точки зрения, код выглядит ужасно}
{Aesthetically, the code looks just horrible}.
\IFRU{Тем не менее, мы здорово сократили количество необходимых сравнений строк}
{Nevertheless, we got rid of a lot of strings comparison calls}! 
\IFRU{Вероятно, для тех ситуаций, когда 
нужно как можно быстрее обрабатывать текстовые строки, такой подход может улучшить ситуацию}
{Apparently, for those cases when strings must be processed fast, such approach may help}.

\subsection{\IFRU{Возврат строки}{String returning}}

\IFRU{Если некая ф-ция должна вернуть строку, имеются такие возможности}
{If a function must return a string, these options are available}:

\begin{itemize}
\item
1: \IFRU{Возврат строки-константы, это самое простое и быстрое}
{Constant string returning, is simplest and fastest}.

\item
2: \IFRU{Возврат строки через глобальный массив символов}
{String returning via global array of characters}. 
\IFRU{Недостаток: массив один и каждый вызов ф-ции перезаписывает его содержимое}
{Shortcoming: there are only one array and each subsequent function call overwrites its contents}.

\item
3: \IFRU{Возврат строки через буфер, заданный в аргументах ф-ции}
{String returning via buffer, pointer to which is passed in the function arguments}.
\IFRU{Недостаток: нужно также передавать и длину буфера, и вообще его длину нельзя зараннее правильно расчитать}
{Shortcoming: buffer length must be passed as well, and also its length cannot be correctly calculated
in before}.

\item
4: \IFRU{Выделяем буфер нужного размера сами, записываем туда строку, возвращаем указатель}
{Allocate buffer of a size we need on our own, write string to it,
return the pointer to the buffer we allocated}.
\IFRU{Недостаток: тратятся ресурсы на выделение памяти}
{Shortcoming: resources spent on memory allocation}.

\item
5: \IFRU{Записываем строку в уже рассмотренный}{Write the string to the} \TT{strbuf}\IFRU{}{ we already mentioned} 
\OrENRU \TT{GString} \IFRU{или иную другую структуру, указатель на которую был
передан в аргументах}{or any other structure, pointer to which was passed in the arguments}.

\end{itemize}

\subsection{1: \IFRU{Возврат строки-константы}{Constant string returning}}

\IFRU{Первый вариант очень прост. Например}{The first option is very simple. E.g.}:

\lstinputlisting{C/strings/return_month_name1.c}

\IFRU{Можно даже еще проще}{Even simpler}:

\lstinputlisting{C/strings/return_month_name2.c}

\subsection{2: \IFRU{Через глобальный массив символов}{Via global array of characters}}

\index{asctime()}
\IFRU{Так делает стандартная ф-ция}{That is how} \TT{asctime()}\IFRU{}{ it does}.
\IFRU{Следует помнить, что нужно использовать возвращенную строку
перед каждым следующим вызовом}{Keep in mind that string should be used before each subsequent call
to} \TT{asctime()}.

\IFRU{Например, это правильно}{For example, this is correct}:

\begin{lstlisting}
printf("date1: %s\n", asctime(&date1));
printf("date2: %s\n", asctime(&date2));
\end{lstlisting}

\IFRU{А это нет}{This is not}:

\begin{lstlisting}
char *date1=asctime(&date1);
char *date2=asctime(&date2);
printf("date1: %s\n", date1);
printf("date2: %s\n", date2);
\end{lstlisting}

... \IFRU{ведь указатели \TT{date1} и \TT{date2} будут указывать на одно и то же место, 
и вывод \TT{printf()} будет одинаковым}
{because \TT{date1} and \TT{date2} pointers will point to one place and \TT{printf()} output will be the same}. \\
\\
\IFRU{В git в \IT{hex.c}}{In \IT{hex.c} of git}\footnote{\url{https://github.com/git/git/blob/master/hex.c}} 
\IFRU{можно найти такое}{we may find this}:

\lstinputlisting{C/strings/git_hex.c}

\IFRU{Строка возвращается фактически через глобальную переменную,
определение её как \TT{static} внутри ф-ции просто напросто
обеспечивает доступ к ней только из этой ф-ции}{In fact, the string is returned via global variable,
\TT{static} declaration makes it visible only from this function}.
\IFRU{Но вот недостаток: после вызова}{Here is a shortcoming: after call to} \IT{sha1\_to\_hex()} 
\IFRU{вы не можете
вызвать её повторно для получения второй строки до тех пор, пока не используете как-то первую, ведь она
затрется}{you cannot call it again for the second string result before you use the first somehow,
because it will be overwritten}.
\IFRU{Для того чтобы решить эту проблему здесь, по видимому, сделали сразу 4 буфера и каждый раз строка
возвращается в следующем}{Apparently, in order to solve the problem, here are 4 buffers, and the string
is returned each time in the next one}.
\IFRU{Но имейте ввиду ~--- так можно делать если только вы уверены в том что вы делаете,
это код на уровне ``грязного хака''}{It is also worth to notice ~--- it is possible to do such things if you
are sure in what you do, the code is on the ``dirty hack'' level}.
\IFRU{Если вы вызовете эту ф-цию 5 раз и вам нужно будет использовать как-то строку полученную при первом вызове, 
это может привести к трудновыявляемой ошибке}{If you will call this function 5 times and will need to 
use the first string somehow, this may lead to hard-to-find bug}.

\IFRU{Кстати, обратите также внимание на то что переменная}{You may also notice that} \IT{bufno} 
\IFRU{не инициализируется}{is not initialized},
\IFRU{потому что используются только 
2 младших её бита}{because only 2 lower bits are used}, 
\IFRU{к тому же, не важно, какое значение переменная будет содержать в самом начале}{aside from that,
it is not important at all, which value it will hold at the program start}.


\subsection{\IFRU{Стандартные ф-ции в Си для работы со строками}{Standard string C functions}}

\index{getcwd()}
\IFRU{Некоторые ф-ции, например, getcwd() не только заполняют буфер, но и возвращают указатель на него}
{Some functions like getcwd() not only filling the buffer, but also returns a pointer to it}.
\IFRU{Это для того чтобы можно было писать что-то вроде}
{It is made for the situations, where it is more compact to write something like}:

\begin{lstlisting}
char buf[256];
do_something (getcwd (buf, sizeof(buf)));
\end{lstlisting}

... \IFRU{вместо}{instead of}:

\begin{lstlisting}
char buf[256];
getcwd (buf, sizeof(buf))
do_something (buf);
\end{lstlisting}

\subsubsection{strstr() \AndENRU memmem()}

\index{strstr()}
strstr() \IFRU{применяется для поиска строки в другой строке, либо чтобы узнать, есть ли там такая строка вообще}
{is intended for searching for a substring in another string, or to get to know,
are there substring present in it anyway}.

\index{memmem()}
memmem() \IFRU{можно применять с этими же целями, но для поиска по буферу, в котором могут быть нули,
либо по части строки}{can be used with the same intentions, but for searching in the buffer which may
contain zeroes, ot in the part of a string}.

\subsubsection{strchr() \AndENRU memchr()}

\index{strchr()}
strchr() \IFRU{применяется для поиска символа в строке, либо чтобы узнать, есть ли там такой символ вообще}
{is used for searching for character in a string or to get to know if there such character present}.

\label{memchr}
\index{memchr()}
memchr() \IFRU{можно применять с этими же целями, но для поиска по части строки}{can be used with the same
intentions, but for searching in the part of a string}.

\subsubsection{atoi(), atof(), strtod(), strtof()}

\index{atoi()}
\index{atof()}
\index{strtod()}
\index{strtof()}
\IFRU{Ф-ции }{}atoi()/atof() \IFRU{не могут сигнализировать об ошибке}{cannot signal an error},
\IFRU{а}{but} strtod()/strtof()
\IFRU{, делая то же самое}{ while doing the same thing} ~--- \IFRU{могут}{can signal}.

\input{C/strings/scanf}

\subsubsection{strspn(), strcspn()}

\index{strspn()}
\TT{strspn()} \IFRU{часто применяется для того чтобы удостовериться, что некая строка полностью состоит из
нужных символов}{is often used to get to be sure that a string has only characters from the list we defined}:
    
\begin{lstlisting}
if (strspn(s, "1234567890") == strlen(s)) ... OK
...
if (strspn(IPv4, "1234567890.") == strlen(IPv4)) ... OK
...
if (strspn(IPv6, "0123456789AaBbCcDdEeFf:.") == strlen(IPv6)) ... OK
\end{lstlisting}

\IFRU{Либо для того чтобы пропустить начало строки}{Or to skip a begin of a string}:

\begin{lstlisting}
const char *whitespaces = " \n\r\t";
*buf += strspn(*buf, whitespaces); // skip whitespaces at start
\end{lstlisting}

\index{strcspn()}
\TT{strcspn()} \IFRU{это обратная ф-ция}{is inverse function},
\IFRU{её можно использовать для пропуска всех символов в начале строки, не попадающих
под множество символов}{it can be used for skipping all symbols at the string beginning,
which are not defined in a set}:

\begin{lstlisting}
s += strcspn(s, whitespaces); // first, skip anything till whitespaces
s += strspn(s, whitespaces); // then skip whitespaces
// here 's' is pointing to the part of string after whitespaces
\end{lstlisting}

\subsubsection{strtok() \AndENRU strpbrk()}

\index{strtok()}
\index{strpbrk()}
\IFRU{Обе ф-ции служат для разбиения строки на подстроки, отделенные друг от друга разделительными символами}
{Both functions are used for delimiting string into substrings, divided by special characters}
\footnote{delimiter}.
\IFRU{Только}{However} strtok() \IFRU{модифицирует исходную строку}{modifies source stirng}
(\IFRU{и таким образом, получаемые подстроки сразу можно использовать как отдельные Си-строки}
{and thus resulting substrings can be used as separated C-strings}), 
\IFRU{а}{but} strpbrk() \IFRU{нет, он только возвращает указатель на следующую подстроку}
{is not, it is only returning a pointer to the next substring}.



\subsection{Unicode}

Unicode \IFRU{это важно}{is important}! \IFRU{Наиболее популярные способы его применения это}
{Most popular approaches are}:

\begin{itemize}
\item UTF-8
\IFRU{Популярно в UNIX-системах}{Popular is UNIX-OS}.
\IFRU{Сильное приемущество: можно продолжать пользоваться стандартными ф-циями для обработки строк}
{Significant advantage: it is still possible to use standard functions for strings processing}.

\item UTF-16
\IFRU{Используется в}{Used in} Windows API.
\end{itemize}

\subsubsection{UTF-16}

\IFRU{Под каждый символ отводят 16-битный тип}{For each character a 16-bit type is assigned:}
\IT{wchar\_t}.

\IFRU{Для объявления строк с таким типом, используется макрос \IT{L}}
{For such typed string definition, \IT{L} macro is used:}:

\begin{lstlisting}
L"hello world"
\end{lstlisting}

Для работы с wchar\_t вместо char, имеется целый класс функций-двойников с символом w в названии,
например: fwprintf(), wcscmp(), wcslen(), iswalpha().

\paragraph{Windows}

В Windows, если некто хочет писать программу сразу в двух версиях, с использованием Unicode и без,
для этого есть тип tchar, в зависимости от объявленной переменной препроцессора UNICODE, 
он будет либо char либо wchar\_t\footnote{Сборка с Unicode и без была популярна во времена популярности
как Windows NT/2000/XP так и Windows 95/98/ME. Вторая линейка плохо поддерживала Unicode}.
Для этого же имеется макрос \TT{\_T(...)}:

\begin{lstlisting}
_T("hello world")
\end{lstlisting}

В зависимости от выставленной переменной препроцессора UNICODE, она будет объявлена как char либо wchar\_t.

В заголовочном файле tchar.h есть масса ф-ций, меняющих свою функцию в зависимости от этой переменной.

\subsection{Списки строк}

Самый простой список строк, это просто набор строк оканчивающийся нулем.
Например, в Windows API, в библиотеке Common Dialogs, 
так\footnote{\url{http://msdn.microsoft.com/en-us/library/windows/desktop/ms646829(v=vs.85).aspx}} 
передаются список допустимых расширений файлов для диалогового окна:

\begin{lstlisting}
// Initialize OPENFILENAME
ZeroMemory(&ofn, sizeof(ofn));
...
ofn.lpstrFilter = "All\0*.*\0Text\0*.TXT\0";
...

// Display the Open dialog box. 

if (GetOpenFileName(&ofn)==TRUE) 
	...
\end{lstlisting}


\section{\IFRU{Ваши собственные структуры данных в Си}{Your own data structures in C}}

\section{Списки в Си.}

Списки это связный набор элементов. Односвязный список --- это когда у каждого элемента есть ссылка на следующий.
Двусвязный список --- когда у элемента есть ссылки на следующий и на предыдущий.

У списков есть серьезное преимущество перед массивами: в список легко добавлять элемент в произвольное место,
так и удалять. В качестве недостатков: тратится много памяти для поддержания самих структур списка, а также
нет возможности индексировать его, как массив.

\subsection{Односвязный список}

Его сделать очень легко. В структуре предназначенной для связывания в список, достаточно добавить где-то
ссылку на следующий элемент, обычно это поле называется next:

\begin{lstlisting}
struct some_object
{
	...
	...
	struct some_object* next;
};
\end{lstlisting}

NULL в next означает что этот элемент является последним.

Операция прохода по такому списку становится очень простой:

\begin{lstlisting}
for (struct some_object *i=list; i!=NULL; i=i->next)
	...
\end{lstlisting}

Для вставки нового элемента, нужно вначале найти последний элемент:

\begin{lstlisting}
for (struct some_object *i=list; i!=NULL; i=i->next);
struct some_object *last_element=i;
\end{lstlisting}

... а затем, создав новую структуру, добавить указатель на нее в next:

\begin{lstlisting}
struct some_object *new_object=calloc(1, sizeof(struct some_object));
// populate new_object with data
last_element->next=new_object;
\end{lstlisting}

calloc() отличается от malloc() тем что обнуляет всё выделенное место, а значит в поле next нового
элемента сразу будет NULL\footnote{Об ``инициализации'' структур, читайте также здесь\ref{COOPInit}.}.

Поиск нужного элемента это просто проход по всему списку до тех пор, пока не найдется то что нужно.

Удаление элемента: найти предыдущий элемент и следующий, 
у предыдущего в next установить указатель на следующий элемент, 
затем освободить блок памяти выделенный для текущего элемента.

Самый первый элемент списка называется ``list head''. Структуру самого первого элемента можно объявлять как локальную
или глобальную переменную. Но тогда удалять первый элемент списка будет неудобно. А с другой стороны,
можно объявлять указатель на первый элемент списка, тогда будет проще этому указателю присвоить другой элемент,
который будет первым.

\subsection{Двусвязный список}

Это почти то же самое, только, помимо указателя на следующий элемент, хранится еще и указатель на предыдущий.
Если элемент первый, то указатель может быть NULL, либо он может указывать сам на себя (кому как удобнее).

Работая с двусвязным списком, легче находить предыдущие элементы, например, когда нужно удалить какой-то элемент.
А также можно перебирать элементы с конца списка до начала.
Но памяти на это тратится немного больше.

\subsection{Windows API}

Здесь, да и много где в ядре Windows, применяются две примитивные структуры:

\begin{lstlisting}
typedef struct _LIST_ENTRY {
   struct _LIST_ENTRY *Flink;
   struct _LIST_ENTRY *Blink;
} LIST_ENTRY, *PLIST_ENTRY, *RESTRICTED_POINTER PRLIST_ENTRY;

typedef struct _SINGLE_LIST_ENTRY {
    struct _SINGLE_LIST_ENTRY *Next;
} SINGLE_LIST_ENTRY, *PSINGLE_LIST_ENTRY;
\end{lstlisting}

Эти структуры нельзя назвать самостоятельными, они скорее предназначены для встраивания в другие структуры.
Например, вам нужно объеденить в список структуру описывающую цвет:

\begin{lstlisting}
struct color
{
	int R;
	int G;
	int B;
	LIST_ENTRY list;
};
\end{lstlisting}

Теперь в вашей структуре есть также и ссылка на предыдущий элемент и на следующий.
Для работы со структурами использующие эти списки, в Windows есть набор ф-ций
\footnote{\url{http://msdn.microsoft.com/en-us/library/windows/hardware/ff563802(v=vs.85).aspx}}.

\subsection{Linux}

В ядре Linux работа с простыми двусвязными списками, описывается в файле /include/linux/list.h
\footnote{\url{http://lxr.free-electrons.com/source/include/linux/list.h}}.

Там это много где используется, в ядре 3.12 по крайней мере ~2900 упоминаний ``struct list\_head''.

\subsection{Glib}

Напрашивается мысль, а нельзя ли выделить отдельную структуру для элемента списка, и не встраивать лишних полей
в свои структуры? Можно, например, так сделано в glist.h
\footnote{\url{https://github.com/GNOME/glib/blob/master/glib/glist.h} 
\url{https://developer.gnome.org/glib/2.37/glib-Doubly-Linked-Lists.html}} в glib:

\begin{lstlisting}
struct _GList
{
  gpointer data;
  GList *next;
  GList *prev;
};
\end{lstlisting}

data может указывать на какой угодно объект, на любую существующую структуру, в которой вы ничего не хотите менять.
Конечно, с эстетической точки зрения, это лучше. Но нельзя забывать, что тогда на каждый элемент вашего списка,
будет приходится уже два выделенных блока памяти + еще затраты на поддержания самих блоков памяти\ref{HeapOverhead}. \\
\\
Таким образом, подобное решение оправдано там, где экономия памяти менее важна.


\subsection{\IFRU{Бинарные деревья в Си}{Binary trees in C}}

\IFRU{Бинарные деревья}{Binary trees} ~--- \IFRU{одна из важнейших структур данных в компьютерных науках}
{are one of the most important structures in computer science}.
\IFRU{Чаще всего они используются для хранения пар}{Most often these are used for}
``\IFRU{ключ-значение}{key-values}''\IFRU{}{ pairs storage}.
\index{C++!STL!map}
\IFRU{Это то что в \CPP \ac{STL} реализовано в std::map}
{This is what implemented in std::map in \CPP \ac{STL}}.

\IFRU{Упрощенно говоря, по сравнению со списками, выборка у деревьев происходит намного быстрее}
{Simply speaking, in comparison with lists, trees offer much faster selection}.
\IFRU{С другой стороны, добавление элемента в дерево может происходить медленнее}
{On the other hand, element insertion may be slower}.

\index{POSIX!tsearch()}
\index{POSIX!twalk()}
\index{POSIX!tfind()}
\index{POSIX!tdelete()}
\IFRU{В стандартных библиотеках Си, нет работы с деревьями, но кое-что есть в}
{There are no C standard functions for working with trees, but some functions are present in} \ac{POSIX}
(tsearch(), twalk(), tfind(), tdelete())
\footnote{\url{http://pubs.opengroup.org/onlinepubs/009696799/functions/tsearch.html}}.

\IFRU{Это семейство ф-ций активно используется в}
{This family of functions are used actively in the} Bash 4.2, BIND 9.9.1, \ac{GCC} ~--- 
\IFRU{там можно посмотреть, как это использовать}{it can be seen there how it can be used}.

\index{Glib!GTree}
\IFRU{В}{The} Glib \IFRU{имеется также свои ф-ции для работы с деревьями, определенные в}
{also has the tree functions declared in the} gtree.h
\footnote{\url{https://github.com/GNOME/glib/blob/master/glib/gtree.h}}.

\index{C++!STL!set}
\IFRU{Множество}{The set} (std::set \InENRU \CPP \ac{STL}) 
\IFRU{можно реализовать так же просто при помощи бинарных деревьев, достаточно просто не хранить значение, а хранить только ключ}
{can be implemented as binary trees as well, one may just choose not to store the value and store the key only}.



\subsection{\IFRU{Еще кое что}{One more thing}}

\IFRU{Структуры данных описывающие какие-либо коллекции, могут также содержать и указатели на ф-ции для работы
с элементами, например, ф-ции сравнения, копирования, итд}
{Data structures related to collections may also contain pointers to the functions working with elements,
like comparison functions, copying, etc}.

\IFRU{К примеру}{For example} \InENRU GTree \InENRU glib:

\begin{lstlisting}[caption=gtree.c]
struct _GTree
{
  GTreeNode        *root;
  GCompareDataFunc  key_compare;
  GDestroyNotify    key_destroy_func;
  GDestroyNotify    value_destroy_func;
  gpointer          key_compare_data;
  guint             nnodes;
  gint              ref_count;
};
\end{lstlisting}

\IFRU{Задав ф-цию сравнения ключей, значений, а также ф-цию освобождения памяти}
{By setting the functions for key/value comparison and also deallocator function} (\InENRU \TT{g\_tree\_new\_full()}),
\IFRU{ф-ции работы с деревьями в glib смогут 
самостоятельно сравнивать два дерева, либо освобождать все структуры связанные с деревом}
{tree functions in glib will be able to compare two trees or to free a tree on its own}.


\label{COOP}
\section{\COOPname}

\IFRU{Как известно, в Си нет поддержки ООП, она есть в Си++, тем не менее, в ``чистом'' Си вполне
можно программировать в стиле ООП}
{Of course, there are no OOP support in C, it is present in C++, nevertheless, it is possible
to program in OOP style in ``pure'' C}.

\IFRU{ООП, коротко говоря, это явное разделение на объекты и методы}
{OOP, in short, is a separation to object and methods}.
\IFRU{В Си структуры легко могут представляться объектами, а обычные ф-ции}
{In C, structures are easily can be represented as objects, and usual functions} ~--- 
\IFRU{методами}{as methods}.

\subsection{\IFRU{Инициализация структур}{Structures initialization}}
\label{COOPInit}

\IFRU{В \CPP у классов имеются конструкторы}{\CPP has class constructors}.
\IFRU{Если вам нужно каким-то особенным образом инициализировать
структуру, вам и в Си придется делать подобную ф-цию}{If one need to initialize structure in some
special way, one would write a special function for it in C as well}.
\index{calloc()}
\index{bzero()}
\IFRU{Но если структура простая, то её можно инициализировать при помощи}
{But if it simple structure, it is possible to initialize it with} calloc()
\footnote{\IFRU{Это тоже самое что и malloc() + заполнение выделенной памяти нулями}
{It is the same as the malloc() + allocated memory filling with zeroes}}
\OrENRU bzero()(\ref{bzero}).

\IFRU{Все int-переменные становятся нулями}{All int-variables are set to zeroes}.
\index{C99!bool}
\index{C++!bool}
\index{Windows API!BOOL}
\IFRU{Нулевое значение}{Zero value} bool \InENRU C99(\ref{C99}) \AndENRU \CPP \IFRU{это}{is} false,
\IFRU{так же как и}{same as} BOOL \InENRU Windows API.
\IFRU{Все указатели становятся}{All pointers are set to} NULL.
\index{IEEE 754}
\IFRU{И даже вещественный ноль представляемый в формате}{And even floating point $0.0$ in} 
IEEE 754 \IFRU{это также все ноли во всех битах}{format is zero bits in all positions}.

\IFRU{Если в структуре присутствуют указатели на другие структуры,
то NULL может означать ``отсутствие объекта''}{If structure has pointers to another structures,
NULL can mean ``object absence''}.

\subsection{\IFRU{Деинициализация структур}{Structures deinitialization}}

\IFRU{Если в структуре есть ссылки на другие структуры, то их нужно освобождать}
{If structure has pointers to another structures, they are also must be freed}.
\IFRU{В простом случае, обычным вызовом}{In simple case, it is just a call to} free().
\index{free()}
\IFRU{Кстати, вот почему free() может принимать на вход NULL,
это чтобы можно было просто писать}
{By the way, that is why NULL is valid argument for free(), it allows to write} 
\TT{free(s->field)} \IFRU{вместо}{instead of}
\TT{if (s->field) free(s->field)}, \IFRU{так короче}{that is shorter}.

\subsection{\IFRU{Копирование структур}{Structures copying}}

\index{memcpy()}
\IFRU{Если структура простая, то её можно копировать обычным побайтовым копированием}
{If the structure is simple, it is possible to copy it with a call to} memcpy()(\ref{memcpy}).
\index{Shallow copy}
\index{Deep copy}
\IFRU{Если в такой манере скопировать структуру, в которой есть указатели на другие структуры,
то это будет называться}
{If to copy structures having pointers to another structures in this manner, it will be called}
``shallow copy''\footnote{\url{https://en.wikipedia.org/wiki/Object_copy}}. 
\IFRU{И напротив}{And in opposite}, \IT{deep copy} ~--- \IFRU{это копирование структуры
плюс всех связанных с ней структур (это дольше)}
{is copying a structure with all connected to it structures (is slower)}.

\IFRU{Вот почему может быть удобнее хранить строку в структуре как
обычный массив символов фиксированной длины}
{That is why it may be more convenient to store a string in the structure as 
an fixed-size array of characters}.
\index{Windows API}
\IFRU{Такого, например, очень много в}{For example, a lot of such cases in the} Windows API.
\IFRU{Такую структуру легко скопировать, её хранение требует меньших накладных расходов
\footnote{overhead} в куче}{Such structure is easier to copy, it requires smaller
memory overhead in the heap}.
\IFRU{Но с другой стороны, придется согласиться с ограничением на длину строки}
{On the other hand, we should accept string length fixedness}.

\IFRU{Помимо всего прочего, структуру можно копировать просто так}
{Aside from that, a structure can be copied just as}: \TT{s1=s2} ~--- 
\IFRU{в итоге генерируется код, копирющий все поля по порядку}{the code generated will copy
each structure filed}.
\IFRU{И это наверное легче читается чем вызов}{Perhaps it is easier to read than a call to} memcpy() 
\IFRU{на этом же месте}{at the same place}.

\subsection{\IFRU{Инкапсуляция}{Encapsulation}}

\CPP \IFRU{предлагает инкапсуляцию (сокрытие информации)}{offers encapsulation (information hiding)}.
\IFRU{Например, вы не можете
написать программу модифицирующую защищенное поле в классе, 
это защита на стадии компиляции}
{For example, you cannot write a program which modify a protected class
field, this is a compile-stage protection}\cite[1.7.3]{REBook}.

\IFRU{В Си этого нет, поэтому тут нужно больше дисциплины}
{There are no such thing in C, it requires more discipline}.

\IFRU{Впрочем, можно попытаться ``защитить'' структуру ``от посторонних глаз''}
{However, it is possible to ``protect'' a structure from ``prying eyes''}.
\index{Glib!GTree}
\IFRU{Например, в Glib, имеется библиотека для работы с деревьями}{For example, Glib has a library
intended for work with trees}.
\IFRU{В заголовочном файле}{In the header file}
gtree.h\footnote{\url{https://github.com/GNOME/glib/blob/master/glib/gtree.h}} 
\IFRU{нет описания самой структуры}{there are no declaration of the structure}
(\IFRU{она есть только в}{it is present only in the} gtree.c
\footnote{\url{https://github.com/GNOME/glib/blob/master/glib/gtree.c}}), 
\IFRU{а есть только}{there are only} forward declaration(\ref{forwarddeclaration}).
\IFRU{Так разработчики Glib могут понадеятся что пользователи GTree 
постараются не пользоваться отдельными полями в структуре напрямую}
{Thus Glib developers may have a hope that GTree users will not try to use specific fields in
the structure directly}.

\IFRU{Но у такого метода есть и обратная сторона}{The technique does have its flip side}: 
\IFRU{могут быть крохотные однострочные ф-ции вроде 
``вернуть длину строки''}{there are may be tiny one-line functions like ``return string length''}
\InENRU strbuf(\ref{strbuf}), \IFRU{например}{e.g.}:

\begin{lstlisting}
typedef struct _strbuf
{
    char *buf;
    unsigned strlen;
    unsigned buflen;
} strbuf;

unsigned strbuf_get_len(strbuf *s)
{
	return s->strlen;
};
\end{lstlisting}

\IFRU{Если компилятору на стадии компиляции доступно и описание структуры и тело ф-ции,
то в каком-то месте,
вместо вызова}{If the compiler on compiling stage have access to structure declaration and the 
function body, instead of call to} strbuf\_get\_len() 
\IFRU{он может сделать эту ф-цию как inline-овую, вставить её тело прямо на том
же место и сэкономить на вызове другой ф-ции}{it may make this function as inline, i.e., insert
its body right at the place and save some resources on call to another function}.
\IFRU{Но если эта информация компилятору недоступна, то он
оставит вызов}{But if this information is not available to compiler, it will leave call to}
strbuf\_get\_len() \IFRU{как есть}{as is}.

\IFRU{То же самое касается поля \TT{buf} в структуре \TT{strbuf}}{The same thing is about \TT{buf}
field in the \TT{strbuf} structure}.
\IFRU{Компилятор может генерировать куда более эффективный
код, если этот сгенерированный машинный код сможет обращаться к полям структур на прямую,
а не вызывать суррогатные функции-``методы''}
{Compiler may generate much more effective code if the generated machine code will access structure
fields directly without calling surrogate functions-``methods''}.


\section{\IFRU{Стандартные библиотеки Си/Си++}{C/C++ standard library}}

\subsection{assert}

\index{assert()}
\IFRU{Как известно, этот макрос часто используется для валидации}
{This macro is commonly used for validating}
\footnote{\IFRU{используется также такой термин как ``инвариант'' и ``sanitization'' в англ.яз.}{
``invariant'' and ``sanitization'' terms are also used}}
\IFRU{заданных значений}{of input values}.
\IFRU{Например, если ваша ф-ция работает с датой, вы, вероятно, захотите написать в её начале что-то вроде}
{For eample, if you have a function working with data, you probably may want to add that code at the beginning}:
\IT{assert (month>=1 \&\& month<=12)}.

\IFRU{Вот то о чем нужно помнить: стандартный макрос assert() доступен только в отладочных (debug) сборках}
{Here is what one should remember: standard assert() macro is available only in debug builds}.
\index{\Preprocessor!NDEBUG}
\IFRU{В release, где объявлена переменная NDEBUG, все выражения как бы исчезают}{In a release build, where NDEBUG
is defined, all statements are ``disappearing''}.
\IFRU{Поэтому писать, например,}{That is why it is not correct to write}
\IT{assert(f=malloc(...))}\IFRU{ неверно}{}.
\IFRU{Впрочем, вы возможно захотите использовать что-то вроде}{However, you may want to write something like}
\IT{assert(object->get\_something()==123)}.

\IFRU{В макросах \IT{assert} можно также указывать небольшие сообщения об ошибках}
{Error messages also can be embedded in an \IT{assert} statements}:
\IFRU{вы увидите их если выражение ``не сойдется''}{you will see it if expression will not be true}.
\IFRU{Например, в исходниках}{For example, in the} LLVM\footnote{\url{http://llvm.org/}}
\IFRU{можно встретить такое}{source code we may find this}:

\index{LLVM}
\begin{lstlisting}
assert(Index < Length && "Invalid index!");
...
assert(i + Count <= M && "Invalid source range");
...
assert(j + Count <= N && "Invalid dest range");
\end{lstlisting}

\IFRU{Текстовая строка имеет тип}{Text string has}
\IT{const char*}\IFRU{, и она никогда не}{ type and it is never} NULL.
\IFRU{Таким образом, можно дописать к любому выражению}{Thus it is possible to add} \IT{... \&\& true} 
\IFRU{не меняя его смысл}{to any expression without changing its sense}.

\IFRU{Макрос }{}assert() \IFRU{можно также вводить и для документации кода}{macro can also be used
for documenting purposes}.

\IFRU{Например}{For example}:

\begin{lstlisting}[caption=GNU Chess]
int my_random_int(int n) {

   int r;

   ASSERT(n>0);

   r = int(floor(my_random_double()*double(n)));
   ASSERT(r>=0&&r<n);

   return r;
}
\end{lstlisting}

\IFRU{Глядя на этот код, можно очень быстро увидеть ``легальные'' значения переменных}
{By reading the code we can quickly see ``legal'' values of the} $n$ \AndENRU $r$\IFRU{}{ variables}.

assert\IFRU{-ы также называют}{-s are also called} ``active comments''\cite{Lakos}.



\subsection{Разница между stdout и stderr}

\IT{stdout} это то что выводится на консоль при помощи вызова \IT{printf()}.
\IT{stdout} это буферизированный вывод,
так что, пользователь, обычно того не зная, видит вывод порциями. Бывает так что программа выдает
что-то используя \IT{printf()} либо \IT{cout} и тут же падает.
Если это попадает в буфер, но буфер не успевает
``сброситься'' (flush) в консоль, то пользователь ничего не увидит. Это бывает неудобно.
Таким образом, для вывода более важной информации, в том числе отладочной, удобнее использовать \IT{stderr}.

\IT{stderr} это не буферизированный вывод, и всё что попадает в этот поток при помощи 
\TT{fprintf(stderr,...)} либо \IT{cerr}, появляется в консоли тут же.

Не следует также забывать, что из-за отсутствия буфера, вывод в \IT{stderr} медленнее.

Чтобы направлять \IT{stderr} в другой файл при запуске процесса, можно указывать:

\begin{lstlisting}
process 2> debug.txt
\end{lstlisting}

... это направит вывод \IT{stderr} в заданный файл (потому что номер этого потока -- 2).

\subsection{UNIX time}

В UNIX-среде очень популярно представление даты и времени в формате UNIX time.
Это просто 32-битное число, показывающее
количество прошедших секунд с 1-го января 1970-го года.

В качестве положительных сторон: 1) очень легко хранить это 32-битное число; 2) очень легко вычислять разницу дат;
3) невозможно закодировать неверные даты и время, такие как 32-е января, 29-е февраля невысокосных годов, 
25 часов 62 минуты.

В качестве отрицательных сторон: 1) нельзя закодировать дату до 1970-го года.

В наше время, если использовать UNIX time, тем не менее, следует помнить что ``срок его действия'' истечет
в 2038-м году, именно тогда 32-битное число переполнится, то есть, пройдет $2^{32}$ секунд с 1970-го года.
Так что, для этого следует использовать 64-битное значение, т.е., time64.

% ? NtQuerySystemTime http://msdn.microsoft.com/en-us/library/windows/desktop/ms724512(v=vs.85).aspx

\label{memcpy}
\subsection{memcpy()}

Поначалу трудно запомнить порядок аргументов в ф-циях memcpy(), strcpy(). Чтобы было легче, можно представлять
знак ``='' (``равно'') между аргументами.

\label{bzero}
\subsection{bzero() и memset()}

bzero() это ф-ция просто обнуляющая блок памяти.
Для этого обычно используют memset(). Но у memset() есть неприятная особенность, легко перепутать второй
и третий аргументы местами, и компилятор промолчит, потому что байт для заполнения всего блока задается как int.

К тому же, имя ф-ции bzero легче читается.

С другой стороны, её нет в стандарте POSIX.

А в Windows API для этих же целей применяется ZeroMemory()
\footnote{\url{http://msdn.microsoft.com/en-us/library/windows/desktop/aa366920(v=vs.85).aspx}}.

\label{printf}
\subsection{printf()}

\subsubsection{\IFRU{Свои собственные модификаторы в printf()}{Your own printf() format-string modifiers}}

\IFRU{Часто можно испытать раздражение, когда было бы логично передать в printf(),
скажем, структуру описывающее комплексное
число, или цвет закодированный в структуре из трех чисел типа int}
{It is often irritating when it is logical to pass to printf(), let's say, 
a structure describing complex number, or a color encoded as 3 int numbers as a single entity}.

\IFRU{Эту проблему в Си++ решают определением ф-ции}
{In C++ this problem is usually solved by definition} \TT{operator<<} \InENRU \TT{ostream} 
\IFRU{для своего типа}{for the own type}, \IFRU{либо введением метода с названием}
{or by a method definition named} \TT{ToString()} (\ref{CPPIO}). \\
\\
\IFRU{В}{In} printk() (printf-\IFRU{подобная ф-ция в ядре Linux}{like function in Linux kernel})
\IFRU{имеются дополнительные модификаторы}{there are additional modifiers exist}
\footnote{\url{http://git.kernel.org/cgit/linux/kernel/git/torvalds/linux.git/tree/Documentation/printk-formats.txt}}, 
\IFRU{такие как}{like}
\TT{\%pM} (Mac-\IFRU{адрес}{address}),
\TT{\%pI4} (IPv4-\IFRU{адрес}{address}),
\TT{\%pUb} (UUID/GUID).

\IFRU{В}{In} GNU Multiple Precision Arithmetic Library \IFRU{есть ф-ция}{there are} gmp\_printf()
\footnote{\url{http://gmplib.org/manual/Formatted-Output-Strings.html}} \IFRU{имеющая нестандартные 
модификаторы нужные для вывода чисел с произвольной точностью}{function having non-standard modifiers for
arbitrary precision numbers outputting}. \\
\\
\IFRU{В \ac{OS}}{In the} Plan9\IFRU{, и в исходниках компилятора Go, можно найти ф-цию}
{ \ac{OS}, and in Go compiler source code, we may find}
fmtinstall()\IFRU{, для объявления нового модификатора printf-строки, например}
{ function for a new printf-string modifier definition, for example}:

\begin{lstlisting}[caption=go\textbackslash{}src\textbackslash{}cmd\textbackslash{}5c\textbackslash{}list.c]
void
listinit(void)
{

	fmtinstall('A', Aconv);
	fmtinstall('P', Pconv);
	fmtinstall('S', Sconv);
	fmtinstall('N', Nconv);
	fmtinstall('B', Bconv);
	fmtinstall('D', Dconv);
	fmtinstall('R', Rconv);
}

...

int
Pconv(Fmt *fp)
{
	char str[STRINGSZ], sc[20];
	Prog *p;
	int a, s;

	p = va_arg(fp->args, Prog*);
	a = p->as;
	s = p->scond;
	strcpy(sc, extra[s & C_SCOND]);
	if(s & C_SBIT)
		strcat(sc, ".S");
	if(s & C_PBIT)
		strcat(sc, ".P");
	if(s & C_WBIT)
		strcat(sc, ".W");
	if(s & C_UBIT)		/* ambiguous with FBIT */
		strcat(sc, ".U");
	if(a == AMOVM) {
		if(p->from.type == D_CONST)
			sprint(str, "	%A%s	%R,%D", a, sc, &p->from, &p->to);
		else
		if(p->to.type == D_CONST)
			sprint(str, "	%A%s	%D,%R", a, sc, &p->from, &p->to);
		else
			sprint(str, "	%A%s	%D,%D", a, sc, &p->from, &p->to);
	} else
	if(a == ADATA)
		sprint(str, "	%A	%D/%d,%D", a, &p->from, p->reg, &p->to);
	else
	if(p->as == ATEXT)
		sprint(str, "	%A	%D,%d,%D", a, &p->from, p->reg, &p->to);
	else
	if(p->reg == NREG)
		sprint(str, "	%A%s	%D,%D", a, sc, &p->from, &p->to);
	else
	if(p->from.type != D_FREG)
		sprint(str, "	%A%s	%D,R%d,%D", a, sc, &p->from, p->reg, &p->to);
	else
		sprint(str, "	%A%s	%D,F%d,%D", a, sc, &p->from, p->reg, &p->to);
	return fmtstrcpy(fp, str);
}
\end{lstlisting}
(\url{http://plan9.bell-labs.com/sources/plan9/sys/src/cmd/5c/list.c})

\IFRU{Ф-ция}{The} Pconv() 
\IFRU{будет вызвана если в строке формата будет встречен \%P}{will be called if \%P modifier
in the format string will be met}.
\IFRU{Затем она копирует созданную строку при помощи}
{Then it copies the string created using} fmtstrcpy().
\IFRU{Кстати, эта ф-ция и сама использует другие объявленные модификаторы, такие как}
{By the way, that function also uses other defined modifiers like} \%A, \%D, \IFRU{итд}{etc}. \\
\\
\IFRU{В}{The} Glibc\footnote{\IFRU{Стандартной библиотеке в Linux}{The Linux standard library}} 
\IFRU{есть нестандартное расширение}{has non-standard extension}
\footnote{\url{http://www.gnu.org/software/libc/manual/html_node/Customizing-Printf.html}}, 
\IFRU{позволяющее объявлять свои модификаторы, но это}{allowing to define our own
modifiers, but it is} \IT{deprecated}.

\IFRU{Попробуем определить свои собственные модификаторы для 
Mac-адреса и для вывода байта в бинарном виде}{Let's try to define our own modifiers for Mac-address
outputting and also for byte outputting in a binary form}:

\lstinputlisting{C/register_printf_function.c}
\footnote{\IFRU{Основа для примера взята отсюда}{The base of example was taken from}:
\url{http://codingrelic.geekhold.com/2008/12/printf-acular.html}}

\IFRU{Это компилируется с предупреждениями}{This compiled with warnings}:

\begin{lstlisting}
1.c: In function 'main':
1.c:48:2: warning: 'register_printf_function' is deprecated (declared at /usr/include/printf.h:106) [-Wdeprecated-declarations]
1.c:49:2: warning: 'register_printf_function' is deprecated (declared at /usr/include/printf.h:106) [-Wdeprecated-declarations]
1.c:51:2: warning: unknown conversion type character 'M' in format [-Wformat]
1.c:52:2: warning: unknown conversion type character 'B' in format [-Wformat]
\end{lstlisting}

\ac{GCC} \IFRU{умеет следить за соответствиями модификаторов в}{is able to track accordance between
modifiers in the} printf-\IFRU{строке и аргументами в вызове}{string and arguments in} printf(),
\IFRU{но здесь ему встречаются незнакомые модификаторы, о чем он предупреждает}
{however, unfamiliar to it modifiers are present here, so it warns us about them}.

\IFRU{Тем не менее, наша программа работает}{Nevertheless, our program works}:

\begin{lstlisting}
$ ./a.out
00:11:22:33:44:55
10101011
\end{lstlisting}



\subsection{atexit()}

При помощи atexit() можно добавить ф-цию, автоматически вызываемую перед выходом из вашей программы.
Кстати, программы на Си++ именно при помощи atexit() добавляют деструкторы глобальных объектов.

Можно попробовать:

\begin{lstlisting}
#include <string>

std::string s="test";

int main()
{
};
\end{lstlisting}

В листинге на ассемблере найдем конструктор этого глобального объекта:

\begin{lstlisting}[caption=MSVC 2010]
??__Es@@YAXXZ PROC					; `dynamic initializer for 's'', COMDAT
; Line 3
	push	ebp
	mov	ebp, esp
	push	OFFSET $SG22192
	mov	ecx, OFFSET ?s@@3V?$basic_string@DU?$char_traits@D@std@@V?$allocator@D@2@@std@@A ; s
	call	??0?$basic_string@DU?$char_traits@D@std@@V?$allocator@D@2@@std@@QAE@PBD@Z ; std::basic_string<char,std::char_traits<char>,std::allocator<char> >::basic_string<char,std::char_traits<char>,std::allocator<char> >
	push	OFFSET ??__Fs@@YAXXZ			; `dynamic atexit destructor for 's''
	call	_atexit
	add	esp, 4
	pop	ebp
	ret	0
??__Es@@YAXXZ ENDP					; `dynamic initializer for 's''

??__Fs@@YAXXZ PROC					; `dynamic atexit destructor for 's'', COMDAT
	push	ebp
	mov	ebp, esp
	mov	ecx, OFFSET ?s@@3V?$basic_string@DU?$char_traits@D@std@@V?$allocator@D@2@@std@@A ; s
	call	??1?$basic_string@DU?$char_traits@D@std@@V?$allocator@D@2@@std@@QAE@XZ ; std::basic_string<char,std::char_traits<char>,std::allocator<char> >::~basic_string<char,std::char_traits<char>,std::allocator<char> >
	pop	ebp
	ret	0
??__Fs@@YAXXZ ENDP					; `dynamic atexit destructor for 's''
\end{lstlisting}

Конструктор, конструируя, также регистрирует деструктор объекта в atexit().

\subsection{bsearch(), lfind()}
\label{bsearch_lfind}

\IFRU{Удобные ф-ции для поиска чего-либо где-либо}{Handy functions for searching for something somewhere}.

\IFRU{Разница между ними только в том что}{The difference between them is that} lfind() 
\IFRU{просто ищет заданное}{just search for data}, \IFRU{а}{buf} bsearch() \IFRU{требует отсортированный массив
данных, но зато может искать быстрее}{requires sorted data array, but may search faster}
\footnote{\IFRU{методом половинного деления, итд}{by bisection method, etc}}
\footnote{\IFRU{разница еще также в том что bsearch() есть в стандарте}{the difference also is that
bsearch() is present in}\cite{C99TC3}, 
\IFRU{а lfind() нет, он есть только в \ac{POSIX} и в \ac{MSVC}}
{but lfind() is not, it is present only in \ac{POSIX} and \ac{MSVC}},
\IFRU{но эти ф-ции достаточно просты чтобы их реализовать самому}{but these functions are simple enough
to implement them on your own}}.

\IFRU{К примеру, поиск строки в массиве указателей на строки}{For example, search for a string
in the array of pointers to a strings}:

\lstinputlisting{C/lfind.c}

\IFRU{Свой собственный}{We need our own} stricmp() \IFRU{нужен потому что}{because} lfind() 
\IFRU{будет передавать в него указатель на искомую строку}{will pass a pointer to the string we looking for},
\IFRU{а также на место в массиве где записан указатель на строку, но не сама строка}
{but also to the place where pointer to the string is stored, not the string itself}.
\IFRU{Если бы это был
массив строк фиксированной длины, тогда можно было бы воспользоваться стандартным stricmp()}
{If it would be an array of fixed-size strings, then usual stricmp() could be used here instead}.

\IFRU{Кстати, точно также ф-ция сравнения задается и для}{By the way, the same comparison function
is used for} qsort().

lfind() \IFRU{возвращает указатель на место в массиве где ф-ция}{returns a pointer to a place in an array
where the} \TT{my\_stricmp()} \IFRU{сработала выдав}{function was triggered returning} $0$.
\IFRU{Далее вычисляем разницу между адресом этого места и адресом начала самого массива}
{Then we compute a difference between the address of that place and the address of the beginning of an array}.
\IFRU{Учитывая арифметику указателей}{Considering pointer arithmetics}\ref{PtrArith}, 
\IFRU{в итоге получается кол-во элементов между этими адресами}
{we get a number of elements between these addresses}.

\IFRU{Реализуя ф-цию сравнивания, можно искать строку в каком угодно массиве}
{By implementing comparison function, we can search a string in any array}.
\IFRU{Пример из}{Example from} OpenWatcom:

\begin{lstlisting}[caption=\textbackslash{}bld\textbackslash{}pbide\textbackslash{}defgen\textbackslash{}scan.c]
int MyComp( const void *p1, const void *p2 ) {

    KeyWord     *ptr;

    ptr = (KeyWord *)p2;
    return( strcmp( p1, ptr->str ) );
}

static int CheckReservedWords( char *tokbuf ) {
    KeyWord     *match;

    match = bsearch( tokbuf, ReservedWords,
                     sizeof( ReservedWords ) / sizeof( KeyWord ),
                     sizeof( KeyWord ), MyComp );
    if( match == NULL ) {
        return( T_NAME );
    } else {
        return( match->tok );
    }
}
\end{lstlisting}

\IFRU{Здесь имеется отсортированный по первому полю массив структур}
{Here we have array of structures sorted by the first field} \IT{ReservedWords},
\IFRU{выглядящий так}{like}:

\begin{lstlisting}[caption=\textbackslash{}bld\textbackslash{}pbide\textbackslash{}defgen\textbackslash{}scan.c]
typedef struct {
    char        *str;
    int         tok;
} KeyWord;

static KeyWord  ReservedWords[] = {
    "__cdecl",          T_CDECL,
    "__export",         T_EXPORT,
    "__far",            T_FAR,
    "__fortran",        T_FORTRAN,
    "__huge",           T_HUGE,
....
};
\end{lstlisting}

bsearch() \IFRU{ищет строку сравнивая её со строкой в первом поле структуры}{searching for the string comparing
it with the string in the first field of structure}.
\IFRU{Здесь можно применить именно bsearch(), потому что массив уже отсортированный}
{bsearch() can be used here because array is already sorted}.
\IFRU{Иначе пришлось бы использовать lfind()}{Otherwise, lfind() should be used}.
\IFRU{Вероятно, несортированный массив можно вначале отсортировать при помощи qsort(), а затем уже
использовать bsearch(), если вам нравится такая идея}
{Probably, unsorted array can be sorted by qsort() before bsearch() usage, if you like the idea}. \\
\\
\IFRU{Точно так же можно искать что угодно, где угодно}{In the same way, it is possible to search anything
anywhere}. \IFRU{Пример из}{The example from} BIND 9.9.1
\footnote{\url{https://www.isc.org/downloads/bind/}}:

\begin{lstlisting}[caption=backtrace.c]
static int
symtbl_compare(const void *addr, const void *entryarg) {
	const isc_backtrace_symmap_t *entry = entryarg;
	const isc_backtrace_symmap_t *end =
		&isc__backtrace_symtable[isc__backtrace_nsymbols - 1];

	if (isc__backtrace_nsymbols == 1 || entry == end) {
		if (addr >= entry->addr) {
			/*
			 * If addr is equal to or larger than that of the last
			 * entry of the table, we cannot be sure if this is
			 * within a valid range so we consider it valid.
			 */
			return (0);
		}
		return (-1);
	}

	/* entry + 1 is a valid entry from now on. */
	if (addr < entry->addr)
		return (-1);
	else if (addr >= (entry + 1)->addr)
		return (1);
	return (0);
}

...

	/*
	 * Search the table for the entry that meets:
	 * entry.addr <= addr < next_entry.addr.
	 */
	found = bsearch(addr, isc__backtrace_symtable, isc__backtrace_nsymbols,
			sizeof(isc__backtrace_symtable[0]), symtbl_compare);
	if (found == NULL)
		result = ISC_R_NOTFOUND;
\end{lstlisting}

\IFRU{Таким образом, можно обойтись без того чтобы писать каждый раз цикл for() для перебирания элементов, итд}
{Thus, it is possible to get rid of for() loop for enumerating elements each time, etc}.



\subsection{setjmp(), longjmp()}

В каком-то смысле, это реализация исключений в Си.

jmp\_buf это просто структура содержащая в себе набор регистров, но самые важные, это адрес текущей инструкции
и адрес указателя стека.
Всё что делает setjmp() это просто записывает текущие регистры процессора в эту структуру.
А всё что делает longjmp() это восстанавливает состояние регистров.

Это часто используется для возврата из каких-то глубоких мест наружу, обычно, в случае ошибок.
Собственно, как и исключения в Си++.

Например, в Oracle RDBMS, когда происходит некая ошибка, и пользователь код ошибки, сообщение об ошибке, итд,
в реальности, там срабатывает longjmp откуда-то из глубины. Для того же это используется и в bash.

Этот механизм даже немного гибче чем исключения в других ЯП --- нет никаких проблем ``устанавливать точки 
возврата'' в каких угодно местах программы и затем возвращаться туда по мере необходимости.

\label{goto}
Пофантазируя, можно даже сказать что longjmp это такой супермега-goto, обходящий блоки, ф-ции, и восстанавливающий
состояние стека.

Однако, нельзя забывать, что всё что восстанавливает longjmp это регистры. Выделенная память остается выделенной,
никаких деструкторов, как в Си++, вызвано не будет. Никакого \ac{RAII} здесь нет. 
С другой стороны, так как часть стека просто аннулируется, 
память выделенная при помощи alloca()\ref{alloca} будет также аннулирована.

\subsection{stdarg.h}

Тут ф-ции для работы с переменным кол-вом аргументов. Как минимум ф-ции семейства printf() и scanf() такие.

\subsubsection{Засада \#1}

Переменную типа va\_list можно использовать только один раз, если нужно больше, их нужно копировать:

\begin{lstlisting}
va_list v1, v2;
va_start(v1, fmt);
va_copy(v2, v1);
// use v1
// use v2
va_end(v2);
va_end(v1);
\end{lstlisting}


\label{C99}
\chapter{Стандарт Си C99}

Текст стандарта: \cite{C99TC3}.

Этот стандарт поддерживается в GCC, но не в MSVC, и не ясно, будет ли он там поддерживаться вообще.

Чтобы включить его поддержку в GCC, нужно добавить ключ компиляции \IT{-stc=c99}.


\chapter{C++}
\section{\IFRU{Элементы языка Си++}{C++ language elements}}
\subsection{references}

\index{C++!references}
\IFRU{Это то же что и указатели (\ref{pointers}), но ``обезопасенные'' (safe), 
потому что работая с ними, труднее сделать ошибку}
{It is the same thing as pointers (\ref{pointers}), but ``safe'', 
because it is harder to make a mistake while working with them}
\cite[8.3.2]{CPP11}.

\IFRU{Например, reference всегда должен указывать объект того же типа и не может быть NULL}
{For example, reference should be always be pointing to the object of corresponding type and can't be NULL}
\cite[8.6]{ParashiftCPPFAQ}.
\IFRU{Более того, reference нельзя менять, нельзя его заставить указывать на другой объект (reseat)}
{Even more, reference cannot be changed, it's not possible to point to to another object (reseat)}
\cite[8.5]{ParashiftCPPFAQ}.

\IFRU{В}{In} \cite[1.7.1]{REBook} \IFRU{продемонстрировано, что на уровне x86-кода, это одно и то же}
{it was demonstrated that on x86-level of code it is the same thing}.

\IFRU{Как и указатели, \IT{reference} также можно возвращать из ф-ций, например}
{Just like pointers, \IT{references} can be returned by functions, e.g.}:

\begin{lstlisting}
#include <iostream>
 
int& use_count()
{
	static int uc=1000; // starting value
	return uc;
};
 
void main()
{
	std::cout << ++use_count() << std::endl;
	std::cout << ++use_count() << std::endl;
	std::cout << ++use_count() << std::endl;
	std::cout << ++use_count() << std::endl;
};
\end{lstlisting}

\IFRU{Это выведет предсказуемое}{This will print predictable}:

\begin{lstlisting}
1001
1002
1003
1004
\end{lstlisting}

\IFRU{Если посмотреть что сгенерировалось на языке ассемблера, то можно увидеть что}
{If to see what was generated in assembly language, it can be seen that} \TT{use\_count()} 
\IFRU{просто возвращает указатель на}{just returns a pointer to} \TT{uc}, 
\IFRU{а в}{and in the} \TT{main()} \IFRU{происходит инкремент значения по этому указателю}
{an increment of a value on this pointer is occuring}.



\section{C++: ввод/вывод}

Часто есть необходимость выводить в ostream целые структуры, а каждый раз выводить их по одному полю это
неудобно. Иногда это решается добавлением метода ToString() в класс. 
Другое решение это сделать ``свободную'' ф-цию (free function)
\footnote{Ф-ции не являющиеся методом какого-либо класса} для вывода вроде:

\begin{lstlisting}
ostream& operator<< (ostream &out, const Object &in)
{
    out << "Object. size=" << in.size << " value=" << in.value << " ";

    return out;
};
\end{lstlisting}

После этого можно отправлять экземпляры класса прямо в ostream:

\begin{lstlisting}
Object o1, o2;
...
cout << "o1=" << o1 << " o2=" << o2 << endl;
\end{lstlisting}

А для того чтобы ф-ция вывода могла обращаться к любым полям класса, её можно сделать friend.


\section{Темплейты}

Темплейты нужны обычно для того чтобы сделать класс универсальным для нескольких типов данных.
К примеру, \TT{std::string} в реальности это \TT{std::basic\_string<char>}, \\ 
а \TT{std::wstring} это \TT{std::basic\_string<wchar\_t>}. \\
\\
Нередко подобное делают и для типов float/double/complex и даже int. 
Некий математический алгоритм может быть описан один раз,
но скомпилирован сразу в нескольких версиях, для работы со всеми этими типами данных. \\
\\
Таким образом, можно описывать алгоритмы только один раз, но работать они будут для разных типов. \\
\\
Простейшие примеры это ф-ции max, min, swap, работающие для любых типов, которые можно сравнивать
и присваивать.



\section{C++ STL}

\subsection{Итераторы}

Итераторы это ``обезопасенные'' указатели.



\chapter{Прочее}

Что хранится в объектных и бинарных (.exe, .dll) файлах?

Обычно только данные (глобальные переменные) и тела ф-ций (включая методы классов).

Информации о типах (классы, структуры, typedef\ref{typedef}-ы) там нет. 
Это может помочь в понимании того как всё устроено.

В этом заключается одна из больших проблем декомпиляции --- отсутствие информации о типах.

О том как всё компилируется в машинный код, подробнее можно почитать здесь:\cite{REBook}.


\chapter*{\IFRU{Послесловие}{Afterword}}
\addcontentsline{toc}{chapter}{\IFRU{Послесловие}{Afterword}}

% \section{\IFRU{Краудфандинг}{Crowdfunding}}

\IFRU{Эта книга является свободной, находится в свободном доступе, и доступна в виде исходных кодов}
{This book is free, available freely and available in source code form}\footnote{\url{https://github.com/dennis714/RE-for-beginners}} (LaTeX), 
\IFRU{и всегда будет оставаться таковой}{and it will be so forever}.

\IFRU{В мои текущие планы насчет этой книги входит добавление информации на эти темы:}
{My current plans for this books is to add a lot of information about} C++11, flex/bison.

\IFRU{Если вы хотите чтобы я продолжал свою работу и писал на эти темы,
вы можете рассмотреть идею краудфандинга}
{If you want me to continue writing on all these topics, you may consider crowdfunding}.

\IFRU{Со способами краудфандинга можно ознакомиться на странице}
{Ways to crowdfund are available on the page:} \url{http://yurichev.com/crowdfunding.html}

%\subsection{\IFRU{Жертвователи}{Donors}}


\section{\IFRU{Вопросы?}{Questions?}}

\IFRU{Совершенно по любым вопросам, вы можете не раздумывая писать автору}
{Do not hesitate to mail any questions to the author}: \TT{<\EMAIL>}
 
\IFRU{Пожалуйста, присылайте мне информацию о замеченных ошибках 
(включая грамматические), итд.}
{Please, also do not hesitate to send me any corrections 
(including grammar ones (you see how horrible my English is?)), etc.}

\chapter*{\IFRU{Список принятых сокращений}{Acronyms used}}

\begin{acronym}
\acro{STL}{Standard Template Library}
\acro{TLS}{Thread Local Storage}
\acro{TIB}{Thread Information Block}
\acro{RAII}{Resource Acquisition Is Initialization}
\IFRU{
	\acro{OS}[ОС]{Операционная Система}
	\acro{PL}[ЯП]{Язык Программирования}
}
{
	\acro{OS}{Operating System}
	\acro{PL}{Programming Language}
}
\acro{MSVC}{Microsoft Visual C++}
\acro{GCC}{GNU Compiler Collection}
\acro{POSIX}{Portable Operating System Interface}
\end{acronym}


%\bibliographystyle{alpha}
\bibliographystyle{plain} % FIXME
\bibliography{books,articles,usenet,misc}

\clearpage
\printindex

\end{document}

