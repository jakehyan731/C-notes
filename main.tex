\documentclass[11pt,a4paper,oneside]{book}

\usepackage{cmap}

\ifdefined\RUSSIAN
\usepackage[english,russian]{babel}
\usepackage[T2A]{fontenc}
\usepackage{paratype}
\renewcommand*\familydefault{\sfdefault}
% http://www.emerson.emory.edu/services/latex/latex_169.html
\newcommand{\lstlistingsize}{\scriptsize}
\else
\usepackage[russian,english]{babel}
\usepackage[T2A]{fontenc}
\usepackage[default]{sourcesanspro}
\newcommand{\lstlistingsize}{\footnotesize}
\fi

\usepackage[utf8]{inputenc}
\usepackage{listings}
\usepackage{ulem}
\usepackage{url}
\usepackage{graphicx}
\usepackage{listingsutf8}
\usepackage{makeidx}
\usepackage{cite}
\usepackage[cm]{fullpage}
\usepackage{color}
\usepackage{fancyvrb}
\usepackage{xspace}
\usepackage{framed}
\usepackage{ccicons}
\usepackage[nottoc]{tocbibind}
\usepackage{amsmath}
\usepackage[table]{xcolor}% http://ctan.org/pkg/xcolor
\usepackage[]{hyperref} % should be last
%\usepackage{tikz}

\definecolor{lstbgcolor}{rgb}{0.94,0.94,0.94}
\makeindex

\newcommand{\TT}[1]{\texttt{#1}}
\newcommand{\IT}[1]{\textit{#1}}
\newcommand{\IFRU}[2]{\iflanguage{russian}{#1}{#2}}


\newcommand{\TITLE}{\IFRU{Заметки о языке программирования Си/Си++}
{C/C++ programming language notes}}
\newcommand{\AUTHOR}{\IFRU{Денис Юричев}{Dennis Yurichev}}
\newcommand{\EMAIL}{dennis@yurichev.com}

\hypersetup{
    pdftex,
    colorlinks=true,
    allcolors=blue,
    pdfauthor={\AUTHOR},
    pdftitle={\TITLE}
    }

\selectlanguage{english}

\lstset{
    backgroundcolor=\color{lstbgcolor},
    basicstyle=\ttfamily\lstlistingsize, 
    breaklines=true,
    frame=single,
    inputencoding=cp1251,
    columns=fullflexible,keepspaces,
}

\begin{document}

\VerbatimFootnotes

\frontmatter

\begin{titlepage}
\begin{center}
\vspace*{\fill}
\LARGE \TITLE

\vspace*{\fill}

\large \AUTHOR

\large \TT{<\EMAIL>}
\vspace*{\fill}
\vfill

\ccbyncnd

\textcopyright 2013, \AUTHOR. 

\IFRU{Это произведение доступно по лицензии Creative Commons «Attribution-NonCommercial-NoDerivs» 
(«Атрибуция — Некоммерческое использование — Без производных произведений») 3.0 Непортированная. 
Чтобы увидеть копию этой лицензии, посетите}
{This work is licensed under the Creative Commons Attribution-NonCommercial-NoDerivs 3.0 Unported License. 
To view a copy of this license, visit} \url{http://creativecommons.org/licenses/by-nc-nd/3.0/}.

\IFRU{Версия этого текста}{Text version} ({\large \today}).

\IFRU{Возможно, более новая версии текста, а так же англоязычная версия, также доступна по ссылке}
{Probably, newer version of this text, and also Russian language version is also accessible at} 
%FIXME URL?
\url{http://yurichev.com/C-notes.html}

\IFRU{Вы также можете подписаться на мой twitter для получения информации о новых версиях этого текста, итд:
\href{https://twitter.com/yurichev_ru}{@yurichev\_ru}}
{You may also subscribe to my twitter, to get information about updates of this text, etc: 
\href{https://twitter.com/yurichev}{@yurichev}}
%FIXME mailing list
\end{center}
\end{titlepage}

\tableofcontents
\cleardoublepage

\begin{center}
\vspace*{\fill}

\IFRU{Эта страница сдается в аренду для рекламы}{This page can be rented for advertisement}.

\TT{<\EMAIL>}

\vspace*{\fill}
\vfill
\end{center}


\cleardoublepage

\chapter{\IFRU{Введение}{Preface}}

Сейчас, в 2013-м году, если некто желает написать 1) как можно более быстро работающую программу; 2) либо как
можно более компактную для встраиваемых систем либо маломощных микроконтроллеров, то выбор небольшой:
Си, Си++ либо ассемблер. И альтернативы этим старым но популярным языкам, в обозримом будущем, 
пока что не видно.

\chapter{\IFRU{Целевая аудитория}{Target audience}}

Этот сборник заметок предназначен не для начинающих, но и не для экспертов, а скорее для тех, 
кто хочет освежить свои знания по Си/Си++.

\chapter{\IFRU{Об авторе}{About author}}

\IFRU{Денис Юричев ~--- опытный программист, свободный для найма как программист, reverse engineer, консультант, тренер. 
С его резюме можно ознакомиться \href{http://yurichev.com/Dennis_Yurichev.pdf}{здесь}.}
{Dennis Yurichev is experienced programmer, available for hire as programmer, reverse engineer, consultant, trainer. 
His CV is available \href{http://yurichev.com/Dennis_Yurichev.pdf}{here}.}

\chapter{\IFRU{Благодарности}{Thanks}}

\IFRU{Андрей ''herm1t'' Баранович, Слава ''Avid'' Казаков}
{Andrey ''herm1t'' Baranovich, Slava ''Avid'' Kazakov}.

\mainmatter

\chapter{\IFRU{Элементы языка Си/Си++}{C/C++ language elements}}
\subsection{\IFRU{Комментарии}{Comments}}

\IFRU{Их иногда удобно вставлять прямо в вызов ф-ции, чтобы где-то на виду держать пометку,
что означает некий аргумент}
{It is sometimes useful to insert them right into a function call, in order to have a visual note
about meaning of an argument}:

\begin{lstlisting}
f (val1, /* a very special flag! */ false, /* another special flag here */ true);
\end{lstlisting}

\IFRU{Целый блок кода можно откомментировать при помощи}
{The whole code block can be commented with the help of} \#if
\footnote{\IFRU{директива препроцессора}{preprocessor directive}}:

\begin{lstlisting}
	ta	= aemif_calc_rate(t->ta, clkrate, TA_MAX);
	rhold	= aemif_calc_rate(t->rhold, clkrate, RHOLD_MAX);
#if 0	
	rstrobe	= aemif_calc_rate(t->rstrobe, clkrate, RSTROBE_MAX);
	rsetup	= aemif_calc_rate(t->rsetup, clkrate, RSETUP_MAX);
	whold	= aemif_calc_rate(t->whold, clkrate, WHOLD_MAX);
#endif	
	wstrobe	= aemif_calc_rate(t->wstrobe, clkrate, WSTROBE_MAX);
	wsetup	= aemif_calc_rate(t->wsetup, clkrate, WSETUP_MAX);
\end{lstlisting}

\IFRU{Это может быть удобнее чем традиционный способ потому что текстовый редактор или \ac{IDE} в этом случае
не ``сломает'' отступы при выравнивании}
{This might be more convenient then usual way because the text editor or \ac{IDE} in this case will not ``break''
indentation while auto-indentation}.


\subsection{\IFRU{Типы данных}{Datatypes}}

\subsubsection{bool}

bool \IFRU{есть в}{is present in the} \CPP, \IFRU{но также он есть и в Си, начиная с}
{but also in the C starting at} C99\ref{C99} (stdbool.h).

\IFRU{В Windows API принят тип BOOL, это синоним \IT{int}}
{There are synonymous to the \IT{int} type in Windows API ~--- BOOL}.

\subsubsection{\IFRU{Знаковые или беззнаковые}{Signed or unsigned}?}

\IFRU{Знаковые типы}{Signed types} (int, char) 
\IFRU{используются куда чаще беззнаковых}{are used much more often than unsigned} (unsigned int, unsigned char).

\IFRU{Но с точки зрения документации кода, если вы объявляете переменную, которая никогда не будет хранить отрицательное
значение, в т.ч., индексы массивов, наверное лучше применять беззнаковый тип}
{However, in the sense of the code self-documenting, if you declare a variable which will not be assigned to
a negative value, including array indices, perhaps, unsigned type is better}.
\IFRU{Например, в LLVM очень часто используется \IT{unsigned}}
{For example, \IT{unsigned} type is often used in LLVM}.

\IFRU{Если вы работаете с байтами, например, с байтами в памяти, то наверное лучше применять именно}
{If you work with bytes, for example, with bytes in memory, then perhaps}
\IT{unsigned char}
\IFRU{}{ is better}.

\IFRU{К тому же, это может немного защититься от ошибок связанных с}
{Aside from that, this may help protecting from the errors related to} integer overflow\cite{Phrack3C0A}.

\IFRU{В качестве очень простого примера}{As a simple example}:

\begin{lstlisting}
#define MAX_BUFFER 1024

void f(int size)
{
	if (size>MAX_BUFFER_SIZE)
		die ("Too large!");
	void *p=malloc (size);
	...
};
\end{lstlisting}

\IFRU{Если}{If} \IT{size} \IFRU{будет, например}{will be, for example}, $-1$, 
\IFRU{то}{then} malloc() \IFRU{вызовется с аргументом}{will be called with an argument}
\TT{0xffffffff} (\IFRU{это}{it is} $4294967295$).
\IFRU{Конечно, нужно было бы добавить вторую проверку}{Of course, we need to add a second sanitizing check}:
\IT{if (size<0)}, \IFRU{но такая проверка выглядит здесь абсурдной}{but such check here will have absurdical look}.

\IFRU{Таким образом, здесь нужно было бы применить}{So, the type} 
\IT{unsigned}\IFRU{, либо даже тип}{should be used here, maybe even} \IT{size\_t}. 
\IT{size\_t} \IFRU{определяет тип, достаточно большой, способный хранить размер любого,
сколько угодно большого блока памяти}{defines a big enough type able to store a size of any, big enough memory block}.
\IFRU{На 32-битных архитектурах это обычно}{It is} \IT{unsigned int}\IFRU{}{ on 32-bit architectures},
\IFRU{а на 64-битных это}{and} \IT{unsigned int64}\IFRU{}{ on 64-bit ones}.

\subsubsection{char \OrENRU uint8\_t \IFRU{вместо}{instead of} int?}

\IFRU{Может показаться что если какая-то переменная всегда будет в пределах}
{One may think that is a value will always be in} $0..100$\IFRU{, то незачем выделять под нее 32-битный}
{ limits, then it is not necessary to allocate the whole 32-bit}
\IT{int}, \IFRU{а можно обойтись типом}{smaller types may be enough like} \IT{char} \OrENRU \IT{unsigned char}.
\IFRU{К тому же, такая переменная будет занимать в памяти в 4 раза меньше}{Besides, it will require
less memory}.

\IFRU{Это не так}{It is not so}.
\IFRU{Из-за выравнивания по 4-байтной границе (а в 64-битных архитектурах ~--- по 8-байтной), 
определяемые переменные типа \IT{char}, занимают столько же места сколько и}
{Because of aligning by 4-bytes border (or by 8-bytes border in 64-bit architectures),
the variables declared with the type \IT{char}, requires as much space as}
\IT{int}.

\IFRU{Конечно, компилятор мог бы отводить под}{Of course the compiler may allocate only 1 bytes for the}
\IT{char}\IFRU{ только один байт}{},
\IFRU{но тогда \ac{CPU} тратил бы больше времени на обращение к ``невыровненным'' по границе байтам}
{but then \ac{CPU} will spent more time for accessing ``unaligned'' by border bytes}.

\IFRU{Работа с отдельными байтами может быть ``дороже'' и медленнее чем работа с 32-битными или 64-битными 
значениями потому что
регистры \ac{CPU} обычно имеют ту же ширину что и разрядность процессора}
{Specific bytes processing may be more ``expensive'' and slower then processing 32-bit or 64-bit values
because \ac{CPU} registers are usually has the same width as CPU bits}.
\IFRU{И даже более того}{Even more than that}, \ac{RISC}-\IFRU{процессоры}{units}
(\IFRU{например ARM}{ARM for example})
\IFRU{вообще могут быть неспособны работать с отдельными байтами внутренне,
потому что имеют только 32-битные регистры}
{may not be able to work with specific bytes internally at all because they have only 32-bit registers}.

\IFRU{Таким образом, если вы раздумываете над типом для локальной переменной, то \IT{int/unsigned int} может быть лучше}
{So if you considering about type for the local variable, \IT{int/unsigned int} may be better}.

\IFRU{С другой стороны, переменные каких типов лучше использовать в структурах}
{On the other hand, which types are better suited for a structures}?
\IFRU{Это вопрос поиск баланса между скоростью и компактностью}
{This is a question of seeking balance between speed and compactness}.
\IFRU{С одной стороны, можно отвести}{On the one hand, one may use} \IT{char} 
\IFRU{под небольшие переменные, под флаги, под enum, итд, но не следует
забывать, что доступ к этим переменным будет чуть медленнее}{for a small variables, flags, bitfields, enums, etc,
but one should not forget that access to these variables will be slower}.
\IFRU{С другой стороны, под все переменные можно отводить}{On the other hand, if to assign}
\IT{int}\IFRU{, тогда работа со структурой будет быстрее, но она будет занимать в памяти больше места}
{ to each variables, working with a structure will be faster, but it will require more space in memory}.

\IFRU{Например}{For example}:

\begin{lstlisting}
struct
{
	char some_flags; // 1 byte
	void* ptr; // 4/8 bytes, offset: +1
} s;
\end{lstlisting}

\IFRU{Если скомпилировать это с упаковкой полей по 1-байтной границе, 
то доступ к}{If to compile this with structure packing by 1-byte border, access to the} \IT{some\_flags}
\IFRU{в памяти будет возможно даже быстрее чем доступ к}{in the memory may be even faster then to} \IT{ptr}, 
\IFRU{потому что первое поле выровнено по 4-байтной границе, а второе нет}
{because the first field is aligned by 4-byte border, while the second is not}.

\IFRU{А если компилировать это с упаковкой полей по умолчанию, то компилятор отведет под первое поле 4 байта и смещение
у второго поля будет +4}{If to compile this by default structure packing, then 4 bytes will be allocated for 
the first field and the offset of the second field will be +4}.

\IFRU{Резюмируя: если компактность и экономия памяти для вас важнее скорости, тогда нужно использовать}
{Summarizing: if compactness and memory footprint is important, then}
\IT{char}, \IT{uint16\_t}, \IFRU{итд}{etc, may be used}.

\subsubsection{x86-64 \OrENRU AMD64}

\IFRU{На новых 64-битных x86-процессорах, тип}{On the new 64-bit x86 CPUs, the} \IT{int/unsigned int} 
\IFRU{оставили 32-битным, вероятно, в целях совместимости}{is still 32-bit, perhaps, for compatibility}.
\IFRU{Так что если вы хотите использовать 64-битные значения, можно использовать}
{So if one need 64-bit variables, one may use} \IT{uint64\_t} \OrENRU \IT{int64\_t}.

\IFRU{А указатели теперь, конечно, 64-битные}{But pointers, of course, has 64-bit width}.

\label{typedef}
\subsubsection{typedef}

\TT{typedef} \IFRU{вводит}{introduces} \IT{\IFRU{синоним}{synonym}} \IFRU{для типа данных}{for a data type}.
\IFRU{Часто это используется для структур, чтобы каждый раз не писать}
{It is often used for structures, for the reason not to write} \IT{struct}
\IFRU{перед её именем, например}{each time before its name, for example}:

\begin{lstlisting}
typedef struct _node
{
	node *prev;
	node *next;
	void *data;
} node;
\end{lstlisting}

\IFRU{Такого очень много в ``заголовочных'' файлах в}
{A lot of such examples may be found in header files in the} Windows SDK (Windows API).

\IFRU{Тем не менее}{Nevertheless}, \IT{typedef} 
\IFRU{также можно использовать не только для структур, но и для обычных, 
интегральных}{can be also used not only for structures, but also for usual integral}
\footnote{\IFRU{приводимых к числу}{which may be converted to a number}}, \IFRU{типов, например}{types like}:

\begin{lstlisting}
typedef int age;
int compute_mean (age wife, age husband);

typedef int coord;
void draw (coord X, coord Y, coord Z);

typedef uint32_t address;
void write_memory (address a, size_t size, byte *buf);
\end{lstlisting}

\IFRU{Как видно}{As we can see}, \IT{typedef} 
\IFRU{здесь может помочь в документировании исходного кода, так он легче читается}{here may help with
code documentation, it is now easier to read}.

\IFRU{Например, тип}{For example, the} \IT{time\_t} 
(\IFRU{время в формате}{The time in the} UNIX time
\IFRU{, то что возвращает стандартная функция}{ format, for example, what the}
localtime()
\IFRU{, например}{returns}), \IFRU{это на самом деле
обычное 32-битное число, но этот тип определен в}{it is in fact 32-bit number, but the type is defined in the}
\IT{time.h} \IFRU{обычно так}{file usually as}:

\begin{lstlisting}
typedef long __time32_t;
\end{lstlisting}

\IFRU{Здесь вполне можно было бы использовать директиву препроцессора}
{A preprocessor directive} \TT{\#define} \IFRU{(многие так и делают),
но это хуже с точки зрения обработки ошибок во время компиляции}{may be used here (many do so),
but it is worse in the sense of errors handling during compilation}.

\paragraph{\IFRU{Критика }{}typedef\IFRU{}{ criticism}}

\IFRU{Тем не менее, такой известный и опытный программист как Линус Торвальдс,
против использования typedef}{Nevertheless, such well-known and experiences programmer as Linus
Torvalds is against typedef usage}:
\cite{Torvalds:2002}.


\subsection{goto}

\index{goto}
\IFRU{Использование}{Usage of} \IT{goto}\footnote{statement} 
\IFRU{считается плохим тоном и вредным вообще}{is considered as bad taste and harmful}
\cite{Dijkstra:1968:LEG:362929.362947}\cite{Dijkstra:1979:GSC:1241515.1241518}, 
\IFRU{тем не менее, использование его в разумных дозах}{nevertheless, its usage in reasonable
doses}\cite{Knuth:1974:SPG:356635.356640} \IFRU{может облегчить жизнь}{may be very helpful}.

\IFRU{Частый пример, это выход из функции}{One frequent example is return from a function}:

\begin{lstlisting}
void f(...)
{
	byte* buf1=malloc(...);
	byte* buf2=malloc(...);

	...

	if (something_goes_wrong_1)
		goto cleanup_and_exit;

	...
	
	if (something_goes_wrong_2)
		goto cleanup_and_exit;

	...

cleanup_and_exit:
	free(buf1);
	free(buf2);
	return;
};
\end{lstlisting}

\IFRU{Более сложный пример}{More complex example}:

\begin{lstlisting}
void f(...)
{
	byte* buf1=malloc(...);
	byte* buf2=malloc(...);

	FILE* f=fopen(...);
	if (f==NULL)
		goto cleanup_and_exit;

	...

	if (something_goes_wrong_1)
		goto close_file_cleanup_and_exit;

	...
	
	if (something_goes_wrong_2)
		goto close_file_cleanup_and_exit;

	...

close_file_cleanup_and_exit:
	fclose(f);

cleanup_and_exit:
	free(buf1);
	free(buf2);
	return;
};
\end{lstlisting}

\IFRU{Если в данных примерах отказаться от}{If to remove all} \IT{goto}\IFRU{, то придется вызывать}
{ in these examples, one will need to call} \IT{free()} \AndENRU \IT{fclose()}
\IFRU{перед каждым выходом из функции}{before each return from the function} 
(\IT{return})\IFRU{, что здорово замусорит весь код}{which adds a lot of mess}.

\index{Linux}
\IFRU{Использование}{Usage of} \IT{goto} 
\IFRU{в таких случаях одобряется, например, в}{is, for example, approved in} \cite{LinuxKernelCodingStyle}.

%Примеры более ``harmful'' но эффективного использования goto можно найти в исходниках nginx.
% example?


\subsection{for}

\IFRU{В}{The} for()\IFRU{, как известно, три выражения:
первое вычисляется перед началом всех итераций,
второе вычисляется перед каждой итерацией,
третье ~--- после каждой итерации}
{ statement, as we know, has 3 expressions:
1st computing before all iterations begin,
2nd computing before each iteration
and the 3rd ~--- after each iteration}.

\IFRU{И конечно же, там можно указывать что-то отличное от обычного счетчика}
{And of course, there might be written something different from the usual counter}.

\subsubsection{\IFRU{Засада}{Caveat} \#1}

\IFRU{Если написать такое}{If to write this}:

\lstinputlisting{common/for_strlen.cpp}

... \IFRU{то это наверное будет ошибкой}{perhaps this is a mistake}:
\TT{strlen(s)} \IFRU{будет вызываться перед каждой итерацией}{will be called before each iteration} 
~--- \IFRU{такой код генерирует}{that is the code} MSVC 2010\IFRU{}{ generated}.
\IFRU{Впрочем}{However}, GCC 4.8.1 \IFRU{вызывает}{calls} \TT{strlen(s)} 
\IFRU{только один раз, в начале цикла}{only once, at the loop beginning}.

\subsubsection{\IFRU{Запятая}{Comma}}

\IFRU{Запятая}{Comma}\cite[6.5.17]{C99TC3} ~--- \IFRU{не самая понятная для всех штука в Си, 
однако, их очень удобно использовать в определениях в for()}
{is not widely understood C feature, however, it is very useful for using in a for() declarations}.

\IFRU{Например, может пригодится использовать в цикле два итератора одновременно}
{For example, it is useful to have two iterators simultaneously}.
\IFRU{Пусть один просто отсчитывает от 0, прибавляя 1 при каждой итерации,
а второй итератор указывает на элемент в списке}
{Let the first iterator just counts from 0 adding 1 at each iteration, and the second iterator
points to the list element}:

\lstinputlisting{common/for_comma.cpp}

\IFRU{Это выдаст предсказуемый результат}{This will dumps predictable result}:

\begin{lstlisting}
0: 123
1: 456
2: 789
3: 1
\end{lstlisting}

\IFRU{Но к сожалению, определять итераторы вместе с типами в теле самого for() вот так нельзя}
{However, it is not possible to declare iterators with its types in for() clause}:

\begin{lstlisting}
	for (int i=0, std::list<int>::iterator it=l.begin(); it!=l.end(); i++, it++)
\end{lstlisting}

\subsubsection{continue}

\IT{continue} \IFRU{это безусловный переход на конец тела цикла}{is unconditional goto to the end
of loop body}.

\IFRU{Это может быть очень полезно, например, в подобном коде}
{This may be very useful, for example, in such code}:

\begin{lstlisting}
for (...)
{
	if (is_element_satisfied_criteria_1(...)==true)
	{
		// do something need in is_element_satisfied_criteria_2()

		if (is_element_satisfied_criteria_2(...)==true)
		{
			do_something_1();
			do_something_2();
			do_something_3();
		};

	};
};
\end{lstlisting}

... \IFRU{всё это можно легко заменить на более опрятное}{it is all can be replaced by neat}:

\begin{lstlisting}
for (...)
{
	if (is_element_satisfied_criteria_1(...)==false)
		continue;

	// do something need in is_element_satisfied_criteria_2()

	if (is_element_satisfied_criteria_2(...)==false)
		continue;

	do_something_1();
	do_something_2();
	do_something_3();
};
\end{lstlisting}


\subsection{sizeof}

\IFRU{Обычно}{Usually}, sizeof() \IFRU{применяют к \glslink{integral type}{интегральным типам}}
{is applied to \glslink{integral type}{integral types}}
\IFRU{либо к структурам}{or to structures},
\IFRU{тем не менее, его можно применять и к массивам, к примеру}
{but nevertheless it is possible to apply it to arrays as well}:

\begin{lstlisting}
	char buf[1024];
	snprintf(buf, sizeof(buf), "...");
\end{lstlisting}

\IFRU{В противном случае, если указывать длину массива}{Otherwise, if to specify array length} ($1024$) 
\IFRU{в двух местах}{in both places} 
(\IFRU{в определении \TT{buf} и как второй аргумент \TT{snprintf()}}
{in \TT{buf} declaration and as a second argument of \TT{snprintf()}}),
\IFRU{то и изменять это значение придется каждый раз в обоих местах, а об этом легко забыть}
{then the value is have to be changed at the both places each time, and it is easy to forget about this}.

\IFRU{Если нужны wide-строки, то}{If one need wide-strings, then} sizeof() 
\IFRU{можно применять к}{can be applied to} \IT{wchar\_t} 
(\IFRU{который, на самом деле, 16-битный тип данных \IT{short}}
{which is in turn, 16-bit data type \IT{short}}):

\begin{lstlisting}
	wchar_t buf[1024];
	swprintf(buf, sizeof(buf)/sizeof(wchar_t), "...");
\end{lstlisting}

sizeof() \IFRU{возвращает длину в байтах, так что здесь он будет равен}{returns the size in bytes, so it will
be here} $1024*2$, \IFRU{т.е.}{i.e.}, $2048$. 
\IFRU{Но мы можем разделить это значение на длину одного элемента массива}
{But we can divide this value by length of one array element} (\IT{wchar\_t})
\IFRU{в байтах ($2$)}{is $2$ in bytes},
\IFRU{чтобы получить количество элементов в массиве}{in order to get elements number in array} ($1024$).

sizeof() \IFRU{можно применять и к массивам структур}{can be applied to array of structures}:

\begin{lstlisting}
struct phonebook_entry
{
	char *name;
	char *surname;
	char *tel;
};

struct phonebook_entry phonebook[]=
{
	{ "Kirk", "Hammett", "555-1234" },
	{ "Lars", "Ulrich", "555-5678" },
	{ "James", "Hetfield", "555-1122" },
	{ "Robert", "Trujillo", "555-7788" }
};

void dump (struct phonebook_entry* input)
{
	for (int i=0; i<sizeof(phonebook)/sizeof(struct phonebook_entry); i++)
		printf ("%s %s - %s\n", input[i].name, input[i].surname, input[i].tel);
};
\end{lstlisting}

sizeof(phonebook) ~--- \IFRU{это размер всего массива структур в байтах}{is a size of the whole array
of structures in bytes}.
\TT{sizeof(struct phonebook\_entry)} ~--- \IFRU{это размер одной структуры в байтах}{is a size of one structure
in bytes}.
\IFRU{Делением мы узнаем количество структур в массиве}{By division we get number of structures in an array}.


\section{\IFRU{Указатели}{Pointers}}

Как однажды сказал Дональд Кнут в интервью\cite{KnuthInterview1993}, то как в Си устроены указатели, это является
очень удачной инновацией в языках программирования по тем временам.

Итак, определимся с терминологией. Указатель это просто адрес какого-то элемента в памяти. Указатели настолько
популярны, потому что в какую-то функцию намного проще передать просто адрес объекта в памяти, вместо того
чтобы передавать весь объект --- ведь это абсурдно. К тому же, вызываемая функция, например, обрабатывающая
ваш массив данных, просто изменит что-то в нем, вместо того чтобы возвращать новый, измененный массив данных, 
что тоже абсурдно.

Возьмем простой пример. Стандартная функция \IT{strtok()} делит строку на подстроки, используя заданный символ
как разделитель. К примеру, мы можем подать на вход строку \TT{The quick brown fox jumps over the lazy dog} 
и задать пробел в качестве разделителя.

\lstinputlisting{src/strtok_ex1.c}

Мы в итоге получим на выходе:

\begin{lstlisting}
The
quick
brown
fox
jumps
over
the
lazy
dog
\end{lstlisting}

Что тут в реальности происходит, это то что ф-ция \IT{strtok()} просто находит в заданной строке следующий пробел 
(либо иной заданный разделитель),
записывает туда $0$ (что по соглашениям текстовых строк в Си является концом строки) и возвращает указатель на это
место.

В качестве недостатка \IT{strtok()} можно отметить, что эта ф-ция ``портит'' входную строку, записывая нули
на месте разделителей.

Но вот что важно заметить: никакие строки или подстроки не копируются в памяти. Входная
строка остается там же где и лежала. В \IT{strtok()} передается только указатель на нее, или, её адрес.
Эта ф-ция затем, после
того как записывает $0$, возвращает \IT{адрес} каждого следующего ``слова''.
Адрес затем подается на вход в \IT{printf()}, где происходит его вывод на консоль.

Обратите также внимание на то что в исходнике присутствует и некорректное объявление \IT{str}. Оно тем некорректное
что в Си строка имеет тип \IT{const char*}, то есть, распологается в константном сегменте данных, защищенным
от записи.
Если так сделать, то \IT{strtok()} не сможет модифицировать строку записывая туда нули и процесс ``упадет''.

Так что, в нашем примере, строка выделяется как массив \IT{char} а не массив \IT{const char}.

Обобщая, скажем что работа со строками в Си происходит только лишь используя адреса этих строк. К примеру,
ф-ция сравнения строк \IT{strcmp()} на вход берет два адреса двух строк и по одному символу сравнивает их.
Было бы очень абсурдно копировать куда-то эти две строки лишний раз, чтобы \IT{strcmp()} обработала их.

Трудность понимания указателей в Си связана с тем, что указатель это ``часть'' объекта. Указатель на строку,
это не сама строка. Сама строка еще должна где-то в памяти хранится, под нее нужно выделять место, итд.

В ЯП более высокого уровня, объект и указатель на него могут быть представлены как единое целое, что облегчает
понимание.
Впрочем, это не значит что в этих ЯП строки и иные объекты неразумно копируются много раз при передаче 
в другие функции,
там точно так же используются указатели, но просто эта механика скрыта от программиста.

\subsection{Синтаксический сахар для a[i]}

Ради упрощения, можно сказать что в Си нет массивов вообще, а есть только синтаксический сахар для выражений
вроде \IT{a[i]}.

К примеру, возможно вы видели такой трюк:

\begin{lstlisting}
printf ("%c", 3["hello"]);
\end{lstlisting}

Это выдаст 'l'. 
Это происходит, потому что любое выражение \IT{a[i]}, на самом деле преобразовывается в \IT{*(a+i)}
\cite[6.5.2.1]{C99TC3}.
\IT{3["hello"]} преобразовывается в \IT{*(3+"hello")}, а \IT{"hello"} это просто указатель на массив символов, 
типа \IT{const char*}.
\IT{3+"hello"} это в итоге указатель на часть строки, то есть, \IT{"lo"}. А \IT{*("lo")} это cимвол 'l'. 
Вот почему это работает.

Но так врядли стоит писать, если вы конечно не готовите программу на конкурс 
The International Obfuscated C Code Contest\footnote{\url{http://www.ioccc.org/}}.
Так что я привел этот пример, чтобы наглядно показать, 
что выражения вроде \IT{a[i]} это синтаксический сахар.

При некотором упорстве, в Си вообще можно обойтись без индексации массивов, хотя выглядеть это будет не очень
эстетично.

Кстати, так легко понять как работают отрицательные индексы массивов. \IT{a[-3]} просто преобразуется в \IT{*(a-3)},
так адресуется элемент лежащий перед самим массивом.
И хотя это вполне возможно, так можно делать только если вы точно знаете, что вы делаете.

В Си массив это, в каком-то смысле, это просто место в памяти под массив плюс указатель, указывающий
на него. 

Вот почему имя массива в Си можно считать за указатель:

Если вы объявите глобальную переменную \IT{int a[10]}, то \IT{a} будет иметь тип \IT{int*}.
Позже, когда где-то в коде
вы укажете \IT{x=a[5]}, выражение будет преобразовано в \IT{x=*(a+5)}. От начала массива (то есть, первого элемента
массива), будет отсчитано 5 элементов, затем оттуда прочитается элемент для записи в \IT{x}.

\subsection{\IFRU{Арифметика указателей}{Pointer arithmetic}}

Простой пример:

\lstinputlisting{src/phonebook1.c}

Мы объяляем глобальный массив из структур. Если скомпилировать это в GCC с ключом \IT{-S} либо в MSVC с ключом
\IT{/Fa}, мы увидим листинг на ассемблере и то, как компилятор расположил эти строки. 

Расположил он их как линейный массив указателей на строки, вот так:

\begin{center}
\begin{tabular}{ | l | l | }
\hline
  ячейка 0    & адрес строки ``Kirk'' \\
  ячейка 1    & адрес строки ``Hammett'' \\
  ячейка 2    & адрес строки ``555-1234'' \\
  ячейка 3    & адрес строки ``Lars'' \\
  ячейка 4    & адрес строки ``Ulrich'' \\
  ячейка 5    & адрес строки ``555-5678'' \\
  ячейка 6    & адрес строки ``James'' \\
  ячейка 7    & адрес строки ``Hetfield'' \\
  ячейка 8    & адрес строки ``555-1122'' \\
  ячейка 9    & адрес строки ``Robert'' \\
  ячейка 10   & адрес строки ``Trujillo'' \\
  ячейка 11   & адрес строки ``555-7788'' \\
  ячейка 12   & 0 \\
  ячейка 13   & 0 \\
  ячейка 14   & 0 \\
\hline
\end{tabular}
\end{center}

Ф-ции \IT{dump1()} и \IT{dump2()} эквивалентны.

Но в первой итератор \IT{i} начинается с 0 и к нему прибавляется 1 на каждой итерации.

Во второй ф-ции итератор \IT{i} указывает на начало массива и затем, к нему прибавляется длина структуры 
(а не 1 байт, как можно поначалу ошибочно подумать),
таким образом, на каждой итерации, \IT{i} указывает на следующий элемент массива.


\subsection{\IFRU{Операторы}{Operators}}

\subsubsection{==}

\IFRU{Очень неприятные ошибки возникают если в условии}
{Somewhat unpleasant mistakes may appear if in} \IT{if(a==3)} 
\IFRU{опечататься и написать}{condition become} \IT{if(a=3)}\IFRU{}{ in result of typo}.
\IFRU{Ведь выражение}{Because the statement} \IT{a=3} ``\IFRU{возвращает}{returns}'' 3,
\IFRU{а}{and} 3 \IFRU{это не}{is not a} 0, \IFRU{поэтому условие \IT{if()} всегда будет 
срабатывать}{so the \IT{if()} condition will always trigger}.

\IFRU{Раньше, для защиты от подобных ошибок, была мода писать наоборот}
{It was fashionable in past to protect from such mistakes by writing}: \IT{if(3==a)}, 
\IFRU{таким образом}{and thus},
\IFRU{если опечататься, выйдет}{we will get a} \IT{if(3=a)}\IFRU{, компилятор тут же выдаст ошибку}
{ in case of typo and the compiler will report error instantly}.

\IFRU{Тем не менее, в наше время, компиляторы обычно предупреждают если написать}
{Nevertheless, in modern times, compilers are usually warns if to write} \IT{if(a=3)}, 
\IFRU{так что, наверное, менять местами элементы выражения уже не обязательно}
{so elements swapping in conditions is probably not necessary these days}.

\subsubsection{Short-circuit evaluation 
\IFRU{и артефакт приоритетов операций}{and operator precedence artefact}}

\IFRU{Разберем что такое}{Let's see what is} \IT{short-circuit}
\IFRU{\footnote{дословный перевод на русский: ``короткое замыкание''}} \IT{evaluation}.

\IFRU{Это когда в выражении}{It is when in the expression} \IT{if(a \&\& b \&\& c)},
\IFRU{часть}{the part} \IT{(b)} \IFRU{будет вычисляться только если}{will be calculated only if} 
\IT{(a)} ~--- \IFRU{истинна}{is true},
\IFRU{а}{and} \IT{(c)}
\IFRU{будет вычисляться только если}{will be calculated only if} \IT{(a)} \AndENRU \IT{(b)} 
~--- \IFRU{оба истинны}{are both true}.
\IFRU{И вычисляться они будут именно в таком порядке, как указано}{and they will be computed
exactly in the same order as specified}.

\IFRU{Иногда можно встретить подобное}{Sometimes we can see expression like}: 
\IT{if (p!=NULL \&\& p->field==123)} ~--- \IFRU{и это совершенно правильно}{and this is completely
correct}.
\IFRU{Поле}{The field} \IT{field} \IFRU{в структуре, на которую указывает}{in the structure to which} 
\IT{(p)}\IFRU{, будет вычисляться только если указатель}{ points will be computed only if the pointer}
\IT{(p)} \IFRU{не равен}{not equals to} \IT{NULL}.

\IFRU{То же касается и операции}{The same story about} ``\IFRU{или}{or}'', 
\IFRU{если в выражении}{if in the expression} \IT{if (a || b || c)} \IFRU{подвыражение}{subexpression} 
\IT{(a)} \IFRU{будет ``истинно''}{will be ``true''},
\IFRU{остальные вычисляться не будут}{others will not be computed}.

\IFRU{Это может быть удобно для вызова нескольких ф-ций}{It is useful when one need to call several
functions}:
\IT{if (get\_flagsA() || get\_flagsB() || get\_flagsC())} ~--- 
\IFRU{если первая или вторая ф-ция вернет}{if first or second will return} \IT{true}, 
\IFRU{остальные даже не будут вызываться}{others will not be called at all}.

\IFRU{Эта особенность есть не только в Си/Си++}{This feature is not unique for C/C++}
\footnote{\IFRU{Здесь список}{Here is a list of} \ac{PL} \IFRU{где присутствует}{where}
\IT{short-circuit evaluation}\IFRU{}{ exist}\url{https://en.wikipedia.org/wiki/Short-circuit_evaluation}.
\IFRU{Кстати, хотя это и не про Си, но все же интересно}{It is not about C, but interesting nevertheless}:
\IFRU{в bash если писать}{if to write in bash} \IT{cmd1 \&\& cmd2 \&\& cmd3}, 
\IFRU{то каждая следующая команда будет исполняться только если предыдущая закончилась с успехом}
{then each next command will be executed only if the previous was executed with success}.
\IFRU{Это также}{It is also} \IT{short-circuit}.}.

\IFRU{Когда-то давно}{Some time ago}\cite{dmr:1995}, 
\IFRU{в языках B и BCPL (предтечи Си) не было операторов}{there was no operators} 
\IT{\&\&} \AndENRU \IT{||}\IFRU{}{ in B and BCPL (C precursors)}, 
\IFRU{но чтобы реализовать в них}{but in order to implement}
\IT{short-circuit evaluation}\IFRU{}{ in them}, 
\IFRU{приоритет операций}{the priority of the operators} \IT{\&} \AndENRU \IT{|} 
\IFRU{сделали больше, чем, например, у}{was made higher than in} \IT{\^} \OrENRU \IT{+}
\footnote{\IFRU{Приоритет операций в Си++}{C++ Operator Precedence}: \url{http://en.cppreference.com/w/cpp/language/operator_precedence}}.

\IFRU{Это позволяло писать что-то вроде}{That allowed to write something like} \IT{if (a==1 \& b==c)} 
\IFRU{используя}{while using} \IT{\&} \IFRU{вместо}{instead of} \IT{\&\&}.
\IFRU{Вот откуда взялся этот артефакт в приоритетах}{That is where that artefact came from}. \\
\\
\IFRU{Так что, нередкая ошибка это забывать о высоком приоритете этих операций и писать, например}
{So one often mistake is to forget about higher priority of these operators and to write e.g.},
\IT{if (a\&1==0)}, \IFRU{в то время как это нужно брать в скобки}{which should be taken
in brackets}: \IT{if ((a\&1)==0)}.

\subsubsection{! \AndENRU \~{}}

\~{} (\IFRU{тильда}{tilde}) \IFRU{это побитовое инвертирование всех бит в значении}
{is a bitwise inversion of all bits in a value}.

\IFRU{Эта операция часто используется для инвертирования результатов действия ф-ций}
{The operation is often used for function results invertion}.
\IFRU{Например}{For example}, strcmp() \IFRU{в случае равенства строк возвращает}{in case of strings
equivalence, returns} 0.
\IFRU{Поэтому можно писать}{So we can write}:

\begin{lstlisting}
if (!strcmp(str1, str2))
{
	// do something in case of strings equivalence
};
\end{lstlisting}

... \IFRU{вместо}{instead of} \TT{if (strcmp (...)==0)}. \\
\\
\IFRU{Также, два подряд восклицательных знака применяется для трасформирования любого значения в тип bool
по правилу}{Also, two consecutive exclamation points can be used for
transforming any value into \IT{bool} type}: 0 ~--- false (0); \IFRU{не ноль}{not zero} ~--- true (1).

\IFRU{Например}{For example}:

\begin{lstlisting}
bool some_object_present=!!struct->object;
\end{lstlisting}

\IFRU{Или}{Or}:

\begin{lstlisting}
#define FLAG 0x00001000
bool FLAG_present=!!(value & FLAG);
\end{lstlisting}

\IFRU{А также}{And also}:

\begin{lstlisting}
bool bit_7_set=!!(value & (1<<7));
\end{lstlisting}


\section{union}

union часто используется, когда в каком-то месте структуры можно хранить разные типы на выбор.
К примеру:

\begin{lstlisting}
union
{
	int i; // 4 bytes
	float f; // 4 bytes
	double d; // 8 bytes
} u;
\end{lstlisting}

Такой union позволяет хранить одну из этих трех переменных на выбор. Занимать он будет места столько же,
сколько максимальный элемент (double) --- 8 байт.

union часто используют для обращения к какому-то типу данных как к другому.

Например, как известно, каждый XMM-регистр в SSE может представлять собой 16 байт, 8 16-битных слов,
4 32-битных слова, 2 64-битных слова, 4 float-значения и 2 double-значения. Так можно описать его:

\begin{lstlisting}
union
{
	double d[2];
	float f[4];
	uint8_t b[16];
	uint16_t w[8];
	uint32_t i[4];
	uint64_t q[2];
} XMM_register;

union XMM_register reg1;

reg.u.d[0]=123.4567;
reg.u.d[1]=89.12345;

// here we can use reg.u.b[...]

\end{lstlisting}

Это также очень удобно использовать вместе со структурой, где поля имеют битовую гранулярность.
Это флаги x86-процессора:

\begin{lstlisting}
typedef struct _s_EFLAGS
{
    unsigned CF : 1;
    unsigned reserved1 : 1;
    unsigned PF : 1;
    unsigned reserved2 : 1;
    unsigned AF : 1;
    unsigned reserved3 : 1;
    unsigned ZF : 1;
    unsigned SF : 1;
    unsigned TF : 1;
    unsigned IF : 1;
    unsigned DF : 1;
    unsigned OF : 1;
    unsigned IOPL : 2;
    unsigned NT : 1;
    unsigned reserved4 : 1;
    unsigned RF : 1;
    unsigned VM : 1;
    unsigned AC : 1;
    unsigned VIF : 1;
    unsigned VIP : 1;
    unsigned ID : 1;
} s_EFLAGS;

typedef union _u_EFLAGS
{
    uint32_t flags;
    s_EFLAGS s;
} u_EFLAGS;
\end{lstlisting}

Можно таким образом загрузить флаги как 32-битное значение в поле flags, а затем из поля s обращаться
к отдельным битам. Либо наоборот, модифицировать биты, затем прочитать поле flags.

\subsection{tagged union}

Это union плюс флаг (tag), определяющий тип union. К примеру, если нам нужна какая-то переменная,
которая может быть как числом, так и числом с плавающей точкой, так и текстовой строкой (как переменные
в динамически-типизированных ЯП), то мы можем объявить такую структуру:

\begin{lstlisting}

enum var_type
{
	INT,
	DOUBLE,
	STRING
};

struct
{
	enum var_type tag; // 4 bytes
	union
	{
		int i; // 4 bytes
		double d; // 8 bytes
		char *string; // 4 bytes (on 32-bit architecture)
	} u;
} variable;
\end{lstlisting}

Суммарная длина такой структуры будет $8+4=12$ байт. В любом случае, это компактнее, чем выделять
поля для переменной каждого возможного типа.



\label{C99}
\chapter{Стандарт Си C99}

Текст стандарта: \cite{C99TC3}.

Этот стандарт поддерживается в GCC, но не в MSVC, и не ясно, будет ли он там поддерживаться вообще.

Чтобы включить его поддержку в GCC, нужно добавить ключ компиляции \IT{-stc=c99}.
 % chapter

\section{\IFRU{Препроцессор}{Preprocessor}}

Препроцессор обрабатывает директивы начинающиеся с \# --- \#define, \#include, \#if, итд.

\section{Стандартные для компиляторов и ОС значения}

\begin{itemize}
\item \TT{\_DEBUG} --- отладочная сборка.
\item \TT{NDEBUG} --- неотладочная (release) сборка.
\item \TT{\_\_linux\_\_} --- ОС Linux.
\item \TT{\_WIN32} --- ОС Windows. Присутствует как и в x86-проектах, так и в x64.
\item \TT{\_WIN64} --- Присутствует в x64-проектах для ОС Windows.
\item \TT{\_\_cplusplus} --- присутствует в Си++ проектах.
\item \TT{\_MSC\_VER} --- компилятор MSVC.
\item \TT{\_\_GNUC\_\_} --- компилятор GCC.
\end{itemize}

Так можно писать разные участки кода для разных компиляторов и ОС.

\subsection{``Пустой'' макрос}

Всем известны макросы не объявляющие никаких значений, например \IT{\_DEBUG}.
Обычно, только проверяется наличие его или отсутствие.
Вот еще пример полезного ``пустого'' макроса:

В заголовочных файлах Windows API мы можем увидеть такое:

\begin{lstlisting}
typedef NTSTATUS
(NTAPI *TDI_REGISTER_CALLBACK)(
  IN PUNICODE_STRING DeviceName,
  OUT HANDLE *TdiHandle);

...

typedef NDIS_STATUS
(NTAPI *CM_CLOSE_CALL_HANDLER)(
  IN NDIS_HANDLE  CallMgrVcContext,
  IN NDIS_HANDLE  CallMgrPartyContext  OPTIONAL,
  IN PVOID  CloseData  OPTIONAL,
  IN UINT  Size  OPTIONAL);
\end{lstlisting}

IN, OUT и OPTIONAL --- это ``пустые'' макросы объявленные так:

\begin{lstlisting}
#ifndef IN
#define IN
#endif
#ifndef OUT
#define OUT
#endif
#ifndef OPTIONAL
#define OPTIONAL
#endif
\end{lstlisting}

Для компилятора они никакой информации не несут, они предназначены только для документирования, показать,
какие параметры ф-ций зачем нужны.

\subsection{Частые ошибки}

\subsubsection{\#1}

К примеру, вы хотите создать макрос для возведения числа в квадрат:

\begin{lstlisting}
#define square(x)      x*x
\end{lstlisting}

Это ошибка, потому что выражение \IT{square(a+b)} в итоге ``развернется'' в $a+b*a+b$, что, разумеется, совсем
не то что хотелось. Поэтому в определнии макроса все переменные, и сам макрос, нужно ``изолировать'' скобками:

\begin{lstlisting}
#define square(x)      ((x)*(x))
\end{lstlisting}

Пример из файла minmax.h из MinGW:

\begin{lstlisting}
#define max(a,b) (((a) > (b)) ? (a) : (b))
...
#define min(a,b) (((a) < (b)) ? (a) : (b))
\end{lstlisting}

\subsubsection{\#2}

Если вы где-то определяете какую-то константу:

\begin{lstlisting}
#define N 1234
\end{lstlisting}

... затем где-то дальше переопределяете её снова, то компилятор промолчит, и это приведет к трудновыявляемой
ошибке.

Поэтому константы желательнее определять как глобальные переменные с модификатором const.

 % chapter

\chapter{Строки в Си}

В Си нет встроенных возможностей для удобной работы со строками, такими, какие имеются в ЯП более
высокого уровня.

Часто жалуются на неудобную
конкатенацию строк (то есть, склеивание) в Си при помощи функции strcat(). Также, многих раздражает sprintf(),
под которых нельзя толком зараннее предсказать, сколько нужно выделять памяти. Копирование строк при помощи
strcpy() также неудобно --- нужно думать, сколько же выделить байт под буфер. Помимо всего прочего, неудобная
работа со строками в Си, это источник огромного количества уязвимостей в ПО, связанных с переполнениями буфера\cite[1.14.2]{REBook}.

Прежде всего, нужно задать себе вопрос, какие операции со строками нам нужны.
Конкатенация (склеивание) нужна чтобы 1) выдавать в лог сообщения; 2) конструировать строки и записывать их куда-то.

Для 1) можно использовать потоки (streams) --- не конструируя строку, выдавать её по порциям, например:

\begin{lstlisting}
printf ("Date: ");
dump_date(stdout, date);
printf (" a=");
dump_a(stdout, a);
printf ("\n");
\end{lstlisting}

Подобное заменяется в Си++ выводом в ostream:

\begin{lstlisting}
cout << "Date: " << Date_ToString(date) << " a=" << a_ToString(a) << "\n";
\end{lstlisting}

Так быстрее и меньше требуется памяти для конструирования строк.

Кстати, ошибкой является писать так:

\begin{lstlisting}
cout << "Date: " + Date_ToString(date) + " a=" + a_ToString(a) + "\n";
\end{lstlisting}

Для неспешного вывода в лог небольшого кол-ва сообщений это нормально, но если таких строк очень много, то будут
накладные расходы на их конкатенацию. \\
\\
Но все же строки иногда конструировать надо.

Есть какие-то библиотеки для этого.
К примеру, в Glib\footnote{\url{https://developer.gnome.org/glib/}} есть 
gstring.h\footnote{\url{https://github.com/GNOME/glib/blob/master/glib/gstring.h}}/
gstring.c\footnote{\url{https://github.com/GNOME/glib/blob/master/glib/gstring.c}}. 

\label{strbuf}
А в исходниках git можно найти strbuf.h\footnote{\url{https://github.com/git/git/blob/master/strbuf.h}}/
strbuf.c\footnote{\url{https://github.com/git/git/blob/master/strbuf.c}}. Собственно,
подобные Си-библиотеки очень похожи: они обеспечивают структуру данных, в которой есть некоторый буфер для строки, текущий размер буфера
и текущий размер строки в буфере. При помощи отдельных функций, можно добавлять новые строки или символы
в буфер, который, в свою очередь, будет автоматически увеличиваться или даже уменьшаться.

В \IT{strbuf.c} из git есть даже ф-ция \IT{strbuf\_addf()}, работающая как \IT{sprintf()}, 
но добавляющая строку-результат в буфер.

Так пользователь освобождается от головной боли связанной с выделением памяти.
При работе с этими библиотеками, практически невозможна ситуация переполнения буфера, если только не начать
работать со структурой самостоятельно.

Типичная последовательность работы с такими библиотеками, выглядит так:

\begin{itemize}
\item
Инициализация структуры strbuf или GString.

\item
Добавление строк и/или символов.

\item
Имеем сконструированную строку. Используем как обычную Си-строку, записываем куда-то в файл, передаем по сети, итд.

\item
Освобождаем структуру.
\end{itemize}

Кстати, конструирование строк чем-то напоминает 
Buffer\footnote{\url{http://docs.oracle.com/javase/7/docs/api/java/nio/Buffer.html}}, 
ByteBuffer\footnote{\url{http://docs.oracle.com/javase/7/docs/api/java/nio/ByteBuffer.html}} и 
CharBuffer\footnote{\url{http://docs.oracle.com/javase/7/docs/api/java/nio/CharBuffer.html}} в Java.

\section{Хранение длины строки}

Всегда хранить длину строки --- это было принято в реализациях ЯП Pascal. 
Не смотря на исходы святых войн\footnote{holy wars} между приверженцами Си и Pascal, все же, почти все библиотеки
для хранения строк и работы с ними, хранят также и текущую длину --- потому что удобства от этого перевешивают
необходимость пересчитывать это значение.

Например, \IT{strlen()} (подсчет длины строки) больше не нужен вообще, длина все время известна.
Конкатенация строк работает намного быстрее, потому что не нужно вычислять длину первой строки.
Ф-ция сравнения строк в самом начале может сравнить длины строк и если они не равны, тут же вернуть false,
не начиная сравнивание самих строк.

В Oracle RDBMS, в сетевых библиотеках, в функции работы со строками, зачастую передается строка и, 
отдельным аргументом, её длина\footnote{\url{http://blog.yurichev.com/node/64}}.
Это не очень эстетично, это выглядит избыточно, зато очень удобно.
Например, у нас есть некоторая ф-ция, которой нужно в начале узнать, какую строку ей передали:

\begin{lstlisting}
void f(char *color)
{
	if (strcmp (color, "red")==0)
		do_red();
	else if (strcmp (color, "green")==0)
		do_green();
	else if (strcmp (color, "blue")==0)
		do_blue();
	else if (strcmp (color, "orange")==0)
		do_orange();
	else if (strcmp (color, "yellow")==0)
		do_yellow();
	printf ("Unknown color!\n");
};
\end{lstlisting}

А вот если бы эта ф-ция имела длину входной строки, её можно было бы переписать так:

\begin{lstlisting}
void f(char *color, int color_len)
{
	switch (color_len)
	{
		case 3:
			if (strcmp (color, "red")==0)
				do_red();
			else 
				goto unknown_color;
			break;
		case 4:
			if (strcmp (color, "blue")==0)
				do_blue();
			else
				goto unknown_color;
			break;
		case 5:
			if (strcmp (color, "green")==0)
				do_green();
			else
				goto unknown_color;
			break;
		case 6:
			if (strcmp (color, "orange")==0)
				do_orange();
			else if (strcmp (color, "yellow")==0)
				do_yellow();
			else
				goto unknown_color;
			break;
		default:
				goto unknown_color;

	};

	return;

unknown_color:
	printf ("Unknown color!\n");
};
\end{lstlisting}

Конечно, с эстетической точки зрения, код выглядит ужасно.
Тем не менее, мы здорово сократили количество необходимых сравнений строк! Вероятно, для тех ситуаций, когда 
нужно как можно быстрее обрабатывать текстовые строки, такой подход может улучшить ситуацию.

\section{Возврат строки}

Если некая ф-ция должна вернуть строку, имеются такие возможности:

\begin{itemize}
\item
Возврат строки-константы, это самое простое и быстрое.

\item
Возврат строки через глобальный массив символов. Недостаток: массив один и каждый вызов ф-ции перезаписывает
его содержимое.

\item
Возврат строки через буфер, заданный в аргументах ф-ции. Недостаток: нужно также передавать и длину буфера.

\item
Выделяем буфер нужного размера сами, записываем туда строку, возвращаем указатель. Недостаток: тратятся ресурсы
на выделение памяти.

\item
Записываем строку в уже рассмотренный strbuf или GString или иную другую структуру, указатель на которую был
передан в аргументах.

\end{itemize}

Первый вариант очень прост. Например:

\begin{lstlisting}
const char* get_month_name (int month)
{
	switch (month)
	{
	case 1: return "January";
	case 2: return "February";
	case 3: return "March";
	case 4: return "April";
	case 5: return "May";
	case 6: return "June";
	case 7: return "July";
	case 8: return "August";
	case 9: return "September";
	case 10: return "October";
	case 11: return "November";
	case 12: return "December";
	default: return "Unknown month!";
	};
};
\end{lstlisting}

Можно даже еще проще:

\begin{lstlisting}
const char* month_names[]={
	"January", "February", "March", "April", "May", "June", "July", "August",
	"September", "October", "November", "December" };

const char* get_month_name (int month)
{
	if (month>=1 && month<=12)
		return month_names[month-1];

	return "Unknown month!";
};
\end{lstlisting}

Насчет второго варианта, в git в hex.c\footnote{\url{https://github.com/git/git/blob/master/hex.c}} можно найти такое:

\begin{lstlisting}
char *sha1_to_hex(const unsigned char *sha1)
{
	static int bufno;
	static char hexbuffer[4][50];
	static const char hex[] = "0123456789abcdef";
	char *buffer = hexbuffer[3 & ++bufno], *buf = buffer;
	int i;

	for (i = 0; i < 20; i++) {
		unsigned int val = *sha1++;
		*buf++ = hex[val >> 4];
		*buf++ = hex[val & 0xf];
	}
	*buf = '\0';

	return buffer;
}
\end{lstlisting}

Строка возвращается фактически через глобальную переменную, объявление её как static внутри ф-ции просто напросто
обеспечивает доступ к ней только из этой ф-ции. Но вот недостаток: после вызова \IT{sha1\_to\_hex()} вы не можете
вызвать её повторно для получения второй строки до тех пор, пока не используете как-то первую, ведь она
затрется! Для того чтобы решить эту проблему здесь, по видимому, сделали сразу 4 буфера и каждый раз строка
возвращается в следующем. Но имейте ввиду --- так можно делать если только вы уверены в том что вы делаете,
это код на уровне ``грязного хака''.
Если вы
вызовете эту ф-цию 5 раз и вам нужно будет использовать как-то строку полученную при первом вызове, это может
привести к трудновыявляемой ошибке.

Кстати, обратите также внимание на то что переменная \IT{bufno} не инициализируется, потому что используются только 
2 младших её бита, к тому же, не важно, какое значение переменная будет содержать в самом начале!

\section{Определение строк}

Малоизвестная возможность Си, длинные строки можно определять так:

\begin{lstlisting}
const char* long_line="line 1"
	"line 2"
	"line 3"
	"line 4"
	"line 5";

...

printf ("Some Utility v0.1\n"
	"Usage: %s parameters\n"
	"\n"
	"Authors:...\n", argv[0]);
\end{lstlisting}

Это отдаленно напоминает ``here document''\footnote{\url{https://en.wikipedia.org/wiki/Here_document}} в 
UNIX-шеллах и Perl.

 % chapter

% \chapter{\IFRU{Ваши собственные структуры данных}{Your own data structures}}
% list, map

\chapter{\IFRU{Стандартные библиотеки Си/Си++}{C/C++ standard library}}

\section{assert}

Как известно, этот макрос часто используется для валидации
\footnote{используется также такой термин как ``инвариант'' и ``sanitization'' в англ.яз.} заданных значений. 
Например, если ваша ф-ция
работает с датой, вы, вероятно, захотите написать в её начале что-то вроде \IT{assert (month>=1 \&\& month<=12)}.

Вот то о чем нужно помнить: стандартный макрос assert() доступен только в отладочных (debug) сборках. В release
все выражения как бы исчезают. Поэтому писать, например, \IT{assert(f=malloc(...))} неверно. Впрочем,
вы возможно захотите использовать что-то вроде \IT{assert(object->get\_something()==123)}.

В макросах assert можно также указывать небольшие сообщения об ошибках: 
вы увидите их если assert() ``не сойдется''. 
Например, в исходниках LLVM\footnote{\url{http://llvm.org/}} можно встретить такое:

\begin{lstlisting}
assert(Index < Length && "Invalid index!");
...
assert(i + Count <= M && "Invalid source range");
...
assert(j + Count <= N && "Invalid dest range");
\end{lstlisting}

Текстовая строка имеет тип \IT{const char*}, и она никогда не NULL. 
Таким образом, можно дописать к любому выражению \IT{... \&\& true} не меняя его смысл.

\section{Разница между stdout и stderr}

\IT{stdout} это то что выводится на консоль при помощи вызова \IT{printf()}.
\IT{stdout} это буферизированный вывод,
так что, пользователь, обычно того не зная, видит вывод порциями. Бывает так что программа выдает
что-то используя \IT{printf()} либо \IT{cout} и тут же падает.
Если это попадает в буфер, но буфер не успевает
``сброситься'' (flush) в консоль, то пользователь ничего не увидит. Это бывает неудобно.
Таким образом, для вывода более важной информации, в том числе отладочной, удобнее использовать \IT{stderr}.

\IT{stderr} это не буферизированный вывод, и всё что попадает в этот поток при помощи 
\TT{fprintf(stderr,...)} либо \IT{cerr}, появляется в консоли тут же.

Не следует также забывать, что из-за отсутствия буфера, вывод в \IT{stderr} медленнее.

Чтобы направлять \IT{stderr} в другой файл при запуске процесса, можно указывать:

\begin{lstlisting}
process 2> debug.txt
\end{lstlisting}

... это направит вывод \IT{stderr} в заданный файл (потому что номер этого потока -- 2).

\section{UNIX time}

В UNIX-среде очень популярно представление даты и времени в формате UNIX time.
Это просто 32-битное число, показывающее
количество прошедших секунд с 1-го января 1970-го года.

В качестве положительных сторон: 1) очень легко хранить это 32-битное число; 2) очень легко вычислять разницу дат;
3) невозможно закодировать неверные даты и время, такие как 32-е января, 29-е февраля невысокосных годов, 
25 часов 62 минуты.

В качестве отрицательных сторон: 1) нельзя закодировать дату до 1970-го года.

В наше время, если использовать UNIX time, тем не менее, следует помнить что ``срок его действия'' истечет
в 2038-м году, именно тогда 32-битное число переполнится, то есть, пройдет $2^{32}$ секунд с 1970-го года.
Так что, для этого следует использовать 64-битное значение, т.е., time64.

% ? NtQuerySystemTime http://msdn.microsoft.com/en-us/library/windows/desktop/ms724512(v=vs.85).aspx

\section{scanf(), fscanf(), sscanf()}

\subsection{Засада \#1}

Если использовать \%d в строке формата, scanf() подразумевает что это 32-битный int. 

Ошибкой является подобное:

\begin{lstlisting}
char a[10];

scanf ("%d %d %d %d", &a[0], &a[1], &a[2], &[3]);
\end{lstlisting}

Символы (или байты) лежат ``в притык'' друг к другу. Когда scanf() будет обрабатывать первое значение, он будет считать
его за 32-битный int, и ``затрет'' остальные три, рядом лежащие. И так далее.



\chapter*{\IFRU{Послесловие}{Afterword}}
\addcontentsline{toc}{chapter}{\IFRU{Послесловие}{Afterword}}

% \section{\IFRU{Краудфандинг}{Crowdfunding}}

\IFRU{Эта книга является свободной, находится в свободном доступе, и доступна в виде исходных кодов}
{This book is free, available freely and available in source code form}\footnote{\url{https://github.com/dennis714/RE-for-beginners}} (LaTeX), 
\IFRU{и всегда будет оставаться таковой}{and it will be so forever}.

\IFRU{В мои текущие планы насчет этой книги входит добавление информации на эти темы:}
{My current plans for this books is to add a lot of information about} C++11, flex/bison.

\IFRU{Если вы хотите чтобы я продолжал свою работу и писал на эти темы,
вы можете рассмотреть идею краудфандинга}
{If you want me to continue writing on all these topics, you may consider crowdfunding}.

\IFRU{Со способами краудфандинга можно ознакомиться на странице}
{Ways to crowdfund are available on the page:} \url{http://yurichev.com/crowdfunding.html}

%\subsection{\IFRU{Жертвователи}{Donors}}


\section{\IFRU{Вопросы?}{Questions?}}

\IFRU{Совершенно по любым вопросам, вы можете не раздумывая писать автору}
{Do not hesitate to mail any questions to the author}: \TT{<\EMAIL>}
 
\IFRU{Пожалуйста, присылайте мне информацию о замеченных ошибках 
(включая грамматические), итд.}
{Please, also do not hesitate to send me any corrections 
(including grammar ones (you see how horrible my English is?)), etc.}


\bibliographystyle{alpha}
\bibliography{books,articles,usenet,misc}

\clearpage
\printindex

\end{document}

