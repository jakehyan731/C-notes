\documentclass[11pt,a4paper,oneside]{book}

\usepackage{cmap}

\ifdefined\RUSSIAN
\usepackage[english,russian]{babel}
\usepackage[T2A]{fontenc}
\usepackage{paratype}
\renewcommand*\familydefault{\sfdefault}
% http://www.emerson.emory.edu/services/latex/latex_169.html
\newcommand{\lstlistingsize}{\scriptsize}
\else
\usepackage[russian,english]{babel}
\usepackage[T2A]{fontenc}
\usepackage[default]{sourcesanspro}
\newcommand{\lstlistingsize}{\footnotesize}
\fi

\usepackage[utf8]{inputenc}
\usepackage{listings}
\usepackage{ulem}
\usepackage{url}
\usepackage{graphicx}
\usepackage{listingsutf8}
\usepackage{makeidx}
\usepackage{cite}
\usepackage[cm]{fullpage}
\usepackage{color}
\usepackage{fancyvrb}
\usepackage{xspace}
\usepackage{framed}
\usepackage{ccicons}
\usepackage[nottoc]{tocbibind}
\usepackage{amsmath}
\usepackage[table]{xcolor}% http://ctan.org/pkg/xcolor
\usepackage[]{hyperref} % should be last
%\usepackage{tikz}

\definecolor{lstbgcolor}{rgb}{0.94,0.94,0.94}
\makeindex

\newcommand{\TT}[1]{\texttt{#1}}
\newcommand{\IT}[1]{\textit{#1}}
\newcommand{\IFRU}[2]{\iflanguage{russian}{#1}{#2}}


\newcommand{\TITLE}{\IFRU{Заметки о языке программирования Си/Си++}
{C/C++ programming language notes}}
\newcommand{\AUTHOR}{\IFRU{Денис Юричев}{Dennis Yurichev}}
\newcommand{\EMAIL}{dennis@yurichev.com}

\hypersetup{
    pdftex,
    colorlinks=true,
    allcolors=blue,
    pdfauthor={\AUTHOR},
    pdftitle={\TITLE}
    }

\selectlanguage{english}

\lstset{
    backgroundcolor=\color{lstbgcolor},
    basicstyle=\ttfamily\lstlistingsize, 
    breaklines=true,
    frame=single,
    inputencoding=cp1251,
    columns=fullflexible,keepspaces,
}

\begin{document}

\VerbatimFootnotes

\frontmatter

\begin{titlepage}
\begin{center}
\vspace*{\fill}
\LARGE \TITLE

\vspace*{\fill}

\large \AUTHOR

\large \TT{<\EMAIL>}
\vspace*{\fill}
\vfill

\ccbyncnd

\textcopyright 2013, \AUTHOR. 

\IFRU{Это произведение доступно по лицензии Creative Commons «Attribution-NonCommercial-NoDerivs» 
(«Атрибуция — Некоммерческое использование — Без производных произведений») 3.0 Непортированная. 
Чтобы увидеть копию этой лицензии, посетите}
{This work is licensed under the Creative Commons Attribution-NonCommercial-NoDerivs 3.0 Unported License. 
To view a copy of this license, visit} \url{http://creativecommons.org/licenses/by-nc-nd/3.0/}.

\IFRU{Версия этого текста}{Text version} ({\large \today}).

\IFRU{Возможно, более новая версии текста, а так же англоязычная версия, также доступна по ссылке}
{Probably, newer version of this text, and also Russian language version is also accessible at} 
%FIXME URL?
\url{http://yurichev.com/C-notes.html}

\IFRU{Вы также можете подписаться на мой twitter для получения информации о новых версиях этого текста, итд:
\href{https://twitter.com/yurichev_ru}{@yurichev\_ru}}
{You may also subscribe to my twitter, to get information about updates of this text, etc: 
\href{https://twitter.com/yurichev}{@yurichev}}
%FIXME mailing list
\end{center}
\end{titlepage}

\tableofcontents
\cleardoublepage

\begin{center}
\vspace*{\fill}

\IFRU{Эта страница сдается в аренду для рекламы}{This page can be rented for advertisement}.

\TT{<\EMAIL>}

\vspace*{\fill}
\vfill
\end{center}


\cleardoublepage

\chapter{\IFRU{Введение}{Preface}}

Сейчас, в 2013-м году, если некто желает написать 1) как можно более быстро работающую программу; 2) либо как
можно более компактную для встраиваемых систем либо маломощных микроконтроллеров, то выбор небольшой:
Си, Си++ либо ассемблер. И альтернативы этим старым но популярным языкам, в обозримом будущем, 
пока что не видно.

\chapter{\IFRU{Целевая аудитория}{Target audience}}

Этот сборник заметок предназначен не для начинающих, но и не для экспертов, а скорее для тех, 
кто хочет освежить свои знания по Си/Си++.

\chapter{\IFRU{Об авторе}{About author}}

\IFRU{Денис Юричев ~--- опытный программист, свободный для найма как программист, reverse engineer, консультант, тренер. 
С его резюме можно ознакомиться \href{http://yurichev.com/Dennis_Yurichev.pdf}{здесь}.}
{Dennis Yurichev is experienced programmer, available for hire as programmer, reverse engineer, consultant, trainer. 
His CV is available \href{http://yurichev.com/Dennis_Yurichev.pdf}{here}.}

\chapter{\IFRU{Благодарности}{Thanks}}

\IFRU{Андрей ''herm1t'' Баранович, Слава ''Avid'' Казаков}
{Andrey ''herm1t'' Baranovich, Slava ''Avid'' Kazakov}.

\mainmatter

\chapter{\IFRU{Элементы языка Си/Си++}{C/C++ language elements}}
\label{typedef}
\subsubsection{typedef}

\TT{typedef} \IFRU{вводит}{introduces} \IT{\IFRU{синоним}{synonym}} \IFRU{для типа данных}{for a data type}.
\IFRU{Часто это используется для структур, чтобы каждый раз не писать}
{It is often used for structures, for the reason not to write} \IT{struct}
\IFRU{перед её именем, например}{each time before its name, for example}:

\begin{lstlisting}
typedef struct _node
{
	node *prev;
	node *next;
	void *data;
} node;
\end{lstlisting}

\IFRU{Такого очень много в ``заголовочных'' файлах в}
{A lot of such examples may be found in header files in the} Windows SDK (Windows API).

\IFRU{Тем не менее}{Nevertheless}, \IT{typedef} 
\IFRU{также можно использовать не только для структур, но и для обычных, 
интегральных}{can be also used not only for structures, but also for usual integral}
\footnote{\IFRU{приводимых к числу}{which may be converted to a number}}, \IFRU{типов, например}{types like}:

\begin{lstlisting}
typedef int age;
int compute_mean (age wife, age husband);

typedef int coord;
void draw (coord X, coord Y, coord Z);

typedef uint32_t address;
void write_memory (address a, size_t size, byte *buf);
\end{lstlisting}

\IFRU{Как видно}{As we can see}, \IT{typedef} 
\IFRU{здесь может помочь в документировании исходного кода, так он легче читается}{here may help with
code documentation, it is now easier to read}.

\IFRU{Например, тип}{For example, the} \IT{time\_t} 
(\IFRU{время в формате}{The time in the} UNIX time
\IFRU{, то что возвращает стандартная функция}{ format, for example, what the}
localtime()
\IFRU{, например}{returns}), \IFRU{это на самом деле
обычное 32-битное число, но этот тип определен в}{it is in fact 32-bit number, but the type is defined in the}
\IT{time.h} \IFRU{обычно так}{file usually as}:

\begin{lstlisting}
typedef long __time32_t;
\end{lstlisting}

\IFRU{Здесь вполне можно было бы использовать директиву препроцессора}
{A preprocessor directive} \TT{\#define} \IFRU{(многие так и делают),
но это хуже с точки зрения обработки ошибок во время компиляции}{may be used here (many do so),
but it is worse in the sense of errors handling during compilation}.

\paragraph{\IFRU{Критика }{}typedef\IFRU{}{ criticism}}

\IFRU{Тем не менее, такой известный и опытный программист как Линус Торвальдс,
против использования typedef}{Nevertheless, such well-known and experiences programmer as Linus
Torvalds is against typedef usage}:
\cite{Torvalds:2002}.


\subsection{goto}

\index{goto}
\IFRU{Использование}{Usage of} \IT{goto}\footnote{statement} 
\IFRU{считается плохим тоном и вредным вообще}{is considered as bad taste and harmful}
\cite{Dijkstra:1968:LEG:362929.362947}\cite{Dijkstra:1979:GSC:1241515.1241518}, 
\IFRU{тем не менее, использование его в разумных дозах}{nevertheless, its usage in reasonable
doses}\cite{Knuth:1974:SPG:356635.356640} \IFRU{может облегчить жизнь}{may be very helpful}.

\IFRU{Частый пример, это выход из функции}{One frequent example is return from a function}:

\begin{lstlisting}
void f(...)
{
	byte* buf1=malloc(...);
	byte* buf2=malloc(...);

	...

	if (something_goes_wrong_1)
		goto cleanup_and_exit;

	...
	
	if (something_goes_wrong_2)
		goto cleanup_and_exit;

	...

cleanup_and_exit:
	free(buf1);
	free(buf2);
	return;
};
\end{lstlisting}

\IFRU{Более сложный пример}{More complex example}:

\begin{lstlisting}
void f(...)
{
	byte* buf1=malloc(...);
	byte* buf2=malloc(...);

	FILE* f=fopen(...);
	if (f==NULL)
		goto cleanup_and_exit;

	...

	if (something_goes_wrong_1)
		goto close_file_cleanup_and_exit;

	...
	
	if (something_goes_wrong_2)
		goto close_file_cleanup_and_exit;

	...

close_file_cleanup_and_exit:
	fclose(f);

cleanup_and_exit:
	free(buf1);
	free(buf2);
	return;
};
\end{lstlisting}

\IFRU{Если в данных примерах отказаться от}{If to remove all} \IT{goto}\IFRU{, то придется вызывать}
{ in these examples, one will need to call} \IT{free()} \AndENRU \IT{fclose()}
\IFRU{перед каждым выходом из функции}{before each return from the function} 
(\IT{return})\IFRU{, что здорово замусорит весь код}{which adds a lot of mess}.

\index{Linux}
\IFRU{Использование}{Usage of} \IT{goto} 
\IFRU{в таких случаях одобряется, например, в}{is, for example, approved in} \cite{LinuxKernelCodingStyle}.

%Примеры более ``harmful'' но эффективного использования goto можно найти в исходниках nginx.
% example?


\subsection{sizeof}

\IFRU{Обычно}{Usually}, sizeof() \IFRU{применяют к \glslink{integral type}{интегральным типам}}
{is applied to \glslink{integral type}{integral types}}
\IFRU{либо к структурам}{or to structures},
\IFRU{тем не менее, его можно применять и к массивам, к примеру}
{but nevertheless it is possible to apply it to arrays as well}:

\begin{lstlisting}
	char buf[1024];
	snprintf(buf, sizeof(buf), "...");
\end{lstlisting}

\IFRU{В противном случае, если указывать длину массива}{Otherwise, if to specify array length} ($1024$) 
\IFRU{в двух местах}{in both places} 
(\IFRU{в определении \TT{buf} и как второй аргумент \TT{snprintf()}}
{in \TT{buf} declaration and as a second argument of \TT{snprintf()}}),
\IFRU{то и изменять это значение придется каждый раз в обоих местах, а об этом легко забыть}
{then the value is have to be changed at the both places each time, and it is easy to forget about this}.

\IFRU{Если нужны wide-строки, то}{If one need wide-strings, then} sizeof() 
\IFRU{можно применять к}{can be applied to} \IT{wchar\_t} 
(\IFRU{который, на самом деле, 16-битный тип данных \IT{short}}
{which is in turn, 16-bit data type \IT{short}}):

\begin{lstlisting}
	wchar_t buf[1024];
	swprintf(buf, sizeof(buf)/sizeof(wchar_t), "...");
\end{lstlisting}

sizeof() \IFRU{возвращает длину в байтах, так что здесь он будет равен}{returns the size in bytes, so it will
be here} $1024*2$, \IFRU{т.е.}{i.e.}, $2048$. 
\IFRU{Но мы можем разделить это значение на длину одного элемента массива}
{But we can divide this value by length of one array element} (\IT{wchar\_t})
\IFRU{в байтах ($2$)}{is $2$ in bytes},
\IFRU{чтобы получить количество элементов в массиве}{in order to get elements number in array} ($1024$).

sizeof() \IFRU{можно применять и к массивам структур}{can be applied to array of structures}:

\begin{lstlisting}
struct phonebook_entry
{
	char *name;
	char *surname;
	char *tel;
};

struct phonebook_entry phonebook[]=
{
	{ "Kirk", "Hammett", "555-1234" },
	{ "Lars", "Ulrich", "555-5678" },
	{ "James", "Hetfield", "555-1122" },
	{ "Robert", "Trujillo", "555-7788" }
};

void dump (struct phonebook_entry* input)
{
	for (int i=0; i<sizeof(phonebook)/sizeof(struct phonebook_entry); i++)
		printf ("%s %s - %s\n", input[i].name, input[i].surname, input[i].tel);
};
\end{lstlisting}

sizeof(phonebook) ~--- \IFRU{это размер всего массива структур в байтах}{is a size of the whole array
of structures in bytes}.
\TT{sizeof(struct phonebook\_entry)} ~--- \IFRU{это размер одной структуры в байтах}{is a size of one structure
in bytes}.
\IFRU{Делением мы узнаем количество структур в массиве}{By division we get number of structures in an array}.


\section{\IFRU{Указатели}{Pointers}}

Как однажды сказал Дональд Кнут в интервью\cite{KnuthInterview1993}, то как в Си устроены указатели, это является
очень удачной инновацией в языках программирования по тем временам.

Итак, определимся с терминологией. Указатель это просто адрес какого-то элемента в памяти. Указатели настолько
популярны, потому что в какую-то функцию намного проще передать просто адрес объекта в памяти, вместо того
чтобы передавать весь объект --- ведь это абсурдно. К тому же, вызываемая функция, например, обрабатывающая
ваш массив данных, просто изменит что-то в нем, вместо того чтобы возвращать новый, измененный массив данных, 
что тоже абсурдно.

Возьмем простой пример. Стандартная функция \IT{strtok()} делит строку на подстроки, используя заданный символ
как разделитель. К примеру, мы можем подать на вход строку \TT{The quick brown fox jumps over the lazy dog} 
и задать пробел в качестве разделителя.

\lstinputlisting{src/strtok_ex1.c}

Мы в итоге получим на выходе:

\begin{lstlisting}
The
quick
brown
fox
jumps
over
the
lazy
dog
\end{lstlisting}

Что тут в реальности происходит, это то что ф-ция \IT{strtok()} просто находит в заданной строке следующий пробел 
(либо иной заданный разделитель),
записывает туда $0$ (что по соглашениям текстовых строк в Си является концом строки) и возвращает указатель на это
место.

В качестве недостатка \IT{strtok()} можно отметить, что эта ф-ция ``портит'' входную строку, записывая нули
на месте разделителей.

Но вот что важно заметить: никакие строки или подстроки не копируются в памяти. Входная
строка остается там же где и лежала. В \IT{strtok()} передается только указатель на нее, или, её адрес.
Эта ф-ция затем, после
того как записывает $0$, возвращает \IT{адрес} каждого следующего ``слова''.
Адрес затем подается на вход в \IT{printf()}, где происходит его вывод на консоль.

Обратите также внимание на то что в исходнике присутствует и некорректное объявление \IT{str}. Оно тем некорректное
что в Си строка имеет тип \IT{const char*}, то есть, распологается в константном сегменте данных, защищенным
от записи.
Если так сделать, то \IT{strtok()} не сможет модифицировать строку записывая туда нули и процесс ``упадет''.

Так что, в нашем примере, строка выделяется как массив \IT{char} а не массив \IT{const char}.

Обобщая, скажем что работа со строками в Си происходит только лишь используя адреса этих строк. К примеру,
ф-ция сравнения строк \IT{strcmp()} на вход берет два адреса двух строк и по одному символу сравнивает их.
Было бы очень абсурдно копировать куда-то эти две строки лишний раз, чтобы \IT{strcmp()} обработала их.

Трудность понимания указателей в Си связана с тем, что указатель это ``часть'' объекта. Указатель на строку,
это не сама строка. Сама строка еще должна где-то в памяти хранится, под нее нужно выделять место, итд.

В ЯП более высокого уровня, объект и указатель на него могут быть представлены как единое целое, что облегчает
понимание.
Впрочем, это не значит что в этих ЯП строки и иные объекты неразумно копируются много раз при передаче 
в другие функции,
там точно так же используются указатели, но просто эта механика скрыта от программиста.

\subsection{Синтаксический сахар для a[i]}

Ради упрощения, можно сказать что в Си нет массивов вообще, а есть только синтаксический сахар для выражений
вроде \IT{a[i]}.

К примеру, возможно вы видели такой трюк:

\begin{lstlisting}
printf ("%c", 3["hello"]);
\end{lstlisting}

Это выдаст 'l'. 
Это происходит, потому что любое выражение \IT{a[i]}, на самом деле преобразовывается в \IT{*(a+i)}
\cite[6.5.2.1]{C99TC3}.
\IT{3["hello"]} преобразовывается в \IT{*(3+"hello")}, а \IT{"hello"} это просто указатель на массив символов, 
типа \IT{const char*}.
\IT{3+"hello"} это в итоге указатель на часть строки, то есть, \IT{"lo"}. А \IT{*("lo")} это cимвол 'l'. 
Вот почему это работает.

Но так врядли стоит писать, если вы конечно не готовите программу на конкурс 
The International Obfuscated C Code Contest\footnote{\url{http://www.ioccc.org/}}.
Так что я привел этот пример, чтобы наглядно показать, 
что выражения вроде \IT{a[i]} это синтаксический сахар.

При некотором упорстве, в Си вообще можно обойтись без индексации массивов, хотя выглядеть это будет не очень
эстетично.

Кстати, так легко понять как работают отрицательные индексы массивов. \IT{a[-3]} просто преобразуется в \IT{*(a-3)},
так адресуется элемент лежащий перед самим массивом.
И хотя это вполне возможно, так можно делать только если вы точно знаете, что вы делаете.

В Си массив это, в каком-то смысле, это просто место в памяти под массив плюс указатель, указывающий
на него. 

Вот почему имя массива в Си можно считать за указатель:

Если вы объявите глобальную переменную \IT{int a[10]}, то \IT{a} будет иметь тип \IT{int*}.
Позже, когда где-то в коде
вы укажете \IT{x=a[5]}, выражение будет преобразовано в \IT{x=*(a+5)}. От начала массива (то есть, первого элемента
массива), будет отсчитано 5 элементов, затем оттуда прочитается элемент для записи в \IT{x}.

\subsection{\IFRU{Арифметика указателей}{Pointer arithmetic}}

Простой пример:

\lstinputlisting{src/phonebook1.c}

Мы объяляем глобальный массив из структур. Если скомпилировать это в GCC с ключом \IT{-S} либо в MSVC с ключом
\IT{/Fa}, мы увидим листинг на ассемблере и то, как компилятор расположил эти строки. 

Расположил он их как линейный массив указателей на строки, вот так:

\begin{center}
\begin{tabular}{ | l | l | }
\hline
  ячейка 0    & адрес строки ``Kirk'' \\
  ячейка 1    & адрес строки ``Hammett'' \\
  ячейка 2    & адрес строки ``555-1234'' \\
  ячейка 3    & адрес строки ``Lars'' \\
  ячейка 4    & адрес строки ``Ulrich'' \\
  ячейка 5    & адрес строки ``555-5678'' \\
  ячейка 6    & адрес строки ``James'' \\
  ячейка 7    & адрес строки ``Hetfield'' \\
  ячейка 8    & адрес строки ``555-1122'' \\
  ячейка 9    & адрес строки ``Robert'' \\
  ячейка 10   & адрес строки ``Trujillo'' \\
  ячейка 11   & адрес строки ``555-7788'' \\
  ячейка 12   & 0 \\
  ячейка 13   & 0 \\
  ячейка 14   & 0 \\
\hline
\end{tabular}
\end{center}

Ф-ции \IT{dump1()} и \IT{dump2()} эквивалентны.

Но в первой итератор \IT{i} начинается с 0 и к нему прибавляется 1 на каждой итерации.

Во второй ф-ции итератор \IT{i} указывает на начало массива и затем, к нему прибавляется длина структуры 
(а не 1 байт, как можно поначалу ошибочно подумать),
таким образом, на каждой итерации, \IT{i} указывает на следующий элемент массива.



\subsection{\IFRU{Операторы}{Operators}}

\subsubsection{==}

\IFRU{Очень неприятные ошибки возникают если в условии}
{Somewhat unpleasant mistakes may appear if in} \IT{if(a==3)} 
\IFRU{опечататься и написать}{condition become} \IT{if(a=3)}\IFRU{}{ in result of typo}.
\IFRU{Ведь выражение}{Because the statement} \IT{a=3} ``\IFRU{возвращает}{returns}'' 3,
\IFRU{а}{and} 3 \IFRU{это не}{is not a} 0, \IFRU{поэтому условие \IT{if()} всегда будет 
срабатывать}{so the \IT{if()} condition will always trigger}.

\IFRU{Раньше, для защиты от подобных ошибок, была мода писать наоборот}
{It was fashionable in past to protect from such mistakes by writing}: \IT{if(3==a)}, 
\IFRU{таким образом}{and thus},
\IFRU{если опечататься, выйдет}{we will get a} \IT{if(3=a)}\IFRU{, компилятор тут же выдаст ошибку}
{ in case of typo and the compiler will report error instantly}.

\IFRU{Тем не менее, в наше время, компиляторы обычно предупреждают если написать}
{Nevertheless, in modern times, compilers are usually warns if to write} \IT{if(a=3)}, 
\IFRU{так что, наверное, менять местами элементы выражения уже не обязательно}
{so elements swapping in conditions is probably not necessary these days}.

\subsubsection{Short-circuit evaluation 
\IFRU{и артефакт приоритетов операций}{and operator precedence artefact}}

\IFRU{Разберем что такое}{Let's see what is} \IT{short-circuit}
\IFRU{\footnote{дословный перевод на русский: ``короткое замыкание''}} \IT{evaluation}.

\IFRU{Это когда в выражении}{It is when in the expression} \IT{if(a \&\& b \&\& c)},
\IFRU{часть}{the part} \IT{(b)} \IFRU{будет вычисляться только если}{will be calculated only if} 
\IT{(a)} ~--- \IFRU{истинна}{is true},
\IFRU{а}{and} \IT{(c)}
\IFRU{будет вычисляться только если}{will be calculated only if} \IT{(a)} \AndENRU \IT{(b)} 
~--- \IFRU{оба истинны}{are both true}.
\IFRU{И вычисляться они будут именно в таком порядке, как указано}{and they will be computed
exactly in the same order as specified}.

\IFRU{Иногда можно встретить подобное}{Sometimes we can see expression like}: 
\IT{if (p!=NULL \&\& p->field==123)} ~--- \IFRU{и это совершенно правильно}{and this is completely
correct}.
\IFRU{Поле}{The field} \IT{field} \IFRU{в структуре, на которую указывает}{in the structure to which} 
\IT{(p)}\IFRU{, будет вычисляться только если указатель}{ points will be computed only if the pointer}
\IT{(p)} \IFRU{не равен}{not equals to} \IT{NULL}.

\IFRU{То же касается и операции}{The same story about} ``\IFRU{или}{or}'', 
\IFRU{если в выражении}{if in the expression} \IT{if (a || b || c)} \IFRU{подвыражение}{subexpression} 
\IT{(a)} \IFRU{будет ``истинно''}{will be ``true''},
\IFRU{остальные вычисляться не будут}{others will not be computed}.

\IFRU{Это может быть удобно для вызова нескольких ф-ций}{It is useful when one need to call several
functions}:
\IT{if (get\_flagsA() || get\_flagsB() || get\_flagsC())} ~--- 
\IFRU{если первая или вторая ф-ция вернет}{if first or second will return} \IT{true}, 
\IFRU{остальные даже не будут вызываться}{others will not be called at all}.

\IFRU{Эта особенность есть не только в Си/Си++}{This feature is not unique for C/C++}
\footnote{\IFRU{Здесь список}{Here is a list of} \ac{PL} \IFRU{где присутствует}{where}
\IT{short-circuit evaluation}\IFRU{}{ exist}\url{https://en.wikipedia.org/wiki/Short-circuit_evaluation}.
\IFRU{Кстати, хотя это и не про Си, но все же интересно}{It is not about C, but interesting nevertheless}:
\IFRU{в bash если писать}{if to write in bash} \IT{cmd1 \&\& cmd2 \&\& cmd3}, 
\IFRU{то каждая следующая команда будет исполняться только если предыдущая закончилась с успехом}
{then each next command will be executed only if the previous was executed with success}.
\IFRU{Это также}{It is also} \IT{short-circuit}.}.

\IFRU{Когда-то давно}{Some time ago}\cite{dmr:1995}, 
\IFRU{в языках B и BCPL (предтечи Си) не было операторов}{there was no operators} 
\IT{\&\&} \AndENRU \IT{||}\IFRU{}{ in B and BCPL (C precursors)}, 
\IFRU{но чтобы реализовать в них}{but in order to implement}
\IT{short-circuit evaluation}\IFRU{}{ in them}, 
\IFRU{приоритет операций}{the priority of the operators} \IT{\&} \AndENRU \IT{|} 
\IFRU{сделали больше, чем, например, у}{was made higher than in} \IT{\^} \OrENRU \IT{+}
\footnote{\IFRU{Приоритет операций в Си++}{C++ Operator Precedence}: \url{http://en.cppreference.com/w/cpp/language/operator_precedence}}.

\IFRU{Это позволяло писать что-то вроде}{That allowed to write something like} \IT{if (a==1 \& b==c)} 
\IFRU{используя}{while using} \IT{\&} \IFRU{вместо}{instead of} \IT{\&\&}.
\IFRU{Вот откуда взялся этот артефакт в приоритетах}{That is where that artefact came from}. \\
\\
\IFRU{Так что, нередкая ошибка это забывать о высоком приоритете этих операций и писать, например}
{So one often mistake is to forget about higher priority of these operators and to write e.g.},
\IT{if (a\&1==0)}, \IFRU{в то время как это нужно брать в скобки}{which should be taken
in brackets}: \IT{if ((a\&1)==0)}.

\subsubsection{! \AndENRU \~{}}

\~{} (\IFRU{тильда}{tilde}) \IFRU{это побитовое инвертирование всех бит в значении}
{is a bitwise inversion of all bits in a value}.

\IFRU{Эта операция часто используется для инвертирования результатов действия ф-ций}
{The operation is often used for function results invertion}.
\IFRU{Например}{For example}, strcmp() \IFRU{в случае равенства строк возвращает}{in case of strings
equivalence, returns} 0.
\IFRU{Поэтому можно писать}{So we can write}:

\begin{lstlisting}
if (!strcmp(str1, str2))
{
	// do something in case of strings equivalence
};
\end{lstlisting}

... \IFRU{вместо}{instead of} \TT{if (strcmp (...)==0)}. \\
\\
\IFRU{Также, два подряд восклицательных знака применяется для трасформирования любого значения в тип bool
по правилу}{Also, two consecutive exclamation points can be used for
transforming any value into \IT{bool} type}: 0 ~--- false (0); \IFRU{не ноль}{not zero} ~--- true (1).

\IFRU{Например}{For example}:

\begin{lstlisting}
bool some_object_present=!!struct->object;
\end{lstlisting}

\IFRU{Или}{Or}:

\begin{lstlisting}
#define FLAG 0x00001000
bool FLAG_present=!!(value & FLAG);
\end{lstlisting}

\IFRU{А также}{And also}:

\begin{lstlisting}
bool bit_7_set=!!(value & (1<<7));
\end{lstlisting}



\chapter{\IFRU{Ваши собственные структуры данных}{Your own data structures}}

\chapter{\IFRU{Стандартные библиотеки Си/Си++}{C/C++ standard library}}

\section{assert}

Как известно, этот макрос часто используется для валидации
\footnote{используется также такой термин как ``инвариант'' и ``sanitization'' в англ.яз.} заданных значений. 
Например, если ваша ф-ция
работает с датой, вы, вероятно, захотите написать в её начале что-то вроде \IT{assert (month>=1 \&\& month<=12)}.

Вот то о чем нужно помнить: стандартный макрос assert() доступен только в отладочных (debug) сборках. В release
все выражения как бы исчезают. Поэтому писать, например, \IT{assert(f=malloc(...))} неверно. Впрочем,
вы возможно захотите использовать что-то вроде \IT{assert(object->get\_something()==123)}.

В макросах assert можно также указывать небольшие сообщения об ошибках: 
вы увидите их если assert() ``не сойдется''. 
Например, в исходниках LLVM\footnote{\url{http://llvm.org/}} можно встретить такое:

\begin{lstlisting}
assert(Index < Length && "Invalid index!");
...
assert(i + Count <= M && "Invalid source range");
...
assert(j + Count <= N && "Invalid dest range");
\end{lstlisting}

Текстовая строка имеет тип \IT{const char*}, и она никогда не NULL. 
Таким образом, можно дописать к любому выражению \IT{... \&\& true} не меняя его смысл.

\section{Разница между stdout и stderr}

\IT{stdout} это то что выводится на консоль при помощи вызова \IT{printf()}.
\IT{stdout} это буферизированный вывод,
так что, пользователь, обычно того не зная, видит вывод порциями. Бывает так что программа выдает
что-то используя \IT{printf()} либо \IT{cout} и тут же падает.
Если это попадает в буфер, но буфер не успевает
``сброситься'' (flush) в консоль, то пользователь ничего не увидит. Это бывает неудобно.
Таким образом, для вывода более важной информации, в том числе отладочной, удобнее использовать \IT{stderr}.

\IT{stderr} это не буферизированный вывод, и всё что попадает в этот поток при помощи 
\TT{fprintf(stderr,...)} либо \IT{cerr}, появляется в консоли тут же.

Не следует также забывать, что из-за отсутствия буфера, вывод в \IT{stderr} медленнее.

Чтобы направлять \IT{stderr} в другой файл при запуске процесса, можно указывать:

\begin{lstlisting}
process 2> debug.txt
\end{lstlisting}

... это направит вывод \IT{stderr} в заданный файл (потому что номер этого потока -- 2).

\section{UNIX time}

В UNIX-среде очень популярно представление даты и времени в формате UNIX time.
Это просто 32-битное число, показывающее
количество прошедших секунд с 1-го января 1970-го года.

В качестве положительных сторон: 1) очень легко хранить это 32-битное число; 2) очень легко вычислять разницу дат;
3) невозможно закодировать неверные даты и время, такие как 32-е января, 29-е февраля невысокосных годов, 
25 часов 62 минуты.

В качестве отрицательных сторон: 1) нельзя закодировать дату до 1970-го года.

В наше время, если использовать UNIX time, тем не менее, следует помнить что ``срок его действия'' истечет
в 2038-м году, именно тогда 32-битное число переполнится, то есть, пройдет $2^{32}$ секунд с 1970-го года.
Так что, для этого следует использовать 64-битное значение, т.е., time64.

% ? NtQuerySystemTime http://msdn.microsoft.com/en-us/library/windows/desktop/ms724512(v=vs.85).aspx

\section{scanf(), fscanf(), sscanf()}

\subsection{Засада \#1}

Если использовать \%d в строке формата, scanf() подразумевает что это 32-битный int. 

Ошибкой является подобное:

\begin{lstlisting}
char a[10];

scanf ("%d %d %d %d", &a[0], &a[1], &a[2], &[3]);
\end{lstlisting}

Символы (или байты) лежат ``в притык'' друг к другу. Когда scanf() будет обрабатывать первое значение, он будет считать
его за 32-битный int, и ``затрет'' остальные три, рядом лежащие. И так далее.



\chapter*{\IFRU{Послесловие}{Afterword}}
\addcontentsline{toc}{chapter}{\IFRU{Послесловие}{Afterword}}

% \section{\IFRU{Краудфандинг}{Crowdfunding}}

\IFRU{Эта книга является свободной, находится в свободном доступе, и доступна в виде исходных кодов}
{This book is free, available freely and available in source code form}\footnote{\url{https://github.com/dennis714/RE-for-beginners}} (LaTeX), 
\IFRU{и всегда будет оставаться таковой}{and it will be so forever}.

\IFRU{В мои текущие планы насчет этой книги входит добавление информации на эти темы:}
{My current plans for this books is to add a lot of information about} C++11, flex/bison.

\IFRU{Если вы хотите чтобы я продолжал свою работу и писал на эти темы,
вы можете рассмотреть идею краудфандинга}
{If you want me to continue writing on all these topics, you may consider crowdfunding}.

\IFRU{Со способами краудфандинга можно ознакомиться на странице}
{Ways to crowdfund are available on the page:} \url{http://yurichev.com/crowdfunding.html}

%\subsection{\IFRU{Жертвователи}{Donors}}


\section{\IFRU{Вопросы?}{Questions?}}

\IFRU{Совершенно по любым вопросам, вы можете не раздумывая писать автору}
{Do not hesitate to mail any questions to the author}: \TT{<\EMAIL>}
 
\IFRU{Пожалуйста, присылайте мне информацию о замеченных ошибках 
(включая грамматические), итд.}
{Please, also do not hesitate to send me any corrections 
(including grammar ones (you see how horrible my English is?)), etc.}


\bibliographystyle{alpha}
\bibliography{books,articles,usenet,misc}

\clearpage
\printindex

\end{document}

