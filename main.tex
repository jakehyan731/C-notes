\documentclass[11pt,a4paper,oneside]{book}

\usepackage{cmap}

\ifdefined\RUSSIAN
\usepackage[english,russian]{babel}
\usepackage[T2A]{fontenc}
\usepackage{paratype}
\renewcommand*\familydefault{\sfdefault}
% http://www.emerson.emory.edu/services/latex/latex_169.html
\newcommand{\lstlistingsize}{\scriptsize}
\else
\usepackage[russian,english]{babel}
\usepackage[T2A]{fontenc}
\usepackage[default]{sourcesanspro}
\newcommand{\lstlistingsize}{\footnotesize}
\fi

\usepackage[utf8]{inputenc}
\usepackage{listings}
\usepackage{ulem}
\usepackage{url}
\usepackage{graphicx}
\usepackage{listingsutf8}
\usepackage{makeidx}
\usepackage{cite}
\usepackage[cm]{fullpage}
\usepackage{color}
\usepackage{fancyvrb}
\usepackage{xspace}
\usepackage{framed}
\usepackage{ccicons}
\usepackage[nottoc]{tocbibind}
\usepackage{amsmath}
\usepackage[table]{xcolor}% http://ctan.org/pkg/xcolor
\usepackage[]{hyperref} % should be last

\definecolor{lstbgcolor}{rgb}{0.94,0.94,0.94}
\makeindex

\newcommand{\TT}[1]{\texttt{#1}}
\newcommand{\IT}[1]{\textit{#1}}
\newcommand{\IFRU}[2]{\iflanguage{russian}{#1}{#2}}


\newcommand{\TITLE}{\IFRU{Заметки о языке программирования Си/Си++}
{C/C++ programming language notes}}
\newcommand{\AUTHOR}{\IFRU{Денис Юричев}{Dennis Yurichev}}
\newcommand{\EMAIL}{dennis@yurichev.com}

\hypersetup{
    pdftex,
    colorlinks=true,
    allcolors=blue,
    pdfauthor={\AUTHOR},
    pdftitle={\TITLE}
    }

\selectlanguage{english}

\lstset{
    backgroundcolor=\color{lstbgcolor},
    basicstyle=\ttfamily\lstlistingsize, 
    breaklines=true,
    frame=single,
    inputencoding=cp1251,
    columns=fullflexible,keepspaces,
}

\begin{document}

\VerbatimFootnotes

\frontmatter

\begin{titlepage}
\begin{center}
\vspace*{\fill}
\LARGE \TITLE

\vspace*{\fill}

\large \AUTHOR

\large \TT{<\EMAIL>}
\vspace*{\fill}
\vfill

\ccbyncnd

\textcopyright 2013, \AUTHOR. 

\IFRU{Это произведение доступно по лицензии Creative Commons «Attribution-NonCommercial-NoDerivs» 
(«Атрибуция — Некоммерческое использование — Без производных произведений») 3.0 Непортированная. 
Чтобы увидеть копию этой лицензии, посетите}
{This work is licensed under the Creative Commons Attribution-NonCommercial-NoDerivs 3.0 Unported License. 
To view a copy of this license, visit} \url{http://creativecommons.org/licenses/by-nc-nd/3.0/}.

\IFRU{Версия этого текста}{Text version} ({\large \today}).

\IFRU{Возможно, более новая версии текста, а так же англоязычная версия, также доступна по ссылке}
{Probably, newer version of this text, and also Russian language version is also accessible at} \url{http://yurichev.com/C-notes.html}

\IFRU{Вы также можете подписаться на мой twitter для получения информации о новых версиях этого текста, итд:
\href{https://twitter.com/yurichev_ru}{@yurichev\_ru}}
{You may also subscribe to my twitter, to get information about updates of this text, etc: 
\href{https://twitter.com/yurichev}{@yurichev}}
\end{center}
\end{titlepage}

\tableofcontents
\cleardoublepage

\begin{center}
\vspace*{\fill}

\IFRU{Эта страница сдается в аренду для рекламы}{This page can be rented for advertisement}.

\TT{<\EMAIL>}

\vspace*{\fill}
\vfill
\end{center}


\cleardoublepage

\chapter{\IFRU{Введение}{Preface}}

Сейчас, в 2013-м году, если некто желает написать как можно более быстро работающую программу, либо как
можно более компактную для встраиваемых систем либо маломощных микроконтроллеров, то выбор небольшой:
Си, Си++ либо ассемблер. И замены этим языкам, в обозримом будущем, похоже не видно.

\chapter{\IFRU{Об авторе}{About author}}

\IFRU{Денис Юричев ~--- опытный программист, свободный для найма как reverse engineer, консультант, тренер. 
С его резюме можно ознакомиться \href{http://yurichev.com/Dennis_Yurichev.pdf}{здесь}.}
{Dennis Yurichev is experienced programmer, available for hire as reverse engineer, consultant, trainer. 
His CV is available \href{http://yurichev.com/Dennis_Yurichev.pdf}{here}.}

\chapter{\IFRU{Благодарности}{Thanks}}

\IFRU{Андрей ''herm1t'' Баранович, Слава ''Avid'' Казаков}
{Andrey ''herm1t'' Baranovich, Slava ''Avid'' Kazakov}.

\mainmatter

\chapter{\IFRU{Элементы языка Си/Си++}{C/C++ language elements}}
\label{typedef}
\subsubsection{typedef}

\TT{typedef} \IFRU{вводит}{introduces} \IT{\IFRU{синоним}{synonym}} \IFRU{для типа данных}{for a data type}.
\IFRU{Часто это используется для структур, чтобы каждый раз не писать}
{It is often used for structures, for the reason not to write} \IT{struct}
\IFRU{перед её именем, например}{each time before its name, for example}:

\begin{lstlisting}
typedef struct _node
{
	node *prev;
	node *next;
	void *data;
} node;
\end{lstlisting}

\IFRU{Такого очень много в ``заголовочных'' файлах в}
{A lot of such examples may be found in header files in the} Windows SDK (Windows API).

\IFRU{Тем не менее}{Nevertheless}, \IT{typedef} 
\IFRU{также можно использовать не только для структур, но и для обычных, 
интегральных}{can be also used not only for structures, but also for usual integral}
\footnote{\IFRU{приводимых к числу}{which may be converted to a number}}, \IFRU{типов, например}{types like}:

\begin{lstlisting}
typedef int age;
int compute_mean (age wife, age husband);

typedef int coord;
void draw (coord X, coord Y, coord Z);

typedef uint32_t address;
void write_memory (address a, size_t size, byte *buf);
\end{lstlisting}

\IFRU{Как видно}{As we can see}, \IT{typedef} 
\IFRU{здесь может помочь в документировании исходного кода, так он легче читается}{here may help with
code documentation, it is now easier to read}.

\IFRU{Например, тип}{For example, the} \IT{time\_t} 
(\IFRU{время в формате}{The time in the} UNIX time
\IFRU{, то что возвращает стандартная функция}{ format, for example, what the}
localtime()
\IFRU{, например}{returns}), \IFRU{это на самом деле
обычное 32-битное число, но этот тип определен в}{it is in fact 32-bit number, but the type is defined in the}
\IT{time.h} \IFRU{обычно так}{file usually as}:

\begin{lstlisting}
typedef long __time32_t;
\end{lstlisting}

\IFRU{Здесь вполне можно было бы использовать директиву препроцессора}
{A preprocessor directive} \TT{\#define} \IFRU{(многие так и делают),
но это хуже с точки зрения обработки ошибок во время компиляции}{may be used here (many do so),
but it is worse in the sense of errors handling during compilation}.

\paragraph{\IFRU{Критика }{}typedef\IFRU{}{ criticism}}

\IFRU{Тем не менее, такой известный и опытный программист как Линус Торвальдс,
против использования typedef}{Nevertheless, such well-known and experiences programmer as Linus
Torvalds is against typedef usage}:
\cite{Torvalds:2002}.


\subsection{goto}

\index{goto}
\IFRU{Использование}{Usage of} \IT{goto}\footnote{statement} 
\IFRU{считается плохим тоном и вредным вообще}{is considered as bad taste and harmful}
\cite{Dijkstra:1968:LEG:362929.362947}\cite{Dijkstra:1979:GSC:1241515.1241518}, 
\IFRU{тем не менее, использование его в разумных дозах}{nevertheless, its usage in reasonable
doses}\cite{Knuth:1974:SPG:356635.356640} \IFRU{может облегчить жизнь}{may be very helpful}.

\IFRU{Частый пример, это выход из функции}{One frequent example is return from a function}:

\begin{lstlisting}
void f(...)
{
	byte* buf1=malloc(...);
	byte* buf2=malloc(...);

	...

	if (something_goes_wrong_1)
		goto cleanup_and_exit;

	...
	
	if (something_goes_wrong_2)
		goto cleanup_and_exit;

	...

cleanup_and_exit:
	free(buf1);
	free(buf2);
	return;
};
\end{lstlisting}

\IFRU{Более сложный пример}{More complex example}:

\begin{lstlisting}
void f(...)
{
	byte* buf1=malloc(...);
	byte* buf2=malloc(...);

	FILE* f=fopen(...);
	if (f==NULL)
		goto cleanup_and_exit;

	...

	if (something_goes_wrong_1)
		goto close_file_cleanup_and_exit;

	...
	
	if (something_goes_wrong_2)
		goto close_file_cleanup_and_exit;

	...

close_file_cleanup_and_exit:
	fclose(f);

cleanup_and_exit:
	free(buf1);
	free(buf2);
	return;
};
\end{lstlisting}

\IFRU{Если в данных примерах отказаться от}{If to remove all} \IT{goto}\IFRU{, то придется вызывать}
{ in these examples, one will need to call} \IT{free()} \AndENRU \IT{fclose()}
\IFRU{перед каждым выходом из функции}{before each return from the function} 
(\IT{return})\IFRU{, что здорово замусорит весь код}{which adds a lot of mess}.

\index{Linux}
\IFRU{Использование}{Usage of} \IT{goto} 
\IFRU{в таких случаях одобряется, например, в}{is, for example, approved in} \cite{LinuxKernelCodingStyle}.

%Примеры более ``harmful'' но эффективного использования goto можно найти в исходниках nginx.
% example?



\chapter{\IFRU{Стандартные библиотеки Си/Си++}{C/C++ standard library}}

\section{assert}

Как известно, этот макрос часто используется для валидации
\footnote{используется также такой термин как ``инвариант'' и ``sanitization'' в англ.яз.} заданных значений. 
Например, если ваша ф-ция
работает с датой, вы, вероятно, захотите написать в её начале что-то вроде \IT{assert (month>=1 \&\& month<=12)}.

Вот то о чем нужно помнить: стандартный макрос assert() доступен только в отладочных (debug) сборках. В release
все выражения как бы исчезают. Поэтому писать, например, \IT{assert(f=malloc(...))} неверно. Впрочем,
вы возможно захотите использовать что-то вроде \IT{assert(object->get\_something()==123)}.

В макросах assert можно также указывать небольшие сообщения об ошибках: 
вы увидите их если assert() ``не сойдется''. 
Например, в исходниках LLVM\footnote{\url{http://llvm.org/}} можно встретить такое:

\begin{lstlisting}
assert(Index < Length && "Invalid index!");
...
assert(i + Count <= M && "Invalid source range");
...
assert(j + Count <= N && "Invalid dest range");
\end{lstlisting}

Текстовая строка имеет тип \IT{const char*}, и она никогда не NULL. 
Таким образом, можно дописать к любому выражению \IT{... \&\& true} не меняя его смысл.

\section{Разница между stdout и stderr}

\IT{stdout} это то что выводится на консоль при помощи вызова \IT{printf()}.
\IT{stdout} это буферизированный вывод,
так что, пользователь, обычно того не зная, видит вывод порциями. Бывает так что программа выдает
что-то используя \IT{printf()} либо \IT{cout} и тут же падает.
Если это попадает в буфер, но буфер не успевает
``сброситься'' (flush) в консоль, то пользователь ничего не увидит. Это бывает неудобно.
Таким образом, для вывода более важной информации, в том числе отладочной, удобнее использовать \IT{stderr}.

\IT{stderr} это не буферизированный вывод, и всё что попадает в этот поток при помощи 
\TT{fprintf(stderr,...)} либо \IT{cerr}, появляется в консоли тут же.

Не следует также забывать, что из-за отсутствия буфера, вывод в \IT{stderr} медленнее.

Чтобы направлять \IT{stderr} в другой файл при запуске процесса, можно указывать:

\begin{lstlisting}
process 2> debug.txt
\end{lstlisting}

... это направит вывод \IT{stderr} в заданный файл (потому что номер этого потока -- 2).

\section{UNIX time}

В UNIX-среде очень популярно представление даты и времени в формате UNIX time.
Это просто 32-битное число, показывающее
количество прошедших секунд с 1-го января 1970-го года.

В качестве положительных сторон: 1) очень легко хранить это 32-битное число; 2) очень легко вычислять разницу дат;
3) невозможно закодировать неверные даты и время, такие как 32-е января, 29-е февраля невысокосных годов, 
25 часов 62 минуты.

В качестве отрицательных сторон: 1) нельзя закодировать дату до 1970-го года.

В наше время, если использовать UNIX time, тем не менее, следует помнить что ``срок его действия'' истечет
в 2038-м году, именно тогда 32-битное число переполнится, то есть, пройдет $2^{32}$ секунд с 1970-го года.
Так что, для этого следует использовать 64-битное значение, т.е., time64.

% ? NtQuerySystemTime http://msdn.microsoft.com/en-us/library/windows/desktop/ms724512(v=vs.85).aspx

\section{scanf(), fscanf(), sscanf()}

\subsection{Засада \#1}

Если использовать \%d в строке формата, scanf() подразумевает что это 32-битный int. 

Ошибкой является подобное:

\begin{lstlisting}
char a[10];

scanf ("%d %d %d %d", &a[0], &a[1], &a[2], &[3]);
\end{lstlisting}

Символы (или байты) лежат ``в притык'' друг к другу. Когда scanf() будет обрабатывать первое значение, он будет считать
его за 32-битный int, и ``затрет'' остальные три, рядом лежащие. И так далее.



\chapter*{\IFRU{Послесловие}{Afterword}}
\addcontentsline{toc}{chapter}{\IFRU{Послесловие}{Afterword}}

% \section{\IFRU{Краудфандинг}{Crowdfunding}}

\IFRU{Эта книга является свободной, находится в свободном доступе, и доступна в виде исходных кодов}
{This book is free, available freely and available in source code form}\footnote{\url{https://github.com/dennis714/RE-for-beginners}} (LaTeX), 
\IFRU{и всегда будет оставаться таковой}{and it will be so forever}.

\IFRU{В мои текущие планы насчет этой книги входит добавление информации на эти темы:}
{My current plans for this books is to add a lot of information about} C++11, flex/bison.

\IFRU{Если вы хотите чтобы я продолжал свою работу и писал на эти темы,
вы можете рассмотреть идею краудфандинга}
{If you want me to continue writing on all these topics, you may consider crowdfunding}.

\IFRU{Со способами краудфандинга можно ознакомиться на странице}
{Ways to crowdfund are available on the page:} \url{http://yurichev.com/crowdfunding.html}

%\subsection{\IFRU{Жертвователи}{Donors}}


\section{\IFRU{Вопросы?}{Questions?}}

\IFRU{Совершенно по любым вопросам, вы можете не раздумывая писать автору}
{Do not hesitate to mail any questions to the author}: \TT{<\EMAIL>}
 
\IFRU{Пожалуйста, присылайте мне информацию о замеченных ошибках 
(включая грамматические), итд.}
{Please, also do not hesitate to send me any corrections 
(including grammar ones (you see how horrible my English is?)), etc.}


\bibliographystyle{alpha}
\bibliography{books,articles,usenet,misc}

\clearpage
\printindex

\end{document}
