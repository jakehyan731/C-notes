\documentclass[11pt,a4paper,oneside]{book}

\usepackage{cmap}

\ifdefined\RUSSIAN
\usepackage[english,russian]{babel}
\usepackage[T2A]{fontenc}
\usepackage{paratype}
\renewcommand*\familydefault{\sfdefault}
% http://www.emerson.emory.edu/services/latex/latex_169.html
\newcommand{\lstlistingsize}{\scriptsize}
\else
\usepackage[russian,english]{babel}
\usepackage[T2A]{fontenc}
\usepackage[default]{sourcesanspro}
\newcommand{\lstlistingsize}{\footnotesize}
\fi

\usepackage[utf8]{inputenc}
\usepackage{listings}
\usepackage{ulem}
\usepackage{url}
\usepackage{graphicx}
\usepackage{listingsutf8}
\usepackage{makeidx}
\usepackage{cite}
\usepackage[cm]{fullpage}
\usepackage{color}
\usepackage{fancyvrb}
\usepackage{xspace}
\usepackage{framed}
\usepackage{ccicons}
\usepackage[nottoc]{tocbibind}
\usepackage{amsmath}
\usepackage[table]{xcolor}% http://ctan.org/pkg/xcolor
\usepackage[]{hyperref} % should be last
%\usepackage{tikz}

\definecolor{lstbgcolor}{rgb}{0.94,0.94,0.94}
\makeindex

\newcommand*{\TT}[1]{\texttt{#1}}
\newcommand*{\IT}[1]{\textit{#1}}
\newcommand*{\IFRU}[2]{\iflanguage{russian}{#1}{#2}}
\newcommand*{\EN}[1]{\iflanguage{english}{#1}{}}
\newcommand*{\RU}[1]{\iflanguage{english}{}{#1}}


\newcommand{\TITLE}{\IFRU{Заметки о языке программирования Си/Си++}
{C/C++ programming language notes}}
\newcommand{\AUTHOR}{\IFRU{Денис Юричев}{Dennis Yurichev}}
\newcommand{\EMAIL}{dennis@yurichev.com}

\hypersetup{
    pdftex,
    colorlinks=true,
    allcolors=blue,
    pdfauthor={\AUTHOR},
    pdftitle={\TITLE}
    }

\selectlanguage{english}

\lstset{
    backgroundcolor=\color{lstbgcolor},
    basicstyle=\ttfamily\lstlistingsize, 
    breaklines=true,
    frame=single,
    inputencoding=cp1251,
    columns=fullflexible,keepspaces,
}

\begin{document}

\VerbatimFootnotes

\frontmatter

\begin{titlepage}
\begin{center}
\vspace*{\fill}
\LARGE \TITLE

\vspace*{\fill}

\large \AUTHOR

\large \TT{<\EMAIL>}
\vspace*{\fill}
\vfill

\ccbyncnd

\textcopyright 2013, \AUTHOR. 

\IFRU{Это произведение доступно по лицензии Creative Commons «Attribution-NonCommercial-NoDerivs» 
(«Атрибуция — Некоммерческое использование — Без производных произведений») 3.0 Непортированная. 
Чтобы увидеть копию этой лицензии, посетите}
{This work is licensed under the Creative Commons Attribution-NonCommercial-NoDerivs 3.0 Unported License. 
To view a copy of this license, visit} \url{http://creativecommons.org/licenses/by-nc-nd/3.0/}.

\IFRU{Версия этого текста}{Text version} ({\large \today}).

\IFRU{Возможно, более новая версии текста, а так же англоязычная версия, также доступна по ссылке}
{Probably, newer version of this text, and also Russian language version is also accessible at} 
%FIXME URL?
\url{http://yurichev.com/C-notes.html}

\IFRU{Вы также можете подписаться на мой twitter для получения информации о новых версиях этого текста, итд:
\href{https://twitter.com/yurichev_ru}{@yurichev\_ru}}
{You may also subscribe to my twitter, to get information about updates of this text, etc: 
\href{https://twitter.com/yurichev}{@yurichev}}
%FIXME mailing list
\end{center}
\end{titlepage}

\tableofcontents
\cleardoublepage

\include{ad}

\cleardoublepage

\chapter{\IFRU{Введение}{Preface}}

Сейчас, в 2013-м году, если некто желает написать 1) как можно более быстро работающую программу; 2) либо как
можно более компактную для встраиваемых систем либо маломощных микроконтроллеров, то выбор небольшой:
Си, Си++ либо ассемблер. И альтернативы этим старым но популярным языкам, в обозримом будущем, 
пока что не видно.

\chapter{\IFRU{Целевая аудитория}{Target audience}}

Этот сборник заметок предназначен не для начинающих, но и не для экспертов, а скорее для тех, 
кто хочет освежить свои знания по Си/Си++.

\chapter{\IFRU{Об авторе}{About author}}

\IFRU{Денис Юричев ~--- опытный программист, свободный для найма как программист, reverse engineer, консультант, тренер. 
С его резюме можно ознакомиться \href{http://yurichev.com/Dennis_Yurichev.pdf}{здесь}.}
{Dennis Yurichev is experienced programmer, available for hire as programmer, reverse engineer, consultant, trainer. 
His CV is available \href{http://yurichev.com/Dennis_Yurichev.pdf}{here}.}

\chapter{\IFRU{Благодарности}{Thanks}}

\IFRU{Андрей ''herm1t'' Баранович, Слава ''Avid'' Казаков}
{Andrey ''herm1t'' Baranovich, Slava ''Avid'' Kazakov}.

\mainmatter

\chapter{\IFRU{Элементы языка Си/Си++}{C/C++ language elements}}
\label{typedef}
\subsubsection{typedef}

typedef вводит \IT{синоним} для типа. Часто это используется для структур, чтобы каждый раз не писать \IT{struct}
перед её именем, например:

\begin{lstlisting}
typedef struct _node
{
	node *prev;
	node *next;
	void *data;
} node;
\end{lstlisting}

Такого очень много в ``заголовочных'' файлах в Windows SDK (Windows API).

Тем не менее, \IT{typedef} также можно использовать не только для структур, но и для обычных, 
интегральных\footnote{приводимых к числу}, типов, например:

\begin{lstlisting}
typedef int age;
int compute_mean (age wife, age husband);

typedef int coord;
void draw (coord X, coord Y, coord Z);

typedef uint32_t address;
void write_memory (address a, size_t size, byte *buf);
\end{lstlisting}

Как видно, \IT{typedef} здесь может помочь в документировании исходного кода, так он легче читается.

Например, тип \IT{time\_t} (время в формате UNIX time, то что возвращает стандартная функция localtime(), 
например), это на самом деле
обычное 32-битное число, но этот тип определен в time.h обычно так:

\begin{lstlisting}
typedef long __time32_t;
\end{lstlisting}

Здесь вполне можно было бы использовать директиву препроцессора \TT{\#define} (многие так и делают),
но это хуже с точки зрения обработки ошибок во время компиляции.

\paragraph{Критика typedef}

Тем не менее, такие известные и опытные программисты как Линус Торвальдс, против использования typedef:
\cite{Torvalds:2002}.


\section{goto}

Использование оператора \IT{goto}\footnote{statement} считается плохим тоном и вредным вообще\cite{Dijkstra:1968:LEG:362929.362947}\cite{Dijkstra:1979:GSC:1241515.1241518}, 
тем не менее, использование его в разумных дозах\cite{Knuth:1974:SPG:356635.356640} может облегчить жизнь.

Частый пример, это выход из функции.

\begin{lstlisting}
void f(...)
{
	byte* buf1=malloc(...);
	byte* buf2=malloc(...);

	...

	if (something_goes_wrong_1)
		goto cleanup_and_exit;

	...
	
	if (something_goes_wrong_2)
		goto cleanup_and_exit;

	...

cleanup_and_exit:
	free(buf1);
	free(buf2);
	return;
};
\end{lstlisting}

Более сложный пример:

\begin{lstlisting}
void f(...)
{
	byte* buf1=malloc(...);
	byte* buf2=malloc(...);

	FILE* f=fopen(...);
	if (f==NULL)
		goto cleanup_and_exit;

	...

	if (something_goes_wrong_1)
		goto close_file_cleanup_and_exit;

	...
	
	if (something_goes_wrong_2)
		goto close_file_cleanup_and_exit;

	...

close_file_cleanup_and_exit:
	fclose(f);

cleanup_and_exit:
	free(buf1);
	free(buf2);
	return;
};
\end{lstlisting}

Если в данных примерах отказаться от \IT{goto}, то придется вызывать \IT{free()} и \IT{fclose()}
перед каждым выходом из функции (\IT{return}), что здорово замусорит весь код.

Использование \IT{goto} в таких случаях одобряется, например, в \cite{LinuxKernelCodingStyle}.


\subsection{for}

\IFRU{В}{The} for()\IFRU{, как известно, три выражения:
первое вычисляется перед началом всех итераций,
второе вычисляется перед каждой итерацией,
третье ~--- после каждой итерации}
{ statement, as we know, has 3 expressions:
1st computing before all iterations begin,
2nd computing before each iteration
and the 3rd ~--- after each iteration}.

\IFRU{И конечно же, там можно указывать что-то отличное от обычного счетчика}
{And of course, there might be written something different from the usual counter}.

\subsubsection{\IFRU{Засада}{Caveat} \#1}

\IFRU{Если написать такое}{If to write this}:

\lstinputlisting{common/for_strlen.cpp}

... \IFRU{то это наверное будет ошибкой}{perhaps this is a mistake}:
\TT{strlen(s)} \IFRU{будет вызываться перед каждой итерацией}{will be called before each iteration} 
~--- \IFRU{такой код генерирует}{that is the code} MSVC 2010\IFRU{}{ generated}.
\IFRU{Впрочем}{However}, GCC 4.8.1 \IFRU{вызывает}{calls} \TT{strlen(s)} 
\IFRU{только один раз, в начале цикла}{only once, at the loop beginning}.

\subsubsection{\IFRU{Запятая}{Comma}}

\IFRU{Запятая}{Comma}\cite[6.5.17]{C99TC3} ~--- \IFRU{не самая понятная для всех штука в Си, 
однако, их очень удобно использовать в определениях в for()}
{is not widely understood C feature, however, it is very useful for using in a for() declarations}.

\IFRU{Например, может пригодится использовать в цикле два итератора одновременно}
{For example, it is useful to have two iterators simultaneously}.
\IFRU{Пусть один просто отсчитывает от 0, прибавляя 1 при каждой итерации,
а второй итератор указывает на элемент в списке}
{Let the first iterator just counts from 0 adding 1 at each iteration, and the second iterator
points to the list element}:

\lstinputlisting{common/for_comma.cpp}

\IFRU{Это выдаст предсказуемый результат}{This will dumps predictable result}:

\begin{lstlisting}
0: 123
1: 456
2: 789
3: 1
\end{lstlisting}

\IFRU{Но к сожалению, определять итераторы вместе с типами в теле самого for() вот так нельзя}
{However, it is not possible to declare iterators with its types in for() clause}:

\begin{lstlisting}
	for (int i=0, std::list<int>::iterator it=l.begin(); it!=l.end(); i++, it++)
\end{lstlisting}

\subsubsection{continue}

\IT{continue} \IFRU{это безусловный переход на конец тела цикла}{is unconditional goto to the end
of loop body}.

\IFRU{Это может быть очень полезно, например, в подобном коде}
{This may be very useful, for example, in such code}:

\begin{lstlisting}
for (...)
{
	if (is_element_satisfied_criteria_1(...)==true)
	{
		// do something need in is_element_satisfied_criteria_2()

		if (is_element_satisfied_criteria_2(...)==true)
		{
			do_something_1();
			do_something_2();
			do_something_3();
		};

	};
};
\end{lstlisting}

... \IFRU{всё это можно легко заменить на более опрятное}{it is all can be replaced by neat}:

\begin{lstlisting}
for (...)
{
	if (is_element_satisfied_criteria_1(...)==false)
		continue;

	// do something need in is_element_satisfied_criteria_2()

	if (is_element_satisfied_criteria_2(...)==false)
		continue;

	do_something_1();
	do_something_2();
	do_something_3();
};
\end{lstlisting}


\section{sizeof}

% array of struct phonebook

Обычно, sizeof() применяют к интегральным\footnote{типы отражающие числа} типам либо к структурам, тем не менее,
его можно применять и к массивам:

% snprintf, wchar_t...

И к массивам структур:

\begin{lstlisting}
struct phonebook_entry
{
	char *name;
	char *surname;
	char *tel;
};

struct phonebook_entry phonebook[]=
{
	{ "Kirk", "Hammett", "555-1234" },
	{ "Lars", "Ulrich", "555-5678" },
	{ "James", "Hetfield", "555-1122" },
	{ "Robert", "Trujillo", "555-7788" }
};

void dump (struct phonebook_entry* input)
{
	for (int i=0; i<sizeof(phonebook)/sizeof(struct phonebook_entry); i++)
		printf ("%s %s - %s\n", input[i].name, input[i].surname, input[i].tel);
};
\end{lstlisting}

sizeof(phonebook) -- это размер всего массива структур в байтах. sizeof(struct phonebook\_entry) -- это размер
одной структуры в байтах. Делением мы узнаем количество структур в массиве.


\section{\IFRU{Указатели}{Pointers}}

Как однажды сказал Дональд Кнут в интервью\cite{KnuthInterview1993}, то как в Си устроены указатели, это является
очень удачной инновацией в языках программирования по тем временам.

Итак, определимся с терминологией. Указатель это просто адрес какого-то элемента в памяти. Указатели настолько
популярны, потому что в какую-то функцию намного проще передать просто адрес объекта в памяти, вместо того
чтобы передавать весь объект --- ведь это абсурдно. К тому же, вызываемая функция, например, обрабатывающая
ваш массив данных, просто изменит что-то в нем, вместо того чтобы возвращать новый, измененный массив данных, 
что тоже абсурдно.

Возьмем простой пример. Стандартная функция \IT{strtok()} делит строку на подстроки, используя заданный символ
как разделитель. К примеру, мы можем подать на вход строку \TT{The quick brown fox jumps over the lazy dog} 
и задать пробел в качестве разделителя.

\lstinputlisting{src/strtok_ex1.c}

Мы в итоге получим на выходе:

\begin{lstlisting}
The
quick
brown
fox
jumps
over
the
lazy
dog
\end{lstlisting}

Что тут в реальности происходит, это то что ф-ция \IT{strtok()} просто находит в заданной строке следующий пробел 
(либо иной заданный разделитель),
записывает туда $0$ (что по соглашениям текстовых строк в Си является концом строки) и возвращает указатель на это
место.

В качестве недостатка \IT{strtok()} можно отметить, что эта ф-ция ``портит'' входную строку, записывая нули
на месте разделителей.

Но вот что важно заметить: никакие строки или подстроки не копируются в памяти. Входная
строка остается там же где и лежала. В \IT{strtok()} передается только указатель на нее, или, её адрес.
Эта ф-ция затем, после
того как записывает $0$, возвращает \IT{адрес} каждого следующего ``слова''.
Адрес затем подается на вход в \IT{printf()}, где происходит его вывод на консоль.

Обратите также внимание на то что в исходнике присутствует и некорректное объявление \IT{str}. Оно тем некорректное
что в Си строка имеет тип \IT{const char*}, то есть, распологается в константном сегменте данных, защищенным
от записи.
Если так сделать, то \IT{strtok()} не сможет модифицировать строку записывая туда нули и процесс ``упадет''.

Так что, в нашем примере, строка выделяется как массив \IT{char} а не массив \IT{const char}.

Обобщая, скажем что работа со строками в Си происходит только лишь используя адреса этих строк. К примеру,
ф-ция сравнения строк \IT{strcmp()} на вход берет два адреса двух строк и по одному символу сравнивает их.
Было бы очень абсурдно копировать куда-то эти две строки лишний раз, чтобы \IT{strcmp()} обработала их.

Трудность понимания указателей в Си связана с тем, что указатель это ``часть'' объекта. Указатель на строку,
это не сама строка. Сама строка еще должна где-то в памяти хранится, под нее нужно выделять место, итд.

В ЯП более высокого уровня, объект и указатель на него могут быть представлены как единое целое, что облегчает
понимание.
Впрочем, это не значит что в этих ЯП строки и иные объекты неразумно копируются много раз при передаче 
в другие функции,
там точно так же используются указатели, но просто эта механика скрыта от программиста.

\subsection{Синтаксический сахар для a[i]}

Ради упрощения, можно сказать что в Си нет массивов вообще, а есть только синтаксический сахар для выражений
вроде \IT{a[i]}.

К примеру, возможно вы видели такой трюк:

\begin{lstlisting}
printf ("%c", 3["hello"]);
\end{lstlisting}

Это выдаст 'l'. 
Это происходит, потому что любое выражение \IT{a[i]}, на самом деле преобразовывается в \IT{*(a+i)}
\cite[6.5.2.1]{C99TC3}.
\IT{3["hello"]} преобразовывается в \IT{*(3+"hello")}, а \IT{"hello"} это просто указатель на массив символов, 
типа \IT{const char*}.
\IT{3+"hello"} это в итоге указатель на часть строки, то есть, \IT{"lo"}. А \IT{*("lo")} это cимвол 'l'. 
Вот почему это работает.

Но так врядли стоит писать, если вы конечно не готовите программу на конкурс 
The International Obfuscated C Code Contest\footnote{\url{http://www.ioccc.org/}}.
Так что я привел этот пример, чтобы наглядно показать, 
что выражения вроде \IT{a[i]} это синтаксический сахар.

При некотором упорстве, в Си вообще можно обойтись без индексации массивов, хотя выглядеть это будет не очень
эстетично.

Кстати, так легко понять как работают отрицательные индексы массивов. \IT{a[-3]} просто преобразуется в \IT{*(a-3)},
так адресуется элемент лежащий перед самим массивом.
И хотя это вполне возможно, так можно делать только если вы точно знаете, что вы делаете.

В Си массив это, в каком-то смысле, это просто место в памяти под массив плюс указатель, указывающий
на него. 

Вот почему имя массива в Си можно считать за указатель:

Если вы объявите глобальную переменную \IT{int a[10]}, то \IT{a} будет иметь тип \IT{int*}.
Позже, когда где-то в коде
вы укажете \IT{x=a[5]}, выражение будет преобразовано в \IT{x=*(a+5)}. От начала массива (то есть, первого элемента
массива), будет отсчитано 5 элементов, затем оттуда прочитается элемент для записи в \IT{x}.

\subsection{\IFRU{Арифметика указателей}{Pointer arithmetic}}

Простой пример:

\lstinputlisting{src/phonebook1.c}

Мы объяляем глобальный массив из структур. Если скомпилировать это в GCC с ключом \IT{-S} либо в MSVC с ключом
\IT{/Fa}, мы увидим листинг на ассемблере и то, как компилятор расположил эти строки. 

Расположил он их как линейный массив указателей на строки, вот так:

\begin{center}
\begin{tabular}{ | l | l | }
\hline
  ячейка 0    & адрес строки ``Kirk'' \\
  ячейка 1    & адрес строки ``Hammett'' \\
  ячейка 2    & адрес строки ``555-1234'' \\
  ячейка 3    & адрес строки ``Lars'' \\
  ячейка 4    & адрес строки ``Ulrich'' \\
  ячейка 5    & адрес строки ``555-5678'' \\
  ячейка 6    & адрес строки ``James'' \\
  ячейка 7    & адрес строки ``Hetfield'' \\
  ячейка 8    & адрес строки ``555-1122'' \\
  ячейка 9    & адрес строки ``Robert'' \\
  ячейка 10   & адрес строки ``Trujillo'' \\
  ячейка 11   & адрес строки ``555-7788'' \\
  ячейка 12   & 0 \\
  ячейка 13   & 0 \\
  ячейка 14   & 0 \\
\hline
\end{tabular}
\end{center}

Ф-ции \IT{dump1()} и \IT{dump2()} эквивалентны.

Но в первой итератор \IT{i} начинается с 0 и к нему прибавляется 1 на каждой итерации.

Во второй ф-ции итератор \IT{i} указывает на начало массива и затем, к нему прибавляется длина структуры 
(а не 1 байт, как можно поначалу ошибочно подумать),
таким образом, на каждой итерации, \IT{i} указывает на следующий элемент массива.


\section{Operators}

\subsection{==}

Очень неприятные ошибки возникают если в условии \IT{if(a==3)} опечататься и написать \IT{if(a=3)}.
Ведь выражение \IT{a=3} ``возвращает'' 3, а 3 это не 0, поэтому условие if() всегда будет 
срабатывать.

Раньше, для защиты от подобных ошибок, была мода писать наоборот: \IT{if(3==a)}, таким образом,
если опечататься, выйдет \IT{if(3=a)}, компилятор тут же выдаст ошибку.

Тем не менее, в наше время, компиляторы обычно предупреждают если написать \IT{if(a==3)}, 
так что, наверное, менять местами элементы выражения уже не обязательно.

\subsection{Short-circuit и артефакты приоритетов операций в Си}

Разберем что такое short-circuit\footnote{дословный перевод на русский: ``короткое замыкание''}.

Это когда в выражении \IT{if(a \&\& b \&\& c)}, часть b будет вычисляться только если a -- истинна,
а c будет вычисляться
только если a и b -- истинны. И вычисляться они будут именно в таком порядке, как указано.

Иногда можно встретить подобное: \IT{if (p!=NULL \&\& p->field==123)} -- и это совершенно правильно.
Поле field в структуре,
на которую указывает p, будет вычисляться только если указатель p не равен NULL.



\chapter{\IFRU{Препроцессор}{Preprocessor}}

Препроцессор обрабатывает директивы начинающиеся с \# --- \#define, \#include, \#if, итд.

\section{Стандартные для компиляторов и ОС значения}

\begin{itemize}
\item \TT{\_DEBUG} --- отладочная сборка.
\item \TT{NDEBUG} --- неотладочная (release) сборка.
\item \TT{\_\_linux\_\_} --- ОС Linux.
\item \TT{\_WIN32} --- ОС Windows. Присутствует как и в x86-проектах, так и в x64.
\item \TT{\_WIN64} --- Присутствует в x64-проектах для ОС Windows.
\item \TT{\_\_cplusplus} --- присутствует в Си++ проектах.
\item \TT{\_MSC\_VER} --- компилятор MSVC.
\item \TT{\_\_GNUC\_\_} --- компилятор GCC.
\end{itemize}

Так можно писать разные участки кода для разных компиляторов и ОС.

\section{``Пустой'' макрос}

Всем известны макросы не объявляющие никаких значений, например \IT{\_DEBUG}.
Обычно, только проверяется наличие его или отсутствие.
Вот еще пример полезного ``пустого'' макроса:

В заголовочных файлах Windows API мы можем увидеть такое:

\begin{lstlisting}
typedef NTSTATUS
(NTAPI *TDI_REGISTER_CALLBACK)(
  IN PUNICODE_STRING DeviceName,
  OUT HANDLE *TdiHandle);

...

typedef NDIS_STATUS
(NTAPI *CM_CLOSE_CALL_HANDLER)(
  IN NDIS_HANDLE  CallMgrVcContext,
  IN NDIS_HANDLE  CallMgrPartyContext  OPTIONAL,
  IN PVOID  CloseData  OPTIONAL,
  IN UINT  Size  OPTIONAL);
\end{lstlisting}

IN, OUT и OPTIONAL --- это ``пустые'' макросы объявленные так:

\begin{lstlisting}
#ifndef IN
#define IN
#endif
#ifndef OUT
#define OUT
#endif
#ifndef OPTIONAL
#define OPTIONAL
#endif
\end{lstlisting}

Для компилятора они никакой информации не несут, они предназначены только для документирования, показать,
какие параметры ф-ций зачем нужны.

\section{Частая ошибка}

К примеру, вы хотите создать макрос для возведения числа в квадрат:

\begin{lstlisting}
#define square(x)      x*x
\end{lstlisting}

Это ошибка, потому что выражение \IT{square(a+b)} в итоге ``развернется'' в $a+b*a+b$, что, разумеется, совсем
не то что хотелось. Поэтому в определнии макроса все переменные, и сам макрос, нужно ``изолировать'' скобками:

\begin{lstlisting}
#define square(x)      ((x)*(x))
\end{lstlisting}

Пример из файла minmax.h из MinGW:

\begin{lstlisting}
#define max(a,b) (((a) > (b)) ? (a) : (b))
...
#define min(a,b) (((a) < (b)) ? (a) : (b))
\end{lstlisting}




\section{Строки в Си}

В Си нет встроенных возможностей для удобной работы со строками, такими, какие имеются в ЯП более
высокого уровня.

Часто жалуются на неудобную
конкатенацию строк (то есть, склеивание) в Си при помощи функции strcat(). Также, многих раздражает sprintf(),
под которых нельзя толком зараннее предсказать, сколько нужно выделять памяти. Копирование строк при помощи
strcpy() также неудобно --- нужно думать, сколько же выделить байт под буфер. Помимо всего прочего, неудобная
работа со строками в Си, это источник огромного количества уязвимостей в ПО, связанных с переполнениями буфера\cite[1.14.2]{REBook}.

Прежде всего, нужно задать себе вопрос, какие операции со строками нам нужны.
Конкатенация (склеивание) нужна чтобы 1) выдавать в лог сообщения; 2) конструировать строки и записывать их куда-то.

Для 1) можно использовать потоки (streams) --- не конструируя строку, выдавать её по порциям, например:

\begin{lstlisting}
printf ("Date: ");
dump_date(stdout, date);
printf (" a=");
dump_a(stdout, a);
printf ("\n");
\end{lstlisting}

Подобное заменяется в Си++ выводом в ostream:

\begin{lstlisting}
cout << "Date: " << Date_ToString(date) << " a=" << a_ToString(a) << "\n";
\end{lstlisting}

Так быстрее и меньше требуется памяти для конструирования строк. Но все же иногда конструировать их надо.

Есть какие-то библиотеки для этого.
К примеру, в Glib есть gstring.h\footnote{\url{https://github.com/GNOME/glib/blob/master/glib/gstring.h}}/
gstring.c\footnote{\url{https://github.com/GNOME/glib/blob/master/glib/gstring.c}}. 

А в исходниках git можно найти strbuf.h\footnote{\url{https://github.com/git/git/blob/master/strbuf.h}}/
strbuf.c\footnote{\url{https://github.com/git/git/blob/master/strbuf.c}}. Собственно,
подобные Си-библиотеки очень похожи: они обеспечивают структуру данных, в которой есть некоторый буфер для строки, текущий размер буфера
и текущий размер строки в буфере. При помощи отдельных функций, можно добавлять новые строки или символы
в буфер, который, в свою очередь, будет автоматически увеличиваться или даже уменьшаться.

В strbuf.c из git есть даже ф-ция \IT{strbuf\_addf()}, работающая как sprintf(), 
но добавляющая строку-результат в буфер.

Так пользователь освобождается от головной боли связанной с выделением памяти.
При работе с этими библиотеками, практически невозможна ситуация переполнения буфера, если только не начать
работать со структурой самостоятельно.

Типичная последовательность работы с такими библиотеками, выглядит так:

\begin{itemize}
\item
Инициализация структуры strbuf или GString.

\item
Добавление строк и/или символов

\item
Имеем сконструированную строку. Используем как обычную Си-строку, записываем куда-то в файл, передаем по сети, итд

\item
Освобождаем структуру.
\end{itemize}

\subsection{Хранение длины строки}

Всегда хранить длину строки --- это было принято в реализациях ЯП Pascal. 
Не смотря на исходы святых войн\footnote{holy wars} между приверженцами Си и Pascal, все же, почти все библиотеки
для хранения строк и работы с ними, хранят также и текущую длину --- потому что удобства от этого перевешивают
необходимость пересчитывать это значение.

Например, \IT{strlen()} (подсчет длины строки) больше не нужен вообще, длина все время известна.
Конкатенация строк работает намного быстрее, потому что не нужно вычислять длину первой строки.
Ф-ция сравнения строк в самом начале может сравнить длины строк и если они не равны, тут же вернуть false,
не начиная сравнивание самих строк.

В Oracle RDBMS, в сетевых библиотеках, в функции работы со строками, зачастую передается строка и, 
отдельным аргументом, её длина\footnote{\url{http://blog.yurichev.com/node/64}}.
Это не очень эстетично, это выглядит избыточно, зато очень удобно.
Например, у нас есть некоторая ф-ция, которой нужно в начале узнать, какую строку ей передали:

\begin{lstlisting}
void f(char *color)
{
	if (strcmp (color, "red")==0)
		do_red();
	else if (strcmp (color, "green")==0)
		do_green();
	else if (strcmp (color, "blue")==0)
		do_blue();
	else if (strcmp (color, "orange")==0)
		do_orange();
	else if (strcmp (color, "yellow")==0)
		do_yellow();
	printf ("Unknown color!\n");
};
\end{lstlisting}

А вот если бы эта ф-ция имела длину входной строки, её можно было бы переписать так:

\begin{lstlisting}
void f(char *color, int color_len)
{
	switch (color_len)
	{
		case 3:
			if (strcmp (color, "red")==0)
				do_red();
			else 
				goto unknown_color;
			break;
		case 4:
			if (strcmp (color, "blue")==0)
				do_blue();
			else
				goto unknown_color;
			break;
		case 5:
			if (strcmp (color, "green")==0)
				do_green();
			else
				goto unknown_color;
			break;
		case 6:
			if (strcmp (color, "orange")==0)
				do_orange();
			else if (strcmp (color, "yellow")==0)
				do_yellow();
			else
				goto unknown_color;
			break;
		default:
				goto unknown_color;

	};

	return;

unknown_color:
	printf ("Unknown color!\n");
};
\end{lstlisting}

Конечно, с эстетической точки зрения, код выглядит ужасно.
Тем не менее, мы здорово сократили количество необходимых сравнений строк! Вероятно, для тех ситуаций, когда 
нужно как можно быстрее обрабатывать текстовые строки, такой подход может улучшить ситуацию.



% \chapter{\IFRU{Ваши собственные структуры данных}{Your own data structures}}
% list, map

\chapter{\IFRU{Стандартные библиотеки Си/Си++}{C/C++ standard library}}

\section{assert}

Как известно, этот макрос часто используется для валидации
\footnote{используется также такой термин как ``инвариант'' и ``sanitization'' в англ.яз.} заданных значений. 
Например, если ваша ф-ция
работает с датой, вы, вероятно, захотите написать в её начале что-то вроде \IT{assert (month>=1 \&\& month<=12)}.

Вот то о чем нужно помнить: стандартный макрос assert() доступен только в отладочных (debug) сборках. В release
все выражения как бы исчезают. Поэтому писать, например, \IT{assert(f=malloc(...))} неверно. Впрочем,
вы возможно захотите использовать что-то вроде \IT{assert(object->get\_something()==123)}.

В макросах assert можно также указывать небольшие сообщения об ошибках: 
вы увидите их если assert() ``не сойдется''. 
Например, в исходниках LLVM\footnote{\url{http://llvm.org/}} можно встретить такое:

\begin{lstlisting}
assert(Index < Length && "Invalid index!");
...
assert(i + Count <= M && "Invalid source range");
...
assert(j + Count <= N && "Invalid dest range");
\end{lstlisting}

Текстовая строка имеет тип \IT{const char*}, и она никогда не NULL. 
Таким образом, можно дописать к любому выражению \IT{... \&\& true} не меняя его смысл.

\section{Разница между stdout и stderr}

\IT{stdout} это то что выводится на консоль при помощи вызова \IT{printf()}.
\IT{stdout} это буферизированный вывод,
так что, пользователь, обычно того не зная, видит вывод порциями. Бывает так что программа выдает
что-то используя \IT{printf()} либо \IT{cout} и тут же падает.
Если это попадает в буфер, но буфер не успевает
``сброситься'' (flush) в консоль, то пользователь ничего не увидит. Это бывает неудобно.
Таким образом, для вывода более важной информации, в том числе отладочной, удобнее использовать \IT{stderr}.

\IT{stderr} это не буферизированный вывод, и всё что попадает в этот поток при помощи 
\TT{fprintf(stderr,...)} либо \IT{cerr}, появляется в консоли тут же.

Не следует также забывать, что из-за отсутствия буфера, вывод в \IT{stderr} медленнее.

Чтобы направлять \IT{stderr} в другой файл при запуске процесса, можно указывать:

\begin{lstlisting}
process 2> debug.txt
\end{lstlisting}

... это направит вывод \IT{stderr} в заданный файл (потому что номер этого потока -- 2).

\section{UNIX time}

В UNIX-среде очень популярно представление даты и времени в формате UNIX time.
Это просто 32-битное число, показывающее
количество прошедших секунд с 1-го января 1970-го года.

В качестве положительных сторон: 1) очень легко хранить это 32-битное число; 2) очень легко вычислять разницу дат;
3) невозможно закодировать неверные даты и время, такие как 32-е января, 29-е февраля невысокосных годов, 
25 часов 62 минуты.

В качестве отрицательных сторон: 1) нельзя закодировать дату до 1970-го года.

В наше время, если использовать UNIX time, тем не менее, следует помнить что ``срок его действия'' истечет
в 2038-м году, именно тогда 32-битное число переполнится, то есть, пройдет $2^{32}$ секунд с 1970-го года.
Так что, для этого следует использовать 64-битное значение, т.е., time64.

% ? NtQuerySystemTime http://msdn.microsoft.com/en-us/library/windows/desktop/ms724512(v=vs.85).aspx

\label{memcpy}
\section{memcpy()}

Поначалу трудно запомнить порядок аргументов в ф-циях memcpy(), strcpy(). Чтобы было легче, можно представлять
знак ``='' (``равно'') между аргументами.

\label{bzero}
\section{bzero() и memset()}

bzero() это ф-ция просто обнуляющая блок памяти.
Для этого обычно используют memset(). Но у memset() есть неприятная особенность, легко перепутать второй
и третий аргументы местами, и компилятор промолчит, потому что байт для заполнения всего блока задается как int.

К тому же, имя ф-ции bzero легче читается.

С другой стороны, её нет в стандарте POSIX.

\label{printf}
\subsection{printf()}

\index{printf()}
\IFRU{Ограничить длину выводимой строки}{Limit output string by size}:

\begin{lstlisting}
printf ("%.*s", size, buf);
\end{lstlisting}

\IFRU{Иногда бывает необходимость делать отступы в N пробелов или символов табуляции перед каждым
выводимым сообщением}
{Sometimes one need to print identation of N spaces or tabulation symbols before
each printing message}:

\begin{lstlisting}
const char* ALOT_OF_SPACES="                                                     ";
const char* ALOT_OF_TABS="\t\t\t\t\t\t\t\t\t\t\t";

printf ("%.*smessage\n", current_nested_level, ALOT_OF_SPACES);
...
printf ("%.*smessage\n", current_nested_level, ALOT_OF_TABS);
\end{lstlisting}

\subsubsection{\IFRU{Засада}{Caveat} \#1}

\IFRU{Этот код}{This code}:

\index{char}
\begin{lstlisting}
char c=0x80;
printf ("%02X", c);
\end{lstlisting}

... \IFRU{выдает}{dumps} \TT{FFFFFF80} \IFRU{потому что}{because} \IT{char} 
\IFRU{по умолчанию знаковый}{is signed by default} (\IFRU{как в}{as in} \ac{GCC} \IFRU{так и в}{and in} \ac{MSVC}).
\IFRU{При передаче в}{While passing it as argument to} \IT{printf()} 
\IFRU{как аргумента, он знаково расширяется до}{it sign-extended to} \IT{int} \IFRU{и мы видим то что видим}
{and we see what we see}.
\IFRU{Правильно так}{This is correct}:

\begin{lstlisting}
char c=0x80;
printf ("%02X", (unsigned char)c);
\end{lstlisting}

\subsubsection{\IFRU{Свои собственные модификаторы в printf()}{Your own printf() format-string modifiers}}

\IFRU{Часто можно испытать раздражение, когда было бы логично передать в printf(),
скажем, структуру описывающее комплексное
число, или цвет закодированный в структуре из трех чисел типа int}
{It is often irritating when it is logical to pass to printf(), let's say, 
a structure describing complex number, or a color encoded as 3 int numbers as a single entity}.

\index{C++!operator<<}
\index{ToString()}
\IFRU{Эту проблему в Си++ решают определением ф-ции}
{In C++ this problem is usually solved by definition} \TT{operator<<} \InENRU \TT{ostream} 
\IFRU{для своего типа}{for the own type}, \IFRU{либо введением метода с названием}
{or by a method definition named} \TT{ToString()} (\ref{CPPIO}). \\
\\
\index{Linux!printk()}
\IFRU{В}{In} printk() (printf-\IFRU{подобная ф-ция в ядре Linux}{like function in Linux kernel})
\IFRU{имеются дополнительные модификаторы}{there are additional modifiers exist}
\footnote{\url{http://git.kernel.org/cgit/linux/kernel/git/torvalds/linux.git/tree/Documentation/printk-formats.txt}}, 
\IFRU{такие как}{like}
\TT{\%pM} (Mac-\IFRU{адрес}{address}),
\TT{\%pI4} (IPv4-\IFRU{адрес}{address}),
\TT{\%pUb} (\ac{UUID}/\ac{GUID}).

\IFRU{В}{In} GNU Multiple Precision Arithmetic Library \IFRU{есть ф-ция}{there are} gmp\_printf()
\footnote{\url{http://gmplib.org/manual/Formatted-Output-Strings.html}} \IFRU{имеющая нестандартные 
модификаторы нужные для вывода \gls{BigInt}-чисел}{function having non-standard modifiers for
\gls{BigInt}-numbers outputting}. \\
\\
\index{Plan9}
\index{Go}
\IFRU{В \ac{OS}}{In the} Plan9\IFRU{, и в исходниках компилятора Go, можно найти ф-цию}
{ \ac{OS}, and in Go compiler source code, we may find}
fmtinstall()\IFRU{, для объявления нового модификатора printf-строки, например}
{ function for a new printf-string modifier definition, e.g.}:

\begin{lstlisting}[caption=go\textbackslash{}src\textbackslash{}cmd\textbackslash{}5c\textbackslash{}list.c]
void
listinit(void)
{

	fmtinstall('A', Aconv);
	fmtinstall('P', Pconv);
	fmtinstall('S', Sconv);
	fmtinstall('N', Nconv);
	fmtinstall('B', Bconv);
	fmtinstall('D', Dconv);
	fmtinstall('R', Rconv);
}

...

int
Pconv(Fmt *fp)
{
	char str[STRINGSZ], sc[20];
	Prog *p;
	int a, s;

	p = va_arg(fp->args, Prog*);
	a = p->as;
	s = p->scond;
	strcpy(sc, extra[s & C_SCOND]);
	if(s & C_SBIT)
		strcat(sc, ".S");
	if(s & C_PBIT)
		strcat(sc, ".P");
	if(s & C_WBIT)
		strcat(sc, ".W");
	if(s & C_UBIT)		/* ambiguous with FBIT */
		strcat(sc, ".U");
	if(a == AMOVM) {
		if(p->from.type == D_CONST)
			sprint(str, "	%A%s	%R,%D", a, sc, &p->from, &p->to);
		else
		if(p->to.type == D_CONST)
			sprint(str, "	%A%s	%D,%R", a, sc, &p->from, &p->to);
		else
			sprint(str, "	%A%s	%D,%D", a, sc, &p->from, &p->to);
	} else
	if(a == ADATA)
		sprint(str, "	%A	%D/%d,%D", a, &p->from, p->reg, &p->to);
	else
	if(p->as == ATEXT)
		sprint(str, "	%A	%D,%d,%D", a, &p->from, p->reg, &p->to);
	else
	if(p->reg == NREG)
		sprint(str, "	%A%s	%D,%D", a, sc, &p->from, &p->to);
	else
	if(p->from.type != D_FREG)
		sprint(str, "	%A%s	%D,R%d,%D", a, sc, &p->from, p->reg, &p->to);
	else
		sprint(str, "	%A%s	%D,F%d,%D", a, sc, &p->from, p->reg, &p->to);
	return fmtstrcpy(fp, str);
}
\end{lstlisting}
(\url{http://plan9.bell-labs.com/sources/plan9/sys/src/cmd/5c/list.c})

\IFRU{Ф-ция}{The} Pconv() 
\IFRU{будет вызвана если в строке формата будет встречен \%P}{will be called if \%P modifier
in the format string will be met}.
\IFRU{Затем она копирует созданную строку при помощи}
{Then it copies the string created using} fmtstrcpy().
\IFRU{Кстати, эта ф-ция и сама использует другие объявленные модификаторы, такие как}
{By the way, that function also uses other defined modifiers like} \%A, \%D, \IFRU{итд}{etc}. \\
\\
\IFRU{В}{The} \gls{glibc}
\IFRU{есть нестандартное расширение}{has non-standard extension}
\footnote{\url{http://www.gnu.org/software/libc/manual/html_node/Customizing-Printf.html}}, 
\IFRU{позволяющее объявлять свои модификаторы, но это}{allowing to define our own
modifiers, but it is} \IT{deprecated}.

\IFRU{Попробуем определить свои собственные модификаторы для 
Mac-адреса и для вывода байта в бинарном виде}{Let's try to define our own modifiers for Mac-address
outputting and also for byte outputting in a binary form}:

\lstinputlisting{C/register_printf_function.c}
\footnote{\IFRU{Основа для примера взята отсюда}{The base of example was taken from}:
\url{http://codingrelic.geekhold.com/2008/12/printf-acular.html}}

\IFRU{Это компилируется с предупреждениями}{This compiled with warnings}:

\begin{lstlisting}
1.c: In function 'main':
1.c:48:2: warning: 'register_printf_function' is deprecated (declared at /usr/include/printf.h:106) [-Wdeprecated-declarations]
1.c:49:2: warning: 'register_printf_function' is deprecated (declared at /usr/include/printf.h:106) [-Wdeprecated-declarations]
1.c:51:2: warning: unknown conversion type character 'M' in format [-Wformat]
1.c:52:2: warning: unknown conversion type character 'B' in format [-Wformat]
\end{lstlisting}

\ac{GCC} \IFRU{умеет следить за соответствиями модификаторов в}{is able to track accordance between
modifiers in the} printf-\IFRU{строке и аргументами в вызове}{string and arguments in} printf(),
\IFRU{но здесь ему встречаются незнакомые модификаторы, о чем он предупреждает}
{however, unfamiliar to it modifiers are present here, so it warns us about them}.

\IFRU{Тем не менее, наша программа работает}{Nevertheless, our program works}:

\begin{lstlisting}
$ ./a.out
00:11:22:33:44:55
10101011
\end{lstlisting}



\section{atexit()}

При помощи atexit() можно добавить ф-цию, автоматически вызываемую перед выходом из вашей программы.
Кстати, программы на Си++ именно при помощи atexit() добавляют деструкторы глобальных объектов.

Можно попробовать:

\begin{lstlisting}
#include <string>

std::string s="test";

int main()
{
};
\end{lstlisting}

В листинге на ассемблере найдем конструктор этого глобального объекта:

\begin{lstlisting}[caption=MSVC 2010]
??__Es@@YAXXZ PROC					; `dynamic initializer for 's'', COMDAT
; Line 3
	push	ebp
	mov	ebp, esp
	push	OFFSET $SG22192
	mov	ecx, OFFSET ?s@@3V?$basic_string@DU?$char_traits@D@std@@V?$allocator@D@2@@std@@A ; s
	call	??0?$basic_string@DU?$char_traits@D@std@@V?$allocator@D@2@@std@@QAE@PBD@Z ; std::basic_string<char,std::char_traits<char>,std::allocator<char> >::basic_string<char,std::char_traits<char>,std::allocator<char> >
	push	OFFSET ??__Fs@@YAXXZ			; `dynamic atexit destructor for 's''
	call	_atexit
	add	esp, 4
	pop	ebp
	ret	0
??__Es@@YAXXZ ENDP					; `dynamic initializer for 's''

??__Fs@@YAXXZ PROC					; `dynamic atexit destructor for 's'', COMDAT
	push	ebp
	mov	ebp, esp
	mov	ecx, OFFSET ?s@@3V?$basic_string@DU?$char_traits@D@std@@V?$allocator@D@2@@std@@A ; s
	call	??1?$basic_string@DU?$char_traits@D@std@@V?$allocator@D@2@@std@@QAE@XZ ; std::basic_string<char,std::char_traits<char>,std::allocator<char> >::~basic_string<char,std::char_traits<char>,std::allocator<char> >
	pop	ebp
	ret	0
??__Fs@@YAXXZ ENDP					; `dynamic atexit destructor for 's''
\end{lstlisting}

Конструктор, конструируя, также регистрирует деструктор объекта в atexit().

\subsection{bsearch(), lfind()}
\label{bsearch_lfind}

\index{bsearch()}
\index{lfind()}
\IFRU{Удобные ф-ции для поиска чего-либо где-либо}{Handy functions for searching for something somewhere}.

\IFRU{Разница между ними только в том что}{The difference between them is that} lfind() 
\IFRU{просто ищет заданное}{just search for data}, \IFRU{а}{buf} bsearch() \IFRU{требует отсортированный массив
данных, но зато может искать быстрее}{requires sorted data array, but may search faster}
\footnote{\IFRU{методом половинного деления, итд}{by bisection method, etc}}
\footnote{\IFRU{разница еще также в том что bsearch() есть в стандарте}{the difference also is that
bsearch() is present in}\cite{C99TC3}, 
\IFRU{а lfind() нет, он есть только в \ac{POSIX} и в \ac{MSVC}}
{but lfind() is not, it is present only in \ac{POSIX} and \ac{MSVC}},
\IFRU{но эти ф-ции достаточно просты чтобы их реализовать самому}{but these functions are simple enough
to implement them on your own}}.

\IFRU{К примеру, поиск строки в массиве указателей на строки}{For example, search for a string
in the array of pointers to a strings}:

\lstinputlisting{C/lfind.c}

\index{stricmp()}
\IFRU{Свой собственный}{We need our own} stricmp() \IFRU{нужен потому что}{because} lfind() 
\IFRU{будет передавать в него указатель на искомую строку}{will pass a pointer to the string we looking for},
\IFRU{а также на место в массиве где записан указатель на строку, но не сама строка}
{but also to the place where pointer to the string is stored, not the string itself}.
\IFRU{Если бы это был
массив строк фиксированной длины, тогда можно было бы воспользоваться стандартным stricmp()}
{If it would be an array of fixed-size strings, then usual stricmp() could be used here instead}.

\index{qsort()}
\IFRU{Кстати, точно также ф-ция сравнения задается и для}{By the way, likewise comparison function
is used for} qsort().

lfind() \IFRU{возвращает указатель на место в массиве где ф-ция}{returns a pointer to a place in an array
where the} \TT{my\_stricmp()} \IFRU{сработала выдав}{function was triggered returning} $0$.
\IFRU{Далее вычисляем разницу между адресом этого места и адресом начала самого массива}
{Then we compute a difference between the address of that place and the address of the beginning of an array}.
\IFRU{Учитывая арифметику указателей}{Considering pointer arithmetics}(\ref{PtrArith}),
\IFRU{в итоге получается кол-во элементов между этими адресами}
{we get a number of elements between these addresses}.

\IFRU{Реализуя ф-цию сравнивания, можно искать строку в каком угодно массиве}
{By implementing comparison function, we can search a string in any array}.
\index{OpenWatcom}
\IFRU{Пример из}{Example from} OpenWatcom:

\begin{lstlisting}[caption=\textbackslash{}bld\textbackslash{}pbide\textbackslash{}defgen\textbackslash{}scan.c]
int MyComp( const void *p1, const void *p2 ) {

    KeyWord     *ptr;

    ptr = (KeyWord *)p2;
    return( strcmp( p1, ptr->str ) );
}

static int CheckReservedWords( char *tokbuf ) {
    KeyWord     *match;

    match = bsearch( tokbuf, ReservedWords,
                     sizeof( ReservedWords ) / sizeof( KeyWord ),
                     sizeof( KeyWord ), MyComp );
    if( match == NULL ) {
        return( T_NAME );
    } else {
        return( match->tok );
    }
}
\end{lstlisting}

\IFRU{Здесь имеется отсортированный по первому полю массив структур}
{Here we have array of structures sorted by the first field} \IT{ReservedWords},
\IFRU{выглядящий так}{like}:

\begin{lstlisting}[caption=\textbackslash{}bld\textbackslash{}pbide\textbackslash{}defgen\textbackslash{}scan.c]
typedef struct {
    char        *str;
    int         tok;
} KeyWord;

static KeyWord  ReservedWords[] = {
    "__cdecl",          T_CDECL,
    "__export",         T_EXPORT,
    "__far",            T_FAR,
    "__fortran",        T_FORTRAN,
    "__huge",           T_HUGE,
....
};
\end{lstlisting}

bsearch() \IFRU{ищет строку сравнивая её со строкой в первом поле структуры}{searching for the string comparing
it with the string in the first field of structure}.
\IFRU{Здесь можно применить именно bsearch(), потому что массив уже отсортированный}
{bsearch() can be used here because array is already sorted}.
\IFRU{Иначе пришлось бы использовать lfind()}{Otherwise, lfind() should be used}.
\IFRU{Вероятно, несортированный массив можно вначале отсортировать при помощи qsort(), а затем уже
использовать bsearch(), если вам нравится такая идея}
{Probably, unsorted array can be sorted by qsort() before bsearch() usage, if you like the idea}. \\
\\
\IFRU{Точно так же можно искать что угодно, где угодно}
{Likewise, it is possible to search anything anywhere}.
\index{BIND}
\IFRU{Пример из}{The example from} BIND 9.9.1
\footnote{\url{https://www.isc.org/downloads/bind/}}:

\begin{lstlisting}[caption=backtrace.c]
static int
symtbl_compare(const void *addr, const void *entryarg) {
	const isc_backtrace_symmap_t *entry = entryarg;
	const isc_backtrace_symmap_t *end =
		&isc__backtrace_symtable[isc__backtrace_nsymbols - 1];

	if (isc__backtrace_nsymbols == 1 || entry == end) {
		if (addr >= entry->addr) {
			/*
			 * If addr is equal to or larger than that of the last
			 * entry of the table, we cannot be sure if this is
			 * within a valid range so we consider it valid.
			 */
			return (0);
		}
		return (-1);
	}

	/* entry + 1 is a valid entry from now on. */
	if (addr < entry->addr)
		return (-1);
	else if (addr >= (entry + 1)->addr)
		return (1);
	return (0);
}

...

	/*
	 * Search the table for the entry that meets:
	 * entry.addr <= addr < next_entry.addr.
	 */
	found = bsearch(addr, isc__backtrace_symtable, isc__backtrace_nsymbols,
			sizeof(isc__backtrace_symtable[0]), symtbl_compare);
	if (found == NULL)
		result = ISC_R_NOTFOUND;
\end{lstlisting}

\IFRU{Таким образом, можно обойтись без того чтобы писать каждый раз цикл for() для перебирания элементов, итд}
{Thus, it is possible to get rid of for() loop for enumerating elements each time, etc}.



\section{setjmp(), longjmp()}

В каком-то смысле, это реализация исключений в Си.

jmp\_buf это просто структура содержащая в себе набор регистров, но самые важные, это адрес текущей инструкции
и адрес указателя стека.
Всё что делает setjmp() это просто записывает текущие регистры процессора в эту структуру.
А всё что делает longjmp() это восстанавливает состояние регистров.

Это часто используется для возврата из каких-то глубоких мест наружу, обычно, в случае ошибок.
Собственно, как и исключения в Си++.

Например, в Oracle RDBMS, когда происходит некая ошибка, и пользователь код ошибки, сообщение об ошибке, итд,
в реальности, там срабатывает longjmp откуда-то из глубины. Для того же это используется и в bash.

Этот механизм даже немного гибче чем исключения в других ЯП --- нет никаких проблем ``устанавливать точки 
возврата'' в каких угодно местах программы и затем возвращаться туда по мере необходимости.

\label{goto}
Пофантазируя, можно даже сказать что longjmp это такой супермега-goto, обходящий блоки, ф-ции, и восстанавливающий
состояние стека.

Однако, нельзя забывать, что всё что восстанавливает longjmp это регистры. Выделенная память остается выделенной,
никаких деструкторов, как в Си++, вызвано не будет. Никакого \ac{RAII} здесь нет. 
С другой стороны, так как часть стека просто аннулируется, 
память выделенная при помощи alloca()\ref{alloca} будет также аннулирована.



\include{afterword}

\bibliographystyle{alpha}
\bibliography{books,articles,usenet,misc}

\clearpage
\printindex

\end{document}

