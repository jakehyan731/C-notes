\documentclass[11pt,a4paper,oneside]{book}

\usepackage{cmap}

\ifdefined\RUSSIAN
\usepackage[english,russian]{babel}
\usepackage[T2A]{fontenc}
\usepackage{paratype}
\renewcommand*\familydefault{\sfdefault}
% http://www.emerson.emory.edu/services/latex/latex_169.html
\newcommand{\lstlistingsize}{\scriptsize}
\else
\usepackage[russian,english]{babel}
\usepackage[T2A]{fontenc}
\usepackage[default]{sourcesanspro}
\newcommand{\lstlistingsize}{\footnotesize}
\fi

\usepackage[utf8]{inputenc}
\usepackage{listings}
\usepackage{ulem}
\usepackage{url} % \url
\usepackage{graphicx}
\usepackage{listingsutf8}
\usepackage{makeidx}
\usepackage{cite}
\usepackage[cm]{fullpage}
\usepackage{color}
\usepackage{fancyvrb}
\usepackage{xspace}
\usepackage{framed}
\usepackage{ccicons}
\usepackage[nottoc]{tocbibind}
\usepackage{amsmath}
\usepackage[footnote,printonlyused,withpage]{acronym}
\usepackage[table]{xcolor}% http://ctan.org/pkg/xcolor
\usepackage[]{hyperref} % \href. should be last
%\usepackage{tikz}

\definecolor{lstbgcolor}{rgb}{0.94,0.94,0.94}
\makeindex

\newcommand{\TT}[1]{\texttt{#1}}
\newcommand{\IT}[1]{\textit{#1}}
\newcommand{\IFRU}[2]{\iflanguage{russian}{#1}{#2}}


\newcommand{\TITLE}{\IFRU{Заметки о языке программирования Си/Си++}
{C/C++ programming language notes}}
\newcommand{\AUTHOR}{\IFRU{Денис Юричев}{Dennis Yurichev}}
\newcommand{\EMAIL}{dennis@yurichev.com}

\hypersetup{
    pdftex,
    colorlinks=true,
    allcolors=blue,
    pdfauthor={\AUTHOR},
    pdftitle={\TITLE}
    }

\selectlanguage{english}

\lstset{
    backgroundcolor=\color{lstbgcolor},
    basicstyle=\ttfamily\lstlistingsize, 
    breaklines=true,
    frame=single,
    inputencoding=cp1251,
    columns=fullflexible,keepspaces,
}

\newcommand{\COOPname}{\IFRU{Объектно-ориентированное программирование в Си}{Object-oriented programming in C}}
\newcommand{\AndENRU}{\IFRU{и}{and}\xspace}
\newcommand{\OrENRU}{\IFRU{или}{or}\xspace}
\newcommand{\InENRU}{\IFRU{в}{in}\xspace}
\newcommand{\CPP}{\IFRU{Си++}{C++}\xspace}



\begin{document}

\VerbatimFootnotes

\frontmatter

\begin{titlepage}
\begin{center}
\vspace*{\fill}
\LARGE \TITLE

\vspace*{\fill}

\large \AUTHOR

\large \TT{<\EMAIL>}
\vspace*{\fill}
\vfill

\ccbyncnd

\textcopyright 2013, \AUTHOR. 

\IFRU{Это произведение доступно по лицензии Creative Commons «Attribution-NonCommercial-NoDerivs» 
(«Атрибуция — Некоммерческое использование — Без производных произведений») 3.0 Непортированная. 
Чтобы увидеть копию этой лицензии, посетите}
{This work is licensed under the Creative Commons Attribution-NonCommercial-NoDerivs 3.0 Unported License. 
To view a copy of this license, visit} \url{http://creativecommons.org/licenses/by-nc-nd/3.0/}.

\IFRU{Версия этого текста}{Text version} ({\large \today}).

\IFRU{Возможно, более новая версии текста, а так же англоязычная версия, также доступна по ссылке}
{Probably, newer version of this text, and also Russian language version is also accessible at} 
\url{http://yurichev.com/C-book.html}

\IFRU{Вы также можете подписаться на мой twitter для получения информации о новых версиях этого текста, итд:
\href{https://twitter.com/yurichev_ru}{@yurichev\_ru}, либо подписаться на \href{http://yurichev.com/mailing_lists.html}{список рассылки}}
{You may also subscribe to my twitter, to get information about updates of this text, etc: 
\href{https://twitter.com/yurichev}{@yurichev}, or to subscribe to \href{http://yurichev.com/mailing_lists.html}{mailing list}}.
\end{center}
\end{titlepage}

\tableofcontents
\cleardoublepage

\begin{center}
\vspace*{\fill}

\IFRU{Эта страница сдается в аренду для рекламы}{This page can be rented for advertisement}.

\TT{<\EMAIL>}

\vspace*{\fill}
\vfill
\end{center}


\cleardoublepage

\chapter{\IFRU{Введение}{Preface}}

Сейчас, в 2013-м году, если некто желает написать 1) как можно более быстро работающую программу; 2) либо как
можно более компактную для встраиваемых систем либо маломощных микроконтроллеров, то выбор небольшой:
Си, Си++ либо ассемблер. И альтернативы этим старым но популярным языкам, в обозримом будущем, 
пока что не видно. \\
\\
О чистом Си также не стоит забывать, огромное количество больших программ продолжаются писаться на нем, 
например, ядро Linux, ядра линейки Windows NT, Oracle RDBMS, итд.

\chapter{\IFRU{Целевая аудитория}{Target audience}}

Этот сборник заметок предназначен не для начинающих, но и не для экспертов, а скорее для тех, 
кто хочет освежить свои знания по Си/Си++.

\chapter{\IFRU{Об авторе}{About author}}

\IFRU{Денис Юричев ~--- опытный программист, свободный для найма как программист, reverse engineer, консультант, тренер. 
С его резюме можно ознакомиться \href{http://yurichev.com/Dennis_Yurichev.pdf}{здесь}.}
{Dennis Yurichev is experienced programmer, available for hire as programmer, reverse engineer, consultant, trainer. 
His CV is available \href{http://yurichev.com/Dennis_Yurichev.pdf}{here}.}

\chapter{\IFRU{Благодарности}{Thanks}}

\IFRU{Андрей ''herm1t'' Баранович, Слава ''Avid'' Казаков}
{Andrey ''herm1t'' Baranovich, Slava ''Avid'' Kazakov}.

\mainmatter

% only chapters here!

\subsection{\IFRU{Описания (definitions)}{Definitions}}
\subsubsection{\IFRU{Определение строк}{String declarations}}

\paragraph{\IFRU{Последовательности символов используемые в строках}{Character sequences used in strings}}

\begin{center}
\begin{tabular}{ | l | l | l | }
\hline
\textbackslash{}0 & 0x00 & \IFRU{нулевой байт}{zero byte} \\
\hline
\textbackslash{}t & 0x09 & \IFRU{табуляция}{tabulation} \\
\hline
\textbackslash{}n & 0x0A & line feed (LF) \\
\hline
\textbackslash{}r & 0x0D & carriage return (CR) \\
\hline
\end{tabular}
\end{center}

\IFRU{Разница между LF и CR в том, что в старых матричных принтерах,
LF означал протягивание бумаги на одну строку вниз, а CR перевод каретки влево до края бумаги}
{The difference between LF and CR is that in old dot-matrix printers LF mean line feed by one line down,
and CR carriage return to the left margin of paper}.
\IFRU{Так что в принтер нужно было передавать оба символа}{So both characters must be transmitted to the
printer}.

\IFRU{Вывод CR без LF дает возможность перезаписывать текующую строку в консоли}{Outputting CR without LF
allows to rewrite current string in the console}:

\begin{lstlisting}
for (;;)
{
	// do something
	// how much we processed?
	percents=ammount_of_work/total_work*100;
	printf ("%d%% complete\r", percents);
};
\end{lstlisting}

\IFRU{Это часто используется например в архиваторах для отображения текущего статуса, или, например}
{This is often used in the file archivers for displaying current status, or, for example}, wget.

\IFRU{В Си}{In C} \AndENRU UNIX \IFRU{принят LF как символ новой строки}
{LF is traditionally accepted as newline symbol}.

\IFRU{В}{In the} DOS \IFRU{и затем в}{and then} Windows ~--- CR+LF.

\paragraph{\IFRU{Строка определенная в нескольких строках}{The string defined as multi-string}}
\label{heredoc}
\IFRU{Малоизвестная возможность Си, длинные строки можно объявлять так}
{Not widely known C feature, long strings can be defined as}:

\begin{lstlisting}
const char* long_line="line 1"
	"line 2"
	"line 3"
	"line 4"
	"line 5";

...

printf ("Some Utility v0.1\n"
	"Usage: %s parameters\n"
	"\n"
	"Authors:...\n", argv[0]);
\end{lstlisting}

\IFRU{Это отдаленно напоминает}{It is somewhat resembling}
``here document''\footnote{\url{https://en.wikipedia.org/wiki/Here_document}} 
\IFRU{в UNIX-шеллах}{in UNIX-shells} \AndENRU Perl.



\section{\IFRU{Объявления в Си/Си++}{C/C++ declarations}}
\subsection{Объявления переменных внутри ф-ции}

Раньше, в Си можно было объявлять переменные только в начале ф-ции. А в Си++ --- где угодно.

К тому же, нельзя было объявлять итератор в for() (а в Си++ также можно было):

\begin{lstlisting}
for(int i=0; i<10; i++)
	...
\end{lstlisting}

Новый стандарт C99\ref{C99} позволяет делать это.

\subsubsection{static}

Обычно глобальные переменные (или ф-ции) объявляются как static, так их область видимости ограничивается 
данным файлом. Но локальные переменные внутри ф-ции также можно объявлять как static, тогда эта переменная
будет не локальной, а глобальной, но её область видимости будет ограничена только этой ф-цией.

Например:

\begin{lstlisting}
void fn(...)
{
	for(int x=0; x<100; x++)
	{
		static int times_executed = 0;
		times_executed++;
	}
};
\end{lstlisting}

К примеру, это помогло бы для реализации strtok(), ведь этой ф-ции что-то нужно хранить у себя между вызовами.

\label{forwarddeclaration}
\subsection{forward declaration}

\IFRU{Как известно, в заголовочных файлах (headers) обычно содержатся декларации ф-ций, то есть, 
имя ф-ции, аргументы и типы, тип возвращаемого значения, но нет тела ф-ций}
{As it is well-known, in the header files (headers) function declarations are usually present,
i.e., function names, arguments and types, returning type, but no function bodies}.
\IFRU{Так делается для того, чтобы компилятор мог знать, с чем имеет дело,
не углубляясь в тонкости реализации ф-ций}{This is done for the compiler so it will have information,
what it is working with, without delving into the intricacies of function implementations}.

\IFRU{То же самое можно делать и для типов}{The same can be done for types}.
\IFRU{Для того чтобы не включать при помощи \#include файл с описаниями
какого либо класса в другой заголовочный файл, можно обойтись указанием, что он вообще существует}
{In order not to include with the help of \#include the file with a class definitions into
the other header file, one can just declare the type presence}.

\IFRU{Например, вы работаете с комплексными числами и у вас где-то есть такая структура}
{For example, you work with complex numbers and you have a such structure somewhere}:

\begin{lstlisting}
struct complex
{
	double real;
	double imag;
};
\end{lstlisting}

\IFRU{И, например, она определена в файле}{And let's say it is defined in the file} my\_complex.h.

\IFRU{Безусловно, вам нужно включить этот файл, если вы собиретесь работать с переменными типа 
\IT{complex}, с отдельными полями структуры}
{Of course, one should include the file if one have intention to work with variables of \IT{complex} type
and specific structure fields}.
\IFRU{Но если вы описываете свои ф-ции для работы с этой структурой в отдельном заголовочном файле,
то включать там}
{But if you declare your functions using the structure in other header file, you may not include}
my\_complex.h \IFRU{не обязательно, компилятору достаточно просто знать что \IT{complex} это структура}
{there, compiler just needs to know that the \IT{complex} is a structre}:

\begin{lstlisting}
struct complex;

void sum(struct complex *x, struct complex *y, struct complex *out);
void pow(struct complex *x, struct complex *y, struct complex *out);
\end{lstlisting}

\IFRU{Это позволяет увеличить скорость компиляции, а также бороться с циркулярными зависимостями, когда
в двух модулях используются типы и ф-ции друг друга}
{This may speed up the compilation process and also solve circular dependencies, when two modules
uses functions and type of each other}.

 % subsection

\subsection{C++11: auto}

Пользуясь \ac{STL}, иногда надоедает каждый раз объявлять тип итератора вроде:

\begin{lstlisting}
for (std::list<int>::iterator it=list.begin(); it!=list.end(); it++)
\end{lstlisting}

Тип it вполне можно получить из list.begin(), поэтому, в начиная со стандарта C++11, можно использовать auto:

\begin{lstlisting}
for (auto it=list.begin(); it!=list.end(); it++)
\end{lstlisting}


\chapter{\IFRU{Элементы языка Си/Си++}{C/C++ language elements}}
\subsection{\IFRU{Комментарии}{Comments}}

\IFRU{Их иногда удобно вставлять прямо в вызов ф-ции, чтобы где-то на виду держать пометку,
что означает некий аргумент}
{It is sometimes useful to insert them right into a function call, in order to have a visual note
about meaning of an argument}:

\begin{lstlisting}
f (val1, /* a very special flag! */ false, /* another special flag here */ true);
\end{lstlisting}

\IFRU{Целый блок кода можно откомментировать при помощи}
{The whole code block can be commented with the help of} \#if
\footnote{\IFRU{директива препроцессора}{preprocessor directive}}:

\begin{lstlisting}
	ta	= aemif_calc_rate(t->ta, clkrate, TA_MAX);
	rhold	= aemif_calc_rate(t->rhold, clkrate, RHOLD_MAX);
#if 0	
	rstrobe	= aemif_calc_rate(t->rstrobe, clkrate, RSTROBE_MAX);
	rsetup	= aemif_calc_rate(t->rsetup, clkrate, RSETUP_MAX);
	whold	= aemif_calc_rate(t->whold, clkrate, WHOLD_MAX);
#endif	
	wstrobe	= aemif_calc_rate(t->wstrobe, clkrate, WSTROBE_MAX);
	wsetup	= aemif_calc_rate(t->wsetup, clkrate, WSETUP_MAX);
\end{lstlisting}

\IFRU{Это может быть удобнее чем традиционный способ потому что текстовый редактор или \ac{IDE} в этом случае
не ``сломает'' отступы при выравнивании}
{This might be more convenient then usual way because the text editor or \ac{IDE} in this case will not ``break''
indentation while auto-indentation}.


\subsection{goto}

\index{goto}
\IFRU{Использование}{Usage of} \IT{goto}\footnote{statement} 
\IFRU{считается плохим тоном и вредным вообще}{is considered as bad taste and harmful}
\cite{Dijkstra:1968:LEG:362929.362947}\cite{Dijkstra:1979:GSC:1241515.1241518}, 
\IFRU{тем не менее, использование его в разумных дозах}{nevertheless, its usage in reasonable
doses}\cite{Knuth:1974:SPG:356635.356640} \IFRU{может облегчить жизнь}{may be very helpful}.

\IFRU{Частый пример, это выход из функции}{One frequent example is return from a function}:

\begin{lstlisting}
void f(...)
{
	byte* buf1=malloc(...);
	byte* buf2=malloc(...);

	...

	if (something_goes_wrong_1)
		goto cleanup_and_exit;

	...
	
	if (something_goes_wrong_2)
		goto cleanup_and_exit;

	...

cleanup_and_exit:
	free(buf1);
	free(buf2);
	return;
};
\end{lstlisting}

\IFRU{Более сложный пример}{More complex example}:

\begin{lstlisting}
void f(...)
{
	byte* buf1=malloc(...);
	byte* buf2=malloc(...);

	FILE* f=fopen(...);
	if (f==NULL)
		goto cleanup_and_exit;

	...

	if (something_goes_wrong_1)
		goto close_file_cleanup_and_exit;

	...
	
	if (something_goes_wrong_2)
		goto close_file_cleanup_and_exit;

	...

close_file_cleanup_and_exit:
	fclose(f);

cleanup_and_exit:
	free(buf1);
	free(buf2);
	return;
};
\end{lstlisting}

\IFRU{Если в данных примерах отказаться от}{If to remove all} \IT{goto}\IFRU{, то придется вызывать}
{ in these examples, one will need to call} \IT{free()} \AndENRU \IT{fclose()}
\IFRU{перед каждым выходом из функции}{before each return from the function} 
(\IT{return})\IFRU{, что здорово замусорит весь код}{which adds a lot of mess}.

\index{Linux}
\IFRU{Использование}{Usage of} \IT{goto} 
\IFRU{в таких случаях одобряется, например, в}{is, for example, approved in} \cite{LinuxKernelCodingStyle}.

%Примеры более ``harmful'' но эффективного использования goto можно найти в исходниках nginx.
% example?


\subsection{for}

\IFRU{В}{The} for()\IFRU{, как известно, три выражения:
первое вычисляется перед началом всех итераций,
второе вычисляется перед каждой итерацией,
третье ~--- после каждой итерации}
{ statement, as we know, has 3 expressions:
1st computing before all iterations begin,
2nd computing before each iteration
and the 3rd ~--- after each iteration}.

\IFRU{И конечно же, там можно указывать что-то отличное от обычного счетчика}
{And of course, there might be written something different from the usual counter}.

\subsubsection{\IFRU{Засада}{Caveat} \#1}

\IFRU{Если написать такое}{If to write this}:

\lstinputlisting{common/for_strlen.cpp}

... \IFRU{то это наверное будет ошибкой}{perhaps this is a mistake}:
\TT{strlen(s)} \IFRU{будет вызываться перед каждой итерацией}{will be called before each iteration} 
~--- \IFRU{такой код генерирует}{that is the code} MSVC 2010\IFRU{}{ generated}.
\IFRU{Впрочем}{However}, GCC 4.8.1 \IFRU{вызывает}{calls} \TT{strlen(s)} 
\IFRU{только один раз, в начале цикла}{only once, at the loop beginning}.

\subsubsection{\IFRU{Запятая}{Comma}}

\IFRU{Запятая}{Comma}\cite[6.5.17]{C99TC3} ~--- \IFRU{не самая понятная для всех штука в Си, 
однако, их очень удобно использовать в определениях в for()}
{is not widely understood C feature, however, it is very useful for using in a for() declarations}.

\IFRU{Например, может пригодится использовать в цикле два итератора одновременно}
{For example, it is useful to have two iterators simultaneously}.
\IFRU{Пусть один просто отсчитывает от 0, прибавляя 1 при каждой итерации,
а второй итератор указывает на элемент в списке}
{Let the first iterator just counts from 0 adding 1 at each iteration, and the second iterator
points to the list element}:

\lstinputlisting{common/for_comma.cpp}

\IFRU{Это выдаст предсказуемый результат}{This will dumps predictable result}:

\begin{lstlisting}
0: 123
1: 456
2: 789
3: 1
\end{lstlisting}

\IFRU{Но к сожалению, определять итераторы вместе с типами в теле самого for() вот так нельзя}
{However, it is not possible to declare iterators with its types in for() clause}:

\begin{lstlisting}
	for (int i=0, std::list<int>::iterator it=l.begin(); it!=l.end(); i++, it++)
\end{lstlisting}

\subsubsection{continue}

\IT{continue} \IFRU{это безусловный переход на конец тела цикла}{is unconditional goto to the end
of loop body}.

\IFRU{Это может быть очень полезно, например, в подобном коде}
{This may be very useful, for example, in such code}:

\begin{lstlisting}
for (...)
{
	if (is_element_satisfied_criteria_1(...)==true)
	{
		// do something need in is_element_satisfied_criteria_2()

		if (is_element_satisfied_criteria_2(...)==true)
		{
			do_something_1();
			do_something_2();
			do_something_3();
		};

	};
};
\end{lstlisting}

... \IFRU{всё это можно легко заменить на более опрятное}{it is all can be replaced by neat}:

\begin{lstlisting}
for (...)
{
	if (is_element_satisfied_criteria_1(...)==false)
		continue;

	// do something need in is_element_satisfied_criteria_2()

	if (is_element_satisfied_criteria_2(...)==false)
		continue;

	do_something_1();
	do_something_2();
	do_something_3();
};
\end{lstlisting}


\subsection{sizeof}

\IFRU{Обычно}{Usually}, sizeof() \IFRU{применяют к \glslink{integral type}{интегральным типам}}
{is applied to \glslink{integral type}{integral types}}
\IFRU{либо к структурам}{or to structures},
\IFRU{тем не менее, его можно применять и к массивам, к примеру}
{but nevertheless it is possible to apply it to arrays as well}:

\begin{lstlisting}
	char buf[1024];
	snprintf(buf, sizeof(buf), "...");
\end{lstlisting}

\IFRU{В противном случае, если указывать длину массива}{Otherwise, if to specify array length} ($1024$) 
\IFRU{в двух местах}{in both places} 
(\IFRU{в определении \TT{buf} и как второй аргумент \TT{snprintf()}}
{in \TT{buf} declaration and as a second argument of \TT{snprintf()}}),
\IFRU{то и изменять это значение придется каждый раз в обоих местах, а об этом легко забыть}
{then the value is have to be changed at the both places each time, and it is easy to forget about this}.

\IFRU{Если нужны wide-строки, то}{If one need wide-strings, then} sizeof() 
\IFRU{можно применять к}{can be applied to} \IT{wchar\_t} 
(\IFRU{который, на самом деле, 16-битный тип данных \IT{short}}
{which is in turn, 16-bit data type \IT{short}}):

\begin{lstlisting}
	wchar_t buf[1024];
	swprintf(buf, sizeof(buf)/sizeof(wchar_t), "...");
\end{lstlisting}

sizeof() \IFRU{возвращает длину в байтах, так что здесь он будет равен}{returns the size in bytes, so it will
be here} $1024*2$, \IFRU{т.е.}{i.e.}, $2048$. 
\IFRU{Но мы можем разделить это значение на длину одного элемента массива}
{But we can divide this value by length of one array element} (\IT{wchar\_t})
\IFRU{в байтах ($2$)}{is $2$ in bytes},
\IFRU{чтобы получить количество элементов в массиве}{in order to get elements number in array} ($1024$).

sizeof() \IFRU{можно применять и к массивам структур}{can be applied to array of structures}:

\begin{lstlisting}
struct phonebook_entry
{
	char *name;
	char *surname;
	char *tel;
};

struct phonebook_entry phonebook[]=
{
	{ "Kirk", "Hammett", "555-1234" },
	{ "Lars", "Ulrich", "555-5678" },
	{ "James", "Hetfield", "555-1122" },
	{ "Robert", "Trujillo", "555-7788" }
};

void dump (struct phonebook_entry* input)
{
	for (int i=0; i<sizeof(phonebook)/sizeof(struct phonebook_entry); i++)
		printf ("%s %s - %s\n", input[i].name, input[i].surname, input[i].tel);
};
\end{lstlisting}

sizeof(phonebook) ~--- \IFRU{это размер всего массива структур в байтах}{is a size of the whole array
of structures in bytes}.
\TT{sizeof(struct phonebook\_entry)} ~--- \IFRU{это размер одной структуры в байтах}{is a size of one structure
in bytes}.
\IFRU{Делением мы узнаем количество структур в массиве}{By division we get number of structures in an array}.


\section{\IFRU{Указатели}{Pointers}}

Как однажды сказал Дональд Кнут в интервью\cite{KnuthInterview1993}, то как в Си устроены указатели, это является
очень удачной инновацией в языках программирования по тем временам.

Итак, определимся с терминологией. Указатель это просто адрес какого-то элемента в памяти. Указатели настолько
популярны, потому что в какую-то функцию намного проще передать просто адрес объекта в памяти, вместо того
чтобы передавать весь объект --- ведь это абсурдно. К тому же, вызываемая функция, например, обрабатывающая
ваш массив данных, просто изменит что-то в нем, вместо того чтобы возвращать новый, измененный массив данных, 
что тоже абсурдно.

Возьмем простой пример. Стандартная функция \IT{strtok()} делит строку на подстроки, используя заданный символ
как разделитель. К примеру, мы можем подать на вход строку \TT{The quick brown fox jumps over the lazy dog} 
и задать пробел в качестве разделителя.

\lstinputlisting{src/strtok_ex1.c}

Мы в итоге получим на выходе:

\begin{lstlisting}
The
quick
brown
fox
jumps
over
the
lazy
dog
\end{lstlisting}

Что тут в реальности происходит, это то что ф-ция \IT{strtok()} просто находит в заданной строке следующий пробел 
(либо иной заданный разделитель),
записывает туда $0$ (что по соглашениям текстовых строк в Си является концом строки) и возвращает указатель на это
место.

В качестве недостатка \IT{strtok()} можно отметить, что эта ф-ция ``портит'' входную строку, записывая нули
на месте разделителей.

Но вот что важно заметить: никакие строки или подстроки не копируются в памяти. Входная
строка остается там же где и лежала. В \IT{strtok()} передается только указатель на нее, или, её адрес.
Эта ф-ция затем, после
того как записывает $0$, возвращает \IT{адрес} каждого следующего ``слова''.
Адрес затем подается на вход в \IT{printf()}, где происходит его вывод на консоль.

Обратите также внимание на то что в исходнике присутствует и некорректное объявление \IT{str}. Оно тем некорректное
что в Си строка имеет тип \IT{const char*}, то есть, распологается в константном сегменте данных, защищенным
от записи.
Если так сделать, то \IT{strtok()} не сможет модифицировать строку записывая туда нули и процесс ``упадет''.

Так что, в нашем примере, строка выделяется как массив \IT{char} а не массив \IT{const char}.

Обобщая, скажем что работа со строками в Си происходит только лишь используя адреса этих строк. К примеру,
ф-ция сравнения строк \IT{strcmp()} на вход берет два адреса двух строк и по одному символу сравнивает их.
Было бы очень абсурдно копировать куда-то эти две строки лишний раз, чтобы \IT{strcmp()} обработала их.

Трудность понимания указателей в Си связана с тем, что указатель это ``часть'' объекта. Указатель на строку,
это не сама строка. Сама строка еще должна где-то в памяти хранится, под нее нужно выделять место, итд.

В ЯП более высокого уровня, объект и указатель на него могут быть представлены как единое целое, что облегчает
понимание.
Впрочем, это не значит что в этих ЯП строки и иные объекты неразумно копируются много раз при передаче 
в другие функции,
там точно так же используются указатели, но просто эта механика скрыта от программиста.

\subsection{Синтаксический сахар для a[i]}

Ради упрощения, можно сказать что в Си нет массивов вообще, а есть только синтаксический сахар для выражений
вроде \IT{a[i]}.

К примеру, возможно вы видели такой трюк:

\begin{lstlisting}
printf ("%c", 3["hello"]);
\end{lstlisting}

Это выдаст 'l'. 
Это происходит, потому что любое выражение \IT{a[i]}, на самом деле преобразовывается в \IT{*(a+i)}
\cite[6.5.2.1]{C99TC3}.
\IT{3["hello"]} преобразовывается в \IT{*(3+"hello")}, а \IT{"hello"} это просто указатель на массив символов, 
типа \IT{const char*}.
\IT{3+"hello"} это в итоге указатель на часть строки, то есть, \IT{"lo"}. А \IT{*("lo")} это cимвол 'l'. 
Вот почему это работает.

Но так врядли стоит писать, если вы конечно не готовите программу на конкурс 
The International Obfuscated C Code Contest\footnote{\url{http://www.ioccc.org/}}.
Так что я привел этот пример, чтобы наглядно показать, 
что выражения вроде \IT{a[i]} это синтаксический сахар.

При некотором упорстве, в Си вообще можно обойтись без индексации массивов, хотя выглядеть это будет не очень
эстетично.

Кстати, так легко понять как работают отрицательные индексы массивов. \IT{a[-3]} просто преобразуется в \IT{*(a-3)},
так адресуется элемент лежащий перед самим массивом.
И хотя это вполне возможно, так можно делать только если вы точно знаете, что вы делаете.

В Си массив это, в каком-то смысле, это просто место в памяти под массив плюс указатель, указывающий
на него. 

Вот почему имя массива в Си можно считать за указатель:

Если вы объявите глобальную переменную \IT{int a[10]}, то \IT{a} будет иметь тип \IT{int*}.
Позже, когда где-то в коде
вы укажете \IT{x=a[5]}, выражение будет преобразовано в \IT{x=*(a+5)}. От начала массива (то есть, первого элемента
массива), будет отсчитано 5 элементов, затем оттуда прочитается элемент для записи в \IT{x}.

\subsection{\IFRU{Арифметика указателей}{Pointer arithmetic}}

Простой пример:

\lstinputlisting{src/phonebook1.c}

Мы объяляем глобальный массив из структур. Если скомпилировать это в GCC с ключом \IT{-S} либо в MSVC с ключом
\IT{/Fa}, мы увидим листинг на ассемблере и то, как компилятор расположил эти строки. 

Расположил он их как линейный массив указателей на строки, вот так:

\begin{center}
\begin{tabular}{ | l | l | }
\hline
  ячейка 0    & адрес строки ``Kirk'' \\
  ячейка 1    & адрес строки ``Hammett'' \\
  ячейка 2    & адрес строки ``555-1234'' \\
  ячейка 3    & адрес строки ``Lars'' \\
  ячейка 4    & адрес строки ``Ulrich'' \\
  ячейка 5    & адрес строки ``555-5678'' \\
  ячейка 6    & адрес строки ``James'' \\
  ячейка 7    & адрес строки ``Hetfield'' \\
  ячейка 8    & адрес строки ``555-1122'' \\
  ячейка 9    & адрес строки ``Robert'' \\
  ячейка 10   & адрес строки ``Trujillo'' \\
  ячейка 11   & адрес строки ``555-7788'' \\
  ячейка 12   & 0 \\
  ячейка 13   & 0 \\
  ячейка 14   & 0 \\
\hline
\end{tabular}
\end{center}

Ф-ции \IT{dump1()} и \IT{dump2()} эквивалентны.

Но в первой итератор \IT{i} начинается с 0 и к нему прибавляется 1 на каждой итерации.

Во второй ф-ции итератор \IT{i} указывает на начало массива и затем, к нему прибавляется длина структуры 
(а не 1 байт, как можно поначалу ошибочно подумать),
таким образом, на каждой итерации, \IT{i} указывает на следующий элемент массива.


\subsection{\IFRU{Операторы}{Operators}}

\subsubsection{==}

\IFRU{Очень неприятные ошибки возникают если в условии}
{Somewhat unpleasant mistakes may appear if in} \IT{if(a==3)} 
\IFRU{опечататься и написать}{condition become} \IT{if(a=3)}\IFRU{}{ in result of typo}.
\IFRU{Ведь выражение}{Because the statement} \IT{a=3} ``\IFRU{возвращает}{returns}'' 3,
\IFRU{а}{and} 3 \IFRU{это не}{is not a} 0, \IFRU{поэтому условие \IT{if()} всегда будет 
срабатывать}{so the \IT{if()} condition will always trigger}.

\IFRU{Раньше, для защиты от подобных ошибок, была мода писать наоборот}
{It was fashionable in past to protect from such mistakes by writing}: \IT{if(3==a)}, 
\IFRU{таким образом}{and thus},
\IFRU{если опечататься, выйдет}{we will get a} \IT{if(3=a)}\IFRU{, компилятор тут же выдаст ошибку}
{ in case of typo and the compiler will report error instantly}.

\IFRU{Тем не менее, в наше время, компиляторы обычно предупреждают если написать}
{Nevertheless, in modern times, compilers are usually warns if to write} \IT{if(a=3)}, 
\IFRU{так что, наверное, менять местами элементы выражения уже не обязательно}
{so elements swapping in conditions is probably not necessary these days}.

\subsubsection{Short-circuit evaluation 
\IFRU{и артефакт приоритетов операций}{and operator precedence artefact}}

\IFRU{Разберем что такое}{Let's see what is} \IT{short-circuit}
\IFRU{\footnote{дословный перевод на русский: ``короткое замыкание''}} \IT{evaluation}.

\IFRU{Это когда в выражении}{It is when in the expression} \IT{if(a \&\& b \&\& c)},
\IFRU{часть}{the part} \IT{(b)} \IFRU{будет вычисляться только если}{will be calculated only if} 
\IT{(a)} ~--- \IFRU{истинна}{is true},
\IFRU{а}{and} \IT{(c)}
\IFRU{будет вычисляться только если}{will be calculated only if} \IT{(a)} \AndENRU \IT{(b)} 
~--- \IFRU{оба истинны}{are both true}.
\IFRU{И вычисляться они будут именно в таком порядке, как указано}{and they will be computed
exactly in the same order as specified}.

\IFRU{Иногда можно встретить подобное}{Sometimes we can see expression like}: 
\IT{if (p!=NULL \&\& p->field==123)} ~--- \IFRU{и это совершенно правильно}{and this is completely
correct}.
\IFRU{Поле}{The field} \IT{field} \IFRU{в структуре, на которую указывает}{in the structure to which} 
\IT{(p)}\IFRU{, будет вычисляться только если указатель}{ points will be computed only if the pointer}
\IT{(p)} \IFRU{не равен}{not equals to} \IT{NULL}.

\IFRU{То же касается и операции}{The same story about} ``\IFRU{или}{or}'', 
\IFRU{если в выражении}{if in the expression} \IT{if (a || b || c)} \IFRU{подвыражение}{subexpression} 
\IT{(a)} \IFRU{будет ``истинно''}{will be ``true''},
\IFRU{остальные вычисляться не будут}{others will not be computed}.

\IFRU{Это может быть удобно для вызова нескольких ф-ций}{It is useful when one need to call several
functions}:
\IT{if (get\_flagsA() || get\_flagsB() || get\_flagsC())} ~--- 
\IFRU{если первая или вторая ф-ция вернет}{if first or second will return} \IT{true}, 
\IFRU{остальные даже не будут вызываться}{others will not be called at all}.

\IFRU{Эта особенность есть не только в Си/Си++}{This feature is not unique for C/C++}
\footnote{\IFRU{Здесь список}{Here is a list of} \ac{PL} \IFRU{где присутствует}{where}
\IT{short-circuit evaluation}\IFRU{}{ exist}\url{https://en.wikipedia.org/wiki/Short-circuit_evaluation}.
\IFRU{Кстати, хотя это и не про Си, но все же интересно}{It is not about C, but interesting nevertheless}:
\IFRU{в bash если писать}{if to write in bash} \IT{cmd1 \&\& cmd2 \&\& cmd3}, 
\IFRU{то каждая следующая команда будет исполняться только если предыдущая закончилась с успехом}
{then each next command will be executed only if the previous was executed with success}.
\IFRU{Это также}{It is also} \IT{short-circuit}.}.

\IFRU{Когда-то давно}{Some time ago}\cite{dmr:1995}, 
\IFRU{в языках B и BCPL (предтечи Си) не было операторов}{there was no operators} 
\IT{\&\&} \AndENRU \IT{||}\IFRU{}{ in B and BCPL (C precursors)}, 
\IFRU{но чтобы реализовать в них}{but in order to implement}
\IT{short-circuit evaluation}\IFRU{}{ in them}, 
\IFRU{приоритет операций}{the priority of the operators} \IT{\&} \AndENRU \IT{|} 
\IFRU{сделали больше, чем, например, у}{was made higher than in} \IT{\^} \OrENRU \IT{+}
\footnote{\IFRU{Приоритет операций в Си++}{C++ Operator Precedence}: \url{http://en.cppreference.com/w/cpp/language/operator_precedence}}.

\IFRU{Это позволяло писать что-то вроде}{That allowed to write something like} \IT{if (a==1 \& b==c)} 
\IFRU{используя}{while using} \IT{\&} \IFRU{вместо}{instead of} \IT{\&\&}.
\IFRU{Вот откуда взялся этот артефакт в приоритетах}{That is where that artefact came from}. \\
\\
\IFRU{Так что, нередкая ошибка это забывать о высоком приоритете этих операций и писать, например}
{So one often mistake is to forget about higher priority of these operators and to write e.g.},
\IT{if (a\&1==0)}, \IFRU{в то время как это нужно брать в скобки}{which should be taken
in brackets}: \IT{if ((a\&1)==0)}.

\subsubsection{! \AndENRU \~{}}

\~{} (\IFRU{тильда}{tilde}) \IFRU{это побитовое инвертирование всех бит в значении}
{is a bitwise inversion of all bits in a value}.

\IFRU{Эта операция часто используется для инвертирования результатов действия ф-ций}
{The operation is often used for function results invertion}.
\IFRU{Например}{For example}, strcmp() \IFRU{в случае равенства строк возвращает}{in case of strings
equivalence, returns} 0.
\IFRU{Поэтому можно писать}{So we can write}:

\begin{lstlisting}
if (!strcmp(str1, str2))
{
	// do something in case of strings equivalence
};
\end{lstlisting}

... \IFRU{вместо}{instead of} \TT{if (strcmp (...)==0)}. \\
\\
\IFRU{Также, два подряд восклицательных знака применяется для трасформирования любого значения в тип bool
по правилу}{Also, two consecutive exclamation points can be used for
transforming any value into \IT{bool} type}: 0 ~--- false (0); \IFRU{не ноль}{not zero} ~--- true (1).

\IFRU{Например}{For example}:

\begin{lstlisting}
bool some_object_present=!!struct->object;
\end{lstlisting}

\IFRU{Или}{Or}:

\begin{lstlisting}
#define FLAG 0x00001000
bool FLAG_present=!!(value & FLAG);
\end{lstlisting}

\IFRU{А также}{And also}:

\begin{lstlisting}
bool bit_7_set=!!(value & (1<<7));
\end{lstlisting}


\subsection{\IFRU{Массивы}{Arrays}}

\IFRU{В}{In} C99(\ref{C99})
\IFRU{можно передавать массив в аргументах ф-ции}{it is possible to pass array in the function arguments}.

\IFRU{Собственно, массив байт можно было передавать и в более старых стандартах Си,
кодируя байты в строке, включая ноль, примерно так}
{Strictly speaking, array of bytes can be passed in the older C standards, by encoding all bytes
including zero in a string}
(\IFRU{узнать, встречается ли байт}{let's determine if the byte} (\IT{c})
\IFRU{в массиве байт}{is present in a byte array})(\ref{memchr}):

\begin{lstlisting}
if (memchr ("\x12\x34\x56\x78\x00\xAB", c, 6))
	...
\end{lstlisting}

\IFRU{Байты после ноля нормально кодируются}{The bytes after zero is encoded finely}.

\IFRU{Но в C99 теперь можно передавать массив значений других типов, например}
{However, it is possible in C99 to pass an array of other types, like}
unsigned int:

\begin{lstlisting}
unsigned int find_max_value (unsigned int *array, size_t array_size);

unsigned int max_value=find_max_value ((unsigned[]){ 0x123, 0x456, 0x789, 0xF00 }, 4);
\end{lstlisting}

\IFRU{Поиск в массиве можно реализовать при помощи ф-ций}
{Search for the element in the array can be implemented with the help of} bsearch() \OrENRU lfind()(\ref{bsearch_lfind}),
\IFRU{поиск и вставку при помощи}{search and insertion with the help of} lsearch()
\footnote{\IFRU{работает также как и}{works like} lfind(), 
\IFRU{но при отсутствии искомого элемента, добавляет его в массив}
{but if the element is absent there, it also inserts it}}.

\subsubsection{\IFRU{Инициализация}{Initialization}}

\IFRU{В GCC можно}{It is possible in GCC}
\footnote{\url{http://gcc.gnu.org/onlinedocs/gcc/Designated-Inits.html}} \IFRU{инициализировать части массивов}
{to initialize array parts}:

\begin{lstlisting}
struct a
{
	int f1;
	int f2;
};
 
struct a tbl[8] =
{
[0x03] =	{ 1,6 },
[0x07] =	{ 5,2 } 
};
\end{lstlisting}

... \IFRU{но это нестандартное расширение}{but it is non-standard extension}.


\section{struct}

В современных x86-микропроцессорах (как Intel, так и AMD) имеется кеш-память разных уровней. 
Самая быстрая кеш-память (первого уровня),
разделена на 64-байтные элементы (кеш-линии) и любое обращение к памяти заполняет сразу всю линию.

Можно сказать, что любое обращение к памяти (по выровненным адресам) подтягивает в кеш сразу 64 байта.

Поэтому, если некая структура данных имеет размер более 64-х байт, очень важно разделить её на две части:
наиболее востребованные поля и менее востребованные. 
Самые востребованные поля желательно разместить в пределах первых 64-х байт. \\
\\
В C99\ref{C99} можно инициализировать отдельные поля структур. Пропущенные будут заполнены нулями. Такого очень
много в ядре Linux. 

\begin{lstlisting}
struct color
{
	int R;
	int G;
	int B;
};

struct color blue={ .B=255 };
\end{lstlisting}


И даже более того, можно создавать структуру прямо в аргументах ф-ции, например:

\begin{lstlisting}
struct color
{
	int R;
	int G;
	int B;
};

void print_color_info (struct color *c)
{
	printf ("%d %d %d\n", c->R, c->G, c->B);
};

int main()
{
	print_color_info(&blue);
	print_color_info(&(struct color){ .G=255 });
};
\end{lstlisting}

Помимо всего прочего, о структурах также много есть в разделе ``\COOPname''\ref{COOP}.


\section{union}

union часто используется, когда в каком-то месте структуры можно хранить разные типы на выбор.
К примеру:

\begin{lstlisting}
union
{
	int i; // 4 bytes
	float f; // 4 bytes
	double d; // 8 bytes
} u;
\end{lstlisting}

Такой union позволяет хранить одну из этих трех переменных на выбор. Занимать он будет места столько же,
сколько максимальный элемент (double) --- 8 байт.

union часто используют для обращения к какому-то типу данных как к другому.

Например, как известно, каждый XMM-регистр в SSE может представлять собой 16 байт, 8 16-битных слов,
4 32-битных слова, 2 64-битных слова, 4 float-значения и 2 double-значения. Так можно описать его:

\begin{lstlisting}
union
{
	double d[2];
	float f[4];
	uint8_t b[16];
	uint16_t w[8];
	uint32_t i[4];
	uint64_t q[2];
} XMM_register;

union XMM_register reg1;

reg.u.d[0]=123.4567;
reg.u.d[1]=89.12345;

// here we can use reg.u.b[...]

\end{lstlisting}

Это также очень удобно использовать вместе со структурой, где поля имеют битовую гранулярность.
Это флаги x86-процессора:

\begin{lstlisting}
typedef struct _s_EFLAGS
{
    unsigned CF : 1;
    unsigned reserved1 : 1;
    unsigned PF : 1;
    unsigned reserved2 : 1;
    unsigned AF : 1;
    unsigned reserved3 : 1;
    unsigned ZF : 1;
    unsigned SF : 1;
    unsigned TF : 1;
    unsigned IF : 1;
    unsigned DF : 1;
    unsigned OF : 1;
    unsigned IOPL : 2;
    unsigned NT : 1;
    unsigned reserved4 : 1;
    unsigned RF : 1;
    unsigned VM : 1;
    unsigned AC : 1;
    unsigned VIF : 1;
    unsigned VIP : 1;
    unsigned ID : 1;
} s_EFLAGS;

typedef union _u_EFLAGS
{
    uint32_t flags;
    s_EFLAGS s;
} u_EFLAGS;
\end{lstlisting}

Можно таким образом загрузить флаги как 32-битное значение в поле flags, а затем из поля s обращаться
к отдельным битам. Либо наоборот, модифицировать биты, затем прочитать поле flags.

\subsection{tagged union}

Это union плюс флаг (tag), определяющий тип union. К примеру, если нам нужна какая-то переменная,
которая может быть как числом, так и числом с плавающей точкой, так и текстовой строкой (как переменные
в динамически-типизированных ЯП), то мы можем объявить такую структуру:

\begin{lstlisting}

enum var_type
{
	INT,
	DOUBLE,
	STRING
};

struct
{
	enum var_type tag; // 4 bytes
	union
	{
		int i; // 4 bytes
		double d; // 8 bytes
		char *string; // 4 bytes (on 32-bit architecture)
	} u;
} variable;
\end{lstlisting}

Суммарная длина такой структуры будет $8+4=12$ байт. В любом случае, это компактнее, чем выделять
поля для переменной каждого возможного типа.



\section{\IFRU{Препроцессор}{Preprocessor}}

Препроцессор обрабатывает директивы начинающиеся с \# --- \#define, \#include, \#if, итд.

\section{Стандартные для компиляторов и ОС значения}

\begin{itemize}
\item \TT{\_DEBUG} --- отладочная сборка.
\item \TT{NDEBUG} --- неотладочная (release) сборка.
\item \TT{\_\_linux\_\_} --- ОС Linux.
\item \TT{\_WIN32} --- ОС Windows. Присутствует как и в x86-проектах, так и в x64.
\item \TT{\_WIN64} --- Присутствует в x64-проектах для ОС Windows.
\item \TT{\_\_cplusplus} --- присутствует в Си++ проектах.
\item \TT{\_MSC\_VER} --- компилятор MSVC.
\item \TT{\_\_GNUC\_\_} --- компилятор GCC.
\end{itemize}

Так можно писать разные участки кода для разных компиляторов и ОС.

\subsection{``Пустой'' макрос}

Всем известны макросы не объявляющие никаких значений, например \IT{\_DEBUG}.
Обычно, только проверяется наличие его или отсутствие.
Вот еще пример полезного ``пустого'' макроса:

В заголовочных файлах Windows API мы можем увидеть такое:

\begin{lstlisting}
typedef NTSTATUS
(NTAPI *TDI_REGISTER_CALLBACK)(
  IN PUNICODE_STRING DeviceName,
  OUT HANDLE *TdiHandle);

...

typedef NDIS_STATUS
(NTAPI *CM_CLOSE_CALL_HANDLER)(
  IN NDIS_HANDLE  CallMgrVcContext,
  IN NDIS_HANDLE  CallMgrPartyContext  OPTIONAL,
  IN PVOID  CloseData  OPTIONAL,
  IN UINT  Size  OPTIONAL);
\end{lstlisting}

IN, OUT и OPTIONAL --- это ``пустые'' макросы объявленные так:

\begin{lstlisting}
#ifndef IN
#define IN
#endif
#ifndef OUT
#define OUT
#endif
#ifndef OPTIONAL
#define OPTIONAL
#endif
\end{lstlisting}

Для компилятора они никакой информации не несут, они предназначены только для документирования, показать,
какие параметры ф-ций зачем нужны.

\subsection{Частые ошибки}

\subsubsection{\#1}

К примеру, вы хотите создать макрос для возведения числа в квадрат:

\begin{lstlisting}
#define square(x)      x*x
\end{lstlisting}

Это ошибка, потому что выражение \IT{square(a+b)} в итоге ``развернется'' в $a+b*a+b$, что, разумеется, совсем
не то что хотелось. Поэтому в определнии макроса все переменные, и сам макрос, нужно ``изолировать'' скобками:

\begin{lstlisting}
#define square(x)      ((x)*(x))
\end{lstlisting}

Пример из файла minmax.h из MinGW:

\begin{lstlisting}
#define max(a,b) (((a) > (b)) ? (a) : (b))
...
#define min(a,b) (((a) < (b)) ? (a) : (b))
\end{lstlisting}

\subsubsection{\#2}

Если вы где-то определяете какую-то константу:

\begin{lstlisting}
#define N 1234
\end{lstlisting}

... затем где-то дальше переопределяете её снова, то компилятор промолчит, и это приведет к трудновыявляемой
ошибке.

Поэтому константы желательнее определять как глобальные переменные с модификатором const.



\label{C99}
\chapter{Стандарт Си C99}

Текст стандарта: \cite{C99TC3}.

Этот стандарт поддерживается в GCC, но не в MSVC, и не ясно, будет ли он там поддерживаться вообще.

Чтобы включить его поддержку в GCC, нужно добавить ключ компиляции \IT{-stc=c99}.

\chapter{Работа с памятью Си}

Есть наверное только два основных типа в памяти, предоставляемых программисту на Си.

\begin{itemize}
\item
Память выделяемая в локальном стеке. Это локальные переменные, память выделенная при помощи alloca().
Обычно это очень быстро выделяемая память.

\item
Куча\footnote{heap}. То что выделается при помощи malloc().
\end{itemize}

\section{Локальный стек}

Когда вы объявляете что-то вроде \TT{char a[1024]}, выделения памяти как такового не происходит, происходит
просто перемещение указателя стека на 1024 байта назад\cite[1.2.3]{REBook}. Это очень быстрая операция.

Освобождать эту память никак не надо, в конце работы ф-ции, это происходит автоматически, с возвратом указателя стека.

В качестве обратной стороны медали, вам нужно знать зараннее, сколько места нужно выделить, а также, размер
этого блока нельзя изменить, освободить и выделить заново его также нельзя.


\subsection{alloca()}

Ф-ция alloca() выделяет память в локальном стеке точно также, отодвигая указатель стека\cite[1.2.4]{REBook}.
Память будет освобождена в конце ф-ции автоматически.

В стандарте C99, использовать alloca() уже не обязательно, там можно просто писать:

\begin{lstlisting}
void f(size_t s, ...)
{
	char a[s];
};
\end{lstlisting}

Впрочем, внутри, это работает так же как и alloca().

Критика: Линус Торвальдс против использования alloca()\cite{Torvalds:2003}.


\subsection{Выделение памяти в куче}

Куча (heap) это какая-то часть памяти выделенная ОС процессу, где процесс может уже сам делить эту часть как хочет.
После заврешения процесса, в т.ч., некорректного, куча автоматически аннулируется и ОС не нужно разбирать 
по одному все выделенные процессом блоки.

Для работы с кучей есть ф-ции malloc(), calloc(), realloc(), free(), а в Си++ --- new и delete.

Очевидно, чтобы поддерживать информацию о выделенных блоках в куче, нужна масса связных друг с другом структур.
Отсюда имеется вполне осязаемые накладные расходы (overhead). Вы можете выделить блок размером 8 байт, 
но еще как минимум 8 байт (MSVC, 32-битная Windows, примерно то же самое и в Linux)
будет задействованы для хранения информации о выделенном блоке\footnote{это еще называют ``метаданными''}.
В 64-битных ОС указатели занимают в два раза больше, так что информация о каждом блоке будет занимать как минимум
16 байт.

Использование кучи требует некоторой программистской дисциплины, без которой легко наделать ошибок.
Возможно поэтому, считается что ЯП с RAII как Си++
либо ЯП со сборщиками мусора (Python, Ruby) легче.

\subsubsection{Одна из основных ошибок: утечки памяти}

Память была выделена, но её забыли освободить через free(). Эта проблема довольно легко решается своей собственной
надстройкой над ф-циями malloc()/free(). Пусть эта надстройка ведет учет выделенных блоков, а также, где и когда
(и для чего) был выделен тот или иной блок.

Я сделал это в своей библиотеке octothorpe\footnote{\url{https://github.com/dennis714/octothorpe/blob/master/dmalloc.c}}. 
Например, макрос DMALLOC вызывает ф-цию dmalloc(), передавая
ей имя файла, имя ф-ции, номер строки, а также комментарий (имя блока). В конце работы программы, вызываем
\TT{dump\_unfreed\_blocks()} и он покажет список блоков, которые забыли освободить.

\begin{lstlisting}
seq_n:2, size: 124, filename: dmalloc_test.c:31, func: main, struct: block124
seq_n:3, size: 12, filename: dmalloc_test.c:33, func: main, struct: block12
seq_n:4, size: 555, filename: dmalloc_test.c:35, func: main, struct: block555
\end{lstlisting}

У каждого блока есть также номер.
Это для того чтобы можно было установить брякпоинт по номеру выделяемого блока --- 
тогда отладчик сработает когда этот блок будет выделяться и вы увидите, где и при каких условиях это происходит.

Писать в коде комментарии для каждого выделяемого блока памяти нудно, но очень полезно. Потом легко увидеть,
под что была выделена память. Я впервые увидел эту идею в Oracle RDBMS. Помимо всего прочего, там еще и ведется
статистика, под что было выделено больше памяти, её можно легко увидеть:

\begin{lstlisting}
SQL> select * from v$sgastat;

POOL         NAME                            BYTES     CON_ID
------------ -------------------------- ---------- ----------
shared pool  AQ Slave list                    1224          1
shared pool  KQR L PO                       653312          2
shared pool  KQR X SO                       635808          2
shared pool  RULEC                           20688          1
shared pool  KQR M SO                         7168          2
shared pool  work area table entry           12240          2
shared pool  kglsim object batch              3864          2
large pool   PX msg pool                    860160          1
large pool   free memory                  30523392          0
large pool   SWRF Metric CHBs              1802240          2
large pool   SWRF Metric Eidbuf             368640          2
\end{lstlisting}

Подобная штука также присутствует и в ядре Windows, там это называется tagging. При выделении памяти
в ядре или драйвере, нужно указывать также 32-битный тег (обычно, четырехбуквенное сокращение, означающее
подсистему Windows). 
Затем в отладчике можно увидеть статистику, под что выделено больше всего памяти:

\begin{lstlisting}
kd> !poolused 4
   Sorting by  Paged Pool Consumed

  Pool Used:
            NonPaged            Paged
 Tag    Allocs     Used    Allocs     Used
 CM25        0        0       935  4124672	Internal Configuration manager allocations , Binary: nt!cm
 Gh05        0        0       268  3291016	GDITAG_HMGR_SPRITE_TYPE , Binary: win32k.sys
 MmSt        0        0      2119  2936752	Mm section object prototype ptes , Binary: nt!mm
 CM35        0        0        91  2150400	Internal Configuration manager allocations , Binary: nt!cm
 vmfb        0        0        13  2148752	UNKNOWN pooltag 'vmfb', please update pooltag.txt
 Ntff        5     1040      1287  1070784	FCB_DATA , Binary: ntfs.sys
 ArbA        0        0       108   442368	ARBITER_ALLOCATION_STATE_TAG , Binary: nt!arb
 NtfF        0        0       457   431408	FCB_INDEX , Binary: ntfs.sys
 CM16        0        0        62   331776	Internal Configuration manager allocations , Binary: nt!cm
 IoNm        0        0      2022   267288	Io parsing names , Binary: nt!io
 Ttfd        0        0       159   253976	TrueType Font driver 
 Ifs         0        0         4   249968	Default file system allocations (user's of ntifs.h) 
 CM29        0        0        26   212992	Internal Configuration manager allocations , Binary: nt!cm
\end{lstlisting}

Конечно, можно возразить, что для этого нужно хранить еще больше информации о выделенных блоках, а еще
и теги, названия блоков. 
И это еще сильнее замедляет работу программы. Конечно. 
Поэтому пусть это будет работать только в отладочных (debug) сборках, 
а в release-сборках, DMALLOC() становится обычной функцией-переходником\footnote{thunk function} для malloc().
Ну а в ядре Windows это вообще по умолчанию отключено, и нужно включать при помощи утилиты GFlags
\footnote{\url{http://msdn.microsoft.com/en-us/library/windows/hardware/ff549557(v=vs.85).aspx}}
Помимо всего прочего, подобное есть и в MSVC
\footnote{читайте больше о ф-циях 
\href{http://msdn.microsoft.com/en-us/library/5at7yxcs.aspx}{\_CrtSetDbgFlag}
и
\href{http://msdn.microsoft.com/en-us/library/d41t22sb.aspx}{\_CrtDumpMemoryLeaks}}.

\subsubsection{Одна из основных ошибок: разрушение кучи}

Нетрудно выделить память под 4 байта, но по ошибке дописать туда пятый. Скорее всего, никак это не проявится,
но фактически это очень опасная мина замедленного действия, опасная, потому что ведет к трудновыявляемым
ошибкам. Байт следующий за выделенным вам блоком может не использоваться вовсе, но также там уже
может начинаться какая-то структура менеджера памяти, отвечающая за учет блоков. Если какую-то из
таких структур сознателно разрушить, перезаписать, то тогда следующие вызовы malloc() или free() не смогут
корректно работать. Иногда это проявляется в выводе ошибок вроде (в Windows):

\begin{lstlisting}
HEAP[Application.exe]: HEAP: Free Heap block 211a10 modified at 211af8 after it was freed
\end{lstlisting}

Подобные ошибки эксплуатируются авторами эксплоитов: если знать что вы можете изменить структуры данных
менеджера памяти нужным вам образом, вы можете добиться какого-то нужного вам поведения программы 
(это называется heap overflow\footnote{\url{https://en.wikipedia.org/wiki/Heap_overflow}}).

Довольно распространенный метод борьбы с подобными ошибками: это просто дописывать ``guard''-ы с обоих сторон
блока, например, 4-байтного размера. Например, я сделал это в своем DMALLOC. При каждом вызове free(),
проверяется целостность guard-ов (это могут быть просто какие-то фиксированные значения вроде 0x12345678),
и если кто-то или что-то затерло один из них, можно тут же сообщить об этом.

%\subsubsection{Приемущества своего собственного менеджера памяти или надстройки над стандартным}
%... может выдать размер выделенного блока, хотя нафик оно надо...

\subsubsection{Одна из основных ошибок: непроверка результата malloc()}

При успешном выполнении, malloc() возвращает указатель на блок, который можно использовать, либо NULL,
если памяти не хватает. Конечно, в наше время дешевой памяти эта проблема становится редкой, тем не менее,
если вы используете много памяти, думать об этом все же надо. Проверять указатель после каждого вызова
malloc() неудобно, так что довольно популярный метод это писать свои функции-переходники с названием 
xmalloc(), xrealloc(), вызывающие malloc()/realloc(), но проверяющие их результат и падающие в случае
ошибки.

Интересно упомянуть, как ведет себя xmalloc() в git:

\begin{lstlisting}
void *xmalloc(size_t size)
{
	void *ret;

	memory_limit_check(size);
	ret = malloc(size);
	if (!ret && !size)
		ret = malloc(1);
	if (!ret) {
		try_to_free_routine(size);
		ret = malloc(size);
		if (!ret && !size)
			ret = malloc(1);
		if (!ret)
			die("Out of memory, malloc failed (tried to allocate %lu bytes)",
			    (unsigned long)size);
	}
#ifdef XMALLOC_POISON
	memset(ret, 0xA5, size);
#endif
	return ret;
}
\end{lstlisting}
\footnote{\url{https://github.com/git/git/blob/master/wrapper.c}}

Если malloc() не успешен, он пытается освободить какие-то уже выделенные (и не очень нужные) 
блоки при помощи try\_to\_free\_routine(), а затем вызвать malloc() снова.

Помимо всего прочего, если определен XMALLOC\_POISON, все байты в выделенном блоке заполняются 0xA5.
Это может помочь визуально, на глаз, увидеть когда вы, например, выделили память под структуру,
а затем используете какое-то поле из нее до того как инициализировали. Значение \TT{0xA5A5A5A5} будет
бросаться в глаза в отладчике, ну или просто если вы захотите где-то в дампе вывести его в шестнадцатеричной
форме. В MSVC для этой же цели служит константа \TT{0xbaadf00d}.

И даже более того: после вызова free(), освобожденный блок может маркироваться уже какой-то другой константой,
чтобы если кто-то захочет использовать что-то оттуда после освобождения блока, это также было видно, хотя
бы визуально.

Примеры констант от Microsoft:

\begin{lstlisting}
* 0xABABABAB : Used by Microsoft's HeapAlloc() to mark "no man's land" guard bytes after allocated heap memory
* 0xABADCAFE : A startup to this value to initialize all free memory to catch errant pointers
* 0xBAADF00D : Used by Microsoft's LocalAlloc(LMEM_FIXED) to mark uninitialised allocated heap memory
* 0xBADCAB1E : Error Code returned to the Microsoft eVC debugger when connection is severed to the debugger
* 0xBEEFCACE : Used by Microsoft .NET as a magic number in resource files
* 0xCCCCCCCC : Used by Microsoft's C++ debugging runtime library to mark uninitialised stack memory
* 0xCDCDCDCD : Used by Microsoft's C++ debugging runtime library to mark uninitialised heap memory
* 0xDEADDEAD : A Microsoft Windows STOP Error code used when the user manually initiates the crash.
* 0xFDFDFDFD : Used by Microsoft's C++ debugging heap to mark "no man's land" guard bytes before and after allocated heap memory
* 0xFEEEFEEE : Used by Microsoft's HeapFree() to mark freed heap memory
\end{lstlisting}
\footnote{\url{https://en.wikipedia.org/wiki/Magic_number_(programming)}}

\subsubsection{Еще частые ошибки}

Если не включить заголовочный файл stdlib.h, 
GCC считает возвращаемое значение неизвестной ф-ции malloc() за int и постоянно ругается на приведение типов.

Еще одна ошибка, которая может попортить нервов, это выделить один и тот же блок памяти в одном месте 
больше одного раза (предыдущие вызовы ``теряются'' из вида).

\subsubsection{Методы борьбы}

Однако, может оказаться так, что ошибки в программе у вас есть, а перекомпилировать её по каким-то причинам
вы не можете. Тогда может помочь, например, valgrind\footnote{\url{http://valgrind.org/}}.



\subsection{Локальный стек или куча?}

Конечно, в локальном стеке выделение памяти происходит быстрее.

Например: в tracer\footnote{\url{http://yurichev.com/tracer-ru.html}} у меня есть дизассемблер\footnote{\url{https://github.com/dennis714/x86_disasm}} 
и эмулятор x86-процессора\footnote{\url{https://github.com/dennis714/bolt/blob/master/X86_emu.c}}.
Когда я писал дизассемблер на Си (я делал это после того как длительное время писал на ЯП 
более высокого уровня --- Python), 
я думал, что было бы неплохо, чтобы он сам выделял память под структуру, заполнял её
и возвращал указатель на нее, а в случае ошибки дизассемблирования, возвращал бы NULL. Эстетически, это 
неплохо смотрится, в стиле высокоуровневых ЯП, к тому же, такой код наверное легче читается. 
Однако, дизассемблер и эмулятор x86-процессора
работают в цикле, огромное количество раз в секунду и эффективность здесь более чем важна.
Так что, основной цикл у меня выглядит примерно так:

\begin{lstlisting}
while(true)
{
	struct disassembled_instruction;

	bool DA_success=disassemble(&disassembled_instruction...);
	if (DA_success==false)
		break;

	bool emulate_success=try_to_emulate(&disassembled_instruction);
	if (emulate_success==false)
		break;

};
\end{lstlisting}

Затрат на выделение памяти под структуры, описывающие дизассемблированную инструкцию, нет вовсе.
А иначе, нужно было бы на каждой итерации цикла вызывать malloc()/free(), каждая из которых, каждый раз,
работала бы со структурами кучи, итд.

Как известно, у x86-инструкций может быть вплоть до трех операндов, так что, в моей структуре, помимо
кода инструкции, есть также и информация о трех операндах. Конечно, можно было бы оформить её примерно так:

\begin{lstlisting}
struct disassembled_instruction
{
	int instruction_code;
	struct operand *op1;
	struct operand *op2;
	struct operand *op3;
};
\end{lstlisting}

... а в случае отсутствия какого либо операнда, пусть там будет NULL. Тем не менее, это снова выделение памяти в куче.

Так что у меня сделано примерно так:

\begin{lstlisting}
struct disassembled_instruction
{
	int instruction_code;
	int operands_total;
	struct operand op[3];
};
\end{lstlisting}

Такая структура занимает больше места в памяти. Ведь, как известно, трехоперандные инструкции очень редки в x86-коде,
а здесь у меня пустой третий операнд хранится всегда. Однако, лишних манипуляций с памятью не происходит.

Ну а если уж так сильно хочется сэкономить на третьем операнде, то можно не хранить третий операнд вовсе: нетрудно
вычислить размер структуры без одного операнда: \TT{sizeof(disassembled\_instruction) - sizeof(struct operand)} 
и скопировать его куда-то, где он должен храниться.
Ведь никто не запрещает нам использовать (и хранить) не всю структуру а только её часть.
А ф-ции работы с этой структурой могут не трогать в памяти третий операнд вовсе и, таким образом, ошибок не будет.

И даже более того: я специально сделал дизассемблер именно так, чтобы он мог принимать на вход не инициализированную
структуру, и мог работать даже если там осталась информация от предыдущих вызовов.

Возможно, это уже слишком, но вы поняли идею.

Таким образом, если вы выделяете память под небольшие структуры зараннее известного размера, или если скорость
очень важна, то лучше подумать насчет выделения в локальном стеке.



\chapter{Строки в Си}

В Си нет встроенных возможностей для удобной работы со строками, такими, какие имеются в ЯП более
высокого уровня.

Часто жалуются на неудобную
конкатенацию строк (то есть, склеивание) в Си при помощи функции strcat(). Также, многих раздражает sprintf(),
под которых нельзя толком зараннее предсказать, сколько нужно выделять памяти. Копирование строк при помощи
strcpy() также неудобно --- нужно думать, сколько же выделить байт под буфер. Помимо всего прочего, неудобная
работа со строками в Си, это источник огромного количества уязвимостей в ПО, связанных с переполнениями буфера\cite[1.14.2]{REBook}.

Прежде всего, нужно задать себе вопрос, какие операции со строками нам нужны.
Конкатенация (склеивание) нужна чтобы 1) выдавать в лог сообщения; 2) конструировать строки и записывать их куда-то.

Для 1) можно использовать потоки (streams) --- не конструируя строку, выдавать её по порциям, например:

\begin{lstlisting}
printf ("Date: ");
dump_date(stdout, date);
printf (" a=");
dump_a(stdout, a);
printf ("\n");
\end{lstlisting}

Подобное заменяется в Си++ выводом в ostream:

\begin{lstlisting}
cout << "Date: " << Date_ToString(date) << " a=" << a_ToString(a) << "\n";
\end{lstlisting}

Так быстрее и меньше требуется памяти для конструирования строк.

Кстати, ошибкой является писать так:

\begin{lstlisting}
cout << "Date: " + Date_ToString(date) + " a=" + a_ToString(a) + "\n";
\end{lstlisting}

Для неспешного вывода в лог небольшого кол-ва сообщений это нормально, но если таких строк очень много, то будут
накладные расходы на их конкатенацию. \\
\\
Но все же строки иногда конструировать надо.

Есть какие-то библиотеки для этого.
К примеру, в Glib\footnote{\url{https://developer.gnome.org/glib/}} есть 
gstring.h\footnote{\url{https://github.com/GNOME/glib/blob/master/glib/gstring.h}}/
gstring.c\footnote{\url{https://github.com/GNOME/glib/blob/master/glib/gstring.c}}. 

\label{strbuf}
А в исходниках git можно найти strbuf.h\footnote{\url{https://github.com/git/git/blob/master/strbuf.h}}/
strbuf.c\footnote{\url{https://github.com/git/git/blob/master/strbuf.c}}. Собственно,
подобные Си-библиотеки очень похожи: они обеспечивают структуру данных, в которой есть некоторый буфер для строки, текущий размер буфера
и текущий размер строки в буфере. При помощи отдельных функций, можно добавлять новые строки или символы
в буфер, который, в свою очередь, будет автоматически увеличиваться или даже уменьшаться.

В \IT{strbuf.c} из git есть даже ф-ция \IT{strbuf\_addf()}, работающая как \IT{sprintf()}, 
но добавляющая строку-результат в буфер.

Так пользователь освобождается от головной боли связанной с выделением памяти.
При работе с этими библиотеками, практически невозможна ситуация переполнения буфера, если только не начать
работать со структурой самостоятельно.

Типичная последовательность работы с такими библиотеками, выглядит так:

\begin{itemize}
\item
Инициализация структуры strbuf или GString.

\item
Добавление строк и/или символов.

\item
Имеем сконструированную строку. Используем как обычную Си-строку, записываем куда-то в файл, передаем по сети, итд.

\item
Освобождаем структуру.
\end{itemize}

Кстати, конструирование строк чем-то напоминает 
Buffer\footnote{\url{http://docs.oracle.com/javase/7/docs/api/java/nio/Buffer.html}}, 
ByteBuffer\footnote{\url{http://docs.oracle.com/javase/7/docs/api/java/nio/ByteBuffer.html}} и 
CharBuffer\footnote{\url{http://docs.oracle.com/javase/7/docs/api/java/nio/CharBuffer.html}} в Java.

\section{Хранение длины строки}

Всегда хранить длину строки --- это было принято в реализациях ЯП Pascal. 
Не смотря на исходы святых войн\footnote{holy wars} между приверженцами Си и Pascal, все же, почти все библиотеки
для хранения строк и работы с ними, хранят также и текущую длину --- потому что удобства от этого перевешивают
необходимость пересчитывать это значение.

Например, \IT{strlen()} (подсчет длины строки) больше не нужен вообще, длина все время известна.
Конкатенация строк работает намного быстрее, потому что не нужно вычислять длину первой строки.
Ф-ция сравнения строк в самом начале может сравнить длины строк и если они не равны, тут же вернуть false,
не начиная сравнивание самих строк.

В Oracle RDBMS, в сетевых библиотеках, в функции работы со строками, зачастую передается строка и, 
отдельным аргументом, её длина\footnote{\url{http://blog.yurichev.com/node/64}}.
Это не очень эстетично, это выглядит избыточно, зато очень удобно.
Например, у нас есть некоторая ф-ция, которой нужно в начале узнать, какую строку ей передали:

\lstinputlisting{C_strings/strcmp1.c}

А вот если бы эта ф-ция имела длину входной строки, её можно было бы переписать так:

\lstinputlisting{C_strings/strcmp2.c}

Конечно, с эстетической точки зрения, код выглядит ужасно.
Тем не менее, мы здорово сократили количество необходимых сравнений строк! Вероятно, для тех ситуаций, когда 
нужно как можно быстрее обрабатывать текстовые строки, такой подход может улучшить ситуацию.

\section{Возврат строки}

Если некая ф-ция должна вернуть строку, имеются такие возможности:

\begin{itemize}
\item
Возврат строки-константы, это самое простое и быстрое.

\item
Возврат строки через глобальный массив символов. Недостаток: массив один и каждый вызов ф-ции перезаписывает
его содержимое.

\item
Возврат строки через буфер, заданный в аргументах ф-ции. Недостаток: нужно также передавать и длину буфера.

\item
Выделяем буфер нужного размера сами, записываем туда строку, возвращаем указатель. Недостаток: тратятся ресурсы
на выделение памяти.

\item
Записываем строку в уже рассмотренный strbuf или GString или иную другую структуру, указатель на которую был
передан в аргументах.

\end{itemize}

\subsection{Через строку-константу}

Первый вариант очень прост. Например:

\begin{lstlisting}
const char* get_month_name (int month)
{
	switch (month)
	{
	case 1: return "January";
	case 2: return "February";
	case 3: return "March";
	case 4: return "April";
	case 5: return "May";
	case 6: return "June";
	case 7: return "July";
	case 8: return "August";
	case 9: return "September";
	case 10: return "October";
	case 11: return "November";
	case 12: return "December";
	default: return "Unknown month!";
	};
};
\end{lstlisting}

Можно даже еще проще:

\begin{lstlisting}
const char* month_names[]={
	"January", "February", "March", "April", "May", "June", "July", "August",
	"September", "October", "November", "December" };

const char* get_month_name (int month)
{
	if (month>=1 && month<=12)
		return month_names[month-1];

	return "Unknown month!";
};
\end{lstlisting}

\subsection{Через глобальный массив символов}

Так делает стандартная ф-ция asctime(). Следует помнить, что нужно использовать возвращенную строку
перед каждым следующим вызовом asctime. 

Например, это правильно:

\begin{lstlisting}
printf("date1: %s\n", asctime(&date1));
printf("date2: %s\n", asctime(&date2));
\end{lstlisting}

А это нет:

\begin{lstlisting}
char *date1=asctime(&date1);
char *date2=asctime(&date2);
printf("date1: %s\n", date1);
printf("date2: %s\n", date2);
\end{lstlisting}

... ведь указатели date1 и date2 будут указывать на одно и то же место, и вывод printf() будет одинаковым. \\
\\
В git в hex.c\footnote{\url{https://github.com/git/git/blob/master/hex.c}} можно найти такое:

\lstinputlisting{C_strings/git_hex.c}

Строка возвращается фактически через глобальную переменную, объявление её как static внутри ф-ции просто напросто
обеспечивает доступ к ней только из этой ф-ции. Но вот недостаток: после вызова \IT{sha1\_to\_hex()} вы не можете
вызвать её повторно для получения второй строки до тех пор, пока не используете как-то первую, ведь она
затрется! Для того чтобы решить эту проблему здесь, по видимому, сделали сразу 4 буфера и каждый раз строка
возвращается в следующем. Но имейте ввиду --- так можно делать если только вы уверены в том что вы делаете,
это код на уровне ``грязного хака''.
Если вы
вызовете эту ф-цию 5 раз и вам нужно будет использовать как-то строку полученную при первом вызове, это может
привести к трудновыявляемой ошибке.

Кстати, обратите также внимание на то что переменная \IT{bufno} не инициализируется, потому что используются только 
2 младших её бита, к тому же, не важно, какое значение переменная будет содержать в самом начале!


\section{Определение строк}

\label{heredoc}
Малоизвестная возможность Си, длинные строки можно определять так:

\begin{lstlisting}
const char* long_line="line 1"
	"line 2"
	"line 3"
	"line 4"
	"line 5";

...

printf ("Some Utility v0.1\n"
	"Usage: %s parameters\n"
	"\n"
	"Authors:...\n", argv[0]);
\end{lstlisting}

Это отдаленно напоминает ``here document''\footnote{\url{https://en.wikipedia.org/wiki/Here_document}} в 
UNIX-шеллах и Perl.

\section{Стандартные ф-ции для работы со строками}

\subsection{strstr() и memmem()}

strstr() применяется для поиска строки в другой строке, либо чтобы узнать, есть ли там такая строка вообще.

memmem() можно применять с этими же целями, но для поиска по буферу, в котором могут быть нули,
либо по части строки.

\label{memchr}
\subsection{strchr() и memchr()}

strchr() применяется для поиска символа в строке, либо чтобы узнать, есть ли там такой символ вообще.

memchr() можно применять с этими же целями, но для поиска по части строки.

\subsection{atoi(), atof(), strtod(), strtof()}

Ф-ции atoi/atof не могут сигнализировать об ошибке, а strtod/strtof, делая то же самое --- могут.

\subsection{scanf(), fscanf(), sscanf()}

Извечный спор, что лучше, текстовые файлы или бинарные. С бинарными быстрее и проще работать, зато текстовые
легче просматривать и редактировать в обычном текстовом редакторе, к тому же, в UNIX имеется огрмоный арсенал
утилит для обработки текстов и строк. Но текстовые файлы нужно парсить.

Ф-ции scanf()\cite[7.19.6.2]{C99TC3} конечно же, регулярные выражения не поддерживают, 
однако при их помощи некоторые простые последовательности строк можно парсить. 

\subsection{Пример \#1}

Генерируемый ядром Linux файл \TT{/proc/meminfo}, начинается примерно так:

\begin{lstlisting}
MemTotal:        1026268 kB
MemFree:          119324 kB
Buffers:          170796 kB
Cached:           263736 kB
SwapCached:        11428 kB
...
\end{lstlisting}

Предположим, нам нужно узнать первое и третье число, игнорируя второе и остальные.
Так это можно сделать:

\begin{lstlisting}
void read_proc_meminfo()
{
	FILE *f=fopen("/proc/meminfo", "r");
	assert(f);
	unsigned result1, result2;
	if (fscanf (f, "MemTotal:\t%d kB\n"
			"MemFree:\t%*d kB\n"
			"Buffers:\t%d kB\n", 
			&result1, &result2)==2)
		printf ("results: %d %d\n", result1, result2);
	fclose(f);
};
\end{lstlisting}

Строка формата расходится на три строки, в реальности это одна\ref{heredoc}.
Обратите внимание на \TT{\textbackslash{}n}, так мы задаем перевод строки.

\TT{*} в модификаторе scanf-строки указывает что число будет прочитано, но никуда записано не будет.
Таким образом, это поле игнорируется. scanf()-функции возвращают кол-во не прочитанных полей (здесь
их будет 3) а кол-во записанных полей (2).

\subsection{Пример \#2}

Имеется текстовый файл с парами в каждой строке (ключ-значение):

\begin{lstlisting}
some_param1=some_value
some_param2=Lazy fox etc etc.
param3=Lorem Ipsum etc etc.
space here=value containing space
too long param, we should fail here=value
\end{lstlisting}

Нужно просто читать оба поля.

\begin{lstlisting}
int main(int argc, char *argv[])
{
	assert(argc==2);
	assert(argv[1]);
	FILE *f=fopen (argv[1], "r");
	assert(f);
	int line=1;
	do
	{
		char param[16];
		char value[60];
		if (fscanf (f, "%16[^=]=%60[^\n]\n", param, value)==2)
		{
			printf ("param=%s\n", param);
			printf ("value=%s\n", value);
		}
		else
		{
			printf ("error at line %d\n", line);
			return 0;
		};
		line++;
	} while (!feof(f) && !ferror(f));
};
\end{lstlisting}

\TT{\%16[\^{}=]} --- это отдаленно напоминает регулярные выражения. Означает, читать 16 любых символов, кроме
знака ``равно'' (=). Затем, мы указываем scanf()-у, что далее должен быть этот самый знак (=). Затем
пусть он читает 60 любых символов, кроме символа перевода строки. В конце читаем символ перевода строки.

Это работает, и поля ограничены длиной 16 и 60 символов. Поэтому на 5-й строке предсказуемо происходит ошибка,
ведь там длина парамера длиннее.

Так можно парсить несложные форматы, даже CSV
\footnote{Comma-separated values: \url{https://en.wikipedia.org/wiki/Comma-separated_values}}.

Однако, нельзя забывать о том что scanf()-функции не способны прочитать пустую строку там где задается \%s.
Поэтому, этим методом невозможно парсить файл с ключами-значениями, где есть отсутствующие ключи или значения.

\subsubsection{Засада \#1}

Если использовать \%d в строке формата, scanf() подразумевает что это 32-битный int. 

Ошибкой является подобное:

\begin{lstlisting}
char a[10];

scanf ("%d %d %d %d", &a[0], &a[1], &a[2], &[3]);
\end{lstlisting}

Символы (или байты) лежат ``в притык'' друг к другу. Когда scanf() будет обрабатывать первое значение, он будет считать
его за 32-битный int, и ``затрет'' остальные три, рядом лежащие. И так далее.


\subsection{strspn(), strcspn()}

strspn() часто применяется для того чтобы удостовериться, что некая строка полностью состоит из
нужных символов:
    
\begin{lstlisting}
if (strspn(s, "1234567890") == strlen(s)) ... OK
...
if (strspn(IPv4, "1234567890.") == strlen(IPv4)) ... OK
...
if (strspn(IPv6, "0123456789AaBbCcDdEeFf:.") == strlen(IPv6)) ... OK
\end{lstlisting}

Либо для того чтобы пропустить начало строки:

\begin{lstlisting}
const char *whitespaces = " \n\r\t";
*buf += strspn(*buf, whitespaces); // skip whitespaces at start
\end{lstlisting}

strcspn() это обратная ф-ция, её можно использовать для пропуска всех символов в начале строки, не попадающих
под множество символов:

\begin{lstlisting}
s += strcspn(s, whitespaces); // first, skip anything till whitespaces
s += strspn(s, whitespaces); // then skip shitespaces
\end{lstlisting}

\subsection{strtok() и strpbrk()}

Обе ф-ции служат для разбиения строки на подстроки, отделенные друг от друга разделительными символами
\footnote{delimiter}.
Только strtok() модифицирует исходную строку (и таким образом, подстроку сразу можно использовать
как отдельную Си-строку), а strpbrk() нет, он только возвращает указатель на следующую подстроку.



\section{Unicode}

Unicode это важно! Наиболее популярные способы его применения это:

\begin{itemize}
\item UTF-8
Популярно в UNIX-системах. Сильное приемущество: можно продолжать пользоваться стандартными ф-циями для
обработки строк.

\item UTF-16
Используется в Windows API.
\end{itemize}

\subsection{UTF-16}

Под каждый символ отводят 16-битный тип \IT{wchar\_t}.

Для объявления строк с таким типом, используется макрос L:

\begin{lstlisting}
L"hello world"
\end{lstlisting}

Для работы с wchar\_t вместо char, имеется целый класс функций-двойников с символом w в названии,
например: fwprintf(), wcscmp(), wcslen(), iswalpha().

\subsubsection{Windows}

В Windows, если некто хочет писать программу сразу в двух версиях, с использованием Unicode и без,
для этого есть тип tchar, в зависимости от объявленной переменной препроцессора UNICODE, 
он будет либо char либо wchar\_t\footnote{Сборка с Unicode и без была популярна во времена популярности
как Windows NT/2000/XP так и Windows 95/98/ME. Вторая линейка плохо поддерживала Unicode}.
Для этого же имеется макрос \TT{\_T(...)}:

\begin{lstlisting}
_T("hello world")
\end{lstlisting}

В зависимости от выставленной переменной препроцессора UNICODE, она будет объявлена как char либо wchar\_t.

В заголовочном файле tchar.h есть масса ф-ций, меняющих свою функцию в зависимости от этой переменной.

\section{Списки строк}

Самый простой список строк, это просто набор строк оканчивающийся нулем.
Например, в Windows API, в библиотеке Common Dialogs, 
так\footnote{\url{http://msdn.microsoft.com/en-us/library/windows/desktop/ms646829(v=vs.85).aspx}} 
передаются список допустимых расширений файлов для диалогового окна:

\begin{lstlisting}
// Initialize OPENFILENAME
ZeroMemory(&ofn, sizeof(ofn));
...
ofn.lpstrFilter = "All\0*.*\0Text\0*.TXT\0";
...

// Display the Open dialog box. 

if (GetOpenFileName(&ofn)==TRUE) 
	...
\end{lstlisting}


\section{\IFRU{Ваши собственные структуры данных в Си}{Your own data structures in C}}

\section{Списки в Си.}

Списки это связный набор элементов. Односвязный список --- это когда у каждого элемента есть ссылка на следующий.
Двусвязный список --- когда у элемента есть ссылки на следующий и на предыдущий.

У списков есть серьезное преимущество перед массивами: в список легко добавлять элемент в произвольное место,
так и удалять. В качестве недостатков: тратится много памяти для поддержания самих структур списка, а также
нет возможности индексировать его, как массив.

\subsection{Односвязный список}

Его сделать очень легко. В структуре предназначенной для связывания в список, достаточно добавить где-то
ссылку на следующий элемент, обычно это поле называется next:

\begin{lstlisting}
struct some_object
{
	...
	...
	struct some_object* next;
};
\end{lstlisting}

NULL в next означает что этот элемент является последним.

Операция прохода по такому списку становится очень простой:

\begin{lstlisting}
for (struct some_object *i=list; i!=NULL; i=i->next)
	...
\end{lstlisting}

Для вставки нового элемента, нужно вначале найти последний элемент:

\begin{lstlisting}
for (struct some_object *i=list; i!=NULL; i=i->next);
struct some_object *last_element=i;
\end{lstlisting}

... а затем, создав новую структуру, добавить указатель на нее в next:

\begin{lstlisting}
struct some_object *new_object=calloc(1, sizeof(struct some_object));
// populate new_object with data
last_element->next=new_object;
\end{lstlisting}

calloc() отличается от malloc() тем что обнуляет всё выделенное место, а значит в поле next нового
элемента сразу будет NULL\footnote{Об ``инициализации'' структур, читайте также здесь\ref{COOPInit}.}.

Поиск нужного элемента это просто проход по всему списку до тех пор, пока не найдется то что нужно.

Удаление элемента: найти предыдущий элемент и следующий, 
у предыдущего в next установить указатель на следующий элемент, 
затем освободить блок памяти выделенный для текущего элемента.

Самый первый элемент списка называется ``list head''. Структуру самого первого элемента можно объявлять как локальную
или глобальную переменную. Но тогда удалять первый элемент списка будет неудобно. А с другой стороны,
можно объявлять указатель на первый элемент списка, тогда будет проще этому указателю присвоить другой элемент,
который будет первым.

\subsection{Двусвязный список}

Это почти то же самое, только, помимо указателя на следующий элемент, хранится еще и указатель на предыдущий.
Если элемент первый, то указатель может быть NULL, либо он может указывать сам на себя (кому как удобнее).

Работая с двусвязным списком, легче находить предыдущие элементы, например, когда нужно удалить какой-то элемент.
А также можно перебирать элементы с конца списка до начала.
Но памяти на это тратится немного больше.

\subsection{Windows API}

Здесь, да и много где в ядре Windows, применяются две примитивные структуры:

\begin{lstlisting}
typedef struct _LIST_ENTRY {
   struct _LIST_ENTRY *Flink;
   struct _LIST_ENTRY *Blink;
} LIST_ENTRY, *PLIST_ENTRY, *RESTRICTED_POINTER PRLIST_ENTRY;

typedef struct _SINGLE_LIST_ENTRY {
    struct _SINGLE_LIST_ENTRY *Next;
} SINGLE_LIST_ENTRY, *PSINGLE_LIST_ENTRY;
\end{lstlisting}

Эти структуры нельзя назвать самостоятельными, они скорее предназначены для встраивания в другие структуры.
Например, вам нужно объеденить в список структуру описывающую цвет:

\begin{lstlisting}
struct color
{
	int R;
	int G;
	int B;
	LIST_ENTRY list;
};
\end{lstlisting}

Теперь в вашей структуре есть также и ссылка на предыдущий элемент и на следующий.
Для работы со структурами использующие эти списки, в Windows есть набор ф-ций
\footnote{\url{http://msdn.microsoft.com/en-us/library/windows/hardware/ff563802(v=vs.85).aspx}}.

\subsection{Linux}

В ядре Linux работа с простыми двусвязными списками, описывается в файле /include/linux/list.h
\footnote{\url{http://lxr.free-electrons.com/source/include/linux/list.h}}.

Там это много где используется, в ядре 3.12 по крайней мере ~2900 упоминаний ``struct list\_head''.

\subsection{Glib}

Напрашивается мысль, а нельзя ли выделить отдельную структуру для элемента списка, и не встраивать лишних полей
в свои структуры? Можно, например, так сделано в glist.h
\footnote{\url{https://github.com/GNOME/glib/blob/master/glib/glist.h} 
\url{https://developer.gnome.org/glib/2.37/glib-Doubly-Linked-Lists.html}} в glib:

\begin{lstlisting}
struct _GList
{
  gpointer data;
  GList *next;
  GList *prev;
};
\end{lstlisting}

data может указывать на какой угодно объект, на любую существующую структуру, в которой вы ничего не хотите менять.
Конечно, с эстетической точки зрения, это лучше. Но нельзя забывать, что тогда на каждый элемент вашего списка,
будет приходится уже два выделенных блока памяти + еще затраты на поддержания самих блоков памяти\ref{HeapOverhead}. \\
\\
Таким образом, подобное решение оправдано там, где экономия памяти менее важна.


\subsection{\IFRU{Бинарные деревья в Си}{Binary trees in C}}

\IFRU{Бинарные деревья}{Binary trees} ~--- \IFRU{одна из важнейших структур данных в компьютерных науках}
{are one of the most important structures in computer science}.
\IFRU{Чаще всего они используются для хранения пар}{Most often these are used for}
``\IFRU{ключ-значение}{key-values}''\IFRU{}{ pairs storage}.
\index{C++!STL!map}
\IFRU{Это то что в \CPP \ac{STL} реализовано в std::map}
{This is what implemented in std::map in \CPP \ac{STL}}.

\IFRU{Упрощенно говоря, по сравнению со списками, выборка у деревьев происходит намного быстрее}
{Simply speaking, in comparison with lists, trees offer much faster selection}.
\IFRU{С другой стороны, добавление элемента в дерево может происходить медленнее}
{On the other hand, element insertion may be slower}.

\index{POSIX!tsearch()}
\index{POSIX!twalk()}
\index{POSIX!tfind()}
\index{POSIX!tdelete()}
\IFRU{В стандартных библиотеках Си, нет работы с деревьями, но кое-что есть в}
{There are no C standard functions for working with trees, but some functions are present in} \ac{POSIX}
(tsearch(), twalk(), tfind(), tdelete())
\footnote{\url{http://pubs.opengroup.org/onlinepubs/009696799/functions/tsearch.html}}.

\IFRU{Это семейство ф-ций активно используется в}
{This family of functions are used actively in the} Bash 4.2, BIND 9.9.1, \ac{GCC} ~--- 
\IFRU{там можно посмотреть, как это использовать}{it can be seen there how it can be used}.

\index{Glib!GTree}
\IFRU{В}{The} Glib \IFRU{имеется также свои ф-ции для работы с деревьями, определенные в}
{also has the tree functions declared in the} gtree.h
\footnote{\url{https://github.com/GNOME/glib/blob/master/glib/gtree.h}}.

\index{C++!STL!set}
\IFRU{Множество}{The set} (std::set \InENRU \CPP \ac{STL}) 
\IFRU{можно реализовать так же просто при помощи бинарных деревьев, достаточно просто не хранить значение, а хранить только ключ}
{can be implemented as binary trees as well, one may just choose not to store the value and store the key only}.



\subsection{\IFRU{Еще кое что}{One more thing}}

\IFRU{Структуры данных описывающие какие-либо коллекции, могут также содержать и указатели на ф-ции для работы
с элементами, например, ф-ции сравнения, копирования, итд}
{Data structures related to collections may also contain pointers to the functions working with elements,
like comparison functions, copying, etc}.

\IFRU{К примеру}{For example} \InENRU GTree \InENRU glib:

\begin{lstlisting}[caption=gtree.c]
struct _GTree
{
  GTreeNode        *root;
  GCompareDataFunc  key_compare;
  GDestroyNotify    key_destroy_func;
  GDestroyNotify    value_destroy_func;
  gpointer          key_compare_data;
  guint             nnodes;
  gint              ref_count;
};
\end{lstlisting}

\IFRU{Задав ф-цию сравнения ключей, значений, а также ф-цию освобождения памяти}
{By setting the functions for key/value comparison and also deallocator function} (\InENRU \TT{g\_tree\_new\_full()}),
\IFRU{ф-ции работы с деревьями в glib смогут 
самостоятельно сравнивать два дерева, либо освобождать все структуры связанные с деревом}
{tree functions in glib will be able to compare two trees or to free a tree on its own}.


\label{COOP}
\chapter{\COOPname}

Как известно, в Си нет поддержки ООП, она есть в Си++, тем не менее, в чистом Си вполне
можно программировать в стиле ООП.

ООП, коротко говоря, это явное разделение на объекты и методы. 
В Си структуры легко могут представляться объектами, а обычные ф-ции --- методами.

\label{COOPInit}
\section{Инициализация структур}

В Си++ у классов имеются конструкторы. Если вам нужно каким-то особенным образом инициализировать
структуру, вам и в Си придется делать подобную ф-цию. Но если структура простая, то её можно
инициализировать при помощи calloc()\footnote{Это тоже самое что и malloc() + заполнение выделенной
памяти нулями} 
либо bzero()\ref{bzero}.

Все int-переменные становятся нулями. Нулевое значение bool в C99\ref{C99} и C++ это false, 
так же как и BOOL в Windows API. Все указатели становятся NULL. И даже вещественный
ноль представляемый в формате IEEE 754 это также все ноли во всех битах.

Если в структуре присутствуют указатели на другие структуры, то NULL может означать ``отсутствие объекта''.

\section{Деинициализация структур}

Если в структуре есть ссылки на другие структуры, то их нужно освобождать. В простом случае,
обычным вызовом free(). Кстати, вот почему free() может принимать на вход NULL и ничего в этом случае
не делать, это чтобы можно было просто писать \TT{free(s->field)} вместо 
\TT{if (s->field) free(s->field)}, так короче.

\section{Копирование структур}

Если структура простая, то её можно копировать обычным побайтовым копированием memcpy()\ref{memcpy}.
Если в такой манере скопировать структуру, в которой есть указатели на другие структуры,
то это будет называться \IT{shallow copy}\footnote{\url{https://en.wikipedia.org/wiki/Object_copy}}. 
И напротив, \IT{deep copy} --- это копирование структуры
плюс всех связанных с ней структур (это дольше).

Вот почему может быть удобнее хранить строку в структуре как обычный массив символов фиксированной длины.
Такого, например, очень много в Windows API. Такую структуру легко скопировать, её хранение
требует меньших накладных расходов\footnote{overhead} в куче. 
Но с другой стороны, придется согласиться с ограничением на длину строки.

Помимо всего прочего, структуру можно копировать просто так: \TT{s1=s2} --- в итоге генерируется код,
копирющий все поля по порядку. И это наверное легче читается чем вызов memcpy() на этом же месте.

\section{Инкапсуляция}

Си++ предлагает инкапсуляцию (сокрытие информации). Например, вы не можете
написать программу модифицирующую защищенное поле в классе, 
это защита на стадии компиляции\cite[1.7.3]{REBook}.

В Си этого нет, поэтому тут нужно больше дисциплины.

Впрочем, можно попытаться ``защитить'' структуру ``от посторонних глаз''. Например, в glib,
имеется библиотека для работы с деревьями. В заголовочном файле 
gtree.h\footnote{\url{https://github.com/GNOME/glib/blob/master/glib/gtree.h}} нет описания самой структуры
(она есть только в gtree.c\footnote{\url{https://github.com/GNOME/glib/blob/master/glib/gtree.c}}), 
а есть только forward declaration\ref{forwarddeclaration}. Так можно понадеятся что
пользователи GTree постараются не пользоваться отдельными полями в структуре.

Но у такого метода есть и обратная сторона: могут быть крохотные однострочные ф-ции вроде 
``вернуть длину строки'' в strbuf\ref{strbuf}, например:

\begin{lstlisting}
typedef struct _strbuf
{
    char *buf;
    unsigned strlen;
    unsigned buflen;
} strbuf;

unsigned strbuf_get_len(strbuf *s)
{
	return s->strlen;
};
\end{lstlisting}

Если компилятору на стадии компиляции доступно и описание структуры и тело ф-ции, то в каком-то месте,
вместо вызова strbuf\_get\_len() он может сделать эту ф-цию как inline-овую, вставить её тело прямо на том
же место и сэкономить на вызове другой ф-ции. Но если эта информация компилятору недоступна, то он
оставит вызов strbuf\_get\_len() как есть.

То же самое касается поля buf в структуре strbuf. Компилятор может генерировать куда более эффективный
код, если этот код сможет обращаться к полям структур на прямую, а не вызывать суррогатные 
функции-``методы''.


\chapter{\IFRU{Стандартные библиотеки Си/Си++}{C/C++ standard library}}

\section{assert}

Как известно, этот макрос часто используется для валидации
\footnote{используется также такой термин как ``инвариант'' и ``sanitization'' в англ.яз.} заданных значений. 
Например, если ваша ф-ция
работает с датой, вы, вероятно, захотите написать в её начале что-то вроде \IT{assert (month>=1 \&\& month<=12)}.

Вот то о чем нужно помнить: стандартный макрос assert() доступен только в отладочных (debug) сборках. В release
все выражения как бы исчезают. Поэтому писать, например, \IT{assert(f=malloc(...))} неверно. Впрочем,
вы возможно захотите использовать что-то вроде \IT{assert(object->get\_something()==123)}.

В макросах assert можно также указывать небольшие сообщения об ошибках: 
вы увидите их если assert() ``не сойдется''. 
Например, в исходниках LLVM\footnote{\url{http://llvm.org/}} можно встретить такое:

\begin{lstlisting}
assert(Index < Length && "Invalid index!");
...
assert(i + Count <= M && "Invalid source range");
...
assert(j + Count <= N && "Invalid dest range");
\end{lstlisting}

Текстовая строка имеет тип \IT{const char*}, и она никогда не NULL. 
Таким образом, можно дописать к любому выражению \IT{... \&\& true} не меняя его смысл.

\section{Разница между stdout и stderr}

\IT{stdout} это то что выводится на консоль при помощи вызова \IT{printf()}.
\IT{stdout} это буферизированный вывод,
так что, пользователь, обычно того не зная, видит вывод порциями. Бывает так что программа выдает
что-то используя \IT{printf()} либо \IT{cout} и тут же падает.
Если это попадает в буфер, но буфер не успевает
``сброситься'' (flush) в консоль, то пользователь ничего не увидит. Это бывает неудобно.
Таким образом, для вывода более важной информации, в том числе отладочной, удобнее использовать \IT{stderr}.

\IT{stderr} это не буферизированный вывод, и всё что попадает в этот поток при помощи 
\TT{fprintf(stderr,...)} либо \IT{cerr}, появляется в консоли тут же.

Не следует также забывать, что из-за отсутствия буфера, вывод в \IT{stderr} медленнее.

Чтобы направлять \IT{stderr} в другой файл при запуске процесса, можно указывать:

\begin{lstlisting}
process 2> debug.txt
\end{lstlisting}

... это направит вывод \IT{stderr} в заданный файл (потому что номер этого потока -- 2).

\section{UNIX time}

В UNIX-среде очень популярно представление даты и времени в формате UNIX time.
Это просто 32-битное число, показывающее
количество прошедших секунд с 1-го января 1970-го года.

В качестве положительных сторон: 1) очень легко хранить это 32-битное число; 2) очень легко вычислять разницу дат;
3) невозможно закодировать неверные даты и время, такие как 32-е января, 29-е февраля невысокосных годов, 
25 часов 62 минуты.

В качестве отрицательных сторон: 1) нельзя закодировать дату до 1970-го года.

В наше время, если использовать UNIX time, тем не менее, следует помнить что ``срок его действия'' истечет
в 2038-м году, именно тогда 32-битное число переполнится, то есть, пройдет $2^{32}$ секунд с 1970-го года.
Так что, для этого следует использовать 64-битное значение, т.е., time64.

% ? NtQuerySystemTime http://msdn.microsoft.com/en-us/library/windows/desktop/ms724512(v=vs.85).aspx

\section{scanf(), fscanf(), sscanf()}

\subsection{Засада \#1}

Если использовать \%d в строке формата, scanf() подразумевает что это 32-битный int. 

Ошибкой является подобное:

\begin{lstlisting}
char a[10];

scanf ("%d %d %d %d", &a[0], &a[1], &a[2], &[3]);
\end{lstlisting}

Символы (или байты) лежат ``в притык'' друг к другу. Когда scanf() будет обрабатывать первое значение, он будет считать
его за 32-битный int, и ``затрет'' остальные три, рядом лежащие. И так далее.

\label{memcpy}
\section{memcpy()}

Поначалу трудно запомнить порядок аргументов в ф-циях memcpy(), strcpy(). Чтобы было легче, можно представлять
знак ``='' (``равно'') между аргументами.

\label{bzero}
\section{bzero() и memset()}

bzero() это ф-ция просто обнуляющая блок памяти.
Для этого обычно используют memset(). Но у memset() есть неприятная особенность, легко перепутать второй
и третий аргументы местами, и компилятор промолчит, потому что байт для заполнения всего блока задается как int.

К тому же, имя ф-ции bzero легче читается.

С другой стороны, её нет в стандарте POSIX.

\label{printf}
\section{printf()}

\subsection{Свои собственные модификаторы в printf()}

Часто можно испытать раздражение, когда было бы логично передать в printf(), скажем, структуру описывающее комплексное
число, или цвет закодированный в структуре из трех чисел типа int.

Эту проблему в Си++ решают определением ф-ции \TT{operator<<} в \TT{ostream} для своего типа, либо введением
метода с названием \TT{ToString()}.

В printk() (printf-подобная ф-ция в ядре Linux) имеются дополнительные модификаторы
\footnote{\url{http://git.kernel.org/cgit/linux/kernel/git/torvalds/linux.git/tree/Documentation/printk-formats.txt}}, 
такие как \TT{\%pM} (Mac-адрес), \TT{\%pI4} (IPv4-адрес), \TT{\%pUb} (UUID/GUID).

В ОС Plan9, и в исходниках компилятора Go, можно найти ф-цию fmtinstall(), объявляющую новый модификатор printf-строки,
например:

\begin{lstlisting}[caption=go\textbackslash{}src\textbackslash{}cmd\textbackslash{}5c\textbackslash{}list.c]
void
listinit(void)
{

	fmtinstall('A', Aconv);
	fmtinstall('P', Pconv);
	fmtinstall('S', Sconv);
	fmtinstall('N', Nconv);
	fmtinstall('B', Bconv);
	fmtinstall('D', Dconv);
	fmtinstall('R', Rconv);
}

...

int
Pconv(Fmt *fp)
{
	char str[STRINGSZ], sc[20];
	Prog *p;
	int a, s;

	p = va_arg(fp->args, Prog*);
	a = p->as;
	s = p->scond;
	strcpy(sc, extra[s & C_SCOND]);
	if(s & C_SBIT)
		strcat(sc, ".S");
	if(s & C_PBIT)
		strcat(sc, ".P");
	if(s & C_WBIT)
		strcat(sc, ".W");
	if(s & C_UBIT)		/* ambiguous with FBIT */
		strcat(sc, ".U");
	if(a == AMOVM) {
		if(p->from.type == D_CONST)
			sprint(str, "	%A%s	%R,%D", a, sc, &p->from, &p->to);
		else
		if(p->to.type == D_CONST)
			sprint(str, "	%A%s	%D,%R", a, sc, &p->from, &p->to);
		else
			sprint(str, "	%A%s	%D,%D", a, sc, &p->from, &p->to);
	} else
	if(a == ADATA)
		sprint(str, "	%A	%D/%d,%D", a, &p->from, p->reg, &p->to);
	else
	if(p->as == ATEXT)
		sprint(str, "	%A	%D,%d,%D", a, &p->from, p->reg, &p->to);
	else
	if(p->reg == NREG)
		sprint(str, "	%A%s	%D,%D", a, sc, &p->from, &p->to);
	else
	if(p->from.type != D_FREG)
		sprint(str, "	%A%s	%D,R%d,%D", a, sc, &p->from, p->reg, &p->to);
	else
		sprint(str, "	%A%s	%D,F%d,%D", a, sc, &p->from, p->reg, &p->to);
	return fmtstrcpy(fp, str);
}
\end{lstlisting}
(\url{http://plan9.bell-labs.com/sources/plan9/sys/src/cmd/5c/list.c})

Ф-ция Pconv() будет вызвана если в строке формата будет встречен \%P. Затем она копирует созданную строку
при помощи fmtstrcpy(). Кстати, эта ф-ция и сама использует другие объявленные модификаторы, такие как \%A, \%D, итд.

В Glibc\footnote{Стандартной библиотеке в Linux} есть нестандартное расширение
\footnote{\url{http://www.gnu.org/software/libc/manual/html_node/Customizing-Printf.html}}, 
позволяющее объявлять свои модификаторы, но это \IT{deprecated}.

Попробуем определить свои собственные модификаторы для Mac-адреса и для вывода байта в бинарном виде:

\lstinputlisting{register_printf_function.c}
\footnote{Основа для примера взята отсюда: \url{http://codingrelic.geekhold.com/2008/12/printf-acular.html}}

Это компилируется с предупреждениями:

\begin{lstlisting}
1.c: In function 'main':
1.c:48:2: warning: 'register_printf_function' is deprecated (declared at /usr/include/printf.h:106) [-Wdeprecated-declarations]
1.c:49:2: warning: 'register_printf_function' is deprecated (declared at /usr/include/printf.h:106) [-Wdeprecated-declarations]
1.c:51:2: warning: unknown conversion type character 'M' in format [-Wformat]
1.c:52:2: warning: unknown conversion type character 'B' in format [-Wformat]
\end{lstlisting}

GCC умеет следить за соответствиями модификаторов в printf-строке и аргументами в вызове printf(), но здесь
ему встречаются незнакомые модификаторы, о чем он предупреждает.

Тем не менее, наша программа работает:

\begin{lstlisting}
$ ./a.out
00:11:22:33:44:55
10101011
\end{lstlisting}



\chapter{Темплейты C++}

Темплейты нужны обычно для того чтобы сделать класс универсальным для нескольких типов данных.
К примеру, \TT{std::string} в реальности это \TT{std::basic\_string<char>}, \\ 
а \TT{std::wstring} это \TT{std::basic\_string<wchar\_t>}. \\
\\
Нередко подобное делают и для float/double. Некий математический алгоритм может быть описан один раз,
но скомпилирован и для float и для double. \\
\\
Таким образом, можно описывать алгоритмы только один раз, но работать они будут для разных типов.


% \chapter{C++ STL}



\chapter{Прочее}

Что хранится в объектных и бинарных (.exe, .dll) файлах?

Обычно только данные (глобальные переменные) и тела ф-ций (включая методы классов).

Информации о типах (классы, структуры, typedef\ref{typedef}-ы) там нет. 
Это может помочь в понимании того как всё устроено.

В этом заключается одна из больших проблем декомпиляции --- отсутствие информации о типах.

О том как всё компилируется в машинный код, подробнее можно почитать здесь:\cite{REBook}.


\chapter*{\IFRU{Послесловие}{Afterword}}
\addcontentsline{toc}{chapter}{\IFRU{Послесловие}{Afterword}}

% \section{\IFRU{Краудфандинг}{Crowdfunding}}

\IFRU{Эта книга является свободной, находится в свободном доступе, и доступна в виде исходных кодов}
{This book is free, available freely and available in source code form}\footnote{\url{https://github.com/dennis714/RE-for-beginners}} (LaTeX), 
\IFRU{и всегда будет оставаться таковой}{and it will be so forever}.

\IFRU{В мои текущие планы насчет этой книги входит добавление информации на эти темы:}
{My current plans for this books is to add a lot of information about} C++11, flex/bison.

\IFRU{Если вы хотите чтобы я продолжал свою работу и писал на эти темы,
вы можете рассмотреть идею краудфандинга}
{If you want me to continue writing on all these topics, you may consider crowdfunding}.

\IFRU{Со способами краудфандинга можно ознакомиться на странице}
{Ways to crowdfund are available on the page:} \url{http://yurichev.com/crowdfunding.html}

%\subsection{\IFRU{Жертвователи}{Donors}}


\section{\IFRU{Вопросы?}{Questions?}}

\IFRU{Совершенно по любым вопросам, вы можете не раздумывая писать автору}
{Do not hesitate to mail any questions to the author}: \TT{<\EMAIL>}
 
\IFRU{Пожалуйста, присылайте мне информацию о замеченных ошибках 
(включая грамматические), итд.}
{Please, also do not hesitate to send me any corrections 
(including grammar ones (you see how horrible my English is?)), etc.}

\chapter*{\IFRU{Список принятых сокращений}{Acronyms used}}

\begin{acronym}
\acro{STL}{Standard Template Library}
\acro{TLS}{Thread Local Storage}
\acro{TIB}{Thread Information Block}
\acro{RAII}{Resource Acquisition Is Initialization}
\IFRU{
	\acro{OS}[ОС]{Операционная Система}
	\acro{PL}[ЯП]{Язык Программирования}
}
{
	\acro{OS}{Operating System}
	\acro{PL}{Programming Language}
}
\acro{MSVC}{Microsoft Visual C++}
\acro{GCC}{GNU Compiler Collection}
\acro{POSIX}{Portable Operating System Interface}
\end{acronym}


%\bibliographystyle{alpha}
\bibliographystyle{plain} % FIXME
\bibliography{books,articles,usenet,misc}

\clearpage
\printindex

\end{document}

