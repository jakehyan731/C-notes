\chapter{\IFRU{Стандартные библиотеки Си/Си++}{C/C++ standard library}}

\section{Разница между stdout и stderr}

\IT{stdout} это то что выводится на консоль при помощи вызова \IT{printf()}.
\IT{stdout} это буферизированный вывод,
так что, пользователь, обычно того не зная, видит вывод порциями. Бывает так что программа выдает
что-то используя \IT{printf()} либо \IT{cout} и тут же падает.
Если это попадает в буфер, но буфер не успевает
``сброситься'' (flush) в консоль, то пользователь ничего не увидит. Это бывает неудобно.
Таким образом, для вывода более важной информации, в том числе отладочной, удобнее использовать \IT{stderr}.

\IT{stderr} это не буферизированный вывод, и всё что попадает в этот поток при помощи 
\TT{fprintf(stderr,...)} либо \IT{cerr}, появляется в консоли тут же.

Не следует также забывать, что из-за отсутствия буфера, вывод в \IT{stderr} медленнее.

Чтобы направлять \IT{stderr} в другой файл при запуске процесса, можно указывать:

\begin{lstlisting}
process 2> debug.txt
\end{lstlisting}

... это направит вывод \IT{stderr} в заданный файл (потому что номер этого потока -- 2).

\section{UNIX time}

В UNIX-среде очень популярно представление даты и времени в формате UNIX time.
Это просто 32-битное число, показывающее
количество прошедших секунд с 1-го января 1970-го года.

В качестве положительных сторон: 1) очень легко хранить это 32-битное число; 2) очень легко вычислять разницу дат;
3) невозможно закодировать неверные даты и время, такие как 32-е января, 29-е февраля невысокосных годов, 
25 часов 62 минуты.

В качестве отрицательных сторон: 1) нельзя закодировать дату до 1970-го года.

В наше время, если использовать UNIX time, тем не менее, следует помнить что ``срок его действия'' истечет
в 2038-м году, именно тогда 32-битное число переполнится, то есть, пройдет $2^{32}$ секунд с 1970-го года.
Так что, для этого следует использовать 64-битное значение, т.е., time64.

% NtQuerySystemTime http://msdn.microsoft.com/en-us/library/windows/desktop/ms724512(v=vs.85).aspx

