\chapter{\IFRU{Введение}{Preface}}

Сейчас, в 2013-м году, если некто желает написать 1) как можно более быстро работающую программу; 2) либо как
можно более компактную для встраиваемых систем либо маломощных микроконтроллеров, то выбор небольшой:
Си, Си++ либо ассемблер. И альтернативы этим старым но популярным языкам, в обозримом будущем, 
пока что не видно. \\
\\
О чистом Си также не стоит забывать, огромное количество больших программ продолжаются писаться на нем, 
например, ядро Linux, ядра линейки Windows NT, Oracle RDBMS, итд.

\section{\IFRU{Целевая аудитория}{Target audience}}

Этот сборник заметок предназначен не для начинающих, но и не для экспертов, а скорее для тех, 
кто хочет освежить свои знания по Си/Си++.

\section{\IFRU{Об авторе}{About author}}

\IFRU{Денис Юричев ~--- опытный программист, свободный для найма как программист, reverse engineer, консультант, (персональный) преподаватель.
С его резюме можно ознакомиться \href{http://yurichev.com/Dennis_Yurichev.pdf}{здесь}.}
{Dennis Yurichev is an experienced programmer, available for hire as programmer, reverse engineer, consultant or (personal) teacher.
His CV is available \href{http://yurichev.com/Dennis_Yurichev.pdf}{here}.}

\section{\IFRU{Благодарности}{Thanks}}

\IFRU{Андрей ''herm1t'' Баранович, Слава ''Avid'' Казаков}
{Andrey ''herm1t'' Baranovich, Slava ''Avid'' Kazakov}.

\input{donate}
