\chapter{\IFRU{Введение}{Preface}}

\IFRU{Сейчас, в 2013-м году, если некто желает написать}{Today, in year 2013, if one wants to write} 
1) \IFRU{как можно более быстро работающую программу}{as fast program as possible};
2) \IFRU{либо как можно более компактную для встраиваемых систем либо маломощных микроконтроллеров}
{or as compact as possible for embedded systems or low-cost microcontrollers},
\IFRU{то выбор очень ограниченный}{the choice is very limited}:
\IFRU{Си}{C}, \CPP \IFRU{либо ассемблер}{or assembly language}.
\IFRU{И альтернативы этим старым но популярным языкам, в обозримом будущем, пока что не видно}
{And as it seems in the near future, there are no alternative to these old but popular programming languages}. \\
\\
\IFRU{О ``чистом Си'' также не стоит забывать, огромное количество больших программ продолжаются писаться на нем, 
например}{``Pure C'' should be still considered, a huge number of large programs are still developed in it, e.g.}
\IFRU{ядро Linux}{Linux kernel}, \IFRU{ядра линейки Windows NT}{Windows NT OS line kernels}, Oracle RDBMS, \IFRU{итд}{etc}.

\section{\IFRU{Целевая аудитория}{Target audience}}

\IFRU{Этот сборник заметок предназначен не для начинающих, но и не для экспертов, а скорее для тех, 
кто хочет освежить свои знания по Си/Си++}{This notes collections is not intended for beginners, neither for experts,
it is rather for those who wants to fresh their C/\CPP knowledge}.

\section{\IFRU{Об авторе}{About author}}

\IFRU{Денис Юричев ~--- опытный программист, reverse engineer.
С его резюме можно ознакомиться \href{http://yurichev.com/Dennis_Yurichev.pdf}{здесь}.}
{Dennis Yurichev is an experienced programmer, reverse engineer.
His CV is available \href{http://yurichev.com/Dennis_Yurichev.pdf}{here}.}

\section{\IFRU{Благодарности}{Thanks}}

\IFRU{Андрей ''herm1t'' Баранович, Слава ''Avid'' Казаков}
{Andrey ''herm1t'' Baranovich, Slava ''Avid'' Kazakov}, Tuta Muniz.

%\section{\IFRU{Краудфандинг}{Crowdfunding}}

\IFRU{Эта книга является свободной, находится в свободном доступе, и доступна в виде исходных кодов}
{This book is free, available freely and available in source code form}\footnote{\url{https://github.com/dennis714/RE-for-beginners}} (LaTeX), 
\IFRU{и всегда будет оставаться таковой}{and it will be so forever}.

\IFRU{В мои текущие планы насчет этой книги входит добавление информации на эти темы:}
{My current plans for this books is to add a lot of information about} C++11, flex/bison.

\IFRU{Если вы хотите чтобы я продолжал свою работу и писал на эти темы,
вы можете рассмотреть идею краудфандинга}
{If you want me to continue writing on all these topics, you may consider crowdfunding}.

\IFRU{Со способами краудфандинга можно ознакомиться на странице}
{Ways to crowdfund are available on the page:} \url{http://yurichev.com/crowdfunding.html}

%\subsection{\IFRU{Жертвователи}{Donors}}

