\chapter{\IFRU{Введение}{Preface}}

Сейчас, в 2013-м году, если некто желает написать 1) как можно более быстро работающую программу; 2) либо как
можно более компактную для встраиваемых систем либо маломощных микроконтроллеров, то выбор небольшой:
Си, Си++ либо ассемблер. И альтернативы этим старым но популярным языкам, в обозримом будущем, 
пока что не видно. \\
\\
О чистом Си также не стоит забывать, огромное количество больших программ продолжаются писаться на нем, 
например, ядро Linux, ядра линейки Windows NT, Oracle RDBMS, итд.

\section{\IFRU{Целевая аудитория}{Target audience}}

Этот сборник заметок предназначен не для начинающих, но и не для экспертов, а скорее для тех, 
кто хочет освежить свои знания по Си/Си++.

\section{\IFRU{Об авторе}{About author}}

\IFRU{Денис Юричев ~--- опытный программист, свободный для найма как программист, reverse engineer, консультант, тренер. 
С его резюме можно ознакомиться \href{http://yurichev.com/Dennis_Yurichev.pdf}{здесь}.}
{Dennis Yurichev is an experienced programmer, available for hire as programmer, reverse engineer, consultant or trainer. 
His CV is available \href{http://yurichev.com/Dennis_Yurichev.pdf}{here}.}

\section{\IFRU{Благодарности}{Thanks}}

\IFRU{Андрей ''herm1t'' Баранович, Слава ''Avid'' Казаков}
{Andrey ''herm1t'' Baranovich, Slava ''Avid'' Kazakov}.

\section{\IFRU{Краудфандинг}{Crowdfunding}}

\IFRU{Эта книга является свободной, находится в свободном доступе, и доступна в виде исходных кодов}
{This book is free, available freely and available in source code form}\footnote{\url{https://github.com/dennis714/RE-for-beginners}} (LaTeX), 
\IFRU{и всегда будет оставаться таковой}{and it will be so forever}.

\IFRU{В мои текущие планы насчет этой книги входит добавление информации на эти темы:}
{My current plans for this books is to add a lot of information about} C++11, flex/bison.

\IFRU{Если вы хотите чтобы я продолжал свою работу и писал на эти темы,
вы можете рассмотреть идею краудфандинга}
{If you want me to continue writing on all these topics, you may consider crowdfunding}.

\IFRU{Со способами краудфандинга можно ознакомиться на странице}
{Ways to crowdfund are available on the page:} \url{http://yurichev.com/crowdfunding.html}

%\subsection{\IFRU{Жертвователи}{Donors}}

