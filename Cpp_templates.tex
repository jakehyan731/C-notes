\chapter{Темплейты C++}

Темплейты нужны обычно для того чтобы сделать класс универсальным для нескольких типов данных.
К примеру, \TT{std::string} в реальности это \TT{std::basic\_string<char>}, \\ 
а \TT{std::wstring} это \TT{std::basic\_string<wchar\_t>}. \\
\\
Нередко подобное делают и для float/double. Некий математический алгоритм может быть описан один раз,
но скомпилирован и для float и для double. \\
\\
Таким образом, можно описывать алгоритмы только один раз, но работать они будут для разных типов.

