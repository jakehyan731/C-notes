\section{Массивы}

В C99\ref{C99} можно передавать массив в аргументах ф-ции.

Собственно, массив байт можно было передавать и раньше, кодируя байты в строке, включая ноль, примерно так
(узнать, встречается ли байт \IT{c} в массиве байт)\ref{memchr}:

\begin{lstlisting}
if (memchr ("\x12\x34\x56\x78\x00\xAB", c, 6))
	...
\end{lstlisting}

Байты после ноля нормально кодируются.

Но в C99 теперь можно передавать массив значений других типов, например unsigned int:

\begin{lstlisting}
unsigned int find_max_value (unsigned int *array, size_t array_size);

unsigned int max_value=find_max_value ((unsigned[]){ 0x123, 0x456, 0x789, 0xF00 }, 4);
\end{lstlisting}

Поиск в массиве можно реализовать при помощи ф-ций bsearch() или lfind()\ref{bsearch_lfind}, 
поиск и вставку при помощи lsearch()\footnote{работает также как и lfind(), но при отсутствии искомого элемента,
добавляет его в массив}.

