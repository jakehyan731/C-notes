\subsection{alloca()}

Ф-ция alloca() выделяет память в локальном стеке точно также, отодвигая указатель стека\cite[1.2.4]{REBook}.
Память будет освобождена в конце ф-ции автоматически.

В стандарте C99\ref{C99}, использовать alloca() уже не обязательно, там можно просто писать:

\begin{lstlisting}
void f(size_t s, ...)
{
	char a[s];
};
\end{lstlisting}

Впрочем, внутри, это работает так же как и alloca().

Критика: Линус Торвальдс против использования alloca()\cite{Torvalds:2003}.
