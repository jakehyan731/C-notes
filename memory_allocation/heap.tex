\subsection{Выделение памяти в куче}

Куча (heap) это какая-то часть памяти выделенная ОС процессу, где процесс может уже сам делить эту часть как хочет.
После заврешения процесса, в т.ч., некорректного, куча автоматически аннулируется и ОС не нужно разбирать 
по одному все выделенные процессом блоки.

Для работы с кучей есть ф-ции malloc(), calloc(), realloc(), free(), а в Си++ --- new и delete.

Очевидно, чтобы поддерживать информацию о выделенных блоках в куче, нужна масса связных друг с другом структур.
Отсюда имеется вполне осязаемые накладные расходы (overhead). Вы можете выделить блок размером 8 байт, 
но еще как минимум 8 байт (MSVC, 32-битная Windows, примерно то же самое и в Linux)
будет задействованы для хранения информации о выделенном блоке\footnote{это еще называют ``метаданными''}.
В 64-битных ОС указатели занимают в два раза больше, так что информация о каждом блоке будет занимать как минимум
16 байт.

Использование кучи требует некоторой программистской дисциплины, без которой легко наделать ошибок.
Возможно поэтому, считается что ЯП с RAII как Си++
либо ЯП со сборщиками мусора (Python, Ruby) легче.

\subsubsection{Одна из основных ошибок: утечки памяти}

Память была выделена, но её забыли освободить через free(). Эта проблема довольно легко решается своей собственной
надстройкой над ф-циями malloc()/free(). Пусть эта надстройка ведет учет выделенных блоков, а также, где и когда
(и для чего) был выделен тот или иной блок.

Я сделал это в своей библиотеке octothorpe\footnote{\url{https://github.com/dennis714/octothorpe/blob/master/dmalloc.c}}. 
Например, макрос DMALLOC вызывает ф-цию dmalloc(), передавая
ей имя файла, имя ф-ции, номер строки, а также комментарий (имя блока). В конце работы программы, вызываем
\TT{dump\_unfreed\_blocks()} и он покажет список блоков, которые забыли освободить.

\begin{lstlisting}
seq_n:2, size: 124, filename: dmalloc_test.c:31, func: main, struct: block124
seq_n:3, size: 12, filename: dmalloc_test.c:33, func: main, struct: block12
seq_n:4, size: 555, filename: dmalloc_test.c:35, func: main, struct: block555
\end{lstlisting}

У каждого блока есть также номер.
Это для того чтобы можно было установить брякпоинт по номеру выделяемого блока --- 
тогда отладчик сработает когда этот блок будет выделяться и вы увидите, где и при каких условиях это происходит.

Писать в коде комментарии для каждого выделяемого блока памяти нудно, но очень полезно. Потом легко увидеть,
под что была выделена память. Я впервые увидел эту идею в Oracle RDBMS. Помимо всего прочего, там еще и ведется
статистика, под что было выделено больше памяти, её можно легко увидеть:

\begin{lstlisting}
SQL> select * from v$sgastat;

POOL         NAME                            BYTES     CON_ID
------------ -------------------------- ---------- ----------
shared pool  AQ Slave list                    1224          1
shared pool  KQR L PO                       653312          2
shared pool  KQR X SO                       635808          2
shared pool  RULEC                           20688          1
shared pool  KQR M SO                         7168          2
shared pool  work area table entry           12240          2
shared pool  kglsim object batch              3864          2
large pool   PX msg pool                    860160          1
large pool   free memory                  30523392          0
large pool   SWRF Metric CHBs              1802240          2
large pool   SWRF Metric Eidbuf             368640          2
\end{lstlisting}

Подобная штука также присутствует и в ядре Windows, там это называется tagging. При выделении памяти
в ядре или драйвере, нужно указывать также 32-битный тег (обычно, четырехбуквенное сокращение, означающее
подсистему Windows). 
Затем в отладчике можно увидеть статистику, под что выделено больше всего памяти:

\begin{lstlisting}
kd> !poolused 4
   Sorting by  Paged Pool Consumed

  Pool Used:
            NonPaged            Paged
 Tag    Allocs     Used    Allocs     Used
 CM25        0        0       935  4124672	Internal Configuration manager allocations , Binary: nt!cm
 Gh05        0        0       268  3291016	GDITAG_HMGR_SPRITE_TYPE , Binary: win32k.sys
 MmSt        0        0      2119  2936752	Mm section object prototype ptes , Binary: nt!mm
 CM35        0        0        91  2150400	Internal Configuration manager allocations , Binary: nt!cm
 vmfb        0        0        13  2148752	UNKNOWN pooltag 'vmfb', please update pooltag.txt
 Ntff        5     1040      1287  1070784	FCB_DATA , Binary: ntfs.sys
 ArbA        0        0       108   442368	ARBITER_ALLOCATION_STATE_TAG , Binary: nt!arb
 NtfF        0        0       457   431408	FCB_INDEX , Binary: ntfs.sys
 CM16        0        0        62   331776	Internal Configuration manager allocations , Binary: nt!cm
 IoNm        0        0      2022   267288	Io parsing names , Binary: nt!io
 Ttfd        0        0       159   253976	TrueType Font driver 
 Ifs         0        0         4   249968	Default file system allocations (user's of ntifs.h) 
 CM29        0        0        26   212992	Internal Configuration manager allocations , Binary: nt!cm
\end{lstlisting}

Конечно, можно возразить, что для этого нужно хранить еще больше информации о выделенных блоках, а еще
и теги, названия блоков. 
И это еще сильнее замедляет работу программы. Конечно. 
Поэтому пусть это будет работать только в отладочных (debug) сборках, 
а в release-сборках, DMALLOC() становится обычной функцией-переходником\footnote{thunk function} для malloc().
Ну а в ядре Windows это вообще по умолчанию отключено, и нужно включать при помощи утилиты GFlags
\footnote{\url{http://msdn.microsoft.com/en-us/library/windows/hardware/ff549557(v=vs.85).aspx}}
Помимо всего прочего, подобное есть и в MSVC
\footnote{читайте больше о ф-циях 
\href{http://msdn.microsoft.com/en-us/library/5at7yxcs.aspx}{\_CrtSetDbgFlag}
и
\href{http://msdn.microsoft.com/en-us/library/d41t22sb.aspx}{\_CrtDumpMemoryLeaks}}.

\subsubsection{Одна из основных ошибок: разрушение кучи}

Нетрудно выделить память под 4 байта, но по ошибке дописать туда пятый. Скорее всего, никак это не проявится,
но фактически это очень опасная мина замедленного действия, опасная, потому что ведет к трудновыявляемым
ошибкам. Байт следующий за выделенным вам блоком может не использоваться вовсе, но также там уже
может начинаться какая-то структура менеджера памяти, отвечающая за учет блоков. Если какую-то из
таких структур сознателно разрушить, перезаписать, то тогда следующие вызовы malloc() или free() не смогут
корректно работать. Иногда это проявляется в выводе ошибок вроде (в Windows):

\begin{lstlisting}
HEAP[Application.exe]: HEAP: Free Heap block 211a10 modified at 211af8 after it was freed
\end{lstlisting}

Подобные ошибки эксплуатируются авторами эксплоитов: если знать что вы можете изменить структуры данных
менеджера памяти нужным вам образом, вы можете добиться какого-то нужного вам поведения программы 
(это называется heap overflow\footnote{\url{https://en.wikipedia.org/wiki/Heap_overflow}}).

Довольно распространенный метод борьбы с подобными ошибками: это просто дописывать ``guard''-ы с обоих сторон
блока, например, 4-байтного размера. Например, я сделал это в своем DMALLOC. При каждом вызове free(),
проверяется целостность guard-ов (это могут быть просто какие-то фиксированные значения вроде 0x12345678),
и если кто-то или что-то затерло один из них, можно тут же сообщить об этом.

%\subsubsection{Приемущества своего собственного менеджера памяти или надстройки над стандартным}
%... может выдать размер выделенного блока, хотя нафик оно надо...

\subsubsection{Одна из основных ошибок: непроверка результата malloc()}

При успешном выполнении, malloc() возвращает указатель на блок, который можно использовать, либо NULL,
если памяти не хватает. Конечно, в наше время дешевой памяти эта проблема становится редкой, тем не менее,
если вы используете много памяти, думать об этом все же надо. Проверять указатель после каждого вызова
malloc() неудобно, так что довольно популярный метод это писать свои функции-переходники с названием 
xmalloc(), xrealloc(), вызывающие malloc()/realloc(), но проверяющие их результат и падающие в случае
ошибки.

Интересно упомянуть, как ведет себя xmalloc() в git:

\begin{lstlisting}
void *xmalloc(size_t size)
{
	void *ret;

	memory_limit_check(size);
	ret = malloc(size);
	if (!ret && !size)
		ret = malloc(1);
	if (!ret) {
		try_to_free_routine(size);
		ret = malloc(size);
		if (!ret && !size)
			ret = malloc(1);
		if (!ret)
			die("Out of memory, malloc failed (tried to allocate %lu bytes)",
			    (unsigned long)size);
	}
#ifdef XMALLOC_POISON
	memset(ret, 0xA5, size);
#endif
	return ret;
}
\end{lstlisting}
\footnote{\url{https://github.com/git/git/blob/master/wrapper.c}}

Если malloc() не успешен, он пытается освободить какие-то уже выделенные (и не очень нужные) 
блоки при помощи try\_to\_free\_routine(), а затем вызвать malloc() снова.

Помимо всего прочего, если определен XMALLOC\_POISON, все байты в выделенном блоке заполняются 0xA5.
Это может помочь визуально, на глаз, увидеть когда вы, например, выделили память под структуру,
а затем используете какое-то поле из нее до того как инициализировали. Значение \TT{0xA5A5A5A5} будет
бросаться в глаза в отладчике, ну или просто если вы захотите где-то в дампе вывести его в шестнадцатеричной
форме. В MSVC для этой же цели служит константа \TT{0xbaadf00d}.

И даже более того: после вызова free(), освобожденный блок может маркироваться уже какой-то другой константой,
чтобы если кто-то захочет использовать что-то оттуда после освобождения блока, это также было видно, хотя
бы визуально.

Примеры констант от Microsoft:

\begin{lstlisting}
* 0xABABABAB : Used by Microsoft's HeapAlloc() to mark "no man's land" guard bytes after allocated heap memory
* 0xABADCAFE : A startup to this value to initialize all free memory to catch errant pointers
* 0xBAADF00D : Used by Microsoft's LocalAlloc(LMEM_FIXED) to mark uninitialised allocated heap memory
* 0xBADCAB1E : Error Code returned to the Microsoft eVC debugger when connection is severed to the debugger
* 0xBEEFCACE : Used by Microsoft .NET as a magic number in resource files
* 0xCCCCCCCC : Used by Microsoft's C++ debugging runtime library to mark uninitialised stack memory
* 0xCDCDCDCD : Used by Microsoft's C++ debugging runtime library to mark uninitialised heap memory
* 0xDEADDEAD : A Microsoft Windows STOP Error code used when the user manually initiates the crash.
* 0xFDFDFDFD : Used by Microsoft's C++ debugging heap to mark "no man's land" guard bytes before and after allocated heap memory
* 0xFEEEFEEE : Used by Microsoft's HeapFree() to mark freed heap memory
\end{lstlisting}
\footnote{\url{https://en.wikipedia.org/wiki/Magic_number_(programming)}}

\subsubsection{Еще частые ошибки}

Если не включить заголовочный файл stdlib.h, 
GCC считает возвращаемое значение неизвестной ф-ции malloc() за int и постоянно ругается на приведение типов.

Еще одна ошибка, которая может попортить нервов, это выделить один и тот же блок памяти в одном месте 
больше одного раза (предыдущие вызовы ``теряются'' из вида).

\subsubsection{Методы борьбы}

Однако, может оказаться так, что ошибки в программе у вас есть, а перекомпилировать её по каким-то причинам
вы не можете. Тогда может помочь, например, valgrind\footnote{\url{http://valgrind.org/}}.

