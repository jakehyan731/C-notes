\section{union}

union часто используется, когда в каком-то месте структуры можно хранить разные типы на выбор.
К примеру:

\begin{lstlisting}
union
{
	int i; // 4 bytes
	float f; // 4 bytes
	double d; // 8 bytes
} u;
\end{lstlisting}

Такой union позволяет хранить одну из этих трех переменных на выбор. Занимать он будет места столько же,
сколько максимальный элемент (double) --- 8 байт.

union часто используют для обращения к какому-то типу данных как к другому.

Например, как известно, каждый XMM-регистр в SSE может представлять собой 16 байт, 8 16-битных слов,
4 32-битных слова, 2 64-битных слова, 4 float-значения и 2 double-значения. Так можно описать его:

\begin{lstlisting}
union
{
	double d[2];
	float f[4];
	uint8_t b[16];
	uint16_t w[8];
	uint32_t i[4];
	uint64_t q[2];
} XMM_register;

union XMM_register reg1;

reg.u.d[0]=123.4567;
reg.u.d[1]=89.12345;

// here we can use reg.u.b[...]

\end{lstlisting}

Это также очень удобно использовать вместе со структурой, где поля имеют битовую гранулярность.
Это флаги x86-процессора:

\begin{lstlisting}
typedef struct _s_EFLAGS
{
    unsigned CF : 1;
    unsigned reserved1 : 1;
    unsigned PF : 1;
    unsigned reserved2 : 1;
    unsigned AF : 1;
    unsigned reserved3 : 1;
    unsigned ZF : 1;
    unsigned SF : 1;
    unsigned TF : 1;
    unsigned IF : 1;
    unsigned DF : 1;
    unsigned OF : 1;
    unsigned IOPL : 2;
    unsigned NT : 1;
    unsigned reserved4 : 1;
    unsigned RF : 1;
    unsigned VM : 1;
    unsigned AC : 1;
    unsigned VIF : 1;
    unsigned VIP : 1;
    unsigned ID : 1;
} s_EFLAGS;

typedef union _u_EFLAGS
{
    uint32_t flags;
    s_EFLAGS s;
} u_EFLAGS;
\end{lstlisting}

Можно таким образом загрузить флаги как 32-битное значение в поле flags, а затем из поля s обращаться
к отдельным битам. Либо наоборот, модифицировать биты, затем прочитать поле flags.

\subsection{tagged union}

Это union плюс флаг (tag), определяющий тип union. К примеру, если нам нужна какая-то переменная,
которая может быть как числом, так и числом с плавающей точкой, так и текстовой строкой (как переменные
в динамически-типизированных ЯП), то мы можем объявить такую структуру:

\begin{lstlisting}

enum var_type
{
	INT,
	DOUBLE,
	STRING
};

struct
{
	enum var_type tag; // 4 bytes
	union
	{
		int i; // 4 bytes
		double d; // 8 bytes
		char *string; // 4 bytes (on 32-bit architecture)
	} u;
} variable;
\end{lstlisting}

Суммарная длина такой структуры будет $8+4=12$ байт. В любом случае, это компактнее, чем выделять
поля для переменной каждого возможного типа.

