\section{\IFRU{Возврат кодов ошибок}{Error codes returning}}

\IFRU{Самый простой способ сообщить вызываемой ф-ции о своем успехе, это вернуть булево значение}
{The simplest way to indicate to caller about success is to return boolean value},
\IT{false} ~--- \IFRU{в случае ошибки}{in case of error},
\AndENRU \IT{true} \IFRU{в случае успеха}{in case of success}.
\index{Windows API}
\IFRU{Такого очень много в}{A lot of such functions are present in the} Windows API.
\IFRU{А если нужно передать больше информации, то оставить код ошибки в \ac{TIB}, откуда затем получить 
это значение вызвав}{And if one need to return more information, the error code may be left in \ac{TIB},
from where it is possible to get it using} GetLastError(). 
\index{errno}
\IFRU{Либо, в}{Or, in the} UNIX-\IFRU{средах, оставлять код ошибки в глобальной переменной}{enviroments,
to leave the error code in the global variable} \IT{errno}.

\subsection{\IFRU{Отрицательные коды ошибок}{Negative error codes}}

\IFRU{Еще один интересный способ передать больше информации в возвращаемом значении}
{Another interesting approach to pass more information in returning value}.
\index{IBM DB2}
\IFRU{Например, в документации}{For example, in the manuals of the} IBM DB2 9.1, \IFRU{мы можем увидеть такое}
{we may spot this}:

\begin{framed}
\begin{quotation}
Regardless of whether the application program provides an SQLCA or a stand-alone variable, SQLCODE is set by DB2 after each SQL statement is executed. DB2 conforms to the ISO/ANSI SQL standard as follows:\\
\\
If SQLCODE = 0, execution was successful.\\
If SQLCODE > 0, execution was successful with a warning.\\
If SQLCODE < 0, execution was not successful.\\
SQLCODE = 100, "no data" was found. For example, a FETCH statement returned no data because the cursor was positioned after the last row of the result table.
\end{quotation}
\end{framed}
\footnote{\href{http://publib.boulder.ibm.com/infocenter/dzichelp/v2r2/index.jsp?topic=\%2Fcom.ibm.db2z9.doc.codes\%2Fsrc\%2Ftpc\%2Fdb2z\_sqlcodes.htm}{SQL codes}}
\footnote{\href{http://publib.boulder.ibm.com/infocenter/dzichelp/v2r2/index.jsp?topic=\%2Fcom.ibm.db2z9.doc.codes\%2Fsrc\%2Ftpc\%2Fdb2z\_n.htm}{SQL error codes}}

\IFRU{В исходниках Linux можно увидеть и такое}{In the Linux kernel source code we may find this}
\footnote{\url{http://lxr.free-electrons.com/source/arch/powerpc/kernel/crash_dump.c}}:

\begin{lstlisting}[caption=arch/arm/kernel/crash\_dump.c]
/**
 * copy_oldmem_page() - copy one page from old kernel memory
 * @pfn: page frame number to be copied
 * @buf: buffer where the copied page is placed
 * @csize: number of bytes to copy
 * @offset: offset in bytes into the page
 * @userbuf: if set, @buf is int he user address space
 *
 * This function copies one page from old kernel memory into buffer pointed by
 * @buf. If @buf is in userspace, set @userbuf to %1. Returns number of bytes
 * copied or negative error in case of failure.
 */
ssize_t copy_oldmem_page(unsigned long pfn, char *buf,
			 size_t csize, unsigned long offset,
			 int userbuf)
{
	void *vaddr;

	if (!csize)
		return 0;

	vaddr = ioremap(pfn << PAGE_SHIFT, PAGE_SIZE);
	if (!vaddr)
		return -ENOMEM;

	if (userbuf) {
		if (copy_to_user(buf, vaddr + offset, csize)) {
			iounmap(vaddr);
			return -EFAULT;
		}
	} else {
		memcpy(buf, vaddr + offset, csize);
	}

	iounmap(vaddr);
	return csize;
}
\end{lstlisting}

\IFRU{Обратите внимание}{Take a look} ~--- 
\IFRU{ф-ция может вернуть как количество байт, так и код ошибки}{a function many return both number of bytes and
the error code}. 
\IFRU{Тип}{The} ssize\_t \IFRU{это}{type is} ``\IFRU{знаковый}{signed}'' size\_t,
\IFRU{то есть, способный принимать отрицательные значения}{i.e., able to store negative values}.
ENOMEM \AndENRU EFAULT \IFRU{это стандартные коды ошибок из}{are standard error codes from the} errno.h.

