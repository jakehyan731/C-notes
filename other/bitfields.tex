\section{\IFRU{Битовые поля}{Bit fields}}

\IFRU{Это очень популярная штука в Си, да и в программировании вообще}{A very popular thing in C, and also
in programming generally}.

\IFRU{Для задания булевых значений}{For defining boolean values} ``true'' \OrENRU ``false'',
\IFRU{можно передавать $1$ или $0$ в байте или в 32-битном регистре}{it is possible to pass $1$ or $0$ in byte or
32-bit register},
\IFRU{или в типе \IT{int}}{or in \IT{int} type}, \IFRU{но это очень не экономно в плане расхода памяти}
{but it is not very compact}.
\IFRU{Намного удобнее передавать такие значения в отдельных битах}{Much more handy is to pass such values
in specific bits}.

\index{findfirst()}
\IFRU{К примеру, стандартная ф-ция}{For example, the standard C function} findfirst() 
\IFRU{возвращает структуру о найденном файле, где аттрибуты файла передаются
такими флагами}{returns a structure about file it found where file attributes are encoded as}:

\begin{lstlisting}
#define _A_NORMAL 0x00
#define _A_RDONLY 0x01
#define _A_HIDDEN 0x02
#define _A_SYSTEM 0x04
#define _A_SUBDIR 0x10
#define _A_ARCH 0x20
\end{lstlisting}

\IFRU{Конечно, передавать каждый аттрибут отдельной переменной типа}{Of course, it would not be very
compact to pass each attribute by a} \IT{bool} \IFRU{было бы очень неэкономично}{variable}.

\index{Windows API!CreateFile()}
\IFRU{И напротив, для указания флагов в ф-цию можно исползовать битовые поля}
{Contrariwise, it is possible to use bit fields for passing flags into the function}.
\IFRU{Например}{For example}, CreateFile()
\footnote{\url{http://msdn.microsoft.com/en-us/library/windows/desktop/aa363858(v=vs.85).aspx}} 
\IFRU{из}{from} Windows API.

\IFRU{Для задания флагов, чтобы не запутаться и не опечататься в значениях каждого бита, можно писать так}
{For flags specifying, in order not to make a typo and mess, they can be defined as}:

\begin{lstlisting}
#define FLAG1 (1<<0)
#define FLAG2 (1<<1)
#define FLAG3 (1<<2)
#define FLAG4 (1<<3)
#define FLAG5 (1<<4)
\end{lstlisting}

(\IFRU{Компилятор все равно всё это легко соптимизирует}{The compiler will optimize it anyway}).

\IFRU{А чтобы было удобнее проверять/выставлять/удалять отдельный бит/флаг,
можно использовать подобные макросы}{These handy macros may be also used for specific bit/flag 
checking/setting/resetting}:

\begin{lstlisting}
#define IS_SET(flag, bit)       (((flag) & (bit)) ? true : false)
#define SET_BIT(var, bit)       ((var) |= (bit))
#define REMOVE_BIT(var, bit)    ((var) &= ~(bit))
\end{lstlisting}

\index{bool}
\IFRU{С другой стороны, необходимо помнить, что операции выделения отдельного бита из значения типа}
{On the other hand, one need to keep in mind that operation of isolation of each bit in the value of type} \IT{int}
\IFRU{обычно даются \ac{CPU} ``дороже'', чем работа с типом \IT{bool} в 32-битном регистре}
{is usually costly for the \ac{CPU} then \IT{bool} type processing in 32-bit register }.
\IFRU{Так что, если скорость для вас намного критичнее чем
экономия памяти, можно попробовать использовать тип}
{So if the speed is more crucial for you then memory footprint, you may try to use}
\IT{bool}.

