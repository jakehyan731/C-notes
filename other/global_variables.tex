\section{\IFRU{Глобальные переменные}{Global variables}}

\IFRU{Мода на }{}\ac{OOP}\IFRU{ и всякие такие штуки утвердила что глобальные переменные это плохо, тем не менее,
в каких-то разумных случаях, это можно использовать (не забывая о мультитредовости), например,
для возврата большего количества информации из ф-ций}
{ hype and other such things tells us that global variables is a bad thing, nevertheless,
sometimes it is worth to use it (keeping in mind thread-awareness),
e.g. for returning large amount of information from functions}.

\IFRU{Так, некоторые стандартные ф-ции Си возвращают код ошибки через глобальную переменную}
{Thus, several C standard library functions are returned error code via global variable} \IT{errno}, 
\IFRU{которая, в наше
время, фактически не глобальная, а находится внутри}{which is not a global anymore in our time,
but is stored in the} \ac{TLS}.

\index{Windows API!GetLastError()}
\IFRU{В}{In} Windows API \IFRU{код ошибки можно узнать вызвав}{the error code can be determined by calling the}
GetLastError(), 
\IFRU{которая, на самом деле, выдает значение из}{which just takes a value from the} \ac{TIB}.

\index{OpenWatcom}
\IFRU{В кодегенераторе компилятора}{In the} OpenWatcom\IFRU{, всё находится в глобальных переменных,
а самая главная ф-ция выглядит так}{ compiler everything is stored in the global variables, so the very
main function looks like}:

\begin{lstlisting}[caption=bld\textbackslash{}cg\textbackslash{}c\textbackslash{}generate.c]
extern  void    Generate( bool routine_done )
/*******************************************/
/* The big one - here's where most of code generation happens.
 * Follow this routine to see the transformation of code unfold.
 */
{
    if( BGInInline() ) return;
    HaveLiveInfo = FALSE;
    HaveDominatorInfo = FALSE;
    #if ( _TARGET & ( _TARG_370 | _TARG_RISC ) ) == 0
        /* if we want to go fast, generate statement at a time */
        if( _IsModel( NO_OPTIMIZATION ) ) {
            if( !BlockByBlock ) {
                InitStackMap();
                BlockByBlock = TRUE;
            }
            LNBlip( SrcLine );
            FlushBlocks( FALSE );
            FreeExtraSyms( LastTemp );
            if( _MemLow ) {
                BlowAwayFreeLists();
            }
            return;
        }
    #endif

    /* if we couldn't get the whole procedure in memory, generate part of it */
    if( BlockByBlock ) {
        if( _MemLow || routine_done ) {
            GenPartialRoutine( routine_done );
        } else {
            BlkTooBig();
        }
        return;
    }

    /* if we're low on memory, go into BlockByBlock mode */
    if( _MemLow ) {
        InitStackMap();
        GenPartialRoutine( routine_done );
        BlowAwayFreeLists();
        return;
    }

    /* check to see that no basic block gets too unwieldy */
    if( routine_done == FALSE ) {
        BlkTooBig();
        return;
    }

    /* The routine is all in memory. Optimize and generate it */
    FixReturns();
    FixEdges();
    Renumber();
    BlockTrim();
    FindReferences();
    TailRecursion();
    NullConflicts( USE_IN_ANOTHER_BLOCK );
    InsDead();
    FixMemRefs();
    FindReferences();
    PreOptimize();
    PropNullInfo();
    MemtoBaseTemp();
    if( _MemCritical ) {
        Panic( FALSE );
        return;
    }
    MakeConflicts();
    if( _IsModel( LOOP_OPTIMIZATION ) ) {
        SplitVars();
    }
    AddCacheRegs();
    MakeLiveInfo();
    HaveLiveInfo = TRUE;
    AxeDeadCode();
    /* AxeDeadCode() may have emptied some blocks. Run BlockTrim() to get rid
     * of useless conditionals, then redo conflicts etc. if any blocks died.
     */
    if( BlockTrim() ) {
        FreeConflicts();
        NullConflicts( EMPTY );
        FindReferences();
        MakeConflicts();
        MakeLiveInfo();
    }
    FixIndex();
    FixSegments();
    FPRegAlloc();
    if( RegAlloc( FALSE ) == FALSE ) {
        Panic( TRUE );
        HaveLiveInfo = FALSE;
        return;
    }
    FPParms();
    FixMemBases();
    PostOptimize();
    InitStackMap();
    AssignTemps();
    FiniStackMap();
    FreeConflicts();
    SortBlocks();
    if( CalcDominatorInfo() ) {
        HaveDominatorInfo = TRUE;
    }
    GenProlog();
    UnFixEdges();
    OptSegs();
    GenObject();
    if( ( CurrProc->prolog_state & GENERATED_EPILOG ) == 0 ) {
        GenEpilog();
    }
    FreeProc();
    HaveLiveInfo = FALSE;
#if _TARGET & _TARG_INTEL
    if( _IsModel( NEW_P5_PROFILING ) ) {
        FlushQueue();
    }
#else
    FlushQueue();
#endif
}
\end{lstlisting}
