\label{COOP}
\section{\COOPname}

\IFRU{Как известно, в Си нет поддержки ООП, она есть в Си++, тем не менее, в ``чистом'' Си вполне
можно программировать в стиле ООП}
{Of course, there are no OOP support in C, it is present in C++, nevertheless, it is possible
to program in OOP style in ``pure'' C}.

\IFRU{ООП, коротко говоря, это явное разделение на объекты и методы}
{OOP, in short, is a separation to object and methods}.
\IFRU{В Си структуры легко могут представляться объектами, а обычные ф-ции}
{In C, structures are easily can be represented as objects, and usual functions} ~--- 
\IFRU{методами}{as methods}.

\subsection{\IFRU{Инициализация структур}{Structures initialization}}
\label{COOPInit}

\IFRU{В \CPP у классов имеются конструкторы}{\CPP has class constructors}.
\IFRU{Если вам нужно каким-то особенным образом инициализировать
структуру, вам и в Си придется делать подобную ф-цию}{If one need to initialize structure in some
special way, one would write a special function for it in C as well}.
\index{calloc()}
\index{bzero()}
\IFRU{Но если структура простая, то её можно инициализировать при помощи}
{But if it simple structure, it is possible to initialize it with} calloc()
\footnote{\IFRU{Это тоже самое что и malloc() + заполнение выделенной памяти нулями}
{It is the same as the malloc() + allocated memory filling with zeroes}}
\OrENRU bzero()(\ref{bzero}).

\IFRU{Все int-переменные становятся нулями}{All int-variables are set to zeroes}.
\index{C99!bool}
\index{C++!bool}
\index{Windows API!BOOL}
\IFRU{Нулевое значение}{Zero value} bool \InENRU C99(\ref{C99}) \AndENRU \CPP \IFRU{это}{is} false,
\IFRU{так же как и}{same as} BOOL \InENRU Windows API.
\IFRU{Все указатели становятся}{All pointers are set to} NULL.
\index{IEEE 754}
\IFRU{И даже вещественный ноль представляемый в формате}{And even floating point $0.0$ in} 
IEEE 754 \IFRU{это также все ноли во всех битах}{format is zero bits in all positions}.

\IFRU{Если в структуре присутствуют указатели на другие структуры,
то NULL может означать ``отсутствие объекта''}{If structure has pointers to another structures,
NULL can mean ``object absence''}.

\IFRU{Помимо всего прочего, неинициализированные глобальные переменные также обнуляются}
{Among other things, not initialized global variables are also zeroed}
\cite[6.7.8.10]{C99TC3}.

\subsection{\IFRU{Деинициализация структур}{Structures deinitialization}}

\IFRU{Если в структуре есть ссылки на другие структуры, то их нужно освобождать}
{If structure has pointers to another structures, they are also must be freed}.
\IFRU{В простом случае, обычным вызовом}{In simple case, it is just a call to} free().
\index{free()}
\IFRU{Кстати, вот почему free() может принимать на вход NULL,
это чтобы можно было просто писать}
{By the way, that is why NULL is valid argument for free(), it allows to write} 
\TT{free(s->field)} \IFRU{вместо}{instead of}
\TT{if (s->field) free(s->field)}, \IFRU{так короче}{that is shorter}.

\subsection{\IFRU{Копирование структур}{Structures copying}}

\index{memcpy()}
\IFRU{Если структура простая, то её можно копировать обычным побайтовым копированием}
{If the structure is simple, it is possible to copy it with a call to} memcpy()(\ref{memcpy}).
\index{Shallow copy}
\index{Deep copy}
\IFRU{Если в такой манере скопировать структуру, в которой есть указатели на другие структуры,
то это будет называться}
{If to copy structures having pointers to another structures in this manner, it will be called}
``shallow copy''\footnote{\url{https://en.wikipedia.org/wiki/Object_copy}}. 
\IFRU{И напротив}{And in opposite}, \IT{deep copy} ~--- \IFRU{это копирование структуры
плюс всех связанных с ней структур (это дольше)}
{is copying a structure with all connected to it structures (is slower)}.

\IFRU{Вот почему может быть удобнее хранить строку в структуре как
обычный массив символов фиксированной длины}
{That is why it may be more convenient to store a string in the structure as 
an fixed-size array of characters}.
\index{Windows API}
\IFRU{Такого, например, очень много в}{For example, a lot of such cases in the} Windows API.
\IFRU{Такую структуру легко скопировать, её хранение требует меньших накладных расходов
\footnote{overhead} в куче}{Such structure is easier to copy, it requires smaller
memory overhead in the heap}.
\IFRU{Но с другой стороны, придется согласиться с ограничением на длину строки}
{On the other hand, we should accept string length fixedness}.

\IFRU{Помимо всего прочего, структуру можно копировать просто так}
{Aside from that, a structure can be copied just as}: \TT{s1=s2} ~--- 
\IFRU{в итоге генерируется код, копирющий все поля по порядку}{the code generated will copy
each structure filed}.
\IFRU{И это наверное легче читается чем вызов}{Perhaps it is easier to read than a call to} memcpy() 
\IFRU{на этом же месте}{at the same place}.

\subsection{\IFRU{Инкапсуляция}{Encapsulation}}

\CPP \IFRU{предлагает инкапсуляцию (сокрытие информации)}{offers encapsulation (information hiding)}.
\IFRU{Например, вы не можете
написать программу модифицирующую защищенное поле в классе, 
это защита на стадии компиляции}
{For example, you cannot write a program which modify a protected class
field, this is a compile-stage protection}\cite[1.7.3]{REBook}.

\IFRU{В Си этого нет, поэтому тут нужно больше дисциплины}
{There are no such thing in C, it requires more discipline}.

\IFRU{Впрочем, можно попытаться ``защитить'' структуру ``от посторонних глаз''}
{However, it is possible to ``protect'' a structure from ``prying eyes''}.
\index{Glib!GTree}
\IFRU{Например, в Glib, имеется библиотека для работы с деревьями}{For example, Glib has a library
intended for work with trees}.
\IFRU{В заголовочном файле}{In the header file}
gtree.h\footnote{\url{https://github.com/GNOME/glib/blob/master/glib/gtree.h}} 
\IFRU{нет описания самой структуры}{there are no declaration of the structure}
(\IFRU{она есть только в}{it is present only in the} gtree.c
\footnote{\url{https://github.com/GNOME/glib/blob/master/glib/gtree.c}}), 
\IFRU{а есть только}{there are only} forward declaration(\ref{forwarddeclaration}).
\IFRU{Так разработчики Glib могут понадеятся что пользователи GTree 
постараются не пользоваться отдельными полями в структуре напрямую}
{Thus Glib developers may have a hope that GTree users will not try to use specific fields in
the structure directly}.

\IFRU{Но у такого метода есть и обратная сторона}{The technique does have its flip side}: 
\IFRU{могут быть крохотные однострочные ф-ции вроде 
``вернуть длину строки''}{there are may be tiny one-line functions like ``return string length''}
\InENRU strbuf(\ref{strbuf}), \IFRU{например}{e.g.}:

\begin{lstlisting}
typedef struct _strbuf
{
    char *buf;
    unsigned strlen;
    unsigned buflen;
} strbuf;

unsigned strbuf_get_len(strbuf *s)
{
	return s->strlen;
};
\end{lstlisting}

\IFRU{Если компилятору на стадии компиляции доступно и описание структуры и тело ф-ции,
то в каком-то месте,
вместо вызова}{If the compiler on compiling stage have access to structure declaration and the 
function body, instead of call to} strbuf\_get\_len() 
\IFRU{он может сделать эту ф-цию как inline-овую, вставить её тело прямо на том
же место и сэкономить на вызове другой ф-ции}{it may make this function as inline, i.e., insert
its body right at the place and save some resources on call to another function}.
\IFRU{Но если эта информация компилятору недоступна, то он
оставит вызов}{But if this information is not available to compiler, it will leave call to}
strbuf\_get\_len() \IFRU{как есть}{as is}.

\IFRU{То же самое касается поля \TT{buf} в структуре \TT{strbuf}}{The same thing is about \TT{buf}
field in the \TT{strbuf} structure}.
\IFRU{Компилятор может генерировать куда более эффективный
код, если этот сгенерированный машинный код сможет обращаться к полям структур на прямую,
а не вызывать суррогатные функции-``методы''}
{Compiler may generate much more effective code if the generated machine code will access structure
fields directly without calling surrogate functions-``methods''}.

