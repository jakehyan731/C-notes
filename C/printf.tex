\label{printf}
\subsection{printf()}

\subsubsection{\IFRU{Свои собственные модификаторы в printf()}{Your own printf() format-string modifiers}}

\IFRU{Часто можно испытать раздражение, когда было бы логично передать в printf(),
скажем, структуру описывающее комплексное
число, или цвет закодированный в структуре из трех чисел типа int}
{It is often irritating when it is logical to pass to printf(), let's say, 
a structure describing complex number, or a color encoded as 3 int numbers as a single entity}.

\IFRU{Эту проблему в Си++ решают определением ф-ции}
{In C++ this problem is usually solved by definition} \TT{operator<<} \InENRU \TT{ostream} 
\IFRU{для своего типа}{for the own type}, \IFRU{либо введением метода с названием}
{or by a method definition named} \TT{ToString()} (\ref{CPPIO}). \\
\\
\IFRU{В}{In} printk() (printf-\IFRU{подобная ф-ция в ядре Linux}{like function in Linux kernel})
\IFRU{имеются дополнительные модификаторы}{there are additional modifiers exist}
\footnote{\url{http://git.kernel.org/cgit/linux/kernel/git/torvalds/linux.git/tree/Documentation/printk-formats.txt}}, 
\IFRU{такие как}{like}
\TT{\%pM} (Mac-\IFRU{адрес}{address}),
\TT{\%pI4} (IPv4-\IFRU{адрес}{address}),
\TT{\%pUb} (UUID/GUID).

\IFRU{В}{In} GNU Multiple Precision Arithmetic Library \IFRU{есть ф-ция}{there are} gmp\_printf()
\footnote{\url{http://gmplib.org/manual/Formatted-Output-Strings.html}} \IFRU{имеющая нестандартные 
модификаторы нужные для вывода чисел с произвольной точностью}{function having non-standard modifiers for
arbitrary precision numbers outputting}. \\
\\
\IFRU{В \ac{OS}}{In the} Plan9\IFRU{, и в исходниках компилятора Go, можно найти ф-цию}
{ \ac{OS}, and in Go compiler source code, we may find}
fmtinstall()\IFRU{, для объявления нового модификатора printf-строки, например}
{ function for a new printf-string modifier definition, for example}:

\begin{lstlisting}[caption=go\textbackslash{}src\textbackslash{}cmd\textbackslash{}5c\textbackslash{}list.c]
void
listinit(void)
{

	fmtinstall('A', Aconv);
	fmtinstall('P', Pconv);
	fmtinstall('S', Sconv);
	fmtinstall('N', Nconv);
	fmtinstall('B', Bconv);
	fmtinstall('D', Dconv);
	fmtinstall('R', Rconv);
}

...

int
Pconv(Fmt *fp)
{
	char str[STRINGSZ], sc[20];
	Prog *p;
	int a, s;

	p = va_arg(fp->args, Prog*);
	a = p->as;
	s = p->scond;
	strcpy(sc, extra[s & C_SCOND]);
	if(s & C_SBIT)
		strcat(sc, ".S");
	if(s & C_PBIT)
		strcat(sc, ".P");
	if(s & C_WBIT)
		strcat(sc, ".W");
	if(s & C_UBIT)		/* ambiguous with FBIT */
		strcat(sc, ".U");
	if(a == AMOVM) {
		if(p->from.type == D_CONST)
			sprint(str, "	%A%s	%R,%D", a, sc, &p->from, &p->to);
		else
		if(p->to.type == D_CONST)
			sprint(str, "	%A%s	%D,%R", a, sc, &p->from, &p->to);
		else
			sprint(str, "	%A%s	%D,%D", a, sc, &p->from, &p->to);
	} else
	if(a == ADATA)
		sprint(str, "	%A	%D/%d,%D", a, &p->from, p->reg, &p->to);
	else
	if(p->as == ATEXT)
		sprint(str, "	%A	%D,%d,%D", a, &p->from, p->reg, &p->to);
	else
	if(p->reg == NREG)
		sprint(str, "	%A%s	%D,%D", a, sc, &p->from, &p->to);
	else
	if(p->from.type != D_FREG)
		sprint(str, "	%A%s	%D,R%d,%D", a, sc, &p->from, p->reg, &p->to);
	else
		sprint(str, "	%A%s	%D,F%d,%D", a, sc, &p->from, p->reg, &p->to);
	return fmtstrcpy(fp, str);
}
\end{lstlisting}
(\url{http://plan9.bell-labs.com/sources/plan9/sys/src/cmd/5c/list.c})

\IFRU{Ф-ция}{The} Pconv() 
\IFRU{будет вызвана если в строке формата будет встречен \%P}{will be called if \%P modifier
in the format string will be met}.
\IFRU{Затем она копирует созданную строку при помощи}
{Then it copies the string created using} fmtstrcpy().
\IFRU{Кстати, эта ф-ция и сама использует другие объявленные модификаторы, такие как}
{By the way, that function also uses other defined modifiers like} \%A, \%D, \IFRU{итд}{etc}. \\
\\
\IFRU{В}{The} Glibc\footnote{\IFRU{Стандартной библиотеке в Linux}{The Linux standard library}} 
\IFRU{есть нестандартное расширение}{has non-standard extension}
\footnote{\url{http://www.gnu.org/software/libc/manual/html_node/Customizing-Printf.html}}, 
\IFRU{позволяющее объявлять свои модификаторы, но это}{allowing to define our own
modifiers, but it is} \IT{deprecated}.

\IFRU{Попробуем определить свои собственные модификаторы для 
Mac-адреса и для вывода байта в бинарном виде}{Let's try to define our own modifiers for Mac-address
outputting and also for byte outputting in a binary form}:

\lstinputlisting{C/register_printf_function.c}
\footnote{\IFRU{Основа для примера взята отсюда}{The base of example was taken from}:
\url{http://codingrelic.geekhold.com/2008/12/printf-acular.html}}

\IFRU{Это компилируется с предупреждениями}{This compiled with warnings}:

\begin{lstlisting}
1.c: In function 'main':
1.c:48:2: warning: 'register_printf_function' is deprecated (declared at /usr/include/printf.h:106) [-Wdeprecated-declarations]
1.c:49:2: warning: 'register_printf_function' is deprecated (declared at /usr/include/printf.h:106) [-Wdeprecated-declarations]
1.c:51:2: warning: unknown conversion type character 'M' in format [-Wformat]
1.c:52:2: warning: unknown conversion type character 'B' in format [-Wformat]
\end{lstlisting}

\ac{GCC} \IFRU{умеет следить за соответствиями модификаторов в}{is able to track accordance between
modifiers in the} printf-\IFRU{строке и аргументами в вызове}{string and arguments in} printf(),
\IFRU{но здесь ему встречаются незнакомые модификаторы, о чем он предупреждает}
{however, unfamiliar to it modifiers are present here, so it warns us about them}.

\IFRU{Тем не менее, наша программа работает}{Nevertheless, our program works}:

\begin{lstlisting}
$ ./a.out
00:11:22:33:44:55
10101011
\end{lstlisting}

