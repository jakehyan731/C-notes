\label{printf}
\section{printf()}

\subsection{Свои собственные модификаторы в printf()}

Часто можно испытать раздражение, когда было бы логично передать в printf(), скажем, структуру описывающее комплексное
число, или цвет закодированный в структуре из трех чисел типа int.

Эту проблему в Си++ решают определением ф-ции \TT{operator<<} в \TT{ostream} для своего типа, либо введением
метода с названием \TT{ToString()}.

В printk() (printf-подобная ф-ция в ядре Linux) имеются дополнительные модификаторы
\footnote{\url{http://git.kernel.org/cgit/linux/kernel/git/torvalds/linux.git/tree/Documentation/printk-formats.txt}}, 
такие как \TT{\%pM} (Mac-адрес), \TT{\%pI4} (IPv4-адрес), \TT{\%pUb} (UUID/GUID).

В ОС Plan9, и в исходниках компилятора Go, можно найти ф-цию fmtinstall(), объявляющую новый модификатор printf-строки,
например:

\begin{lstlisting}[caption=go\textbackslash{}src\textbackslash{}cmd\textbackslash{}5c\textbackslash{}list.c]
void
listinit(void)
{

	fmtinstall('A', Aconv);
	fmtinstall('P', Pconv);
	fmtinstall('S', Sconv);
	fmtinstall('N', Nconv);
	fmtinstall('B', Bconv);
	fmtinstall('D', Dconv);
	fmtinstall('R', Rconv);
}

...

int
Pconv(Fmt *fp)
{
	char str[STRINGSZ], sc[20];
	Prog *p;
	int a, s;

	p = va_arg(fp->args, Prog*);
	a = p->as;
	s = p->scond;
	strcpy(sc, extra[s & C_SCOND]);
	if(s & C_SBIT)
		strcat(sc, ".S");
	if(s & C_PBIT)
		strcat(sc, ".P");
	if(s & C_WBIT)
		strcat(sc, ".W");
	if(s & C_UBIT)		/* ambiguous with FBIT */
		strcat(sc, ".U");
	if(a == AMOVM) {
		if(p->from.type == D_CONST)
			sprint(str, "	%A%s	%R,%D", a, sc, &p->from, &p->to);
		else
		if(p->to.type == D_CONST)
			sprint(str, "	%A%s	%D,%R", a, sc, &p->from, &p->to);
		else
			sprint(str, "	%A%s	%D,%D", a, sc, &p->from, &p->to);
	} else
	if(a == ADATA)
		sprint(str, "	%A	%D/%d,%D", a, &p->from, p->reg, &p->to);
	else
	if(p->as == ATEXT)
		sprint(str, "	%A	%D,%d,%D", a, &p->from, p->reg, &p->to);
	else
	if(p->reg == NREG)
		sprint(str, "	%A%s	%D,%D", a, sc, &p->from, &p->to);
	else
	if(p->from.type != D_FREG)
		sprint(str, "	%A%s	%D,R%d,%D", a, sc, &p->from, p->reg, &p->to);
	else
		sprint(str, "	%A%s	%D,F%d,%D", a, sc, &p->from, p->reg, &p->to);
	return fmtstrcpy(fp, str);
}
\end{lstlisting}
(\url{http://plan9.bell-labs.com/sources/plan9/sys/src/cmd/5c/list.c})

Ф-ция Pconv() будет вызвана если в строке формата будет встречен \%P. Затем она копирует созданную строку
при помощи fmtstrcpy(). Кстати, эта ф-ция и сама использует другие объявленные модификаторы, такие как \%A, \%D, итд.

В Glibc\footnote{Стандартной библиотеке в Linux} есть нестандартное расширение
\footnote{\url{http://www.gnu.org/software/libc/manual/html_node/Customizing-Printf.html}}, 
позволяющее объявлять свои модификаторы, но это \IT{deprecated}.

Попробуем определить свои собственные модификаторы для Mac-адреса и для вывода байта в бинарном виде:

\lstinputlisting{register_printf_function.c}
\footnote{Основа для примера взята отсюда: \url{http://codingrelic.geekhold.com/2008/12/printf-acular.html}}

Это компилируется с предупреждениями:

\begin{lstlisting}
1.c: In function 'main':
1.c:48:2: warning: 'register_printf_function' is deprecated (declared at /usr/include/printf.h:106) [-Wdeprecated-declarations]
1.c:49:2: warning: 'register_printf_function' is deprecated (declared at /usr/include/printf.h:106) [-Wdeprecated-declarations]
1.c:51:2: warning: unknown conversion type character 'M' in format [-Wformat]
1.c:52:2: warning: unknown conversion type character 'B' in format [-Wformat]
\end{lstlisting}

GCC умеет следить за соответствиями модификаторов в printf-строке и аргументами в вызове printf(), но здесь
ему встречаются незнакомые модификаторы, о чем он предупреждает.

Тем не менее, наша программа работает:

\begin{lstlisting}
$ ./a.out
00:11:22:33:44:55
10101011
\end{lstlisting}

