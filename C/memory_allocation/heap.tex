\subsection{\IFRU{Выделение памяти в куче}{Allocating memory in heap}}

\IFRU{Куча (heap) это какая-то часть памяти выделенная \ac{OS} процессу,
где процесс может уже сам делить эту часть как хочет}
{The heap is an area of memory allocated by \ac{OS} to the process,
where it can divide it within its sole discretion}.
\IFRU{После заврешения процесса (в т.ч., некорректного),
куча автоматически аннулируется и \ac{OS} не нужно разбирать 
по одному все выделенные процессом блоки}
{After terminating of the process (including process crash),
the heap is annuled automatically and \ac{OS} will not need to free
all allocated blocks one by one}.

\IFRU{Для работы с кучей есть стандартные библиотечные ф-ции}
{There are standard C functions to work with the heap:}
malloc(), calloc(), realloc(), free(), 
\IFRU{а в Си++ --- new/delete}{and new/delete in C++}.

\label{HeapOverhead}
\IFRU{Очевидно, чтобы поддерживать информацию о выделенных блоках в куче,
нужна масса связных друг с другом структур}
{Apparently, heap manager must use a lot of interconnected structures
in order to preserve information about allocated blocks}.
\IFRU{Отсюда имеется вполне осязаемые накладные расходы (overhead)}
{So that's why quite tangible overhead is present}.
\IFRU{Вы можете выделить блок размером 8 байт, 
но еще как минимум 8 байт\footnote{MSVC, 32-битная Windows, 
примерно то же самое и в Linux}
будет задействованы для хранения информации о выделенном блоке}
{
You can allocate memory block of size 8 bytes,
but at least more 8 bytes\footnote{MSVC, 32-bit Windows, 
almost the same in Linux}
will be used for preserving information about allocated block
}
\footnote{\IFRU{Это еще называют ``метаданными'', то есть, данные о данных}
{It is also called ``metadata'', i.e., data about data}}.
\IFRU{В 64-битных ОС указатели занимают в два раза больше,
так что информация о каждом блоке будет занимать как минимум
16 байт}{In 64-bit \ac{OS} pointers requiring twices as much space,
so information about each block will require at least 16 bytes}.
\IFRU{В свете этого, чтобы эффективнее использовать память компьютера,
блоки должны быть побольше, либо
сама организация данных должна быть иная}
{In the light of this, in order to effectively use as much memory as possible,
the blocks should be as large as possible, or, the orgranization of data
must be different}.

\IFRU{Использование кучи требует некоторой программистской дисциплины,
без которой легко наделать ошибок}
{Heap using require a programmer's discipline, it is easy to make a lot
of mistakes without one}.
\IFRU{Возможно поэтому, считается что ЯП с \ac{RAII} как Си++
либо ЯП со сборщиками мусора (Python, Ruby) легче}
{Probably because of this, it is widely considered that \ac{PL} with 
\ac{RAII} like C++
or PL with garbage collector (Python, Ruby) are easier}.

\subsubsection{\IFRU{Одна из основных ошибок: утечки памяти}
{One of the common mistakes: memory leaks}}

\IFRU{Память была выделена, но её забыли освободить через}
{Memory was allocated, but we forgot to free it via} free().
\IFRU{Эта проблема довольно легко решается своей собственной
надстройкой над ф-циями malloc()/free()}
{This problem is easily solved by thunk functions on top of
malloc()/free()}.
\IFRU{Пусть эта надстройка ведет учет выделенных блоков, а также, где и когда
(и для чего) был выделен тот или иной блок}
{Let this thunk to keep a records about blocks allocated, and also,
where and when (and for what) each block was allocated}.

\IFRU{Я сделал это в своей библиотеке octothorpe}{I made this in my octothorpe library}
\footnote{\url{https://github.com/dennis714/octothorpe/blob/master/dmalloc.c}}. 
\IFRU{Макрос DMALLOC вызывает ф-цию dmalloc(), передавая
ей имя файла, имя вызывающей ф-ции, номер строки, а также комментарий (имя блока)}
{DMALLOC macro calls dmalloc() function passing it the file name, name of the calling function,
line number and comment (block name)}.
\IFRU{В конце работы программы, вызываем}
{At the end of program, we call}
\TT{dump\_unfreed\_blocks()} \IFRU{и он покажет список блоков, которые забыли освободить}
{and it will dump the list of blocks we forgot to free}:

\begin{lstlisting}
seq_n:2, size: 124, filename: dmalloc_test.c:31, func: main, struct: block124
seq_n:3, size: 12, filename: dmalloc_test.c:33, func: main, struct: block12
seq_n:4, size: 555, filename: dmalloc_test.c:35, func: main, struct: block555
\end{lstlisting}

\IFRU{У каждого блока есть также номер}{Each block also has a number}.
\IFRU{Это для того чтобы можно было установить брякпоинт по номеру выделяемого блока ~--- 
тогда отладчик сработает в тот момент, когда этот блок будет выделяться и вы увидите,
где и при каких условиях это происходит}
{This is helpful because one can set a breakpoint by a block number and debugger will trigger at the moment
the block is being allocated, and you can see, where and under what conditions it is occurring}.

\IFRU{Писать в коде комментарии для каждого выделяемого блока памяти нудно, но очень полезно}
{It is boring to write a comment for each block allocated, but very useful}.
\IFRU{Потом легко увидеть, под что была выделена память}{Then it is easy to see, what was memory allocated for}.
\IFRU{Я впервые увидел эту идею в}{I first saw this idea in the} Oracle RDBMS.
\IFRU{Помимо всего прочего, там еще и ведется
статистика, под какие блоки было выделено больше памяти, её можно легко увидеть}
{Aside from that, it also keeps statistics of block types,
how many memory was allocated for each, and it is easy to see it}:

\begin{lstlisting}
SQL> select * from v$sgastat;

POOL         NAME                            BYTES     CON_ID
------------ -------------------------- ---------- ----------
shared pool  AQ Slave list                    1224          1
shared pool  KQR L PO                       653312          2
shared pool  KQR X SO                       635808          2
shared pool  RULEC                           20688          1
shared pool  KQR M SO                         7168          2
shared pool  work area table entry           12240          2
shared pool  kglsim object batch              3864          2
large pool   PX msg pool                    860160          1
large pool   free memory                  30523392          0
large pool   SWRF Metric CHBs              1802240          2
large pool   SWRF Metric Eidbuf             368640          2
\end{lstlisting}

\IFRU{Подобная штука также присутствует и в ядре Windows, там это называется}
{The same thing present in the Windows kernel, it is called there} \IT{tagging}.

\IFRU{При выделении памяти
в ядре или драйвере, нужно указывать также 32-битный тег (обычно, четырехбуквенное сокращение, означающее
подсистему Windows)}
{When one allocates memory in the kernel or dirver, a 32-bit tag may be set (usually, it is a four-letter
abbreviation, indicating Windows subsystem)}.
\IFRU{Затем в отладчике можно увидеть статистику, под что выделено больше всего памяти}
{Then it is possible to see a statistics in a debugger, how much memory is allocated what for}:

\begin{lstlisting}
kd> !poolused 4
   Sorting by  Paged Pool Consumed

  Pool Used:
            NonPaged            Paged
 Tag    Allocs     Used    Allocs     Used
 CM25        0        0       935  4124672	Internal Configuration manager allocations , Binary: nt!cm
 Gh05        0        0       268  3291016	GDITAG_HMGR_SPRITE_TYPE , Binary: win32k.sys
 MmSt        0        0      2119  2936752	Mm section object prototype ptes , Binary: nt!mm
 CM35        0        0        91  2150400	Internal Configuration manager allocations , Binary: nt!cm
 vmfb        0        0        13  2148752	UNKNOWN pooltag 'vmfb', please update pooltag.txt
 Ntff        5     1040      1287  1070784	FCB_DATA , Binary: ntfs.sys
 ArbA        0        0       108   442368	ARBITER_ALLOCATION_STATE_TAG , Binary: nt!arb
 NtfF        0        0       457   431408	FCB_INDEX , Binary: ntfs.sys
 CM16        0        0        62   331776	Internal Configuration manager allocations , Binary: nt!cm
 IoNm        0        0      2022   267288	Io parsing names , Binary: nt!io
 Ttfd        0        0       159   253976	TrueType Font driver 
 Ifs         0        0         4   249968	Default file system allocations (user's of ntifs.h) 
 CM29        0        0        26   212992	Internal Configuration manager allocations , Binary: nt!cm
\end{lstlisting}

\IFRU{Конечно, можно возразить, что для этого нужно хранить еще больше информации о выделенных блоках, а еще
и теги, названия блоков}
{Of course, one may argue the heap manager will require much more space about allocated blocks,
including their names or tags}.
\IFRU{И это еще сильнее замедляет работу программы. Конечно.}
{And it is slowing down the program much more. That is for sure.}
\IFRU{Поэтому пусть это будет работать только в отладочных (debug) сборках, 
а в release-сборках, DMALLOC() становится обычной \IT{пустой} функцией-переходником\footnote{thunk function} для}
{Then we may use it only in debug builds, and in the release-builds DMALLOC() will be
simple \IT{empty} thunk-function for} malloc().
\IFRU{Ну а в ядре Windows это вообще по умолчанию отключено, и нужно включать при помощи утилиты GFlags}
{It is also turned off by default in Windows and it should be turned on with the help of GFlags utility}
\footnote{\url{http://msdn.microsoft.com/en-us/library/windows/hardware/ff549557(v=vs.85).aspx}}
\IFRU{Помимо всего прочего, подобное есть и в MSVC}{Aside from that, something similar present in MSVC}
\footnote{\IFRU{читайте больше о ф-циях}{read more about the} 
\href{http://msdn.microsoft.com/en-us/library/5at7yxcs.aspx}{\_CrtSetDbgFlag}
\IFRU{и}{and}
\href{http://msdn.microsoft.com/en-us/library/d41t22sb.aspx}{\_CrtDumpMemoryLeaks}\IFRU{}{ functions}}.

\subsubsection{\IFRU{Одна из основных ошибок: разрушение кучи}
{One of the common mistakes: heap corruption}}

\IFRU{Нетрудно выделить память под 4 байта, но по ошибке дописать туда пятый}
{It is easy to allocate a memory for 4 bytes, but write there fifth by accident}.
\IFRU{Скорее всего, сразу это никак не проявится, но фактически это очень опасная мина замедленного действия,
опасная, потому что ведет к трудновыявляемым ошибкам}
{Most likely, it will not come out instantly, but in fact, it is a very dangerous time bomb, dangerous because
it is hard-to-find bug}.
\IFRU{Байт следующий за выделенным вами блоком может не использоваться вовсе, но также там уже
может начинаться какая-то структура менеджера памяти, хранящая информацию о каком-то выделенном блоке,
а может даже об этом самом}
{The byte next after block you allocated, most likely, is not used at all,
but there may begin a heap manager structure, keeping the information about some other allocated block, or maybe
even that block}.
\IFRU{Если какую-то из таких структур сознателно разрушить,
перезаписать, то тогда следующие вызовы malloc() или free() не смогут корректно работать}
{If some of these structures to corrupt or rewrite intentionally, consequent malloc() or free() calls
will not work properly}.
\IFRU{Иногда это проявляется в выводе ошибок вроде}{Sometimes it is manifested in errors like} 
(\IFRU{в}{in} Windows):

\begin{lstlisting}
HEAP[Application.exe]: HEAP: Free Heap block 211a10 modified at 211af8 after it was freed
\end{lstlisting}

\IFRU{Подобные ошибки эксплуатируются авторами эксплоитов}
{Such errors are exploited by exploit authors}: 
\IFRU{если знать что вы можете изменить структуры данных
менеджера памяти нужным вам образом, вы можете добиться какого-то нужного вам поведения программы}
{if to know you can alter heap manager structures in a way you need, you may achieve some 
specific program behaviour you need}
(\IFRU{это называется}{this is called} heap overflow
\footnote{\url{https://en.wikipedia.org/wiki/Heap_overflow}}).

\IFRU{Довольно распространенный метод борьбы с подобными ошибками: это просто дописывать ``guard''-ы с обоих сторон
блока, например, 4-байтного размера}
{Widely used protection from such errors: just to write ``guard'' values (of 32-bit size, for example) 
at the both sides of the block}.
\IFRU{Например, я сделал это в своем}{For example, I did it in} DMALLOC.
\IFRU{При каждом вызове free(),
проверяется целостность guard-ов (это могут быть просто какие-то фиксированные значения вроде 0x12345678),
и если кто-то или что-то затерло один из них, можно тут же сообщить об этом}
{At each free() call, integrity of both guards are checked (these may be a fixed values like 0x12345678),
and if something or someone writed to it, that fact can be reported instantly}.

%\subsubsection{Приемущества своего собственного менеджера памяти или надстройки над стандартным}
%... может выдать размер выделенного блока, хотя нафик оно надо...

\subsubsection{\IFRU{Одна из основных ошибок: непроверка результата malloc()}
{One of the common mistakes: not checking malloc() result}}

\IFRU{При успешном выполнении, malloc() возвращает указатель на только что выделенный блок,
который можно использовать, либо NULL если памяти не хватает}
{If malloc() finishes successfully, it returns the pointer to the newly allocated block can be used,
or NULL in case of memory shortage}.
\IFRU{Конечно, в наше время дешевой памяти эта проблема становится редкой, тем не менее,
если вы используете много памяти, думать об этом все же надо}
{Of course, in our time of cheap memory, this is rare problem, nevertheless, if one use it a lot,
one should this about it}.
\IFRU{Проверять возвращаемый указатель после каждого вызова
malloc() неудобно, так что довольно популярный метод это писать свои функции-переходники с названием 
xmalloc(), xrealloc(), вызывающие malloc()/realloc(), но проверяющие их результат и падающие в случае
ошибки}
{It is not handy to check the returned pointer after each malloc() call, so there are a popular method
to write own thunk functions named xmalloc(), xrealloc() calling malloc()/realloc(), which checks
returning result and exiting in case of error}.

\IFRU{Интересно упомянуть, как ведет себя}{It is interesting to note how} xmalloc() \IFRU{в}{behaves in} git:

\begin{lstlisting}
void *xmalloc(size_t size)
{
	void *ret;

	memory_limit_check(size);
	ret = malloc(size);
	if (!ret && !size)
		ret = malloc(1);
	if (!ret) {
		try_to_free_routine(size);
		ret = malloc(size);
		if (!ret && !size)
			ret = malloc(1);
		if (!ret)
			die("Out of memory, malloc failed (tried to allocate %lu bytes)",
			    (unsigned long)size);
	}
#ifdef XMALLOC_POISON
	memset(ret, 0xA5, size);
#endif
	return ret;
}
\end{lstlisting}
\footnote{\url{https://github.com/git/git/blob/master/wrapper.c}}

\IFRU{Если}{If} malloc() \IFRU{не успешен, он пытается освободить какие-то уже выделенные (и не очень нужные) 
блоки при помощи}{is not successful, it tries to free some already allocated (but not very needed)
blocks with the help of}
try\_to\_free\_routine(), \IFRU{а затем вызвать}{and then to call} malloc() \IFRU{снова}{again}.

\IFRU{Помимо всего прочего, если определен}
{Aside from that, if} XMALLOC\_POISON\IFRU{, все байты в выделенном блоке заполняются}
{ is defined, all bytes in the block allocated is filled with} 0xA5.

\IFRU{Это может помочь визуально, на глаз, увидеть когда вы, например, выделили память под структуру,
а затем используете какое-то поле из нее до того как инициализировали}
{This may help to see, visually, when you use some value from the block before its initialization}.

\IFRU{Значение}{The value of} \TT{0xA5A5A5A5} \IFRU{будет
бросаться в глаза в отладчике, ну или просто если вы захотите где-то в дампе вывести его в шестнадцатеричной
форме}{will easily be spotted in the debugger, or, in some place in dump where it will be printed in hexadecimal
form}.
\IFRU{В MSVC для этой же цели служит константа}{There are the constant for the same purpose in MSVC:}
\TT{0xbaadf00d}.

\IFRU{И даже более того: после вызова free(), освобожденный блок может маркироваться уже какой-то другой константой,
чтобы если кто-то захочет использовать что-то оттуда после освобождения блока, это также было видно, хотя
бы визуально}
{Even more than that: after call of free(), freed block may be marked by some other constant, in order
to spot visually if someone attempting to use some data from the block after it has been freed}.

\IFRU{Некоторые константы от}{Some constants from} Microsoft:

\begin{lstlisting}
* 0xABABABAB : Used by Microsoft's HeapAlloc() to mark "no man's land" guard bytes after allocated heap memory
* 0xABADCAFE : A startup to this value to initialize all free memory to catch errant pointers
* 0xBAADF00D : Used by Microsoft's LocalAlloc(LMEM_FIXED) to mark uninitialised allocated heap memory
* 0xCCCCCCCC : Used by Microsoft's C++ debugging runtime library to mark uninitialised stack memory
* 0xCDCDCDCD : Used by Microsoft's C++ debugging runtime library to mark uninitialised heap memory
* 0xFDFDFDFD : Used by Microsoft's C++ debugging heap to mark "no man's land" guard bytes before and after allocated heap memory
* 0xFEEEFEEE : Used by Microsoft's HeapFree() to mark freed heap memory
\end{lstlisting}
\footnote{\url{https://en.wikipedia.org/wiki/Magic_number_(programming)}}

\subsubsection{\IFRU{Еще частые ошибки}{Other common mistakes}}

\IFRU{Если не включить заголовочный файл}{If not to include header file} stdlib.h, 
GCC \IFRU{считает возвращаемое значение неизвестной ф-ции}{treat returning value of} malloc() 
\IFRU{за}{as} int \IFRU{и постоянно ругается на приведение типов}{and warns about types}.

\IFRU{Еще одна ошибка, которая может попортить нервов, это выделить один и тот же блок памяти в одном месте 
больше одного раза (предыдущие вызовы ``теряются'' из вида)}
{Another mistake may cost your nerves is to allocate the same block of memory at one place 
more than one time (previous calls are ``hiding'' from the sight)}.

\subsubsection{\IFRU{Еще методы борьбы}{Other bug hunting methods}}

\IFRU{Однако, может оказаться так, что ошибки в программе у вас есть, а перекомпилировать её по каким-то причинам
вы не можете}
{However, you may be in the situation, when there are bugs in program, but you are not able to recompile it
for some reason}. \IFRU{Тогда может помочь, например, valgrind}
{As an example, valgrind\footnote{\url{http://valgrind.org/}} may help then}.

