\subsection{\IFRU{Локальный стек или куча}{Local stack or heap}?}

\IFRU{Конечно, в локальном стеке выделение памяти происходит намного быстрее}
{Of course, allocation in the local stack is much faster}.

\IFRU{Например: в}{For example: in} tracer
\footnote{\url{\IFRU{http://yurichev.com/tracer-ru.html}{http://yurichev.com/tracer-en.html}}}
\IFRU{у меня есть дизассемблер}{i have disassembler}\footnote{\url{https://github.com/dennis714/x86_disasm}} 
\IFRU{и эмулятор x86-процессора}{and also x86 CPU emulator}
\footnote{\url{https://github.com/dennis714/bolt/blob/master/X86_emu.c}}.
\IFRU{Когда я писал дизассемблер на Си (я делал это после того как длительное время писал на \ac{PL} 
более высокого уровня ~--- Python), 
я думал, что было бы неплохо, чтобы он сам выделял память под структуру, заполнял её
и возвращал указатель на нее, а в случае ошибки дизассемблирования, возвращал бы NULL}
{When I was writing disassembled in C (I did it after a long period of programming in higher level 
\ac{PL} ~---
Python), I thought, it is a good idea for disassembler to allocate the memory for the structure it returns,
fill it and return a pointer to it, or NULL in case of disassembling error}.

\IFRU{Эстетически, это 
неплохо смотрится, в стиле высокоуровневых \ac{PL}, к тому же, такой код наверное легче читается}
{\AE{}sthetically it looks good, in style of high-level \ac{PL}, aside from that, such code is easier to read}.
\IFRU{Однако, дизассемблер и эмулятор x86-процессора
работают в цикле, огромное количество раз в секунду и эффективность здесь более чем важна}
{However, disassembler and x86 CPU emulator works in loop, a huge number of iterations per second and efficiency
is crucial}.
\IFRU{Так что, основной цикл у меня выглядит примерно так}
{So the main loop I wrote in that way}:

\begin{lstlisting}
while(true)
{
	struct disassembled_instruction DA;

	bool DA_success=disassemble(&DA...);
	if (DA_success==false)
		break;

	bool emulate_success=try_to_emulate(&dDA);
	if (emulate_success==false)
		break;

};
\end{lstlisting}

\IFRU{Затрат на выделение памяти под структуры, описывающие дизассемблированную инструкцию, нет вовсе}
{There are no costs for allocating disassembler structures at all}.
\IFRU{А иначе, нужно было бы на каждой итерации цикла вызывать malloc()/free(), каждая из которых, каждый раз,
работала бы со структурами кучи, и т.д.}
{Otherwise, we need to call malloc()/free() at teach loop iteration, each of which will also work with heap
data structures, etc}.

\IFRU{Как известно, у x86-инструкций может быть вплоть до трех операндов, так что, в моей структуре, помимо
кода инструкции, есть также и информация о трех операндах}
{As we know, x86-instructions may have up to 3 operands, so, in my structure, aside from instruction code,
there are also information about 3 operands}.
\IFRU{Конечно, можно было бы оформить её примерно так}{Of course, I could do it like}:

\begin{lstlisting}
struct disassembled_instruction
{
	int instruction_code;
	struct operand *op1;
	struct operand *op2;
	struct operand *op3;
};
\end{lstlisting}

... \IFRU{а в случае отсутствия какого либо операнда, пусть там будет NULL}
{and let it be NULL there in case of absence of some operand}.
\IFRU{Тем не менее, это снова выделение памяти в куче}{Nevertheless, it is still heap memory allocations}.

\IFRU{Так что у меня сделано примерно так}{So I did it like}:

\begin{lstlisting}
struct disassembled_instruction
{
	int instruction_code;
	int operands_total;
	struct operand op[3];
};
\end{lstlisting}

\IFRU{Такая структура занимает больше места в памяти}
{Such structure requires more memory}.
\IFRU{К тому же, трехоперандные инструкции очень редки в x86-коде,
а здесь у меня пустой третий операнд хранится всегда}
{Aside from that, 3-operand instructions are rare in x86-code, but third operand is stored here always}.
\IFRU{Однако, лишних манипуляций с памятью не происходит}
{However, there are no extra manipulations with memory}.

\IFRU{Ну а если уж так сильно хочется сэкономить на третьем операнде,
то можно не хранить третий операнд вовсе}{But if one like to save a space by not storing third operator,
it may be not stored at all}: \IFRU{нетрудно
вычислить размер структуры без одного операнда}{it is easy to calculate structure size
without one operatnd}: \TT{sizeof(disassembled\_instruction) - sizeof(struct operand)} 
\IFRU{и скопировать её куда-то, где она должен храниться}{and copy it to the some place where it must be
stored}.
\IFRU{Ведь никто не запрещает нам использовать (и хранить) не всю структуру а только её часть}
{Because no one prohibits to use (and store) not the whole structure, but only its part}.
\IFRU{А ф-ции работы с этой структурой могут не трогать в памяти третий операнд вовсе и,
таким образом, ошибок не будет}{Besides, the functions which works with the structure, may not touch
third operand at all, and that will work correctly}.

\IFRU{И даже более того: я специально сделал свой дизассемблер именно так,
чтобы он мог принимать на вход не инициализированную
структуру, и мог работать даже если там осталась информация от предыдущих вызовов}
{Even more than that: I made my disassembler intentionally in that way that it can take
not initialized structure and may work even if there are still some information leaved from
the previous calls}.

\IFRU{Возможно, это уже слишком, но вы поняли идею}{Maybe it is overkill, but you got the idea}.

\IFRU{Таким образом, если вы выделяете память под небольшие структуры заранее известного размера, или если скорость
очень важна, то лучше подумать насчет выделения в локальном стеке}
{Thus, if you allocate small structures of known size and if speed is crucial, you may consider allocating
them in the local stack}.

