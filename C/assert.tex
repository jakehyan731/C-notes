\subsection{assert}

\index{assert()}
\IFRU{Как известно, этот макрос часто используется для валидации}
{This macro is commonly used for validating}
\footnote{\IFRU{используется также такой термин как ``инвариант'' и ``sanitization'' в англ.яз.}{
``invariant'' and ``sanitization'' terms are also used}}
\IFRU{заданных значений}{of input values}.
\IFRU{Например, если ваша ф-ция работает с датой, вы, вероятно, захотите написать в её начале что-то вроде}
{For eample, if you have a function working with data, you probably may want to add that code to the begin}:
\IT{assert (month>=1 \&\& month<=12)}.

\IFRU{Вот то о чем нужно помнить: стандартный макрос assert() доступен только в отладочных (debug) сборках}
{Here is what one should remember: standard assert() macro is available only in debug builds}.
\index{\Preprocessor!NDEBUG}
\IFRU{В release, где объявлена переменная NDEBUG, все выражения как бы исчезают}{In a release build, where NDEBUG
is defined, all statements are ``disappearing''}.
\IFRU{Поэтому писать, например,}{That is why it is not correct to write}
\IT{assert(f=malloc(...))}\IFRU{ неверно}{}.
\IFRU{Впрочем, вы возможно захотите использовать что-то вроде}{However, you may want to write something like}
\IT{assert(object->get\_something()==123)}.

\IFRU{В макросах \IT{assert} можно также указывать небольшие сообщения об ошибках}
{Error messages also can be embedded in an \IT{assert} statements}:
\IFRU{вы увидите их если выражение ``не сойдется''}{you will see it if expression will not be true}.
\IFRU{Например, в исходниках}{For example, in the} LLVM\footnote{\url{http://llvm.org/}}
\IFRU{можно встретить такое}{source code we may find this}:

\index{LLVM}
\begin{lstlisting}
assert(Index < Length && "Invalid index!");
...
assert(i + Count <= M && "Invalid source range");
...
assert(j + Count <= N && "Invalid dest range");
\end{lstlisting}

\IFRU{Текстовая строка имеет тип}{Text string has}
\IT{const char*}\IFRU{, и она никогда не}{ type and it is never} NULL.
\IFRU{Таким образом, можно дописать к любому выражению}{Thus it is possible to add} \IT{... \&\& true} 
\IFRU{не меняя его смысл}{to any expression without changing its sense}.

\IFRU{Макрос }{}assert() \IFRU{можно также вводить и для документации кода}{macro can also be used
for documenting purposes}.

\IFRU{Например}{For example}:

\begin{lstlisting}[caption=GNU Chess]
int my_random_int(int n) {

   int r;

   ASSERT(n>0);

   r = int(floor(my_random_double()*double(n)));
   ASSERT(r>=0&&r<n);

   return r;
}
\end{lstlisting}

\IFRU{Глядя на этот код, можно очень быстро увидеть ``легальные'' значения переменных}
{By reading the code we can quickly see ``legal'' values of the} $n$ \AndENRU $r$\IFRU{}{ variables}.

assert\IFRU{-ы также называют}{-s are also called} ``active comments''\cite{Lakos}.

