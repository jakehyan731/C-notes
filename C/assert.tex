\subsection{assert}

Как известно, этот макрос часто используется для валидации
\footnote{используется также такой термин как ``инвариант'' и ``sanitization'' в англ.яз.} заданных значений. 
Например, если ваша ф-ция
работает с датой, вы, вероятно, захотите написать в её начале что-то вроде \IT{assert (month>=1 \&\& month<=12)}.

Вот то о чем нужно помнить: стандартный макрос assert() доступен только в отладочных (debug) сборках. 
В release, где объявлена переменная NDEBUG,
все выражения как бы исчезают. Поэтому писать, например, \IT{assert(f=malloc(...))} неверно. Впрочем,
вы возможно захотите использовать что-то вроде \IT{assert(object->get\_something()==123)}.

В макросах assert можно также указывать небольшие сообщения об ошибках: 
вы увидите их если assert() ``не сойдется''. 
Например, в исходниках LLVM\footnote{\url{http://llvm.org/}} можно встретить такое:

\begin{lstlisting}
assert(Index < Length && "Invalid index!");
...
assert(i + Count <= M && "Invalid source range");
...
assert(j + Count <= N && "Invalid dest range");
\end{lstlisting}

Текстовая строка имеет тип \IT{const char*}, и она никогда не NULL. 
Таким образом, можно дописать к любому выражению \IT{... \&\& true} не меняя его смысл.

Макрос assert() можно также вводить и для документации кода.

Например:

\begin{lstlisting}[caption=GNU Chess]
int my_random_int(int n) {

   int r;

   ASSERT(n>0);

   r = int(floor(my_random_double()*double(n)));
   ASSERT(r>=0&&r<n);

   return r;
}
\end{lstlisting}

Глядя на этот код, можно очень быстро увидеть ``легальные'' значения переменных n и r.

assert-ы также называют ``active comments''\cite{Lakos}.

