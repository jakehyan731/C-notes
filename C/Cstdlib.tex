\chapter{\IFRU{Стандартные библиотеки Си/Си++}{C/C++ standard library}}

\section{assert}

Как известно, этот макрос часто используется для валидации
\footnote{используется также такой термин как ``инвариант'' и ``sanitization'' в англ.яз.} заданных значений. 
Например, если ваша ф-ция
работает с датой, вы, вероятно, захотите написать в её начале что-то вроде \IT{assert (month>=1 \&\& month<=12)}.

Вот то о чем нужно помнить: стандартный макрос assert() доступен только в отладочных (debug) сборках. В release
все выражения как бы исчезают. Поэтому писать, например, \IT{assert(f=malloc(...))} неверно. Впрочем,
вы возможно захотите использовать что-то вроде \IT{assert(object->get\_something()==123)}.

В макросах assert можно также указывать небольшие сообщения об ошибках: 
вы увидите их если assert() ``не сойдется''. 
Например, в исходниках LLVM\footnote{\url{http://llvm.org/}} можно встретить такое:

\begin{lstlisting}
assert(Index < Length && "Invalid index!");
...
assert(i + Count <= M && "Invalid source range");
...
assert(j + Count <= N && "Invalid dest range");
\end{lstlisting}

Текстовая строка имеет тип \IT{const char*}, и она никогда не NULL. 
Таким образом, можно дописать к любому выражению \IT{... \&\& true} не меняя его смысл.

\section{Разница между stdout и stderr}

\IT{stdout} это то что выводится на консоль при помощи вызова \IT{printf()}.
\IT{stdout} это буферизированный вывод,
так что, пользователь, обычно того не зная, видит вывод порциями. Бывает так что программа выдает
что-то используя \IT{printf()} либо \IT{cout} и тут же падает.
Если это попадает в буфер, но буфер не успевает
``сброситься'' (flush) в консоль, то пользователь ничего не увидит. Это бывает неудобно.
Таким образом, для вывода более важной информации, в том числе отладочной, удобнее использовать \IT{stderr}.

\IT{stderr} это не буферизированный вывод, и всё что попадает в этот поток при помощи 
\TT{fprintf(stderr,...)} либо \IT{cerr}, появляется в консоли тут же.

Не следует также забывать, что из-за отсутствия буфера, вывод в \IT{stderr} медленнее.

Чтобы направлять \IT{stderr} в другой файл при запуске процесса, можно указывать:

\begin{lstlisting}
process 2> debug.txt
\end{lstlisting}

... это направит вывод \IT{stderr} в заданный файл (потому что номер этого потока -- 2).

\section{UNIX time}

В UNIX-среде очень популярно представление даты и времени в формате UNIX time.
Это просто 32-битное число, показывающее
количество прошедших секунд с 1-го января 1970-го года.

В качестве положительных сторон: 1) очень легко хранить это 32-битное число; 2) очень легко вычислять разницу дат;
3) невозможно закодировать неверные даты и время, такие как 32-е января, 29-е февраля невысокосных годов, 
25 часов 62 минуты.

В качестве отрицательных сторон: 1) нельзя закодировать дату до 1970-го года.

В наше время, если использовать UNIX time, тем не менее, следует помнить что ``срок его действия'' истечет
в 2038-м году, именно тогда 32-битное число переполнится, то есть, пройдет $2^{32}$ секунд с 1970-го года.
Так что, для этого следует использовать 64-битное значение, т.е., time64.

% ? NtQuerySystemTime http://msdn.microsoft.com/en-us/library/windows/desktop/ms724512(v=vs.85).aspx

\section{scanf(), fscanf(), sscanf()}

Извечный спор, что лучше, текстовые файлы или бинарные. С бинарными быстрее и проще работать, зато текстовые
легче просматривать и редактировать в обычном текстовом редакторе, к тому же, в UNIX имеется огрмоный арсенал
утилит для обработки текстов и строк. Но текстовые файлы нужно парсить.

Ф-ции scanf()\cite[7.19.6.2]{C99TC3} конечно же, регулярные выражения не поддерживают, 
однако при их помощи некоторые простые последовательности строк можно парсить. 

\subsection{Пример \#1}

Генерируемый ядром Linux файл \TT{/proc/meminfo}, начинается примерно так:

\begin{lstlisting}
MemTotal:        1026268 kB
MemFree:          119324 kB
Buffers:          170796 kB
Cached:           263736 kB
SwapCached:        11428 kB
...
\end{lstlisting}

Предположим, нам нужно узнать первое и третье число, игнорируя второе и остальные.
Так это можно сделать:

\begin{lstlisting}
void read_proc_meminfo()
{
	FILE *f=fopen("/proc/meminfo", "r");
	assert(f);
	unsigned result1, result2;
	if (fscanf (f, "MemTotal:\t%d kB\n"
			"MemFree:\t%*d kB\n"
			"Buffers:\t%d kB\n", 
			&result1, &result2)==2)
		printf ("results: %d %d\n", result1, result2);
	fclose(f);
};
\end{lstlisting}

Строка формата расходится на три строки, в реальности это одна\ref{heredoc}.
Обратите внимание на \TT{\textbackslash{}n}, так мы задаем перевод строки.

\TT{*} в модификаторе scanf-строки указывает что число будет прочитано, но никуда записано не будет.
Таким образом, это поле игнорируется. scanf()-функции возвращают кол-во не прочитанных полей (здесь
их будет 3) а кол-во записанных полей (2).

\subsection{Пример \#2}

Имеется текстовый файл с парами в каждой строке (ключ-значение):

\begin{lstlisting}
some_param1=some_value
some_param2=Lazy fox etc etc.
param3=Lorem Ipsum etc etc.
space here=value containing space
too long param, we should fail here=value
\end{lstlisting}

Нужно просто читать оба поля.

\begin{lstlisting}
int main(int argc, char *argv[])
{
	assert(argc==2);
	assert(argv[1]);
	FILE *f=fopen (argv[1], "r");
	assert(f);
	int line=1;
	do
	{
		char param[16];
		char value[60];
		if (fscanf (f, "%16[^=]=%60[^\n]\n", param, value)==2)
		{
			printf ("param=%s\n", param);
			printf ("value=%s\n", value);
		}
		else
		{
			printf ("error at line %d\n", line);
			return 0;
		};
		line++;
	} while (!feof(f) && !ferror(f));
};
\end{lstlisting}

\TT{\%16[\^{}=]} --- это отдаленно напоминает регулярные выражения. Означает, читать 16 любых символов, кроме
знака ``равно'' (=). Затем, мы указываем scanf()-у, что далее должен быть этот самый знак (=). Затем
пусть он читает 60 любых символов, кроме символа перевода строки. В конце читаем символ перевода строки.

Это работает, и поля ограничены длиной 16 и 60 символов. Поэтому на 5-й строке предсказуемо происходит ошибка,
ведь там длина парамера длиннее.

Так можно парсить несложные форматы, даже CSV
\footnote{Comma-separated values: \url{https://en.wikipedia.org/wiki/Comma-separated_values}}.

Однако, нельзя забывать о том что scanf()-функции не способны прочитать пустую строку там где задается \%s.
Поэтому, этим методом невозможно парсить файл с ключами-значениями, где есть отсутствующие ключи или значения.

\subsection{Засада \#1}

Если использовать \%d в строке формата, scanf() подразумевает что это 32-битный int. 

Ошибкой является подобное:

\begin{lstlisting}
char a[10];

scanf ("%d %d %d %d", &a[0], &a[1], &a[2], &[3]);
\end{lstlisting}

Символы (или байты) лежат ``в притык'' друг к другу. Когда scanf() будет обрабатывать первое значение, он будет считать
его за 32-битный int, и ``затрет'' остальные три, рядом лежащие. И так далее.


\label{memcpy}
\section{memcpy()}

Поначалу трудно запомнить порядок аргументов в ф-циях memcpy(), strcpy(). Чтобы было легче, можно представлять
знак ``='' (``равно'') между аргументами.

\label{bzero}
\section{bzero() и memset()}

bzero() это ф-ция просто обнуляющая блок памяти.
Для этого обычно используют memset(). Но у memset() есть неприятная особенность, легко перепутать второй
и третий аргументы местами, и компилятор промолчит, потому что байт для заполнения всего блока задается как int.

К тому же, имя ф-ции bzero легче читается.

С другой стороны, её нет в стандарте POSIX.

\label{printf}
\subsection{printf()}

\subsubsection{\IFRU{Свои собственные модификаторы в printf()}{Your own printf() format-string modifiers}}

\IFRU{Часто можно испытать раздражение, когда было бы логично передать в printf(),
скажем, структуру описывающее комплексное
число, или цвет закодированный в структуре из трех чисел типа int}
{It is often irritating when it is logical to pass to printf(), let's say, 
a structure describing complex number, or a color encoded as 3 int numbers as a single entity}.

\IFRU{Эту проблему в Си++ решают определением ф-ции}
{In C++ this problem is usually solved by definition} \TT{operator<<} \InENRU \TT{ostream} 
\IFRU{для своего типа}{for the own type}, \IFRU{либо введением метода с названием}
{or by a method definition named} \TT{ToString()} (\ref{CPPIO}). \\
\\
\IFRU{В}{In} printk() (printf-\IFRU{подобная ф-ция в ядре Linux}{like function in Linux kernel})
\IFRU{имеются дополнительные модификаторы}{there are additional modifiers exist}
\footnote{\url{http://git.kernel.org/cgit/linux/kernel/git/torvalds/linux.git/tree/Documentation/printk-formats.txt}}, 
\IFRU{такие как}{like}
\TT{\%pM} (Mac-\IFRU{адрес}{address}),
\TT{\%pI4} (IPv4-\IFRU{адрес}{address}),
\TT{\%pUb} (UUID/GUID).

\IFRU{В}{In} GNU Multiple Precision Arithmetic Library \IFRU{есть ф-ция}{there are} gmp\_printf()
\footnote{\url{http://gmplib.org/manual/Formatted-Output-Strings.html}} \IFRU{имеющая нестандартные 
модификаторы нужные для вывода чисел с произвольной точностью}{function having non-standard modifiers for
arbitrary precision numbers outputting}. \\
\\
\IFRU{В \ac{OS}}{In the} Plan9\IFRU{, и в исходниках компилятора Go, можно найти ф-цию}
{ \ac{OS}, and in Go compiler source code, we may find}
fmtinstall()\IFRU{, для объявления нового модификатора printf-строки, например}
{ function for a new printf-string modifier definition, for example}:

\begin{lstlisting}[caption=go\textbackslash{}src\textbackslash{}cmd\textbackslash{}5c\textbackslash{}list.c]
void
listinit(void)
{

	fmtinstall('A', Aconv);
	fmtinstall('P', Pconv);
	fmtinstall('S', Sconv);
	fmtinstall('N', Nconv);
	fmtinstall('B', Bconv);
	fmtinstall('D', Dconv);
	fmtinstall('R', Rconv);
}

...

int
Pconv(Fmt *fp)
{
	char str[STRINGSZ], sc[20];
	Prog *p;
	int a, s;

	p = va_arg(fp->args, Prog*);
	a = p->as;
	s = p->scond;
	strcpy(sc, extra[s & C_SCOND]);
	if(s & C_SBIT)
		strcat(sc, ".S");
	if(s & C_PBIT)
		strcat(sc, ".P");
	if(s & C_WBIT)
		strcat(sc, ".W");
	if(s & C_UBIT)		/* ambiguous with FBIT */
		strcat(sc, ".U");
	if(a == AMOVM) {
		if(p->from.type == D_CONST)
			sprint(str, "	%A%s	%R,%D", a, sc, &p->from, &p->to);
		else
		if(p->to.type == D_CONST)
			sprint(str, "	%A%s	%D,%R", a, sc, &p->from, &p->to);
		else
			sprint(str, "	%A%s	%D,%D", a, sc, &p->from, &p->to);
	} else
	if(a == ADATA)
		sprint(str, "	%A	%D/%d,%D", a, &p->from, p->reg, &p->to);
	else
	if(p->as == ATEXT)
		sprint(str, "	%A	%D,%d,%D", a, &p->from, p->reg, &p->to);
	else
	if(p->reg == NREG)
		sprint(str, "	%A%s	%D,%D", a, sc, &p->from, &p->to);
	else
	if(p->from.type != D_FREG)
		sprint(str, "	%A%s	%D,R%d,%D", a, sc, &p->from, p->reg, &p->to);
	else
		sprint(str, "	%A%s	%D,F%d,%D", a, sc, &p->from, p->reg, &p->to);
	return fmtstrcpy(fp, str);
}
\end{lstlisting}
(\url{http://plan9.bell-labs.com/sources/plan9/sys/src/cmd/5c/list.c})

\IFRU{Ф-ция}{The} Pconv() 
\IFRU{будет вызвана если в строке формата будет встречен \%P}{will be called if \%P modifier
in the format string will be met}.
\IFRU{Затем она копирует созданную строку при помощи}
{Then it copies the string created using} fmtstrcpy().
\IFRU{Кстати, эта ф-ция и сама использует другие объявленные модификаторы, такие как}
{By the way, that function also uses other defined modifiers like} \%A, \%D, \IFRU{итд}{etc}. \\
\\
\IFRU{В}{The} Glibc\footnote{\IFRU{Стандартной библиотеке в Linux}{The Linux standard library}} 
\IFRU{есть нестандартное расширение}{has non-standard extension}
\footnote{\url{http://www.gnu.org/software/libc/manual/html_node/Customizing-Printf.html}}, 
\IFRU{позволяющее объявлять свои модификаторы, но это}{allowing to define our own
modifiers, but it is} \IT{deprecated}.

\IFRU{Попробуем определить свои собственные модификаторы для 
Mac-адреса и для вывода байта в бинарном виде}{Let's try to define our own modifiers for Mac-address
outputting and also for byte outputting in a binary form}:

\lstinputlisting{C/register_printf_function.c}
\footnote{\IFRU{Основа для примера взята отсюда}{The base of example was taken from}:
\url{http://codingrelic.geekhold.com/2008/12/printf-acular.html}}

\IFRU{Это компилируется с предупреждениями}{This compiled with warnings}:

\begin{lstlisting}
1.c: In function 'main':
1.c:48:2: warning: 'register_printf_function' is deprecated (declared at /usr/include/printf.h:106) [-Wdeprecated-declarations]
1.c:49:2: warning: 'register_printf_function' is deprecated (declared at /usr/include/printf.h:106) [-Wdeprecated-declarations]
1.c:51:2: warning: unknown conversion type character 'M' in format [-Wformat]
1.c:52:2: warning: unknown conversion type character 'B' in format [-Wformat]
\end{lstlisting}

\ac{GCC} \IFRU{умеет следить за соответствиями модификаторов в}{is able to track accordance between
modifiers in the} printf-\IFRU{строке и аргументами в вызове}{string and arguments in} printf(),
\IFRU{но здесь ему встречаются незнакомые модификаторы, о чем он предупреждает}
{however, unfamiliar to it modifiers are present here, so it warns us about them}.

\IFRU{Тем не менее, наша программа работает}{Nevertheless, our program works}:

\begin{lstlisting}
$ ./a.out
00:11:22:33:44:55
10101011
\end{lstlisting}



\section{strstr() и memmem()}

strstr() применяется для поиска строки в другой строке, либо чтобы узнать, есть ли там такая строка вообще.

memmem() можно применять с этими же целями, но для поиска по буферу, в котором могут быть нули,
либо по части строки.

\label{memchr}
\section{strchr() и memchr()}

strchr() применяется для поиска символа в строке, либо чтобы узнать, есть ли там такой символ вообще.

memchr() можно применять с этими же целями, но для поиска по части строки.

\section{atexit()}

При помощи atexit() можно добавить ф-цию, автоматически вызываемую перед выходом из вашей программы.
Кстати, программы на Си++ именно при помощи atexit() добавляют деструкторы глобальных объектов.

Можно попробовать:

\begin{lstlisting}
#include <string>

std::string s="test";

int main()
{
};
\end{lstlisting}

В листинге на ассемблере найдем конструктор этого глобального объекта:

\begin{lstlisting}[caption=MSVC 2010]
??__Es@@YAXXZ PROC					; `dynamic initializer for 's'', COMDAT
; Line 3
	push	ebp
	mov	ebp, esp
	push	OFFSET $SG22192
	mov	ecx, OFFSET ?s@@3V?$basic_string@DU?$char_traits@D@std@@V?$allocator@D@2@@std@@A ; s
	call	??0?$basic_string@DU?$char_traits@D@std@@V?$allocator@D@2@@std@@QAE@PBD@Z ; std::basic_string<char,std::char_traits<char>,std::allocator<char> >::basic_string<char,std::char_traits<char>,std::allocator<char> >
	push	OFFSET ??__Fs@@YAXXZ			; `dynamic atexit destructor for 's''
	call	_atexit
	add	esp, 4
	pop	ebp
	ret	0
??__Es@@YAXXZ ENDP					; `dynamic initializer for 's''

??__Fs@@YAXXZ PROC					; `dynamic atexit destructor for 's'', COMDAT
	push	ebp
	mov	ebp, esp
	mov	ecx, OFFSET ?s@@3V?$basic_string@DU?$char_traits@D@std@@V?$allocator@D@2@@std@@A ; s
	call	??1?$basic_string@DU?$char_traits@D@std@@V?$allocator@D@2@@std@@QAE@XZ ; std::basic_string<char,std::char_traits<char>,std::allocator<char> >::~basic_string<char,std::char_traits<char>,std::allocator<char> >
	pop	ebp
	ret	0
??__Fs@@YAXXZ ENDP					; `dynamic atexit destructor for 's''
\end{lstlisting}

Конструктор, конструируя, также регистрирует деструктор объекта в atexit().

\section{atoi(), atof(), strtod(), strtof()}

Ф-ции atoi/atof не могут сигнализировать об ошибке, а strtod/strtof, делая то же самое --- могут.

\subsection{bsearch(), lfind()}
\label{bsearch_lfind}

\IFRU{Удобные ф-ции для поиска чего-либо где-либо}{Handy functions for searching for something somewhere}.

\IFRU{Разница между ними только в том что}{The difference between them is that} lfind() 
\IFRU{просто ищет заданное}{just search for data}, \IFRU{а}{buf} bsearch() \IFRU{требует отсортированный массив
данных, но зато может искать быстрее}{requires sorted data array, but may search faster}
\footnote{\IFRU{методом половинного деления, итд}{by bisection method, etc}}
\footnote{\IFRU{разница еще также в том что bsearch() есть в стандарте}{the difference also is that
bsearch() is present in}\cite{C99TC3}, 
\IFRU{а lfind() нет, он есть только в \ac{POSIX} и в \ac{MSVC}}
{but lfind() is not, it is present only in \ac{POSIX} and \ac{MSVC}},
\IFRU{но эти ф-ции достаточно просты чтобы их реализовать самому}{but these functions are simple enough
to implement them on your own}}.

\IFRU{К примеру, поиск строки в массиве указателей на строки}{For example, search for a string
in the array of pointers to a strings}:

\lstinputlisting{C/lfind.c}

\IFRU{Свой собственный}{We need our own} stricmp() \IFRU{нужен потому что}{because} lfind() 
\IFRU{будет передавать в него указатель на искомую строку}{will pass a pointer to the string we looking for},
\IFRU{а также на место в массиве где записан указатель на строку, но не сама строка}
{but also to the place where pointer to the string is stored, not the string itself}.
\IFRU{Если бы это был
массив строк фиксированной длины, тогда можно было бы воспользоваться стандартным stricmp()}
{If it would be an array of fixed-size strings, then usual stricmp() could be used here instead}.

\IFRU{Кстати, точно также ф-ция сравнения задается и для}{By the way, the same comparison function
is used for} qsort().

lfind() \IFRU{возвращает указатель на место в массиве где ф-ция}{returns a pointer to a place in an array
where the} \TT{my\_stricmp()} \IFRU{сработала выдав}{function was triggered returning} $0$.
\IFRU{Далее вычисляем разницу между адресом этого места и адресом начала самого массива}
{Then we compute a difference between the address of that place and the address of the beginning of an array}.
\IFRU{Учитывая арифметику указателей}{Considering pointer arithmetics}\ref{PtrArith}, 
\IFRU{в итоге получается кол-во элементов между этими адресами}
{we get a number of elements between these addresses}.

\IFRU{Реализуя ф-цию сравнивания, можно искать строку в каком угодно массиве}
{By implementing comparison function, we can search a string in any array}.
\IFRU{Пример из}{Example from} OpenWatcom:

\begin{lstlisting}[caption=\textbackslash{}bld\textbackslash{}pbide\textbackslash{}defgen\textbackslash{}scan.c]
int MyComp( const void *p1, const void *p2 ) {

    KeyWord     *ptr;

    ptr = (KeyWord *)p2;
    return( strcmp( p1, ptr->str ) );
}

static int CheckReservedWords( char *tokbuf ) {
    KeyWord     *match;

    match = bsearch( tokbuf, ReservedWords,
                     sizeof( ReservedWords ) / sizeof( KeyWord ),
                     sizeof( KeyWord ), MyComp );
    if( match == NULL ) {
        return( T_NAME );
    } else {
        return( match->tok );
    }
}
\end{lstlisting}

\IFRU{Здесь имеется отсортированный по первому полю массив структур}
{Here we have array of structures sorted by the first field} \IT{ReservedWords},
\IFRU{выглядящий так}{like}:

\begin{lstlisting}[caption=\textbackslash{}bld\textbackslash{}pbide\textbackslash{}defgen\textbackslash{}scan.c]
typedef struct {
    char        *str;
    int         tok;
} KeyWord;

static KeyWord  ReservedWords[] = {
    "__cdecl",          T_CDECL,
    "__export",         T_EXPORT,
    "__far",            T_FAR,
    "__fortran",        T_FORTRAN,
    "__huge",           T_HUGE,
....
};
\end{lstlisting}

bsearch() \IFRU{ищет строку сравнивая её со строкой в первом поле структуры}{searching for the string comparing
it with the string in the first field of structure}.
\IFRU{Здесь можно применить именно bsearch(), потому что массив уже отсортированный}
{bsearch() can be used here because array is already sorted}.
\IFRU{Иначе пришлось бы использовать lfind()}{Otherwise, lfind() should be used}.
\IFRU{Вероятно, несортированный массив можно вначале отсортировать при помощи qsort(), а затем уже
использовать bsearch(), если вам нравится такая идея}
{Probably, unsorted array can be sorted by qsort() before bsearch() usage, if you like the idea}. \\
\\
\IFRU{Точно так же можно искать что угодно, где угодно}{In the same way, it is possible to search anything
anywhere}. \IFRU{Пример из}{The example from} BIND 9.9.1
\footnote{\url{https://www.isc.org/downloads/bind/}}:

\begin{lstlisting}[caption=backtrace.c]
static int
symtbl_compare(const void *addr, const void *entryarg) {
	const isc_backtrace_symmap_t *entry = entryarg;
	const isc_backtrace_symmap_t *end =
		&isc__backtrace_symtable[isc__backtrace_nsymbols - 1];

	if (isc__backtrace_nsymbols == 1 || entry == end) {
		if (addr >= entry->addr) {
			/*
			 * If addr is equal to or larger than that of the last
			 * entry of the table, we cannot be sure if this is
			 * within a valid range so we consider it valid.
			 */
			return (0);
		}
		return (-1);
	}

	/* entry + 1 is a valid entry from now on. */
	if (addr < entry->addr)
		return (-1);
	else if (addr >= (entry + 1)->addr)
		return (1);
	return (0);
}

...

	/*
	 * Search the table for the entry that meets:
	 * entry.addr <= addr < next_entry.addr.
	 */
	found = bsearch(addr, isc__backtrace_symtable, isc__backtrace_nsymbols,
			sizeof(isc__backtrace_symtable[0]), symtbl_compare);
	if (found == NULL)
		result = ISC_R_NOTFOUND;
\end{lstlisting}

\IFRU{Таким образом, можно обойтись без того чтобы писать каждый раз цикл for() для перебирания элементов, итд}
{Thus, it is possible to get rid of for() loop for enumerating elements each time, etc}.



\section{setjmp(), longjmp()}

В каком-то смысле, это реализация исключений в Си.

jmp\_buf это просто структура содержащая в себе набор регистров, но самые важные, это адрес текущей инструкции
и адрес указателя стека.
Всё что делает setjmp() это просто записывает текущие регистры процессора в эту структуру.
А всё что делает longjmp() это восстанавливает состояние регистров.

Это часто используется для возврата из каких-то глубоких мест наружу, обычно, в случае ошибок.
Собственно, как и исключения в Си++.

Например, в Oracle RDBMS, когда происходит некая ошибка, и пользователь код ошибки, сообщение об ошибке, итд,
в реальности, там срабатывает longjmp откуда-то из глубины. Для того же это используется и в bash.

Этот механизм даже немного гибче чем исключения в других ЯП --- нет никаких проблем ``устанавливать точки 
возврата'' в каких угодно местах программы и затем возвращаться туда по мере необходимости.

\label{goto}
Пофантазируя, можно даже сказать что longjmp это такой супермега-goto, обходящий блоки, ф-ции, и восстанавливающий
состояние стека.

Однако, нельзя забывать, что всё что восстанавливает longjmp это регистры. Выделенная память остается выделенной,
никаких деструкторов, как в Си++, вызвано не будет. Никакого \ac{RAII} здесь нет. 
С другой стороны, так как часть стека просто аннулируется, 
память выделенная при помощи alloca()\ref{alloca} будет также аннулирована.

