\section{Бинарные деревья в Си}

Бинарные деревья --- одна из важнейших структур данных в компьютерных науках.
Чаще всего они используются для хранения пар ``ключ-значение''. Это то что в Си++ STL реализовано в std::map.

Упрощенно говоря, по сравнению со списками, выборка у деревьев происходит намного быстрее.
С другой стороны, добавление элемента в дерево происходит чуть медленнее.

В стандартных библиотеках Си, нет работы с деревьями, но кое-что есть в POSIX 
(tsearch(), twalk(), tfind(), tdelete())
\footnote{\url{http://pubs.opengroup.org/onlinepubs/009696799/functions/tsearch.html}}.

Это семейство ф-ций активно используется в bash 4.2, BIND 9.9.1, GCC --- там можно посмотреть, как использовать
это.

Множество (std::set в Си++ STL) можно реализовать так же просто при помощи бинарных деревьев, достаточно
просто не хранить значение, а хранить только ключ.

