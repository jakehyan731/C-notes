\subsection{\IFRU{Бинарные деревья в Си}{Binary trees in C}}

\IFRU{Бинарные деревья}{Binary trees} ~--- \IFRU{одна из важнейших структур данных в компьютерных науках}
{are the one of the most important structures in computer science}.
\IFRU{Чаще всего они используются для хранения пар}{Most often these are used for}
``\IFRU{ключ-значение}{key-values}''\IFRU{}{ pairs storage}.
\IFRU{Это то что в \CPP \ac{STL} реализовано в std::map}
{This is what implemented in std::map in \CPP \ac{STL}}.

\IFRU{Упрощенно говоря, по сравнению со списками, выборка у деревьев происходит намного быстрее}
{Simply speaking, in comparison with lists, trees offers much faster selection}.
\IFRU{С другой стороны, добавление элемента в дерево может происходить медленнее}
{On the other hand, element insertion may be slower}.

\IFRU{В стандартных библиотеках Си, нет работы с деревьями, но кое-что есть в}
{There are no C standard functions working with a trees, but some things are present in} \ac{POSIX}
(tsearch(), twalk(), tfind(), tdelete())
\footnote{\url{http://pubs.opengroup.org/onlinepubs/009696799/functions/tsearch.html}}.

\IFRU{Это семейство ф-ций активно используется в}
{This family of functions are used actively in the} Bash 4.2, BIND 9.9.1, \ac{GCC} ~--- 
\IFRU{там можно посмотреть, как это использовать}{it can be seen there how it can be used}.

\IFRU{В}{The} Glib \IFRU{имеется также свои ф-ции для работы с деревьями, объявленные в}
{also has the tree functions declared in the} gtree.h
\footnote{\url{https://github.com/GNOME/glib/blob/master/glib/gtree.h}}.

\IFRU{Множество}{The set} (std::set \InENRU \CPP \ac{STL}) 
\IFRU{можно реализовать так же просто при помощи бинарных деревьев, достаточно просто не хранить значение, а хранить только ключ}
{can be implemented as binary trees as well, one may just not to store value and store only key}.

