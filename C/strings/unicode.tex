\subsection{Unicode}

Unicode \IFRU{в наше время это важно}{is important these days}. 
\IFRU{Наиболее популярные способы его применения это}{Most popular approaches are}:

\begin{itemize}
\index{UTF-8}
\item UTF-8
\IFRU{Популярно в UNIX-системах}{Popular in Unices}.
\IFRU{Сильное приемущество: можно продолжать пользоваться многими стандартными (и не только) 
ф-циями для обработки строк}
{Significant advantage: it is still possible to use many (but not limited to) 
standard functions for strings processing}.

\index{UTF-16}
\item UTF-16
\IFRU{Используется в}{Used in} Windows API.
\end{itemize}

\subsubsection{UTF-16}

\IFRU{Под каждый символ отводят 16-битный тип}{For each character a 16-bit type is assigned:}
\index{wchar\_t}
\IT{wchar\_t}.

\IFRU{Для объявления строк с таким типом, используется макрос \IT{L}}
{For such typed string definition, \IT{L} macro is used:}:

\begin{lstlisting}
L"hello world"
\end{lstlisting}

\IFRU{Для работы с wchar\_t вместо char, имеется целый класс функций-двойников с символом w в названии,
например}
{There is a special class of ``twin'' functions with the ``w'' in name, intended for work with 
wchar\_t instead of char}:
\index{fwprintf()}
\index{wcscmp()}
\index{wcslen()}
\index{iswalpha()}
fwprintf(), wcscmp(), wcslen(), iswalpha().

\paragraph{Windows}

\index{\Preprocessor!UNICODE}
\IFRU{В Windows, если некто хочет писать программу сразу в двух версиях, с использованием Unicode и без,
для этого есть тип \TT{tchar}, в зависимости от объявленной переменной препроцессора \TT{UNICODE},
он будет либо \TT{char} либо \TT{wchar\_t}}
{There is \TT{tchar} type in the Windows API which helps us to write a program in two builds: with Unicode
and without, depending on \TT{UNICODE} preprocessor variable definition, it will be \TT{char} or \TT{wchar\_t}}
\footnote{\IFRU{Одновременные сборки с Unicode и без были популярна во времена популярности
как}{Simultenous builds with Unicode and without were popular in the time of popularity of both} 
Windows NT/2000/XP \IFRU{так и}{and} Windows 95/98/ME\IFRU{}{ lines}.
\IFRU{Вторая линейка плохо поддерживала Unicode}
{Unicode support in the second was not very good}}.
\IFRU{Для этого же имеется макрос}{ }\TT{\_T(...)}\IFRU{}{ macro is also intended for this}:

\begin{lstlisting}
_T("hello world")
\end{lstlisting}

\IFRU{В зависимости от выставленной переменной препроцессора \TT{UNICODE}, она будет определена как}
{Depending on \TT{UNICODE} preprocessor macro definition, it will be} \TT{char} \OrENRU \TT{wchar\_t}.

\index{tchar.h}
\IFRU{В заголовочном файле \TT{tchar.h} есть масса ф-ций, меняющих свое поведение в зависимости от этой переменной}
{In the \TT{tchar.h} header file, there are a lot of functions, 
changing its behaviour depending on this variable}.
