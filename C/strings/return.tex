\subsection{\IFRU{Возврат строки}{String returning}}

\IFRU{Если некая ф-ция должна вернуть строку, имеются такие возможности}
{If a function must return a string, these options are available}:

\begin{itemize}
\item
1: \IFRU{Возврат строки-константы, это самое простое и быстрое}
{Constant string returning, is simplest and fastest}.

\item
2: \IFRU{Возврат строки через глобальный массив символов}
{String returning via global array of characters}. 
\IFRU{Недостаток: массив один и каждый вызов ф-ции перезаписывает его содержимое}
{Shortcoming: there is only one array and each subsequent function call overwrites its contents}.

\item
3: \IFRU{Возврат строки через буфер, заданный в аргументах ф-ции}
{String returning via buffer, pointer to which is passed in the function arguments}.
\IFRU{Недостаток: нужно также передавать и длину буфера, и вообще его длину нельзя заранее правильно расчитать}
{Shortcoming: buffer length must be passed as well, and also its length cannot be correctly calculated
in before}.

\item
4: \IFRU{Выделяем буфер нужного размера сами, записываем туда строку, возвращаем указатель}
{Allocate buffer of a size we need on our own, write string to it,
return the pointer to the buffer we allocated}.
\IFRU{Недостаток: тратятся ресурсы на выделение памяти}
{Shortcoming: resources spent on memory allocation}.

\item
5: \IFRU{Записываем строку в уже рассмотренный}{Write the string to the} \TT{strbuf}\IFRU{}{ we already mentioned} 
\OrENRU \TT{GString} \IFRU{или иную другую структуру, указатель на которую был
передан в аргументах}{or any other structure, pointer to which was passed in the arguments}.

\end{itemize}

\subsection{1: \IFRU{Возврат строки-константы}{Constant string returning}}

\IFRU{Первый вариант очень прост. Например}{The first option is very simple. E.g.}:

\lstinputlisting{C/strings/return_month_name1.c}

\IFRU{Можно даже еще проще}{Even simpler}:

\lstinputlisting{C/strings/return_month_name2.c}

\subsection{2: \IFRU{Через глобальный массив символов}{Via global array of characters}}

\index{asctime()}
\IFRU{Так делает стандартная ф-ция}{That is how} \TT{asctime()}\IFRU{}{ does it}.
\IFRU{Следует помнить, что нужно использовать возвращенную строку
перед каждым следующим вызовом}{Keep in mind that the string should be used before each subsequent call
to} \TT{asctime()}.

\IFRU{Например, это правильно}{For example, this is correct}:

\begin{lstlisting}
printf("date1: %s\n", asctime(&date1));
printf("date2: %s\n", asctime(&date2));
\end{lstlisting}

\IFRU{А это нет}{This is not}:

\begin{lstlisting}
char *date1=asctime(&date1);
char *date2=asctime(&date2);
printf("date1: %s\n", date1);
printf("date2: %s\n", date2);
\end{lstlisting}

... \IFRU{ведь указатели \TT{date1} и \TT{date2} будут указывать на одно и то же место, 
и вывод \TT{printf()} будет одинаковым}
{because \TT{date1} and \TT{date2} pointers will point to one place and \TT{printf()} output will be the same}. \\
\\
\IFRU{В git в \IT{hex.c}}{In \IT{hex.c} of git}\footnote{\url{https://github.com/git/git/blob/master/hex.c}} 
\IFRU{можно найти такое}{we may find this}:

\lstinputlisting{C/strings/git_hex.c}

\IFRU{Строка возвращается фактически через глобальную переменную,
определение её как \TT{static} внутри ф-ции просто напросто
обеспечивает доступ к ней только из этой ф-ции}{In fact, the string is returned via global variable,
\TT{static} declaration makes it visible only from this function}.
\IFRU{Но вот недостаток: после вызова}{Here is a shortcoming: after call to} \IT{sha1\_to\_hex()} 
\IFRU{вы не можете
вызвать её повторно для получения второй строки до тех пор, пока не используете как-то первую, ведь она
затрется}{you cannot call it again for the second string result before you use the first somehow,
because it will be overwritten}.
\IFRU{Для того чтобы решить эту проблему здесь, по видимому, сделали сразу 4 буфера и каждый раз строка
возвращается в следующем}{Apparently, in order to solve the problem, there are 4 buffers, and the string
is returned each time in the next one}.
\IFRU{Но имейте ввиду ~--- так можно делать если только вы уверены в том что вы делаете,
это код на уровне ``грязного хака''}{It is also worth to notice ~--- it is possible to do such things if you
are sure of what you do, the code is on the ``dirty hack'' level}.
\IFRU{Если вы вызовете эту ф-цию 5 раз и вам нужно будет использовать как-то строку полученную при первом вызове, 
это может привести к трудновыявляемой ошибке}{If you will call this function 5 times and will need to 
use the first string somehow, this may lead to hard-to-find bug}.

\IFRU{Кстати, обратите также внимание на то что переменная}{You may also notice that} \IT{bufno} 
\IFRU{не инициализируется}{is not initialized},
\IFRU{потому что используются только 
2 младших её бита}{because only 2 lower bits are used}, 
\IFRU{к тому же, не важно, какое значение переменная будет содержать в самом начале}{aside from that,
it is not important at all, which value it will hold at the program start}.

