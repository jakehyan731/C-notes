\subsection{Возврат строки}

Если некая ф-ция должна вернуть строку, имеются такие возможности:

\begin{itemize}
\item
Возврат строки-константы, это самое простое и быстрое.

\item
Возврат строки через глобальный массив символов. Недостаток: массив один и каждый вызов ф-ции перезаписывает
его содержимое.

\item
Возврат строки через буфер, заданный в аргументах ф-ции. Недостаток: нужно также передавать и длину буфера.

\item
Выделяем буфер нужного размера сами, записываем туда строку, возвращаем указатель. Недостаток: тратятся ресурсы
на выделение памяти.

\item
Записываем строку в уже рассмотренный strbuf или GString или иную другую структуру, указатель на которую был
передан в аргументах.

\end{itemize}

\subsection{Через строку-константу}

Первый вариант очень прост. Например:

\begin{lstlisting}
const char* get_month_name (int month)
{
	switch (month)
	{
	case 1: return "January";
	case 2: return "February";
	case 3: return "March";
	case 4: return "April";
	case 5: return "May";
	case 6: return "June";
	case 7: return "July";
	case 8: return "August";
	case 9: return "September";
	case 10: return "October";
	case 11: return "November";
	case 12: return "December";
	default: return "Unknown month!";
	};
};
\end{lstlisting}

Можно даже еще проще:

\begin{lstlisting}
const char* month_names[]={
	"January", "February", "March", "April", "May", "June", "July", "August",
	"September", "October", "November", "December" };

const char* get_month_name (int month)
{
	if (month>=1 && month<=12)
		return month_names[month-1];

	return "Unknown month!";
};
\end{lstlisting}

\subsection{Через глобальный массив символов}

Так делает стандартная ф-ция asctime(). Следует помнить, что нужно использовать возвращенную строку
перед каждым следующим вызовом asctime. 

Например, это правильно:

\begin{lstlisting}
printf("date1: %s\n", asctime(&date1));
printf("date2: %s\n", asctime(&date2));
\end{lstlisting}

А это нет:

\begin{lstlisting}
char *date1=asctime(&date1);
char *date2=asctime(&date2);
printf("date1: %s\n", date1);
printf("date2: %s\n", date2);
\end{lstlisting}

... ведь указатели date1 и date2 будут указывать на одно и то же место, и вывод printf() будет одинаковым. \\
\\
В git в hex.c\footnote{\url{https://github.com/git/git/blob/master/hex.c}} можно найти такое:

\lstinputlisting{C/strings/git_hex.c}

Строка возвращается фактически через глобальную переменную, объявление её как static внутри ф-ции просто напросто
обеспечивает доступ к ней только из этой ф-ции. Но вот недостаток: после вызова \IT{sha1\_to\_hex()} вы не можете
вызвать её повторно для получения второй строки до тех пор, пока не используете как-то первую, ведь она
затрется! Для того чтобы решить эту проблему здесь, по видимому, сделали сразу 4 буфера и каждый раз строка
возвращается в следующем. Но имейте ввиду --- так можно делать если только вы уверены в том что вы делаете,
это код на уровне ``грязного хака''.
Если вы
вызовете эту ф-цию 5 раз и вам нужно будет использовать как-то строку полученную при первом вызове, это может
привести к трудновыявляемой ошибке.

Кстати, обратите также внимание на то что переменная \IT{bufno} не инициализируется, потому что используются только 
2 младших её бита, к тому же, не важно, какое значение переменная будет содержать в самом начале!
