\subsubsection{scanf(), fscanf(), sscanf()}
\index{scanf()}

\IFRU{Извечный спор, что лучше, текстовые файлы или бинарные}
{A well-known holy-war, is text files are better than binary files or otherwise}.
\IFRU{Бинарные файлы быстрее и проще обрабатывать, зато текстовые
легче просматривать и редактировать в любом текстовом редакторе, к тому же, в UNIX имеется огромный арсенал
утилит для обработки текстов и строк}{It is easier and faster to process binary files,
however, text files are easier to view and edit in any text editor, beside, UNIX has
a lot of utilities for text and strings processing}.
\IFRU{Но текстовые файлы нужно парсить}{But text files must be parsed}.

\IFRU{Ф-ции }{}scanf()\IFRU{}{ function}\cite[7.19.6/2]{C99TC3} 
\IFRU{конечно же, регулярные выражения не поддерживают, 
однако при их помощи некоторые простые последовательности строк можно парсить}
{of course, does not support regular expressions, however, some simple sequences can be parsed by it}.

\paragraph{\IFRU{Пример}{Example} \#1}

\IFRU{Генерируемый ядром Linux файл}{The} 
\TT{/proc/meminfo}\IFRU{, начинается примерно так}{file generated by Linux kernel, beginning as}:

\begin{lstlisting}
MemTotal:        1026268 kB
MemFree:          119324 kB
Buffers:          170796 kB
Cached:           263736 kB
SwapCached:        11428 kB
...
\end{lstlisting}

\IFRU{Предположим, нам нужно узнать первое и третье число, игнорируя второе и остальные}
{Let's consider, we need to get first and third numbers, ignoring second and rest}.
\IFRU{Так это можно сделать}{That is how it can be done}:

\begin{lstlisting}
void read_proc_meminfo()
{
	FILE *f=fopen("/proc/meminfo", "r");
	assert(f);
	unsigned result1, result2;
	if (fscanf (f, "MemTotal:\t%d kB\n"
			"MemFree:\t%*d kB\n"
			"Buffers:\t%d kB\n", 
			&result1, &result2)==2)
		printf ("results: %d %d\n", result1, result2);
	fclose(f);
};
\end{lstlisting}

\IFRU{Строка формата расходится на три строки, в реальности это одна}
{The format string is defined in three lines, it is one in fact}: (\ref{heredoc}).
N.B. \IFRU{Перевод строки задается как}{The newline is defined as} \TT{\textbackslash{}n}.

\TT{*} \IFRU{в модификаторе scanf-строки указывает что число будет прочитано, но никуда записано не будет}
{in the scanf-string modifier pointing out that the number will be read, but will not be stored}.
\IFRU{Таким образом, это поле игнорируется}{Thus, the field is being ignored}. 
scanf()-\IFRU{функции возвращают кол-во не прочитанных полей (здесь
их будет 3) а кол-во записанных полей (2)}{functions are returning not a number of fields read (3 will be here),
buf number of fields stored (2 will be here)}.

\paragraph{\IFRU{Пример}{Example} \#2}

\IFRU{Имеется текстовый файл с парами в каждой строке (ключ-значение)}
{There a text file containing key-value pairs in each string}:

\begin{lstlisting}
some_param1=some_value
some_param2=Lazy fox etc etc.
param3=Lorem Ipsum etc etc.
space here=value containing space
too long param, we should fail here=value
\end{lstlisting}

\IFRU{Нужно просто читать оба поля}{We should just read two fields}:

\begin{lstlisting}
int main(int argc, char *argv[])
{
	assert(argc==2);
	assert(argv[1]);
	FILE *f=fopen (argv[1], "r");
	assert(f);
	int line=1;
	do
	{
		char param[16];
		char value[60];
		if (fscanf (f, "%16[^=]=%60[^\n]\n", param, value)==2)
		{
			printf ("param=%s\n", param);
			printf ("value=%s\n", value);
		}
		else
		{
			printf ("error at line %d\n", line);
			return 0;
		};
		line++;
	} while (!feof(f) && !ferror(f));
};
\end{lstlisting}

\index{\IFRU{Регулярные выражения}{Regular expressions}}
\TT{\%16[\^{}=]} ~--- \IFRU{это отдаленно напоминает регулярные выражения}
{is somewhat looks like regular expression}.
\IFRU{Означает, читать 16 любых символов, кроме
знака ``равно'' (=)}{Meaning, to read any 16 characters, except ``equal'' (=) sign}.
\IFRU{Затем, мы указываем scanf()-у, что далее должен быть этот самый знак (=)}{Then we point to scanf() that
there must be this sign (=)}.
\IFRU{Затем
пусть он читает 60 любых символов, кроме символа перевода строки}
{Then let him to read any 60 characters}. \IFRU{В конце читаем символ перевода строки}{We read newline character
at the end}.

\IFRU{Это работает, и поля ограничены длиной 16 и 60 символов}{This works, and field lengths are limited
to 16 and 60 characters}.
\IFRU{Поэтому на 5-й строке предсказуемо происходит ошибка,
ведь там длина парамера (первое поле) длиннее}{That is why error predictabily occuring on the fifth string, because
it has larger length of parameter (first field)}.

\IFRU{Так можно парсить несложные форматы, даже}{Thus it is possible to parse simple file formats, even}
\ac{CSV}.

\IFRU{Однако, нельзя забывать о том что scanf()-функции не способны прочитать пустую строку там где задается 
модификатор \%s}
{However, it should be noted that scanf()-functions are not able to read empty string where 
\%s modifier is defined}.
\IFRU{Поэтому, этим методом невозможно парсить файл с ключами-значениями,
где есть отсутствующие ключи или значения}{Thus it is not possible to parse a key-value file with absent keys
or values}.

\paragraph{\IFRU{Засада}{Caveat} \#1}

\IFRU{Если использовать \%d в строке формата, scanf() подразумевает что это 32-битный \TT{int} и на x86 и на
x64 процессорах}
{scanf() treat \%d modifier in the format string as 32-bit \TT{int} on both x86 and x64 CPUs}.

\IFRU{Частой ошибкой является писать нечто подобное}{It is a common mistake to write}:

\begin{lstlisting}
char a[10];

scanf ("%d %d %d %d", &a[0], &a[1], &a[2], &[3]);
\end{lstlisting}

\IFRU{Символы (или байты) лежат ``в притык'' друг к другу}
{Characters (or bytes) are placed adjacently to each other}.
\IFRU{Когда}{When} scanf() \IFRU{будет обрабатывать первое значение, он будет считать
его за 32-битный \TT{int}, и ``затрет'' остальные три, рядом лежащие}
{will process first value, it will treat it as 32-bit \TT{int} and overwrite other 3 located near}.
\IFRU{И так далее}{And so on}.

