\section{\IFRU{Строки в Си}{Strings in C}}

\IFRU{Причина, почему формат строки в Си именно такой (оканчивающийся нулем) вероятно историческая}
{The reason why C string format is as it is (zero-terminating) is apparently historical}.
\IFRU{В}{In} \cite{Ritchie79} \IFRU{мы можем прочитать}{we can read}:

\begin{framed}
\begin{quotation}
A minor difference was that the unit of I/O was the word, not the byte, because the PDP-7 was a word-addressed
machine. In practice this meant merely that all programs dealing with character streams ignored null
characters, because null was used to pad a file to an even number of characters.
\end{quotation}
\end{framed}

\IFRU{В Си нет встроенных возможностей для удобной работы со строками, такими, какие имеются в 
\ac{PL} более
высокого уровня как конкатенация}{There are no features in C to handle strings like those present 
in higher level PLs like concatenation}.

\index{strcat()}
\index{sprintf()}
\IFRU{Часто жалуются на неудобную
конкатенацию строк (то есть, склеивание) в Си при помощи функции strcat().
Также, многих раздражает sprintf(), под который нельзя толком заранее предсказать, 
сколько нужно выделять памяти}{People often complain about awkward string concatenation (i.e., glueling together).
Also irritating sprintf(), for which is hard to predict how much space will need}.

\index{strcpy()}
\IFRU{Копирование строк при помощи
strcpy() также неудобно ~--- нужно думать, сколько же выделить байт под буфер}
{String copying with strcpy() is not easy as well ~--- one needs to think ahead how many bytes must be allocated
for buffer}.
\IFRU{Помимо всего прочего, неудобная
работа со строками в Си, это источник огромного количества уязвимостей в ПО, связанных с переполнениями буфера}
{Aside from that, awkward C strings is the source of huge number of vulnerabilities related
to buffer overflow}\cite[1.14.2]{REBook}.

\IFRU{Прежде всего, нужно задать себе вопрос, какие операции со строками нам нужны}
{In the first place, we should ask ourselves, which string operations we need}.
\IFRU{Конкатенация (склеивание) нужна чтобы}{Concatenation (glueling) is needed for} 
1) \IFRU{выдавать в лог сообщения}{output messages to log};
2) \IFRU{конструировать строки и затем передавать (или записывать) их куда-то}
{construction of strings and then to pass (or write) them to some place}.

\IFRU{Для 1) можно использовать потоки (streams) ~--- не конструируя строку, выдавать её по порциям, например}
{For 1) it is possible to use streams ~--- without string construction just to output it by portions, e.g.}:

\begin{lstlisting}
printf ("Date: ");
dump_date(stdout, date);
printf (" a=");
dump_a(stdout, a);
printf ("\n");
\end{lstlisting}

\index{C++!ostream}
\IFRU{Подобное заменяется в \CPP выводом в \IT{ostream}}
{This is what \IT{ostream} in C++ is intended for}:

\begin{lstlisting}
cout << "Date: " << Date_ToString(date) << " a=" << a_ToString(a) << "\n";
\end{lstlisting}

\IFRU{Так быстрее и меньше требуется памяти для конструирования строк}
{It is faster, and requries less memory for string construction}.

\IFRU{Кстати, ошибкой является писать так}{By the way, it is a mistake to write like}:

\begin{lstlisting}
cout << "Date: " + Date_ToString(date) + " a=" + a_ToString(a) + "\n";
\end{lstlisting}

\IFRU{Для неспешного вывода в лог небольшого кол-ва сообщений это нормально,
но если таких сообщений очень много, то будут накладные расходы на их конкатенацию}
{At an easy pace, it is good enough to write messages to log, however, if there are a lot
of such messages, there may be string concatenation overhead}. \\
\\
\IFRU{Но все же строки иногда конструировать надо}{Anyway, sometimes strings must be constructed}.

\index{Glib}
\IFRU{Есть какие-то библиотеки для этого}{There are some libraries for this}.
\IFRU{К примеру, в}{For example, in} Glib\footnote{\url{https://developer.gnome.org/glib/}} 
\IFRU{есть}{there exist}
gstring.h\footnote{\url{https://github.com/GNOME/glib/blob/master/glib/gstring.h}}/
gstring.c\footnote{\url{https://github.com/GNOME/glib/blob/master/glib/gstring.c}}. 

\index{git}
\label{strbuf}
\IFRU{А в исходниках git можно найти}
{In the git source code we may find} strbuf.h\footnote{\url{https://github.com/git/git/blob/master/strbuf.h}}/
strbuf.c\footnote{\url{https://github.com/git/git/blob/master/strbuf.c}}. 
\IFRU{Собственно,
подобные Си-библиотеки очень похожи: они обеспечивают структуру данных, 
в которой есть некоторый буфер для строки, текущий размер буфера
и текущий размер строки в буфере}
{Strictly speaking, such C-libraries are very similar: they provide a data structure with a string buffer
in it, current buffer length and current string in buffer length}.
\IFRU{При помощи отдельных функций, можно добавлять новые строки или символы
в буфер, который, в свою очередь, будет автоматически увеличиваться или даже уменьшаться}
{With the help of various functions, it is possible to add to buffer other string or characters,
which, in turn, will grow or shrink}.

\index{sprintf()}
\IFRU{В}{In} \IT{strbuf.c} \IFRU{из}{from} git 
\IFRU{есть в том числе и ф-ция}{there is also a function} \IT{strbuf\_addf()}, 
\IFRU{работающая как}{working just like} \IT{sprintf()}, 
\IFRU{но добавляющая строку-результат в буфер}{but adding resulting string into the buffer}.

\IFRU{Так программист освобождается от головной боли связанной с выделением памяти}
{Thus a programmer may get rid of headache related to memory allocation}.
\IFRU{При работе с этими библиотеками, практически невозможна ситуация переполнения буфера, если только не
работать со структурой данных самостоятельно}{While using such libraries,
buffer overflows are virtually impossibly if not to work with the structures by himself}.

\IFRU{Типичная последовательность работы с такими библиотеками, выглядит так}
{The typical sequence of using such libraries looks like}:

\begin{itemize}
\item
\index{Glib!GString}
\IFRU{Инициализация структуры}{Structure} strbuf \IFRU{или}{or} GString\IFRU{}{ initialization}.

\item
\IFRU{Добавление строк и/или символов}{Adding strings and/or characters}.

\item
\IFRU{Имеем сконструированную строку}{Now we have constructed string}.

\item
\IFRU{Модифицируем её если нужно}{Modifying it if need}.

\item
\IFRU{Используем её как обычную Си-строку, записываем куда-то в файл, передаем по сети, и т.д.}
{Using it as usual C-string, writing it to to some file, send it by network, etc}.

\item
\IFRU{Освобождаем структуру}{Structure deinitialization}.
\end{itemize}

\index{Java}
\IFRU{Кстати, конструирование строк чем-то напоминает}{By the way, string construction resembles somehow}
Buffer\footnote{\url{http://docs.oracle.com/javase/7/docs/api/java/nio/Buffer.html}}, 
ByteBuffer\footnote{\url{http://docs.oracle.com/javase/7/docs/api/java/nio/ByteBuffer.html}} \AndENRU
CharBuffer\footnote{\url{http://docs.oracle.com/javase/7/docs/api/java/nio/CharBuffer.html}} \InENRU Java.

\subsection{\IFRU{Хранение длины строки}{String length storage}}

\index{Pascal}
\IFRU{Всегда хранить длину строки ~--- это было принято в реализациях \ac{PL} Pascal}
{String length is always stored ~--- it was done in Pascal \ac{PL} implementations}.
\IFRU{Не смотря на исходы святых войн\footnote{holy wars} между приверженцами Си и Pascal}
{Aside from holy wars outcomes between both PL devotees}, 
\IFRU{все же, почти все библиотеки
для хранения строк и работы с ними, хранят также и текущую длину}
{nevertheless, almost all string libraries keep current string length} ~--- 
\IFRU{просто потому что удобства от этого перевешивают необходимость пересчитывать это значение после
каждой модификации}
{just because conveniences outweigh the need of length value recalculation after each modification}.

\index{strlen()}
\IFRU{Например}{For example}, \IT{strlen()}
\footnote{\IFRU{подсчёт длины строки}{string length calculation}}
\IFRU{больше не нужен вообще, длина строки известна всегда}{is not needed at all, string length is always known}.
\IFRU{Конкатенация строк работает намного быстрее, потому что не нужно вычислять длину первой строки}
{String concatenation is also much faster, because we do not need to calculate length of the first string}.
\IFRU{Ф-ция сравнения строк в самом начале может сравнить длины строк и если они не равны, тут же вернуть 
\IT{false},
не начиная сравнивание символов в строках}
{The function of strings comparing may just compare string lengths 
at the beginning and if they are not equal to each other, return \IT{false} without starting to compare
characters in the strings}.

\index{Oracle RDBMS}
\IFRU{В всетевых библиотеках}{In the network libraries of} Oracle RDBMS, 
\IFRU{в функции работы со строками, зачастую передается строка и, 
отдельным аргументом, её длина}
{to the various string functions often passed string with its length, as separate argument}
\footnote{\url{http://blog.yurichev.com/node/64}}.
\IFRU{Это не очень эстетично, это выглядит избыточно, зато очень удобно}
{Not very \ae{}sthetical, looks redundant, but very useful}.
\IFRU{Например, у нас есть некоторая ф-ция, которой нужно в начале узнать, какую строку ей передали}
{For example, we have a function, which needs to know, which string was passed to it}:

\lstinputlisting{C/strings/strcmp1.c}

\IFRU{А вот если бы эта ф-ция имела длину входной строки, её можно было бы переписать так}
{However, if this function have length of the input string, it may be rewritten like}:

\lstinputlisting{C/strings/strcmp2.c}

\IFRU{Конечно, с эстетической точки зрения, код выглядит ужасно}
{\AE{}sthetically, the code looks just horrible}.
\IFRU{Тем не менее, мы здорово сократили количество необходимых сравнений строк}
{Nevertheless, we got rid of a lot of strings comparison calls}! 
\IFRU{Вероятно, для тех ситуаций, когда 
нужно как можно быстрее обрабатывать текстовые строки, такой подход может улучшить ситуацию}
{Apparently, for those cases when strings must be processed fast, such approach may help}.

\subsection{\IFRU{Возврат строки}{String returning}}

\IFRU{Если некая ф-ция должна вернуть строку, имеются такие возможности}
{If a function must return a string, these options are available}:

\begin{itemize}
\item
1: \IFRU{Возврат строки-константы, это самое простое и быстрое}
{Constant string returning, is simplest and fastest}.

\item
2: \IFRU{Возврат строки через глобальный массив символов}
{String returning via global array of characters}. 
\IFRU{Недостаток: массив один и каждый вызов ф-ции перезаписывает его содержимое}
{Shortcoming: there are only one array and each subsequent function call overwrites its contents}.

\item
3: \IFRU{Возврат строки через буфер, заданный в аргументах ф-ции}
{String returning via buffer, pointer to which is passed in the function arguments}.
\IFRU{Недостаток: нужно также передавать и длину буфера, и вообще его длину нельзя зараннее правильно расчитать}
{Shortcoming: buffer length must be passed as well, and also its length cannot be correctly calculated
in before}.

\item
4: \IFRU{Выделяем буфер нужного размера сами, записываем туда строку, возвращаем указатель}
{Allocate buffer of a size we need on our own, write string to it,
return the pointer to the buffer we allocated}.
\IFRU{Недостаток: тратятся ресурсы на выделение памяти}
{Shortcoming: resources spent on memory allocation}.

\item
5: \IFRU{Записываем строку в уже рассмотренный}{Write the string to the} \TT{strbuf}\IFRU{}{ we already mentioned} 
\OrENRU \TT{GString} \IFRU{или иную другую структуру, указатель на которую был
передан в аргументах}{or any other structure, pointer to which was passed in the arguments}.

\end{itemize}

\subsection{1: \IFRU{Возврат строки-константы}{Constant string returning}}

\IFRU{Первый вариант очень прост. Например}{The first option is very simple. E.g.}:

\lstinputlisting{C/strings/return_month_name1.c}

\IFRU{Можно даже еще проще}{Even simpler}:

\lstinputlisting{C/strings/return_month_name2.c}

\subsection{2: \IFRU{Через глобальный массив символов}{Via global array of characters}}

\index{asctime()}
\IFRU{Так делает стандартная ф-ция}{That is how} \TT{asctime()}\IFRU{}{ it does}.
\IFRU{Следует помнить, что нужно использовать возвращенную строку
перед каждым следующим вызовом}{Keep in mind that string should be used before each subsequent call
to} \TT{asctime()}.

\IFRU{Например, это правильно}{For example, this is correct}:

\begin{lstlisting}
printf("date1: %s\n", asctime(&date1));
printf("date2: %s\n", asctime(&date2));
\end{lstlisting}

\IFRU{А это нет}{This is not}:

\begin{lstlisting}
char *date1=asctime(&date1);
char *date2=asctime(&date2);
printf("date1: %s\n", date1);
printf("date2: %s\n", date2);
\end{lstlisting}

... \IFRU{ведь указатели \TT{date1} и \TT{date2} будут указывать на одно и то же место, 
и вывод \TT{printf()} будет одинаковым}
{because \TT{date1} and \TT{date2} pointers will point to one place and \TT{printf()} output will be the same}. \\
\\
\IFRU{В git в \IT{hex.c}}{In \IT{hex.c} of git}\footnote{\url{https://github.com/git/git/blob/master/hex.c}} 
\IFRU{можно найти такое}{we may find this}:

\lstinputlisting{C/strings/git_hex.c}

\IFRU{Строка возвращается фактически через глобальную переменную,
определение её как \TT{static} внутри ф-ции просто напросто
обеспечивает доступ к ней только из этой ф-ции}{In fact, the string is returned via global variable,
\TT{static} declaration makes it visible only from this function}.
\IFRU{Но вот недостаток: после вызова}{Here is a shortcoming: after call to} \IT{sha1\_to\_hex()} 
\IFRU{вы не можете
вызвать её повторно для получения второй строки до тех пор, пока не используете как-то первую, ведь она
затрется}{you cannot call it again for the second string result before you use the first somehow,
because it will be overwritten}.
\IFRU{Для того чтобы решить эту проблему здесь, по видимому, сделали сразу 4 буфера и каждый раз строка
возвращается в следующем}{Apparently, in order to solve the problem, here are 4 buffers, and the string
is returned each time in the next one}.
\IFRU{Но имейте ввиду ~--- так можно делать если только вы уверены в том что вы делаете,
это код на уровне ``грязного хака''}{It is also worth to notice ~--- it is possible to do such things if you
are sure in what you do, the code is on the ``dirty hack'' level}.
\IFRU{Если вы вызовете эту ф-цию 5 раз и вам нужно будет использовать как-то строку полученную при первом вызове, 
это может привести к трудновыявляемой ошибке}{If you will call this function 5 times and will need to 
use the first string somehow, this may lead to hard-to-find bug}.

\IFRU{Кстати, обратите также внимание на то что переменная}{You may also notice that} \IT{bufno} 
\IFRU{не инициализируется}{is not initialized},
\IFRU{потому что используются только 
2 младших её бита}{because only 2 lower bits are used}, 
\IFRU{к тому же, не важно, какое значение переменная будет содержать в самом начале}{aside from that,
it is not important at all, which value it will hold at the program start}.


\subsection{\IFRU{Стандартные ф-ции в Си для работы со строками}{Standard string C functions}}

\index{getcwd()}
\IFRU{Некоторые ф-ции, например, getcwd() не только заполняют буфер, но и возвращают указатель на него}
{Some functions like getcwd() not only filling the buffer, but also returns a pointer to it}.
\IFRU{Это для того чтобы можно было писать что-то вроде}
{It is made for the situations, where it is more compact to write something like}:

\begin{lstlisting}
char buf[256];
do_something (getcwd (buf, sizeof(buf)));
\end{lstlisting}

... \IFRU{вместо}{instead of}:

\begin{lstlisting}
char buf[256];
getcwd (buf, sizeof(buf))
do_something (buf);
\end{lstlisting}

\subsubsection{strstr() \AndENRU memmem()}

\index{strstr()}
strstr() \IFRU{применяется для поиска строки в другой строке, либо чтобы узнать, есть ли там такая строка вообще}
{is intended for searching for a substring in another string, or to get to know,
are there substring present in it anyway}.

\index{memmem()}
memmem() \IFRU{можно применять с этими же целями, но для поиска по буферу, в котором могут быть нули,
либо по части строки}{can be used with the same intentions, but for searching in the buffer which may
contain zeroes, ot in the part of a string}.

\subsubsection{strchr() \AndENRU memchr()}

\index{strchr()}
strchr() \IFRU{применяется для поиска символа в строке, либо чтобы узнать, есть ли там такой символ вообще}
{is used for searching for character in a string or to get to know if there such character present}.

\label{memchr}
\index{memchr()}
memchr() \IFRU{можно применять с этими же целями, но для поиска по части строки}{can be used with the same
intentions, but for searching in the part of a string}.

\subsubsection{atoi(), atof(), strtod(), strtof()}

\index{atoi()}
\index{atof()}
\index{strtod()}
\index{strtof()}
\IFRU{Ф-ции }{}atoi()/atof() \IFRU{не могут сигнализировать об ошибке}{cannot signal an error},
\IFRU{а}{but} strtod()/strtof()
\IFRU{, делая то же самое}{ while doing the same thing} ~--- \IFRU{могут}{can signal}.

\subsubsection{scanf(), fscanf(), sscanf()}

\IFRU{Извечный спор, что лучше, текстовые файлы или бинарные}
{A well-known holy-war, is text files are better than binary files or otherwise}.
\IFRU{Бинарные файлы быстрее и проще обрабатывать, зато текстовые
легче просматривать и редактировать в любом текстовом редакторе, к тому же, в UNIX имеется огромный арсенал
утилит для обработки текстов и строк}{It is easier and faster to process binary files,
however, text files are easier to view and edit in any text editor, beside, UNIX has
a lot of utilities for text and strings processing}.
\IFRU{Но текстовые файлы нужно парсить}{But text files must be parsed}.

\IFRU{Ф-ции }{}scanf()\IFRU{}{ function}\cite[7.19.6.2]{C99TC3} 
\IFRU{конечно же, регулярные выражения не поддерживают, 
однако при их помощи некоторые простые последовательности строк можно парсить}
{of course, does not support regular expressions, however, some simple sequences can be parsed by it}.

\paragraph{\IFRU{Пример}{Example} \#1}

\IFRU{Генерируемый ядром Linux файл}{The} 
\TT{/proc/meminfo}\IFRU{, начинается примерно так}{file generated by Linux kernel, beginning as}:

\begin{lstlisting}
MemTotal:        1026268 kB
MemFree:          119324 kB
Buffers:          170796 kB
Cached:           263736 kB
SwapCached:        11428 kB
...
\end{lstlisting}

\IFRU{Предположим, нам нужно узнать первое и третье число, игнорируя второе и остальные}
{Let's consider, we need to get first and third numbers, ignoring second and rest}.
\IFRU{Так это можно сделать}{That is how it can be done}:

\begin{lstlisting}
void read_proc_meminfo()
{
	FILE *f=fopen("/proc/meminfo", "r");
	assert(f);
	unsigned result1, result2;
	if (fscanf (f, "MemTotal:\t%d kB\n"
			"MemFree:\t%*d kB\n"
			"Buffers:\t%d kB\n", 
			&result1, &result2)==2)
		printf ("results: %d %d\n", result1, result2);
	fclose(f);
};
\end{lstlisting}

\IFRU{Строка формата расходится на три строки, в реальности это одна}
{The format string is defined in three lines, it is one in fact}: \ref{heredoc}.
\IFRU{Обратите внимание на}{Please also note} \TT{\textbackslash{}n}, \IFRU{так мы задаем перевод строки}
{that is how newline is defined}.

\TT{*} \IFRU{в модификаторе scanf-строки указывает что число будет прочитано, но никуда записано не будет}
{in the scanf-string modifier pointing out that the number will be read, but will not be stored}.
\IFRU{Таким образом, это поле игнорируется}{Thus, the field is being ignored}. 
scanf()-\IFRU{функции возвращают кол-во не прочитанных полей (здесь
их будет 3) а кол-во записанных полей (2)}{functions are returning not a number of fields read (3 will be here),
buf number of fields stored (2 will be here)}.

\paragraph{\IFRU{Пример}{Example} \#2}

\IFRU{Имеется текстовый файл с парами в каждой строке (ключ-значение)}
{There a text file containing key-value pairs in each string}:

\begin{lstlisting}
some_param1=some_value
some_param2=Lazy fox etc etc.
param3=Lorem Ipsum etc etc.
space here=value containing space
too long param, we should fail here=value
\end{lstlisting}

\IFRU{Нужно просто читать оба поля}{We should just read two fields}:

\begin{lstlisting}
int main(int argc, char *argv[])
{
	assert(argc==2);
	assert(argv[1]);
	FILE *f=fopen (argv[1], "r");
	assert(f);
	int line=1;
	do
	{
		char param[16];
		char value[60];
		if (fscanf (f, "%16[^=]=%60[^\n]\n", param, value)==2)
		{
			printf ("param=%s\n", param);
			printf ("value=%s\n", value);
		}
		else
		{
			printf ("error at line %d\n", line);
			return 0;
		};
		line++;
	} while (!feof(f) && !ferror(f));
};
\end{lstlisting}

\TT{\%16[\^{}=]} ~--- \IFRU{это отдаленно напоминает регулярные выражения}
{is somewhat looks like regular expression}.
\IFRU{Означает, читать 16 любых символов, кроме
знака ``равно'' (=)}{Meaning, to read any 16 characters, except ``equal'' (=) sign}.
\IFRU{Затем, мы указываем scanf()-у, что далее должен быть этот самый знак (=)}{Then we point to scanf() that
there must be this sign (=)}.
\IFRU{Затем
пусть он читает 60 любых символов, кроме символа перевода строки}
{Then let him to read any 60 characters}. \IFRU{В конце читаем символ перевода строки}{We read newline character
at the end}.

\IFRU{Это работает, и поля ограничены длиной 16 и 60 символов}{This works, and field lengths are limited
to 16 and 60 characters}.
\IFRU{Поэтому на 5-й строке предсказуемо происходит ошибка,
ведь там длина парамера (первое поле) длиннее}{That is why error predictabily occuring on the fifth string, because
it has larger length of parameter (first field)}.

\IFRU{Так можно парсить несложные форматы, даже}{Thus it is possible to parse simple file formats, even} CSV
\footnote{Comma-separated values: \url{https://en.wikipedia.org/wiki/Comma-separated_values}}.

\IFRU{Однако, нельзя забывать о том что scanf()-функции не способны прочитать пустую строку там где задается 
модификатор \%s}
{However, it should be noted that scanf()-functions are not able to read empty string where 
\%s modifier is defined}.
\IFRU{Поэтому, этим методом невозможно парсить файл с ключами-значениями,
где есть отсутствующие ключи или значения}{Thus it is not possible to parse a key-value file with absent keys
or values}.

\paragraph{\IFRU{Засада}{Caveat} \#1}

\IFRU{Если использовать \%d в строке формата, scanf() подразумевает что это 32-битный \TT{int} и на x86 и на
x64 процессорах}
{scanf() treat \%d modifier in the format string as 32-bit \TT{int} on both x86 and x64 CPUs}.

\IFRU{Частой ошибкой является писать нечто подобное}{It is a common mistake to write}:

\begin{lstlisting}
char a[10];

scanf ("%d %d %d %d", &a[0], &a[1], &a[2], &[3]);
\end{lstlisting}

\IFRU{Символы (или байты) лежат ``в притык'' друг к другу}
{Characters (or bytes) are placed adjacently to each other}.
\IFRU{Когда}{When} scanf() \IFRU{будет обрабатывать первое значение, он будет считать
его за 32-битный \TT{int}, и ``затрет'' остальные три, рядом лежащие}
{will process first value, it will treat it as 32-bit \TT{int} and overwrite other 3 located near}.
\IFRU{И так далее}{And so on}.



\subsubsection{strspn(), strcspn()}

\index{strspn()}
\TT{strspn()} \IFRU{часто применяется для того чтобы удостовериться, что некая строка полностью состоит из
нужных символов}{is often used to get to be sure that a string has only characters from the list we defined}:
    
\begin{lstlisting}
if (strspn(s, "1234567890") == strlen(s)) ... OK
...
if (strspn(IPv4, "1234567890.") == strlen(IPv4)) ... OK
...
if (strspn(IPv6, "0123456789AaBbCcDdEeFf:.") == strlen(IPv6)) ... OK
\end{lstlisting}

\IFRU{Либо для того чтобы пропустить начало строки}{Or to skip a begin of a string}:

\begin{lstlisting}
const char *whitespaces = " \n\r\t";
*buf += strspn(*buf, whitespaces); // skip whitespaces at start
\end{lstlisting}

\index{strcspn()}
\TT{strcspn()} \IFRU{это обратная ф-ция}{is inverse function},
\IFRU{её можно использовать для пропуска всех символов в начале строки, не попадающих
под множество символов}{it can be used for skipping all symbols at the string beginning,
which are not defined in a set}:

\begin{lstlisting}
s += strcspn(s, whitespaces); // first, skip anything till whitespaces
s += strspn(s, whitespaces); // then skip whitespaces
// here 's' is pointing to the part of string after whitespaces
\end{lstlisting}

\subsubsection{strtok() \AndENRU strpbrk()}

\index{strtok()}
\index{strpbrk()}
\IFRU{Обе ф-ции служат для разбиения строки на подстроки, отделенные друг от друга разделительными символами}
{Both functions are used for delimiting string into substrings, divided by special characters}
\footnote{delimiter}.
\IFRU{Только}{However} strtok() \IFRU{модифицирует исходную строку}{modifies source stirng}
(\IFRU{и таким образом, получаемые подстроки сразу можно использовать как отдельные Си-строки}
{and thus resulting substrings can be used as separated C-strings}), 
\IFRU{а}{but} strpbrk() \IFRU{нет, он только возвращает указатель на следующую подстроку}
{is not, it is only returning a pointer to the next substring}.


\subsection{Unicode}

Unicode \IFRU{в наше время это важно}{is important these days}. 
\IFRU{Наиболее популярные способы его применения это}{Most popular approaches are}:

\begin{itemize}
\index{UTF-8}
\item UTF-8
\IFRU{Популярно в UNIX-системах}{Popular in Unices}.
\IFRU{Сильное приемущество: можно продолжать пользоваться многими стандартными (и не только) 
ф-циями для обработки строк}
{Significant advantage: it is still possible to use many (but not limited to) 
standard functions for strings processing}.

\index{UTF-16}
\item UTF-16
\IFRU{Используется в}{Used in} Windows API.
\end{itemize}

\subsubsection{UTF-16}

\IFRU{Под каждый символ отводят 16-битный тип}{For each character a 16-bit type is assigned:}
\index{wchar\_t}
\IT{wchar\_t}.

\IFRU{Для объявления строк с таким типом, используется макрос \IT{L}}
{For such typed string definition, \IT{L} macro is used:}:

\begin{lstlisting}
L"hello world"
\end{lstlisting}

\IFRU{Для работы с wchar\_t вместо char, имеется целый класс функций-двойников с символом w в названии,
например}
{There are a special class of ``twin'' functions with the ``w'' in name, intended for work with 
wchar\_t instead of char}:
\index{fwprintf()}
\index{wcscmp()}
\index{wcslen()}
\index{iswalpha()}
fwprintf(), wcscmp(), wcslen(), iswalpha().

\paragraph{Windows}

\index{\Preprocessor!UNICODE}
\IFRU{В Windows, если некто хочет писать программу сразу в двух версиях, с использованием Unicode и без,
для этого есть тип \TT{tchar}, в зависимости от объявленной переменной препроцессора \TT{UNICODE},
он будет либо \TT{char} либо \TT{wchar\_t}}
{There are \TT{tchar} type in the Windows API which helps us to write a program in two builds: with Unicode
and without, depending on \TT{UNICODE} preprocessor variable definition, it will be \TT{char} or \TT{wchar\_t}}
\footnote{\IFRU{Одновременные сборки с Unicode и без были популярна во времена популярности
как}{Simultenous builds with Unicode and without were popular in the time of popularity of both} 
Windows NT/2000/XP \IFRU{так и}{and} Windows 95/98/ME\IFRU{}{ lines}.
\IFRU{Вторая линейка плохо поддерживала Unicode}
{Unicode support in the second was not very good}}.
\IFRU{Для этого же имеется макрос}{ }\TT{\_T(...)}\IFRU{}{ macro is also intended for this}:

\begin{lstlisting}
_T("hello world")
\end{lstlisting}

\IFRU{В зависимости от выставленной переменной препроцессора \TT{UNICODE}, она будет определена как}
{Depending on \TT{UNICODE} preprocessor macro definition, it will be} \TT{char} \OrENRU \TT{wchar\_t}.

\index{tchar.h}
\IFRU{В заголовочном файле \TT{tchar.h} есть масса ф-ций, меняющих свое поведение в зависимости от этой переменной}
{In the \TT{tchar.h} header file, there are a lot of functions, 
changing its behaviour depending on this variable}.


\subsection{\IFRU{Списки строк}{Lists of strings}}

\IFRU{Самый простой список строк, это просто набор строк оканчивающийся нулем}{The simplest list of strings
is just a strings set ending with the zero}.
\IFRU{Например}{For example}, \InENRU Windows API, \IFRU{в библиотеке}{in the} Common Dialogs
\IFRU{так}{library, thus}
\footnote{\url{http://msdn.microsoft.com/en-us/library/windows/desktop/ms646829(v=vs.85).aspx}} 
\IFRU{передаются список допустимых расширений файлов для диалогового окна}
{a list of available file extensions for dialog box are passed}:

\begin{lstlisting}
// Initialize OPENFILENAME
ZeroMemory(&ofn, sizeof(ofn));
...
ofn.lpstrFilter = "All\0*.*\0Text\0*.TXT\0";
...

// Display the Open dialog box. 

if (GetOpenFileName(&ofn)==TRUE) 
	...
\end{lstlisting}

