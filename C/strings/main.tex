\section{Строки в Си}

В Си нет встроенных возможностей для удобной работы со строками, такими, какие имеются в ЯП более
высокого уровня.

Часто жалуются на неудобную
конкатенацию строк (то есть, склеивание) в Си при помощи функции strcat(). Также, многих раздражает sprintf(),
под которых нельзя толком зараннее предсказать, сколько нужно выделять памяти. Копирование строк при помощи
strcpy() также неудобно --- нужно думать, сколько же выделить байт под буфер. Помимо всего прочего, неудобная
работа со строками в Си, это источник огромного количества уязвимостей в ПО, связанных с переполнениями буфера\cite[1.14.2]{REBook}.

Прежде всего, нужно задать себе вопрос, какие операции со строками нам нужны.
Конкатенация (склеивание) нужна чтобы 1) выдавать в лог сообщения; 2) конструировать строки и записывать их куда-то.

Для 1) можно использовать потоки (streams) --- не конструируя строку, выдавать её по порциям, например:

\begin{lstlisting}
printf ("Date: ");
dump_date(stdout, date);
printf (" a=");
dump_a(stdout, a);
printf ("\n");
\end{lstlisting}

Подобное заменяется в Си++ выводом в ostream:

\begin{lstlisting}
cout << "Date: " << Date_ToString(date) << " a=" << a_ToString(a) << "\n";
\end{lstlisting}

Так быстрее и меньше требуется памяти для конструирования строк.

Кстати, ошибкой является писать так:

\begin{lstlisting}
cout << "Date: " + Date_ToString(date) + " a=" + a_ToString(a) + "\n";
\end{lstlisting}

Для неспешного вывода в лог небольшого кол-ва сообщений это нормально, но если таких строк очень много, то будут
накладные расходы на их конкатенацию. \\
\\
Но все же строки иногда конструировать надо.

Есть какие-то библиотеки для этого.
К примеру, в Glib\footnote{\url{https://developer.gnome.org/glib/}} есть 
gstring.h\footnote{\url{https://github.com/GNOME/glib/blob/master/glib/gstring.h}}/
gstring.c\footnote{\url{https://github.com/GNOME/glib/blob/master/glib/gstring.c}}. 

\label{strbuf}
А в исходниках git можно найти strbuf.h\footnote{\url{https://github.com/git/git/blob/master/strbuf.h}}/
strbuf.c\footnote{\url{https://github.com/git/git/blob/master/strbuf.c}}. Собственно,
подобные Си-библиотеки очень похожи: они обеспечивают структуру данных, в которой есть некоторый буфер для строки, текущий размер буфера
и текущий размер строки в буфере. При помощи отдельных функций, можно добавлять новые строки или символы
в буфер, который, в свою очередь, будет автоматически увеличиваться или даже уменьшаться.

В \IT{strbuf.c} из git есть даже ф-ция \IT{strbuf\_addf()}, работающая как \IT{sprintf()}, 
но добавляющая строку-результат в буфер.

Так пользователь освобождается от головной боли связанной с выделением памяти.
При работе с этими библиотеками, практически невозможна ситуация переполнения буфера, если только не начать
работать со структурой самостоятельно.

Типичная последовательность работы с такими библиотеками, выглядит так:

\begin{itemize}
\item
Инициализация структуры strbuf или GString.

\item
Добавление строк и/или символов.

\item
Имеем сконструированную строку. Используем как обычную Си-строку, записываем куда-то в файл, передаем по сети, итд.

\item
Освобождаем структуру.
\end{itemize}

Кстати, конструирование строк чем-то напоминает 
Buffer\footnote{\url{http://docs.oracle.com/javase/7/docs/api/java/nio/Buffer.html}}, 
ByteBuffer\footnote{\url{http://docs.oracle.com/javase/7/docs/api/java/nio/ByteBuffer.html}} и 
CharBuffer\footnote{\url{http://docs.oracle.com/javase/7/docs/api/java/nio/CharBuffer.html}} в Java.

\subsection{Хранение длины строки}

Всегда хранить длину строки --- это было принято в реализациях ЯП Pascal. 
Не смотря на исходы святых войн\footnote{holy wars} между приверженцами Си и Pascal, все же, почти все библиотеки
для хранения строк и работы с ними, хранят также и текущую длину --- потому что удобства от этого перевешивают
необходимость пересчитывать это значение.

Например, \IT{strlen()} (подсчет длины строки) больше не нужен вообще, длина все время известна.
Конкатенация строк работает намного быстрее, потому что не нужно вычислять длину первой строки.
Ф-ция сравнения строк в самом начале может сравнить длины строк и если они не равны, тут же вернуть false,
не начиная сравнивание самих строк.

В Oracle RDBMS, в сетевых библиотеках, в функции работы со строками, зачастую передается строка и, 
отдельным аргументом, её длина\footnote{\url{http://blog.yurichev.com/node/64}}.
Это не очень эстетично, это выглядит избыточно, зато очень удобно.
Например, у нас есть некоторая ф-ция, которой нужно в начале узнать, какую строку ей передали:

\lstinputlisting{C/strings/strcmp1.c}

А вот если бы эта ф-ция имела длину входной строки, её можно было бы переписать так:

\lstinputlisting{C/strings/strcmp2.c}

Конечно, с эстетической точки зрения, код выглядит ужасно.
Тем не менее, мы здорово сократили количество необходимых сравнений строк! Вероятно, для тех ситуаций, когда 
нужно как можно быстрее обрабатывать текстовые строки, такой подход может улучшить ситуацию.

\subsection{\IFRU{Возврат строки}{String returning}}

\IFRU{Если некая ф-ция должна вернуть строку, имеются такие возможности}
{If a function must return a string, these options are available}:

\begin{itemize}
\item
1: \IFRU{Возврат строки-константы, это самое простое и быстрое}
{Constant string returning, is simplest and fastest}.

\item
2: \IFRU{Возврат строки через глобальный массив символов}
{String returning via global array of characters}. 
\IFRU{Недостаток: массив один и каждый вызов ф-ции перезаписывает его содержимое}
{Shortcoming: there are only one array and each subsequent function call overwrites its contents}.

\item
3: \IFRU{Возврат строки через буфер, заданный в аргументах ф-ции}
{String returning via buffer, pointer to which is passed in the function arguments}.
\IFRU{Недостаток: нужно также передавать и длину буфера, и вообще его длину нельзя зараннее правильно расчитать}
{Shortcoming: buffer length must be passed as well, and also its length cannot be correctly calculated
in before}.

\item
4: \IFRU{Выделяем буфер нужного размера сами, записываем туда строку, возвращаем указатель}
{Allocate buffer of a size we need on our own, write string to it,
return the pointer to the buffer we allocated}.
\IFRU{Недостаток: тратятся ресурсы на выделение памяти}
{Shortcoming: resources spent on memory allocation}.

\item
5: \IFRU{Записываем строку в уже рассмотренный}{Write the string to the} \TT{strbuf}\IFRU{}{ we already mentioned} 
\OrENRU \TT{GString} \IFRU{или иную другую структуру, указатель на которую был
передан в аргументах}{or any other structure, pointer to which was passed in the arguments}.

\end{itemize}

\subsection{1: \IFRU{Возврат строки-константы}{Constant string returning}}

\IFRU{Первый вариант очень прост. Например}{The first option is very simple. E.g.}:

\lstinputlisting{C/strings/return_month_name1.c}

\IFRU{Можно даже еще проще}{Even simpler}:

\lstinputlisting{C/strings/return_month_name2.c}

\subsection{2: \IFRU{Через глобальный массив символов}{Via global array of characters}}

\index{asctime()}
\IFRU{Так делает стандартная ф-ция}{That is how} \TT{asctime()}\IFRU{}{ it does}.
\IFRU{Следует помнить, что нужно использовать возвращенную строку
перед каждым следующим вызовом}{Keep in mind that string should be used before each subsequent call
to} \TT{asctime()}.

\IFRU{Например, это правильно}{For example, this is correct}:

\begin{lstlisting}
printf("date1: %s\n", asctime(&date1));
printf("date2: %s\n", asctime(&date2));
\end{lstlisting}

\IFRU{А это нет}{This is not}:

\begin{lstlisting}
char *date1=asctime(&date1);
char *date2=asctime(&date2);
printf("date1: %s\n", date1);
printf("date2: %s\n", date2);
\end{lstlisting}

... \IFRU{ведь указатели \TT{date1} и \TT{date2} будут указывать на одно и то же место, 
и вывод \TT{printf()} будет одинаковым}
{because \TT{date1} and \TT{date2} pointers will point to one place and \TT{printf()} output will be the same}. \\
\\
\IFRU{В git в \IT{hex.c}}{In \IT{hex.c} of git}\footnote{\url{https://github.com/git/git/blob/master/hex.c}} 
\IFRU{можно найти такое}{we may find this}:

\lstinputlisting{C/strings/git_hex.c}

\IFRU{Строка возвращается фактически через глобальную переменную,
определение её как \TT{static} внутри ф-ции просто напросто
обеспечивает доступ к ней только из этой ф-ции}{In fact, the string is returned via global variable,
\TT{static} declaration makes it visible only from this function}.
\IFRU{Но вот недостаток: после вызова}{Here is a shortcoming: after call to} \IT{sha1\_to\_hex()} 
\IFRU{вы не можете
вызвать её повторно для получения второй строки до тех пор, пока не используете как-то первую, ведь она
затрется}{you cannot call it again for the second string result before you use the first somehow,
because it will be overwritten}.
\IFRU{Для того чтобы решить эту проблему здесь, по видимому, сделали сразу 4 буфера и каждый раз строка
возвращается в следующем}{Apparently, in order to solve the problem, here are 4 buffers, and the string
is returned each time in the next one}.
\IFRU{Но имейте ввиду ~--- так можно делать если только вы уверены в том что вы делаете,
это код на уровне ``грязного хака''}{It is also worth to notice ~--- it is possible to do such things if you
are sure in what you do, the code is on the ``dirty hack'' level}.
\IFRU{Если вы вызовете эту ф-цию 5 раз и вам нужно будет использовать как-то строку полученную при первом вызове, 
это может привести к трудновыявляемой ошибке}{If you will call this function 5 times and will need to 
use the first string somehow, this may lead to hard-to-find bug}.

\IFRU{Кстати, обратите также внимание на то что переменная}{You may also notice that} \IT{bufno} 
\IFRU{не инициализируется}{is not initialized},
\IFRU{потому что используются только 
2 младших её бита}{because only 2 lower bits are used}, 
\IFRU{к тому же, не важно, какое значение переменная будет содержать в самом начале}{aside from that,
it is not important at all, which value it will hold at the program start}.


\subsection{\IFRU{Стандартные ф-ции в Си для работы со строками}{Standard string C functions}}

\index{getcwd()}
\IFRU{Некоторые ф-ции, например, getcwd() не только заполняют буфер, но и возвращают указатель на него}
{Some functions like getcwd() not only filling the buffer, but also returns a pointer to it}.
\IFRU{Это для того чтобы можно было писать что-то вроде}
{It is made for the situations, where it is more compact to write something like}:

\begin{lstlisting}
char buf[256];
do_something (getcwd (buf, sizeof(buf)));
\end{lstlisting}

... \IFRU{вместо}{instead of}:

\begin{lstlisting}
char buf[256];
getcwd (buf, sizeof(buf))
do_something (buf);
\end{lstlisting}

\subsubsection{strstr() \AndENRU memmem()}

\index{strstr()}
strstr() \IFRU{применяется для поиска строки в другой строке, либо чтобы узнать, есть ли там такая строка вообще}
{is intended for searching for a substring in another string, or to get to know,
are there substring present in it anyway}.

\index{memmem()}
memmem() \IFRU{можно применять с этими же целями, но для поиска по буферу, в котором могут быть нули,
либо по части строки}{can be used with the same intentions, but for searching in the buffer which may
contain zeroes, ot in the part of a string}.

\subsubsection{strchr() \AndENRU memchr()}

\index{strchr()}
strchr() \IFRU{применяется для поиска символа в строке, либо чтобы узнать, есть ли там такой символ вообще}
{is used for searching for character in a string or to get to know if there such character present}.

\label{memchr}
\index{memchr()}
memchr() \IFRU{можно применять с этими же целями, но для поиска по части строки}{can be used with the same
intentions, but for searching in the part of a string}.

\subsubsection{atoi(), atof(), strtod(), strtof()}

\index{atoi()}
\index{atof()}
\index{strtod()}
\index{strtof()}
\IFRU{Ф-ции }{}atoi()/atof() \IFRU{не могут сигнализировать об ошибке}{cannot signal an error},
\IFRU{а}{but} strtod()/strtof()
\IFRU{, делая то же самое}{ while doing the same thing} ~--- \IFRU{могут}{can signal}.

\subsubsection{scanf(), fscanf(), sscanf()}

\IFRU{Извечный спор, что лучше, текстовые файлы или бинарные}
{A well-known holy-war, is text files are better than binary files or otherwise}.
\IFRU{Бинарные файлы быстрее и проще обрабатывать, зато текстовые
легче просматривать и редактировать в любом текстовом редакторе, к тому же, в UNIX имеется огромный арсенал
утилит для обработки текстов и строк}{It is easier and faster to process binary files,
however, text files are easier to view and edit in any text editor, beside, UNIX has
a lot of utilities for text and strings processing}.
\IFRU{Но текстовые файлы нужно парсить}{But text files must be parsed}.

\IFRU{Ф-ции }{}scanf()\IFRU{}{ function}\cite[7.19.6.2]{C99TC3} 
\IFRU{конечно же, регулярные выражения не поддерживают, 
однако при их помощи некоторые простые последовательности строк можно парсить}
{of course, does not support regular expressions, however, some simple sequences can be parsed by it}.

\paragraph{\IFRU{Пример}{Example} \#1}

\IFRU{Генерируемый ядром Linux файл}{The} 
\TT{/proc/meminfo}\IFRU{, начинается примерно так}{file generated by Linux kernel, beginning as}:

\begin{lstlisting}
MemTotal:        1026268 kB
MemFree:          119324 kB
Buffers:          170796 kB
Cached:           263736 kB
SwapCached:        11428 kB
...
\end{lstlisting}

\IFRU{Предположим, нам нужно узнать первое и третье число, игнорируя второе и остальные}
{Let's consider, we need to get first and third numbers, ignoring second and rest}.
\IFRU{Так это можно сделать}{That is how it can be done}:

\begin{lstlisting}
void read_proc_meminfo()
{
	FILE *f=fopen("/proc/meminfo", "r");
	assert(f);
	unsigned result1, result2;
	if (fscanf (f, "MemTotal:\t%d kB\n"
			"MemFree:\t%*d kB\n"
			"Buffers:\t%d kB\n", 
			&result1, &result2)==2)
		printf ("results: %d %d\n", result1, result2);
	fclose(f);
};
\end{lstlisting}

\IFRU{Строка формата расходится на три строки, в реальности это одна}
{The format string is defined in three lines, it is one in fact}: \ref{heredoc}.
\IFRU{Обратите внимание на}{Please also note} \TT{\textbackslash{}n}, \IFRU{так мы задаем перевод строки}
{that is how newline is defined}.

\TT{*} \IFRU{в модификаторе scanf-строки указывает что число будет прочитано, но никуда записано не будет}
{in the scanf-string modifier pointing out that the number will be read, but will not be stored}.
\IFRU{Таким образом, это поле игнорируется}{Thus, the field is being ignored}. 
scanf()-\IFRU{функции возвращают кол-во не прочитанных полей (здесь
их будет 3) а кол-во записанных полей (2)}{functions are returning not a number of fields read (3 will be here),
buf number of fields stored (2 will be here)}.

\paragraph{\IFRU{Пример}{Example} \#2}

\IFRU{Имеется текстовый файл с парами в каждой строке (ключ-значение)}
{There a text file containing key-value pairs in each string}:

\begin{lstlisting}
some_param1=some_value
some_param2=Lazy fox etc etc.
param3=Lorem Ipsum etc etc.
space here=value containing space
too long param, we should fail here=value
\end{lstlisting}

\IFRU{Нужно просто читать оба поля}{We should just read two fields}:

\begin{lstlisting}
int main(int argc, char *argv[])
{
	assert(argc==2);
	assert(argv[1]);
	FILE *f=fopen (argv[1], "r");
	assert(f);
	int line=1;
	do
	{
		char param[16];
		char value[60];
		if (fscanf (f, "%16[^=]=%60[^\n]\n", param, value)==2)
		{
			printf ("param=%s\n", param);
			printf ("value=%s\n", value);
		}
		else
		{
			printf ("error at line %d\n", line);
			return 0;
		};
		line++;
	} while (!feof(f) && !ferror(f));
};
\end{lstlisting}

\TT{\%16[\^{}=]} ~--- \IFRU{это отдаленно напоминает регулярные выражения}
{is somewhat looks like regular expression}.
\IFRU{Означает, читать 16 любых символов, кроме
знака ``равно'' (=)}{Meaning, to read any 16 characters, except ``equal'' (=) sign}.
\IFRU{Затем, мы указываем scanf()-у, что далее должен быть этот самый знак (=)}{Then we point to scanf() that
there must be this sign (=)}.
\IFRU{Затем
пусть он читает 60 любых символов, кроме символа перевода строки}
{Then let him to read any 60 characters}. \IFRU{В конце читаем символ перевода строки}{We read newline character
at the end}.

\IFRU{Это работает, и поля ограничены длиной 16 и 60 символов}{This works, and field lengths are limited
to 16 and 60 characters}.
\IFRU{Поэтому на 5-й строке предсказуемо происходит ошибка,
ведь там длина парамера (первое поле) длиннее}{That is why error predictabily occuring on the fifth string, because
it has larger length of parameter (first field)}.

\IFRU{Так можно парсить несложные форматы, даже}{Thus it is possible to parse simple file formats, even} CSV
\footnote{Comma-separated values: \url{https://en.wikipedia.org/wiki/Comma-separated_values}}.

\IFRU{Однако, нельзя забывать о том что scanf()-функции не способны прочитать пустую строку там где задается 
модификатор \%s}
{However, it should be noted that scanf()-functions are not able to read empty string where 
\%s modifier is defined}.
\IFRU{Поэтому, этим методом невозможно парсить файл с ключами-значениями,
где есть отсутствующие ключи или значения}{Thus it is not possible to parse a key-value file with absent keys
or values}.

\paragraph{\IFRU{Засада}{Caveat} \#1}

\IFRU{Если использовать \%d в строке формата, scanf() подразумевает что это 32-битный \TT{int} и на x86 и на
x64 процессорах}
{scanf() treat \%d modifier in the format string as 32-bit \TT{int} on both x86 and x64 CPUs}.

\IFRU{Частой ошибкой является писать нечто подобное}{It is a common mistake to write}:

\begin{lstlisting}
char a[10];

scanf ("%d %d %d %d", &a[0], &a[1], &a[2], &[3]);
\end{lstlisting}

\IFRU{Символы (или байты) лежат ``в притык'' друг к другу}
{Characters (or bytes) are placed adjacently to each other}.
\IFRU{Когда}{When} scanf() \IFRU{будет обрабатывать первое значение, он будет считать
его за 32-битный \TT{int}, и ``затрет'' остальные три, рядом лежащие}
{will process first value, it will treat it as 32-bit \TT{int} and overwrite other 3 located near}.
\IFRU{И так далее}{And so on}.



\subsubsection{strspn(), strcspn()}

\index{strspn()}
\TT{strspn()} \IFRU{часто применяется для того чтобы удостовериться, что некая строка полностью состоит из
нужных символов}{is often used to get to be sure that a string has only characters from the list we defined}:
    
\begin{lstlisting}
if (strspn(s, "1234567890") == strlen(s)) ... OK
...
if (strspn(IPv4, "1234567890.") == strlen(IPv4)) ... OK
...
if (strspn(IPv6, "0123456789AaBbCcDdEeFf:.") == strlen(IPv6)) ... OK
\end{lstlisting}

\IFRU{Либо для того чтобы пропустить начало строки}{Or to skip a begin of a string}:

\begin{lstlisting}
const char *whitespaces = " \n\r\t";
*buf += strspn(*buf, whitespaces); // skip whitespaces at start
\end{lstlisting}

\index{strcspn()}
\TT{strcspn()} \IFRU{это обратная ф-ция}{is inverse function},
\IFRU{её можно использовать для пропуска всех символов в начале строки, не попадающих
под множество символов}{it can be used for skipping all symbols at the string beginning,
which are not defined in a set}:

\begin{lstlisting}
s += strcspn(s, whitespaces); // first, skip anything till whitespaces
s += strspn(s, whitespaces); // then skip whitespaces
// here 's' is pointing to the part of string after whitespaces
\end{lstlisting}

\subsubsection{strtok() \AndENRU strpbrk()}

\index{strtok()}
\index{strpbrk()}
\IFRU{Обе ф-ции служат для разбиения строки на подстроки, отделенные друг от друга разделительными символами}
{Both functions are used for delimiting string into substrings, divided by special characters}
\footnote{delimiter}.
\IFRU{Только}{However} strtok() \IFRU{модифицирует исходную строку}{modifies source stirng}
(\IFRU{и таким образом, получаемые подстроки сразу можно использовать как отдельные Си-строки}
{and thus resulting substrings can be used as separated C-strings}), 
\IFRU{а}{but} strpbrk() \IFRU{нет, он только возвращает указатель на следующую подстроку}
{is not, it is only returning a pointer to the next substring}.



\subsection{Unicode}

Unicode это важно! Наиболее популярные способы его применения это:

\begin{itemize}
\item UTF-8
Популярно в UNIX-системах. Сильное приемущество: можно продолжать пользоваться стандартными ф-циями для
обработки строк.

\item UTF-16
Используется в Windows API.
\end{itemize}

\subsubsection{UTF-16}

Под каждый символ отводят 16-битный тип \IT{wchar\_t}.

Для объявления строк с таким типом, используется макрос L:

\begin{lstlisting}
L"hello world"
\end{lstlisting}

Для работы с wchar\_t вместо char, имеется целый класс функций-двойников с символом w в названии,
например: fwprintf(), wcscmp(), wcslen(), iswalpha().

\paragraph{Windows}

В Windows, если некто хочет писать программу сразу в двух версиях, с использованием Unicode и без,
для этого есть тип tchar, в зависимости от объявленной переменной препроцессора UNICODE, 
он будет либо char либо wchar\_t\footnote{Сборка с Unicode и без была популярна во времена популярности
как Windows NT/2000/XP так и Windows 95/98/ME. Вторая линейка плохо поддерживала Unicode}.
Для этого же имеется макрос \TT{\_T(...)}:

\begin{lstlisting}
_T("hello world")
\end{lstlisting}

В зависимости от выставленной переменной препроцессора UNICODE, она будет объявлена как char либо wchar\_t.

В заголовочном файле tchar.h есть масса ф-ций, меняющих свою функцию в зависимости от этой переменной.

\subsection{Списки строк}

Самый простой список строк, это просто набор строк оканчивающийся нулем.
Например, в Windows API, в библиотеке Common Dialogs, 
так\footnote{\url{http://msdn.microsoft.com/en-us/library/windows/desktop/ms646829(v=vs.85).aspx}} 
передаются список допустимых расширений файлов для диалогового окна:

\begin{lstlisting}
// Initialize OPENFILENAME
ZeroMemory(&ofn, sizeof(ofn));
...
ofn.lpstrFilter = "All\0*.*\0Text\0*.TXT\0";
...

// Display the Open dialog box. 

if (GetOpenFileName(&ofn)==TRUE) 
	...
\end{lstlisting}

