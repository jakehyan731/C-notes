\subsection{\IFRU{Списки в Си}{Lists in C}}

\IFRU{Списки это связный набор элементов}{Lists are linked set of elements}.
\IFRU{Односвязный список}{Singly-linked list} ~--- \IFRU{это когда у каждого элемента есть ссылка на следующий}
{it is when each element has pointer to the next one}.
\IFRU{Двусвязный список}{Doubly-linked list} ~--- \IFRU{когда у элемента есть ссылки на следующий и на предыдущий}
{is when each element has pointers to the both previous element and the next one}.

\IFRU{У списков есть серьезное преимущество перед массивами: в список легко добавлять элемент в произвольное место,
так и удалять}
{In comparison with arrays, one significant advantage is ease of new element adding at the random place}.
\IFRU{В качестве недостатков: тратится много памяти для поддержания самих структур списка, а также
нет возможности индексировать список как массив}
{As disadvantages: list supporting data structures consumes some memory overhead, and also it is not
possible to index a list as an array}.

\subsubsection{\IFRU{Односвязный список}{Singly-linked list}}

\IFRU{Самый легкий для реализации}{Simplest to implement}.
\IFRU{В структуре предназначенной для связывания в список, достаточно добавить где-то
ссылку на следующий элемент, обычно это поле называется}
{In the structure intended for linking into a list, it is enough just to add somewhere
a link to the next element, usually this field called} \TT{next}:

\begin{lstlisting}
struct some_object
{
	...
	...
	struct some_object* next;
};
\end{lstlisting}

\TT{NULL} \IFRU{в}{in the} \TT{next} \IFRU{означает что этот элемент является последним в списке}
{meaning that the element is last in the list}.

\IFRU{Операция прохода по такому списку становится очень простой}
{Elements enumerating in this list is very simple}:

\begin{lstlisting}
for (struct some_object *i=list; i!=NULL; i=i->next)
	...
\end{lstlisting}

\IFRU{Для вставки нового элемента, нужно вначале найти последний элемент}
{One need first to find the last element}:

\begin{lstlisting}
for (struct some_object *i=list; i!=NULL; i=i->next);
struct some_object *last_element=i;
\end{lstlisting}

... \IFRU{а затем, создав новую структуру, записать указатель на нее в}
{and then, after creating new structure, store the pointer to it in} \TT{next}:

\begin{lstlisting}
struct some_object *new_object=calloc(1, sizeof(struct some_object));
// populate new_object with data
last_element->next=new_object;
\end{lstlisting}

calloc() \IFRU{отличается от}{is different from the} malloc() 
\IFRU{тем что обнуляет всё выделенное место, а значит в поле \TT{next} нового
элемента сразу будет NULL}
{in that sense that all allocated space will be cleared and consequently, there will be NULL
in the \TT{next} field}
\footnote{\IFRU{Об ``инициализации'' структур, читайте также здесь}
{More about structures ``initialization'', read here}: (\ref{COOPInit}).}.

\IFRU{Поиск нужного элемента это просто проход по всему списку и сравнение каждого элемента с искомым,
до тех пор, пока не найдется то что нужно}
{Searching for the needed element is just enumerating all elements in the list with comparing them
with the sought-for element until it is found}.

\IFRU{Удаление элемента}{Element deletion}: \IFRU{найти предыдущий элемент и следующий, 
у предыдущего в \TT{next} установить указатель на следующий элемент, 
затем освободить блок памяти выделенный для текущего элемента}
{find previous the element and the next, set the \TT{next} pointer in the previous element to the next one,
then free memory block allocated for the current element}.

\IFRU{Самый первый элемент списка называется}{The very first list element is also called} ``list head''.
\IFRU{Структуру самого первого элемента можно определить как локальную
или глобальную переменную}{The very first element structure can be declared as a local or global
variable}.
\IFRU{Но тогда удалять первый элемент списка будет неудобно}
{But it will be harder to delete first element}. 
\IFRU{А с другой стороны,
можно определить указатель на первый элемент списка, 
тогда будет проще этому указателю присвоить другой элемент,
который будет первым}{On the other hand, it is possible to declare the pointer to the first
list element, then it will be easier to assign other element to this pointer (which will be first)}.

\subsubsection{\IFRU{Двусвязный список}{Doubly-linked list}}

\IFRU{Это почти то же самое, только, помимо указателя на следующий элемент,
хранится еще и указатель на предыдущий}{Almost the same, but, aside from the pointer to a next
element, also pointer to a previous element is stored}.
\IFRU{Если элемент первый, то указатель на предыдущий элемент может быть NULL, 
либо он может указывать сам на себя (кому как удобнее)}
{If the element is first, the pointer to the previous element may be NULL, or it may
point to itself (whatever you like)}.

\IFRU{Работая с двусвязным списком, легче находить предыдущие элементы,
например, когда нужно удалить какой-то элемент}
{When working with doubly-linked list, it is easier to find previous elements, e.g.,
in case of element deletion}.
\IFRU{А также можно перебирать элементы с конца списка до начала}{It is easier to enumerate
elements backwards from the end of the list}.
\IFRU{Но памяти на это тратится немного больше}{But the memory overhead is slightly larger}.

\subsubsection{Windows API}

\IFRU{Здесь, да и много где в ядре Windows, применяются две примитивные структуры}
{Here, and also in a lot of places of Windows kernel, two primitive data structures are used}:

\begin{lstlisting}
typedef struct _LIST_ENTRY {
   struct _LIST_ENTRY *Flink;
   struct _LIST_ENTRY *Blink;
} LIST_ENTRY, *PLIST_ENTRY, *RESTRICTED_POINTER PRLIST_ENTRY;

typedef struct _SINGLE_LIST_ENTRY {
    struct _SINGLE_LIST_ENTRY *Next;
} SINGLE_LIST_ENTRY, *PSINGLE_LIST_ENTRY;
\end{lstlisting}

\IFRU{Эти структуры нельзя назвать самостоятельными,
они скорее предназначены для встраивания в другие структуры}
{These structures are not intended for independent use, but rather they are intended for embedding
into another structures}.
\IFRU{Например, нам нужно объеденить в список структуру описывающую цвет}
{For example, we need to unite a color-describing structure into a list}:

\begin{lstlisting}
struct color
{
	int R;
	int G;
	int B;
	LIST_ENTRY list;
};
\end{lstlisting}

\IFRU{Теперь в вашей структуре есть также и ссылка на предыдущий элемент и на следующий}
{Now we have a pointers to the both next and previous elements}.
\IFRU{Для работы со структурами использующие эти списки, в Windows есть набор ф-ций}
{There is a small API present in Windows API using these structures}
\footnote{\url{http://msdn.microsoft.com/en-us/library/windows/hardware/ff563802(v=vs.85).aspx}}.

\subsubsection{Linux}

\IFRU{В ядре Linux работа с простыми двусвязными списками, описывается в файле}
{Doubly-linked list routines in Linux kernel are declared in the file} /include/linux/list.h
\footnote{\url{http://lxr.free-electrons.com/source/include/linux/list.h}}.

\IFRU{Там это много где используется, в ядре версии 3.12 по крайней мере ~2900 упоминаний}
{It is heavily used there, in the kernel version 3.12 there are at least ~2900 references to} 
``struct list\_head''.

\subsubsection{Glib}

\IFRU{Напрашивается мысль, а нельзя ли выделить отдельную структуру для элемента списка, 
и не встраивать лишних полей в свои структуры}
{One might ask, is not it possible to declare a perticular structure for the list element,
and not to embed it to own structures}?
\IFRU{Можно, например, так сделано в}{Yes, for example, that is how it is done in} glist.h
\footnote{\url{https://github.com/GNOME/glib/blob/master/glib/glist.h} 
\url{https://developer.gnome.org/glib/2.37/glib-Doubly-Linked-Lists.html}} \InENRU glib:

\begin{lstlisting}
struct _GList
{
  gpointer data;
  GList *next;
  GList *prev;
};
\end{lstlisting}

\TT{data} \IFRU{может указывать на какой угодно объект, на любую существующую структуру,
в которой вы ничего не хотите менять, это также называется}
{may point to any object you like, to any existing structure in which you want not to change
anything, this is also called} ``opaque pointer''.
\IFRU{Конечно, с эстетической точки зрения, это лучше}{Of course, \ae{}sthetically it is better}.
\IFRU{Но нельзя забывать, что тогда на каждый элемент вашего списка,
будет приходится уже два выделенных блока памяти}{But one should remember that there will be
two allocated memory block for each element of list} + 
\IFRU{еще затраты на поддержания самих блоков памяти в куче}{memory overhead for supporting
allocated blocks in heap}(\ref{HeapOverhead}). \\
\\
\IFRU{Таким образом, подобное решение оправдано там, где экономия памяти менее важна}
{Thus, this approach is acceptable if memory footprint is not important}.

