\section{bsearch(), lfind()}
\label{bsearch_lfind}

Удобные ф-ции для поиска чего-либо где-либо. 

Разница между ними только в том что lfind() просто ищет заданное, а bsearch() требует отсортированный массив
данных, но зато может искать быстрее\footnote{методом половинного деления, итд}
\footnote{разница еще также в том что bsearch() есть в стандарте\cite{C99TC3}, 
а lfind() нет, он есть только в POSIX и в MSVC, но эти ф-ции достаточно просты чтобы их реализовать самому}.

К примеру, поиск строки в массиве указателей на строки:

\lstinputlisting{C/lfind.c}

Свой собственный stricmp() нужен потому что lfind() будет передавать в него указатель на искомую строку,
а также на место в массиве где записан указатель на строку, но не сама строка. Если бы это был
массив строк фиксированной длины, тогда можно было бы воспользоваться стандартным stricmp().

Кстати, точно также ф-ция сравнивания задается и для qsort().

lfind() возвращает место в массиве где ф-ция \TT{my\_stricmp()} сработала выдав 0. Далее вычисляем разницу
между адресом этого места и адресом начала самого массива. Учитывая арифметику указателей\ref{PtrArith}, 
в итоге получается кол-во элементов между этими адресами.

Переделав ф-цию сравнивания, можно искать строку в каком угодно массиве.
Пример из OpenWatcom:

\begin{lstlisting}[caption=\textbackslash{}bld\textbackslash{}pbide\textbackslash{}defgen\textbackslash{}scan.c]
int MyComp( const void *p1, const void *p2 ) {

    KeyWord     *ptr;

    ptr = (KeyWord *)p2;
    return( strcmp( p1, ptr->str ) );
}

static int CheckReservedWords( char *tokbuf ) {
    KeyWord     *match;

    match = bsearch( tokbuf, ReservedWords,
                     sizeof( ReservedWords ) / sizeof( KeyWord ),
                     sizeof( KeyWord ), MyComp );
    if( match == NULL ) {
        return( T_NAME );
    } else {
        return( match->tok );
    }
}
\end{lstlisting}

Здесь имеется отсортированный по первому полю массив структур ReservedWords, выглядящий так:

\begin{lstlisting}[caption=\textbackslash{}bld\textbackslash{}pbide\textbackslash{}defgen\textbackslash{}scan.c]
typedef struct {
    char        *str;
    int         tok;
} KeyWord;

static KeyWord  ReservedWords[] = {
    "__cdecl",          T_CDECL,
    "__export",         T_EXPORT,
    "__far",            T_FAR,
    "__fortran",        T_FORTRAN,
    "__huge",           T_HUGE,
....
};
\end{lstlisting}

bsearch() ищет строку сравнивая её со строкой в первом поле структуры. Здесь можно применить именно bsearch(),
потому что массив уже отсортированный. Иначе пришлось бы использовать lfind().\\
\\
Точно так же можно искать что угодно, где угодно. Пример из BIND 9.9.1
\footnote{\url{https://www.isc.org/downloads/bind/}}:

\begin{lstlisting}[caption=backtrace.c]
static int
symtbl_compare(const void *addr, const void *entryarg) {
	const isc_backtrace_symmap_t *entry = entryarg;
	const isc_backtrace_symmap_t *end =
		&isc__backtrace_symtable[isc__backtrace_nsymbols - 1];

	if (isc__backtrace_nsymbols == 1 || entry == end) {
		if (addr >= entry->addr) {
			/*
			 * If addr is equal to or larger than that of the last
			 * entry of the table, we cannot be sure if this is
			 * within a valid range so we consider it valid.
			 */
			return (0);
		}
		return (-1);
	}

	/* entry + 1 is a valid entry from now on. */
	if (addr < entry->addr)
		return (-1);
	else if (addr >= (entry + 1)->addr)
		return (1);
	return (0);
}

...

	/*
	 * Search the table for the entry that meets:
	 * entry.addr <= addr < next_entry.addr.
	 */
	found = bsearch(addr, isc__backtrace_symtable, isc__backtrace_nsymbols,
			sizeof(isc__backtrace_symtable[0]), symtbl_compare);
	if (found == NULL)
		result = ISC_R_NOTFOUND;
\end{lstlisting}

Таким образом, можно обойтись без того чтобы писать каждый раз цикл for() для перебирания элементов, итд.

