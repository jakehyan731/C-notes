\subsection{bsearch(), lfind()}
\label{bsearch_lfind}

\IFRU{Удобные ф-ции для поиска чего-либо где-либо}{Handy functions for searching for something somewhere}.

\IFRU{Разница между ними только в том что}{The difference between them is that} lfind() 
\IFRU{просто ищет заданное}{just search for data}, \IFRU{а}{buf} bsearch() \IFRU{требует отсортированный массив
данных, но зато может искать быстрее}{requires sorted data array, but may search faster}
\footnote{\IFRU{методом половинного деления, итд}{by bisection method, etc}}
\footnote{\IFRU{разница еще также в том что bsearch() есть в стандарте}{the difference also is that
bsearch() is present in}\cite{C99TC3}, 
\IFRU{а lfind() нет, он есть только в \ac{POSIX} и в \ac{MSVC}}
{but lfind() is not, it is present only in \ac{POSIX} and \ac{MSVC}},
\IFRU{но эти ф-ции достаточно просты чтобы их реализовать самому}{but these functions are simple enough
to implement them on your own}}.

\IFRU{К примеру, поиск строки в массиве указателей на строки}{For example, search for a string
in the array of pointers to a strings}:

\lstinputlisting{C/lfind.c}

\IFRU{Свой собственный}{We need our own} stricmp() \IFRU{нужен потому что}{because} lfind() 
\IFRU{будет передавать в него указатель на искомую строку}{will pass a pointer to the string we looking for},
\IFRU{а также на место в массиве где записан указатель на строку, но не сама строка}
{but also to the place where pointer to the string is stored, not the string itself}.
\IFRU{Если бы это был
массив строк фиксированной длины, тогда можно было бы воспользоваться стандартным stricmp()}
{If it would be an array of fixed-size strings, then usual stricmp() could be used here instead}.

\IFRU{Кстати, точно также ф-ция сравнения задается и для}{By the way, likewise comparison function
is used for} qsort().

lfind() \IFRU{возвращает указатель на место в массиве где ф-ция}{returns a pointer to a place in an array
where the} \TT{my\_stricmp()} \IFRU{сработала выдав}{function was triggered returning} $0$.
\IFRU{Далее вычисляем разницу между адресом этого места и адресом начала самого массива}
{Then we compute a difference between the address of that place and the address of the beginning of an array}.
\IFRU{Учитывая арифметику указателей}{Considering pointer arithmetics}(\ref{PtrArith}),
\IFRU{в итоге получается кол-во элементов между этими адресами}
{we get a number of elements between these addresses}.

\IFRU{Реализуя ф-цию сравнивания, можно искать строку в каком угодно массиве}
{By implementing comparison function, we can search a string in any array}.
\IFRU{Пример из}{Example from} OpenWatcom:

\begin{lstlisting}[caption=\textbackslash{}bld\textbackslash{}pbide\textbackslash{}defgen\textbackslash{}scan.c]
int MyComp( const void *p1, const void *p2 ) {

    KeyWord     *ptr;

    ptr = (KeyWord *)p2;
    return( strcmp( p1, ptr->str ) );
}

static int CheckReservedWords( char *tokbuf ) {
    KeyWord     *match;

    match = bsearch( tokbuf, ReservedWords,
                     sizeof( ReservedWords ) / sizeof( KeyWord ),
                     sizeof( KeyWord ), MyComp );
    if( match == NULL ) {
        return( T_NAME );
    } else {
        return( match->tok );
    }
}
\end{lstlisting}

\IFRU{Здесь имеется отсортированный по первому полю массив структур}
{Here we have array of structures sorted by the first field} \IT{ReservedWords},
\IFRU{выглядящий так}{like}:

\begin{lstlisting}[caption=\textbackslash{}bld\textbackslash{}pbide\textbackslash{}defgen\textbackslash{}scan.c]
typedef struct {
    char        *str;
    int         tok;
} KeyWord;

static KeyWord  ReservedWords[] = {
    "__cdecl",          T_CDECL,
    "__export",         T_EXPORT,
    "__far",            T_FAR,
    "__fortran",        T_FORTRAN,
    "__huge",           T_HUGE,
....
};
\end{lstlisting}

bsearch() \IFRU{ищет строку сравнивая её со строкой в первом поле структуры}{searching for the string comparing
it with the string in the first field of structure}.
\IFRU{Здесь можно применить именно bsearch(), потому что массив уже отсортированный}
{bsearch() can be used here because array is already sorted}.
\IFRU{Иначе пришлось бы использовать lfind()}{Otherwise, lfind() should be used}.
\IFRU{Вероятно, несортированный массив можно вначале отсортировать при помощи qsort(), а затем уже
использовать bsearch(), если вам нравится такая идея}
{Probably, unsorted array can be sorted by qsort() before bsearch() usage, if you like the idea}. \\
\\
\IFRU{Точно так же можно искать что угодно, где угодно}
{Likewise, it is possible to search anything anywhere}. \IFRU{Пример из}{The example from} BIND 9.9.1
\footnote{\url{https://www.isc.org/downloads/bind/}}:

\begin{lstlisting}[caption=backtrace.c]
static int
symtbl_compare(const void *addr, const void *entryarg) {
	const isc_backtrace_symmap_t *entry = entryarg;
	const isc_backtrace_symmap_t *end =
		&isc__backtrace_symtable[isc__backtrace_nsymbols - 1];

	if (isc__backtrace_nsymbols == 1 || entry == end) {
		if (addr >= entry->addr) {
			/*
			 * If addr is equal to or larger than that of the last
			 * entry of the table, we cannot be sure if this is
			 * within a valid range so we consider it valid.
			 */
			return (0);
		}
		return (-1);
	}

	/* entry + 1 is a valid entry from now on. */
	if (addr < entry->addr)
		return (-1);
	else if (addr >= (entry + 1)->addr)
		return (1);
	return (0);
}

...

	/*
	 * Search the table for the entry that meets:
	 * entry.addr <= addr < next_entry.addr.
	 */
	found = bsearch(addr, isc__backtrace_symtable, isc__backtrace_nsymbols,
			sizeof(isc__backtrace_symtable[0]), symtbl_compare);
	if (found == NULL)
		result = ISC_R_NOTFOUND;
\end{lstlisting}

\IFRU{Таким образом, можно обойтись без того чтобы писать каждый раз цикл for() для перебирания элементов, итд}
{Thus, it is possible to get rid of for() loop for enumerating elements each time, etc}.

