\label{name_mangling}
\section{\IFRU{Кодирование имен}{Name mangling}}

\IFRU{В Си к именам прибавляется спереди символ подчеркивания, и ваша ф-ция с именем}
{In the C a underscore symbol is to be prepended before each function name, so} \TT{function} 
\IFRU{может иметь имя в объектом файле}{may in fact have the following name in object file:} \TT{\_function}. \\
\\
\IFRU{В}{The} \CPP 
\IFRU{возможна перегрузка операторов, следовательно, несколько ф-ций могут иметь одно и то же
имя, но разные типы}{has operator overloading, so, several functions may share one name but different
types}.
\IFRU{С другой стороны, загрузчик}{On the other hand,} \ac{OS} 
\IFRU{и линкер могут работать только с обычными именами ф-ций
(или символов) и ничего про \CPP не знают}{are not aware of \CPP and works with plain function names (or symbols)}.
\IFRU{Следовательно, имеется потребность кодировать имя функции,
типы аргументов, тип возвращаемого значения, возможно имя класса, в одну строку}{As a consequence,
there is a need to encode function name, argument types, return value type, and probably a class name into
one line}. \\
\\
\IFRU{К примеру, конструктор класса}{For example, that is how the} \IT{box} 
\IFRU{объявленный следующим образом}{class constructor defined as}: \\

\begin{lstlisting}
box::box(int color, int width, int height, int depth)
\end{lstlisting}

... \IFRU{в соглашениях}{In the} \ac{MSVC} 
\IFRU{будет иметь следующее внутреннее имя}{conventions will have the following name}: 
\TT{??0box@@QAE@HHHH@Z} ~--- \IFRU{например, четыре символа \IT{H} 
означают четыре подряд идущих аргумента типа}{for example, four consequtive \IT{H} characters
means for consequtive argument types} \IT{int}. \\
\\
\IFRU{Это называется}{This is what called} \IT{name mangling}.\\
\\
\IFRU{Вот зачем в заголовочных файлах нужна директива}{That is why header files may contain} \TT{extern "C"}:

\begin{lstlisting}
#ifdef  __cplusplus
extern "C" {
#endif

   void foo(int a, int b);

#ifdef  __cplusplus
}
#endif
\end{lstlisting}

\IFRU{Это означает что}{This mean that} \TT{foo()} \IFRU{написана на Си}{is written in C},
\IFRU{скомпилирована как ф-ция Си и будет иметь имя в объектных файлах}{compiled as C function
and will have the following name in object files:} \TT{\_foo}.

\IFRU{Если этот заголовочный файл подключить к проекту на \CPP, то компилятор будет знать, что имя ф-ции}
{If to include that header in the \CPP project, the compiler will treat its internal name as} \TT{\_foo}.
\IFRU{Без этого объявления, компилятор будет искать в объектных файлах ф-цию с внутренним именем}
{Without this directive, the compiler will look for the function named} \TT{?foo@@YAXHH@Z}. \\
\\
\IFRU{Таким образом, эта директива необходима для подключения Си-библиотек к \CPP-проектам}
{Therefore, this directive is needed for linking C libraries to \CPP-projects}.

\IFRU{А}{And} \IT{ifdef} \IFRU{делает эту директиву видимой только в}{makes this directive
visible only in} \CPP.

\IFRU{Еще о}{More about} \IT{name mangling}: \cite[1.17]{REBook}.
