\subsection{references}

Это то же что и указатели, но обезопасенные (safe), потому что работая с ними, труднее сделать ошибку
\cite[8.3.2]{CPP11}.

\IFRU{Например, reference всегда должен указывать объект того же типа и не может быть NULL}
{For example, reference should be always be pointing to the object of corresponding type and can't be NULL}
\cite[8.6]{ParashiftCPPFAQ}.
\IFRU{Более того, reference нельзя менять, нельзя его заставить указывать на другой объект (reseat)}
{Even more, reference cannot be changed, it's not possible to point to to another object (reseat)}
\cite[8.5]{ParashiftCPPFAQ}.

В \cite[1.7.1]{REBook} можно увидеть, что на уровне x86-кода, это одно и то же.

Как и указатели, reference-ы также можно возвращать из ф-ций, например:

\begin{lstlisting}
#include <iostream>
 
int& use_count()
{
	static int uc=1000; // starting value
	return uc;
};
 
void main()
{
	std::cout << ++use_count() << std::endl;
	std::cout << ++use_count() << std::endl;
	std::cout << ++use_count() << std::endl;
	std::cout << ++use_count() << std::endl;
};
\end{lstlisting}

Это выведет предсказуемое:

\begin{lstlisting}
1001
1002
1003
1004
\end{lstlisting}

Если посмотреть что там сгенерировалось на ассемблере, то можно увидеть что \TT{use\_count()} просто возвращает
указатель на \TT{uc}, а в \TT{main()} происходит инкремент значения по этому указателю.

