\section{Ввод/вывод}

Часто есть необходимость выводить в ostream целые структуры, а каждый раз выводить их по одному полю это
неудобно. Иногда это решается добавлением метода ToString() в класс. 
Другое решение это сделать ``свободную'' ф-цию (free function)
\footnote{Ф-ции не являющиеся методом какого-либо класса} для вывода вроде:

\begin{lstlisting}
ostream& operator<< (ostream &out, const Object &in)
{
    out << "Object. size=" << in.size << " value=" << in.value << " ";

    return out;
};
\end{lstlisting}

После этого можно отправлять экземпляры класса прямо в ostream:

\begin{lstlisting}
Object o1, o2;
...
cout << "o1=" << o1 << " o2=" << o2 << endl;
\end{lstlisting}

А для того чтобы ф-ция вывода могла обращаться к любым полям класса, её можно сделать friend.
Впрочем, имеется также такая точка зрения, что подобные ф-ции не следует делать friend
для сохранения энкапсуляции
\cite[Item 23 Prefer non-member non-friend functions to member functions]{EffectiveCPP}.
