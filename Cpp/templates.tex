\section{\IFRU{Темплейты}{Templates}}

\IFRU{Темплейты нужны обычно для того чтобы сделать класс универсальным для нескольких типов данных}
{Templates are usually necessary in order to make a class universal for several data types}.
\IFRU{К примеру}{For example}, \TT{std::string} \IFRU{в реальности это}{is in fact} \TT{std::basic\_string<char>}, \\ 
\IFRU{а}{and} \TT{std::wstring} \IFRU{это}{is} \TT{std::basic\_string<wchar\_t>}. \\
\\
\IFRU{Нередко подобное делают и для типов}{It is often done for data types like} \IT{float/double/complex} 
\IFRU{и даже}{and even} \IT{int}.
\IFRU{Некий математический алгоритм может быть описан один раз,
но скомпилирован сразу в нескольких версиях, для работы со всеми этими типами данных}
{Some mathematical algorithm can be defined only once, but be compiled in several versions for all
these data types}. \\
\\
\IFRU{Таким образом, можно описывать алгоритмы только один раз, но работать они будут для разных типов}
{Thus it is possible to define algorithms only once, but they will work for several data types}. \\
\\
\IFRU{Простейшие примеры это ф-ции}{Simplest examples are the} \IT{max, min, swap}
\IFRU{, работающие для любых типов, переменные которых можно сравнивать и присваивать}
{ functions working for any type, variables of which can be compared and assigned}.
\IFRU{Потом, вы можете написать свою реализацию}{Then you may want to write your own} \gls{BigInt}\IFRU{}{ implementation},
\IFRU{и если там присутствует 
оператор сравнения двух объектов}{and if there is a comparison operator} (\TT{operator<})\IFRU{}{ is present},
\IFRU{то написанные раннее}{then written earlier} \IT{max/min} \IFRU{будут работать и для нового класса}
{will work for the new class as well}.\\
\\
\IFRU{Вот почему списки и прочие контейнеры в}{That is why lists and other containers in the} \ac{STL} 
\IFRU{это именно темплейты}{are exactly templates}: \IFRU{они как бы}{it can be said, they}
``\IFRU{встраивают}{embedds}'' \IFRU{в ваш класс возможность объеденять в списки, итд}{the possibility of be united
into list or collection to your class}.

