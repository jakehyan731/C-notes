\section{Темплейты}

Темплейты нужны обычно для того чтобы сделать класс универсальным для нескольких типов данных.
К примеру, \TT{std::string} в реальности это \TT{std::basic\_string<char>}, \\ 
а \TT{std::wstring} это \TT{std::basic\_string<wchar\_t>}. \\
\\
Нередко подобное делают и для типов float/double/complex и даже int. 
Некий математический алгоритм может быть описан один раз,
но скомпилирован сразу в нескольких версиях, для работы со всеми этими типами данных. \\
\\
Таким образом, можно описывать алгоритмы только один раз, но работать они будут для разных типов. \\
\\
Простейшие примеры это ф-ции max, min, swap, работающие для любых типов, которые можно сравнивать
и присваивать. Потом, вы можете написать свою реализацию BigInt
\footnote{так обычно называют библиотеки для работы с числами произвольной точности}, 
и если там присутствует 
оператор сравнения двух объектов (\TT{operator<}),
то написанные раннее max/min будут работать и для нового класса.\\
\\
Вот почему списки и прочие контейнеры в \ac{STL} это именно темплейты: они как бы 
``встраивают'' в ваш класс возможность объеденять в списки, итд.

