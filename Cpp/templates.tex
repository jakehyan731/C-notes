\section{Темплейты}

Темплейты нужны обычно для того чтобы сделать класс универсальным для нескольких типов данных.
К примеру, \TT{std::string} в реальности это \TT{std::basic\_string<char>}, \\ 
а \TT{std::wstring} это \TT{std::basic\_string<wchar\_t>}. \\
\\
Нередко подобное делают и для типов float/double/complex и даже int. 
Некий математический алгоритм может быть описан один раз,
но скомпилирован сразу в нескольких версиях, для работы со всеми этими типами данных. \\
\\
Таким образом, можно описывать алгоритмы только один раз, но работать они будут для разных типов. \\
\\
Простейшие примеры это ф-ции max, min, swap, работающие для любых типов, которые можно сравнивать
и присваивать.


