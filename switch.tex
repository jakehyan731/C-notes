\section{switch}

Иногда можно устать писать одно и то же:

\begin{lstlisting}
switch(...)
{
	case 0:
	case 1:
	case 2:
	case 3:
		fn1();
		break;
	case 4:
	case 5:
	case 6:
	case 7:
		fn2();
		break;
};
\end{lstlisting}

Это нестандартное расширение GCC
\footnote{\url{http://gcc.gnu.org/onlinedocs/gcc/Case-Ranges.html}} может немного всё упростить:

\begin{lstlisting}
switch(...)
{
	case 0 ... 3:
		fn1();
		break;
	case 4 ... 7:
		fn2();
		break;
};
\end{lstlisting}

Так что если в планах имеется использовать только компилятор GCC, то можно делать так.

